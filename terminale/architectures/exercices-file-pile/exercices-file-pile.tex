\documentclass[a4paper,11pt]{article}
\input{/home/tof/Documents/Cozy/latex-include/preambule_lua.tex}
\newcommand{\showprof}{show them}  % comment this line if you don't want to see todo environment
\fancyhead[L]{Exercices file - pile}
\newdate{madate}{10}{09}{2020}
\fancyhead[R]{Terminale - NSI} %\today
\fancyfoot[L]{~\\Christophe Viroulaud}
\fancyfoot[C]{\textbf{Page \thepage}}
\fancyfoot[R]{\includegraphics[width=2cm,align=t]{/home/tof/Documents/Cozy/latex-include/cc.png}}

\begin{document}
\begin{Form}
\begin{exo}
Dans une version simplifiée, nous pourrions utiliser les structures proposées par Python pour implémenter les piles et files.
\begin{enumerate}
\item Créer une classe \emph{Pile} qui possédera un attribut \emph{donnees}. Cet attribut sera une liste Python vide.
\item Implémenter les méthodes d'une pile.
\item Implémenter une classe \emph{File} sur le même modèle.
\end{enumerate}
\end{exo}
\begin{exo}
L'historique d'un navigateur web conserve la page récemment visitée. De plus, le bouton \emph{retour arrière} permet de revenir à la page précédemment explorée.
\begin{enumerate}
\item Quelle structure de données permet d'implémenter le comportement de l'historique de navigation?
\item Le fichier \emph{historique.csv} contient une liste d'adresses web récemment visitées. 
\end{enumerate}
\end{exo}
\end{Form}
\end{document}