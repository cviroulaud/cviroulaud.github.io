\documentclass[a4paper,11pt]{article}
\input{/home/tof/Documents/Cozy/latex-include/preambule_doc.tex}
\input{/home/tof/Documents/Cozy/latex-include/preambule_commun.tex}
\newcommand{\showprof}{show them}  % comment this line if you don't want to see todo environment
\setlength{\fboxrule}{0.8pt}
\fancyhead[L]{\fbox{\Large{\textbf{DonRep 12}}}}
\fancyhead[C]{\textbf{Bilan types construits}}
\newdate{madate}{10}{09}{2020}
%\fancyhead[R]{\displaydate{madate}} %\today
\fancyhead[R]{Première - NSI}
\fancyfoot[L]{\vspace{1mm}Christophe Viroulaud}
\AtEndDocument{\label{lastpage}}
\fancyfoot[C]{\textbf{Page \thepage/\pageref{lastpage}}}
\fancyfoot[R]{\includegraphics[width=2cm,align=t]{/home/tof/Documents/Cozy/latex-include/cc.png}}

\begin{document}
\section{Tuple}
Un tuple est un n-uplet non mutable.
\begin{lstlisting}[language=Python  , xleftmargin=2em, xrightmargin=2em]
tup = (3, 5, 9, 8)
# lire un élément
tup[2] # renvoie l'entier 9

# Il n'est pas possible de modifier le contenu d'un tuple.
\end{lstlisting}
\section{Tableau (list)}
Un tableau contient des éléments de même type (entiers, booléens\dots) repérés par leur indice (position) dans le tableau. Les indices commencent à zéro.
\begin{lstlisting}[language=Python  , xleftmargin=2em, xrightmargin=2em]
tab = [3, 5, 9, 8]
# lire un élément
tab[2] # renvoie l'entier 9

# écrire un élément
tab[2] = 10 # le 9 est remplacé par 10
\end{lstlisting}
En Python, les tableaux se nomment des \textbf{\texttt{list}}.
\section{Dictionnaire}
Un dictionnaire contient des éléments repérés par une clé. Une clé est un élément non mutable: entier, chaîne de caractères, tuple.
\begin{lstlisting}[language=Python  , xleftmargin=2em, xrightmargin=2em]
dico = {"prems": 18, "deuz": 13, "troiz": 9}
# lire un élément
dico["deuz"] # renvoie l'entier 13

# écrire un élément
tab["deuz"] = 10 # le 13 est remplacé par 10
\end{lstlisting}

\section{Construction par compréhension}
En Python il est possible de construire une structure de données de manière rapide et efficace.
\begin{center}
    \begin{lstlisting}[language=Python  , xleftmargin=2em, xrightmargin=2em]
tup = tuple(0 for i in range(5))
# tup = (0, 0, 0, 0, 0)

tab = [0 for i in range(5)]
# tab = [0, 0, 0, 0, 0]

dico = {i: 0 for i in range(5)}
# dico = {0:0, 1:0, 2:0, 3:0, 4:0}
\end{lstlisting}
    \captionof{code}{Construction par compréhension}
    \label{CODE}
\end{center}
\end{document}