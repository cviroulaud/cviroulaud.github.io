\documentclass[a4paper,11pt]{article}
\input{/home/tof/Documents/Cozy/latex-include/preambule_doc.tex}
\input{/home/tof/Documents/Cozy/latex-include/preambule_commun.tex}
\newcommand{\showprof}{show them}  % comment this line if you don't want to see todo environment
\setlength{\fboxrule}{0.8pt}
\fancyhead[L]{\fbox{\Large{\textbf{Loc 02}}}}
\fancyhead[C]{\textbf{Géoportail - correction}}
\newdate{madate}{10}{09}{2020}
%\fancyhead[R]{\displaydate{madate}} %\today
\fancyhead[R]{Seconde - SNT}
%\fancyhead[R]{Première - NSI}
%\fancyhead[R]{Terminale - NSI}
\fancyfoot[L]{\vspace{1mm}Christophe Viroulaud}
\AtEndDocument{\label{lastpage}}
\fancyfoot[C]{\textbf{Page \thepage/\pageref{lastpage}}}
\fancyfoot[R]{\includegraphics[width=2cm,align=t]{/home/tof/Documents/Cozy/latex-include/cc.png}}

\begin{document}
\begin{activite}
    Le fond \emph{Plan IGN} fournit de nombreuses informations:
    \begin{itemize}
        \item noms des villes, des lieu-dits\dots
        \item tracés des routes nationales, départementales, autoroutes.
        \item indications sur le relief (zones ombrées).
        \item selon le niveau de zoom, on peut également voir les bâtiments.
    \end{itemize}
Le fond \emph{Parcelles cadastrales} donnent le découpage des propriétés (ou parcelles).
\end{activite}
\begin{activite}
Le lycée est sur la parcelle 0338, sur une surface de 16800m². Les coordonnées GPS de l'internat sont:
\begin{itemize}
    \item Latitude: 45,181330°N
    \item Longitude: 0,710249°E
\end{itemize}
Les indications Nord et Est précise la position du lieu par rapport à l'équateur et au méridien de Greenwich (voir cours précédent). Il est à noter que les coordonnées GPS sont très précises: un décalage infime de la souris sur l'écran donne des valeurs légèrement différentes.
\end{activite}
\begin{activite}
Plus les courbes de niveau sont rapprochées, plus le terrain est pentue. C'est une indication qu'il est important de repérer lors d'une randonnée. Le professeur a roulé sur les communes de Périgueux et Trélissac. Il a longé la commune de Champcevinel.
\end{activite}
\end{document}