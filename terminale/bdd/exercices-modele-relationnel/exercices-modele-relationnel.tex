\documentclass[a4paper,11pt]{article}
\input{/home/tof/Documents/Cozy/latex-include/preambule_doc.tex}
\input{/home/tof/Documents/Cozy/latex-include/preambule_commun.tex}
\newcommand{\showprof}{show them}  % comment this line if you don't want to see todo environment
\setlength{\fboxrule}{0.8pt}
\fancyhead[L]{\fbox{\Large{\textbf{BDD 02}}}}
\fancyhead[C]{\textbf{Exercices modèle relationnel}}
\newdate{madate}{10}{09}{2020}
%\fancyhead[R]{\displaydate{madate}} %\today
\fancyhead[R]{Terminale - NSI}
\fancyfoot[L]{\vspace{1mm}Christophe Viroulaud}
\AtEndDocument{\label{lastpage}}
\fancyfoot[C]{\textbf{Page \thepage/\pageref{lastpage}}}
\fancyfoot[R]{\includegraphics[width=2cm,align=t]{/home/tof/Documents/Cozy/latex-include/cc.png}}

\begin{document}
\begin{exo}
On souhaite modéliser un annuaire téléphonique simple dans lequel chaque personne, identifiée par son nom et son prénom, est associée à son numéro de téléphone.
\begin{enumerate}
\item Proposer un modèle relationnel de cet annuaire.
\item Dire si chacun des ensembles suivants est une relation valide pour le schéma \emph{Annuaire}:
\begin{enumerate}
\item $\{\}$
\item $\{("Dupont","Jean","012345678")\}$
\item $\{("Dupont","Jean","012345678"), ("Durant","Jacques","012345678")\}$
\item $\{("Dupont","Jean","012345678"), ("Dupont","Jean","896789")\}$
\item $\{("Dupont","Jean","012345678"), ("Durant","Jacques")\}$
\item $\{("Dupont","Jean",896789)\}$
\end{enumerate}
\end{enumerate}
\end{exo}
\begin{exo}
On se propose de décrire le schéma d'un bulletin scolaire d'élève par le modèle relationnel suivant:
\begin{itemize}
\item Eleves(\underline{nom \emph{String}, prenom \emph{String}})
\item Matieres(intitule \emph{String}, \underline{id \emph{Integer}})
\item Notes(\underline{\dashuline{nom\_eleve} \emph{String}, \dashuline{id\_matiere} \emph{Integer}}, notes \emph{Integer})
\end{itemize}
\begin{enumerate}
\item Relever et corriger les erreurs de ce schéma.
\item Remplir chaque relation avec au moins deux entités.
\end{enumerate}
\end{exo}
\begin{exo}
Métro parisien
\begin{enumerate}
\item Construire un modèle relationnel du métro parisien. Ce-dernier est composé de \emph{stations} qui forment une \emph{ligne}. Il peut circuler plusieurs lignes dans une même station.
\item Ajouter une relation qui contiendra les horaires de passage de chaque ligne, pour chaque station.
\end{enumerate}

\end{exo}
\end{document}