\documentclass[svgnames,11pt]{beamer}
\input{/home/tof/Documents/Cozy/latex-include/preambule_commun.tex}
\input{/home/tof/Documents/Cozy/latex-include/preambule_beamer.tex}
%\usepackage{pgfpages} \setbeameroption{show notes on second screen=left}
\author[]{Christophe Viroulaud}
\title{TP Terminus}
\date{\framebox{\textbf{ArchMat 10}}}
%\logo{}
\institute{Première - NSI}

\begin{document}
\begin{frame}
\titlepage
\end{frame}
\begin{frame}
    \frametitle{}

    Nous avons l'habitude d'interagir avec le système d'exploitation grâce à son interface graphique (bureau, explorateur\dots). Il reste cependant pertinent de savoir utiliser les commandes \emph{en mode console} lors de l'intervention sur un serveur à distance par exemple.

\end{frame}
\begin{frame}
    \frametitle{}

    \begin{framed}
        \centering Quelles commandes principales permettent d'interagir avec le système d'exploitation?
    \end{framed}

\end{frame}
\section{Commandes de base}
\subsection{Se déplacer}
\begin{frame}[fragile]
    \frametitle{Commandes de base - Se déplacer}

    Pour interagir avec le système de fichiers, il faut le connaître:
\begin{center}
\begin{lstlisting}[language=Bash , basicstyle=\ttfamily\small, xleftmargin=1em, xrightmargin=1em]
# Lire le contenu d'un fichier
cat mon_fichier

# Lister le contenu d'un répertoire
ls

# Se déplacer dans les répertoires
cd nom_repertoire
\end{lstlisting}
\end{center}

\end{frame}
\begin{frame}
    \frametitle{}

    \begin{activite}
    \begin{enumerate}
        \item Ouvrir la machine virtuelle.
        \item Ouvrir un \textbf{terminal}.
    \end{enumerate}
    \end{activite}

\end{frame}
\end{document}