\documentclass[a4paper,11pt]{article}
\input{/home/tof/Documents/Cozy/latex-include/preambule_lua.tex}
\newcommand{\showprof}{show them}  % comment this line if you don't want to see todo environment
\fancyhead[L]{Correction exercices requêtes avancées}
\newdate{madate}{10}{09}{2020}
%\fancyhead[R]{\displaydate{madate}} %\today
%\fancyhead[R]{Seconde - SNT}
%\fancyhead[R]{Première - NSI}
\fancyhead[R]{Terminale - NSI}
\fancyfoot[L]{~\\Christophe Viroulaud}
\AtEndDocument{\label{lastpage}}
\fancyfoot[C]{\textbf{Page \thepage/\pageref{lastpage}}}
\fancyfoot[R]{\includegraphics[width=2cm,align=t]{/home/tof/Documents/Cozy/latex-include/cc.png}}

\begin{document}
\begin{Form}
\begin{exo}
La clinique vétérinaire utilise la base de données construite précédemment.
\begin{enumerate}
\item Ouvrir la base \emph{animaux.db} .
\item Écrire les requêtes simples pour:
\begin{enumerate}
\item \begin{lstlisting}[language=SQL]
INSERT INTO Especes(nom) VALUES ("poisson");
\end{lstlisting}
\item \begin{lstlisting}[language=SQL]
INSERT INTO Soins(id_animal,soin) VALUES (3,"patte cassée");
\end{lstlisting}
\item \begin{lstlisting}[language=SQL]
SELECT * FROM Animaux WHERE age >= 10 ORDER BY nom;
\end{lstlisting}
\item 
\begin{lstlisting}[language=SQL]
SELECT DISTINCT(soin) FROM Soins;
\end{lstlisting}
\item 
\begin{lstlisting}[language=SQL]
SELECT COUNT(soin) FROM Soins;
\end{lstlisting}
\item 
\begin{lstlisting}[language=SQL]
SELECT COUNT(DISTINCT(soin)) FROM Soins;
\end{lstlisting}
\item 
\begin{lstlisting}[language=SQL]
SELECT COUNT(soin) FROM Soins WHERE soin = "stérilisation";
\end{lstlisting}
\item 
\begin{lstlisting}[language=SQL]
UPDATE Animaux SET nom="charlie" WHERE nom="charly";
\end{lstlisting}
\end{enumerate}
\item Écrire les requêtes pour:
\begin{enumerate}
\item 
\begin{lstlisting}[language=SQL]
SELECT Especes.nom FROM Especes
JOIN Animaux ON Especes.id = Animaux.id_espece
WHERE Animaux.nom="zappy";
\end{lstlisting}
\item 
\begin{lstlisting}[language=SQL]
SELECT Animaux.nom FROM Animaux
JOIN Especes ON Especes.id = Animaux.id_espece
WHERE Especes.nom="chat"
ORDER BY Animaux.nom;
\end{lstlisting}
\item 
\begin{lstlisting}[language=SQL]
SELECT COUNT(Animaux.nom) FROM Animaux
JOIN Especes ON Especes.id = Animaux.id_espece
WHERE Especes.nom="chien";
\end{lstlisting}
\item 
\begin{lstlisting}[language=SQL]
SELECT COUNT(Soins.soin) FROM Soins
JOIN Animaux ON Animaux.id = Soins.id_animal
WHERE Animaux.nom="chouchou";
\end{lstlisting}
\item Il faut remarquer l'utilisation du mot-clé \textbf{AS} pour définir un alias.
\begin{lstlisting}[language=SQL]
SELECT Animaux.nom AS n FROM Animaux
JOIN Soins ON Animaux.id = Soins.id_animal
JOIN Especes ON Animaux.id_espece = Especes.id
WHERE Especes.nom="chat" AND Soins.soin="stérilisation"
ORDER BY n;
\end{lstlisting}
\item
\begin{lstlisting}[language=SQL]
INSERT INTO Animaux(nom,age,id_espece) VALUES ("bubulle",3,(SELECT id FROM Especes WHERE nom="poisson")); 
\end{lstlisting},
\item
\begin{lstlisting}[language=SQL]
INSERT INTO Soins(id_animal, soin) VALUES ((SELECT id FROM Animaux WHERE nom="cartman"),"stérilisation");
\end{lstlisting}
\end{enumerate}
\item Créer une table \emph{Types\_soins} qui référence les différentes possibilités de soins. Il faut alors remplacer l'attribut\emph{soin} de la table \emph{Soins} par une clé étrangère sur l'\emph{id} de \emph{Types\_soins}.
\item 
\begin{itemize}
\item Especes(\underline{id Integer}, nom String)
\item Animaux(\underline{id Integer}, nom String, age Integer, \dashuline{id\_espece Integer})
\item Soins(\underline{id Integer}, \dashuline{id\_animal Integer}, \dashuline{id\_soin Integer})
\item Types\_soins(\underline{id Integer}, soin String)
\end{itemize}


\end{enumerate}
\end{exo}
\end{Form}
\end{document}