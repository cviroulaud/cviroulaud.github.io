\documentclass[a4paper,11pt]{article}
\input{/home/tof/Documents/Cozy/latex-include/preambule_lua.tex}
\newcommand{\showprof}{show them}  % comment this line if you don't want to see todo environment
\fancyhead[L]{Exercices arbre}
\newdate{madate}{10}{09}{2020}
%\fancyhead[R]{\displaydate{madate}} %\today
%\fancyhead[R]{Seconde - SNT}
%\fancyhead[R]{Première - NSI}
\fancyhead[R]{Terminale - NSI}
\fancyfoot[L]{~\\Christophe Viroulaud}
\AtEndDocument{\label{lastpage}}
\fancyfoot[C]{\textbf{Page \thepage/\pageref{lastpage}}}
\fancyfoot[R]{\includegraphics[width=2cm,align=t]{/home/tof/Documents/Cozy/latex-include/cc.png}}

\begin{document}
\begin{Form}
\begin{exo}
Un site web de cuisine stocke chaque recette sous forme d'un fichier \emph{json (JavaScript Object Notation)}.
\begin{enumerate}
\item Télécharger l'annexe \emph{annexe-exercice-arbre.zip} et extraire \emph{recette-fondant.json} .
\item Ouvrir le fichier à l'aide d'un éditeur de texte et construire sur papier l'arbre correspondant à la recette. Il faut remarquer que dans le \emph{json} il n'y a pas de nœud racine.
\end{enumerate}
\end{exo}
\end{Form}
\end{document}