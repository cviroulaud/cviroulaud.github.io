\documentclass[svgnames,11pt]{beamer}
\input{/home/tof/Documents/Cozy/latex-include/preambule_commun.tex}
\input{/home/tof/Documents/Cozy/latex-include/preambule_beamer.tex}
%\usepackage{pgfpages} \setbeameroption{show notes on second screen=left}
\author[]{Christophe Viroulaud}
\title{Exercices requêtes avancées\\correction}
\date{\framebox{\textbf{BDD 07}}}
%\logo{}
\institute{Terminale - NSI}

\begin{document}
\begin{frame}
\titlepage
\end{frame}
\begin{frame}[fragile]
    \frametitle{}

\begin{center}
\begin{lstlisting}[language=SQL , basicstyle=\ttfamily\small, xleftmargin=1em, xrightmargin=0em]
INSERT INTO Especes(nom) VALUES ("poisson");
\end{lstlisting}
\captionof{code}{Insérer l'espèce \emph{poisson}}
\label{CODE}
\end{center}

\end{frame}
\begin{frame}[fragile]
    \frametitle{}

\begin{center}
\begin{lstlisting}[language=SQL , basicstyle=\ttfamily\small, xleftmargin=1em, xrightmargin=0em]
INSERT INTO Soins(id_animal,soin) VALUES (3,"patte cassée");
\end{lstlisting}
\captionof{code}{Insérer le soin \emph{patte cassée}}
\label{CODE}
\end{center}

\end{frame}
\begin{frame}[fragile]
    \frametitle{}

\begin{center}
\begin{lstlisting}[language=SQL , basicstyle=\ttfamily\small, xleftmargin=1em, xrightmargin=0em]
SELECT * FROM Animaux WHERE age >= 10 ORDER BY nom;
\end{lstlisting}
\captionof{code}{Sélectionner les animaux de plus de 10 ans}
\label{CODE}
\end{center}

\end{frame}

\begin{frame}[fragile]
    \frametitle{}

\begin{center}
\begin{lstlisting}[language=SQL , basicstyle=\ttfamily\small, xleftmargin=1em, xrightmargin=0em]
SELECT DISTINCT(soin) FROM Soins;
\end{lstlisting}
\captionof{code}{Sélectionner les soins distincts}
\label{CODE}
\end{center}

\end{frame} 
\begin{frame}[fragile]
    \frametitle{}

\begin{center}
\begin{lstlisting}[language=SQL , basicstyle=\ttfamily\small, xleftmargin=1em, xrightmargin=0em]
SELECT COUNT(soin) FROM Soins;
\end{lstlisting}
\captionof{code}{Compter les soins}
\label{CODE}
\end{center}

\end{frame} 
\begin{frame}[fragile]
    \frametitle{}

\begin{center}
\begin{lstlisting}[language=SQL , basicstyle=\ttfamily\small, xleftmargin=1em, xrightmargin=0em]
SELECT COUNT(DISTINCT(soin)) FROM Soins;
\end{lstlisting}
\captionof{code}{Compter tous les soins distincts}
\label{CODE}
\end{center}

\end{frame} 
\begin{frame}[fragile]
    \frametitle{}

\begin{center}
\begin{lstlisting}[language=SQL , basicstyle=\ttfamily\small, xleftmargin=1em, xrightmargin=0em]
SELECT COUNT(soin) FROM Soins WHERE soin = "stérilisation";
\end{lstlisting}
\captionof{code}{Compter les stérilisations}
\label{CODE}
\end{center}

\end{frame} 
\begin{frame}[fragile]
    \frametitle{}

\begin{center}
\begin{lstlisting}[language=SQL , basicstyle=\ttfamily\small, xleftmargin=.5em, xrightmargin=0em]
UPDATE Animaux SET nom="charlie" WHERE nom="charly";
\end{lstlisting}
\captionof{code}{Renommer Charly}
\label{CODE}
\end{center}

\end{frame} 
\end{document}