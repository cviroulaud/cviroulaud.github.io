\documentclass[a4paper,11pt]{article}
\input{/home/tof/Documents/Cozy/latex-include/preambule_lua.tex}
\newcommand{\showprof}{show them}  % comment this line if you don't want to see todo environment
\fancyhead[L]{Exercices constructions élémentaires}
\newdate{madate}{10}{09}{2020}
\fancyhead[R]{Première - NSI} %\today
\fancyfoot[L]{~\\Christophe Viroulaud}
\fancyfoot[C]{\textbf{Page \thepage}}
\fancyfoot[R]{\includegraphics[width=2cm,align=t]{/home/tof/Documents/Cozy/latex-include/cc.png}}

\begin{document}
\begin{Form}
\begin{exo}
Répondre aux questions sans ordinateur.
\begin{enumerate}
\item Quelle est la valeur affichée par l'interprète après la séquence suivante?
\begin{lstlisting}
>>> a = 3
>>> a = 4
>>> a = a+2
>>> a
\end{lstlisting}
\item Même question
\begin{lstlisting}
>>> a = 2
>>> b = a*a
>>> b = a*b
>>> b = b*b
>>> b
\end{lstlisting}
\item Qu'affichent les instructions suivantes?
\begin{lstlisting}
print("i+")
print(i+)
\end{lstlisting}
\item Que fait la séquence suivante?
\begin{lstlisting}
a = 2
b = 3
tmp = a
a = b
b = tmp
\end{lstlisting}
\begin{commentprof}
évoquer
\begin{lstlisting}
a,b = b,a
\end{lstlisting}
\end{commentprof}
\end{enumerate}
\end{exo}
\begin{center}
\shadowbox{\parbox{15cm}{\centering Dans l'espace personnel ou sur le lycée connecté, créer un dossier \emph{exercices-constructions-elementaires}. Chaque nouveau programme sera crée dans un nouveau fichier avec un nom explicite. D'une manière générale, éviter accents et espaces.}}
\end{center}
\begin{exo}
Écrire un programme qui demande à l'utilisateur les longueurs des côtés d'un rectangle et qui affiche son aire. Les longueurs seront des valeurs entières.
\end{exo}
\begin{exo}
Écrire un programme qui demande l'âge de l'utilisateur et qui affiche s'il est majeur ou mineur.
\end{exo}
\begin{exo}
Écrire un programme qui demande l'âge de l'utilisateur et qui renvoie le prix de l'abonnement de la carte cinéma à payer:
\begin{itemize}
\item 10€ si strictement moins de 16 ans,
\item 15€ si entre 16 et 25 ans,
\item 25€ si entre 26 et 59 ans,
\item 15€ si 60 ans ou plus.
\end{itemize}
\end{exo}
\begin{exo}
\begin{enumerate}
\item Tester les instructions ci-après et expliquer ce qu'elles renvoient.
\begin{lstlisting}
>>> 20/3
>>> 20//3
>>> 20%3
\end{lstlisting}
\item Écrire un programme qui demande un nombre de secondes et affiche le nombre d'heures, minutes, secondes correspondantes.
\item Améliorer le programme pour qu'il affiche chaque résultat sur deux digits (ajouter un 0 si le résultat est inférieur à 10).
\end{enumerate}
\end{exo}
\begin{exo}
Écrire un programme qui demande un nombre entre 1 et 10 et affiche la table de multiplication correspondante.
\end{exo}
\begin{exo}
Écrire un programme qui affiche un compte à rebours en partant de 10 jusqu'à 0.
\end{exo}
\begin{exo}
Écrire un programme qui affiche tous les nombres pairs entre 2 et 25.
\end{exo}
\begin{exo}
\begin{enumerate}
\item Écrire un programme qui demande 10 notes sur 20 à l'utilisateur et renvoie la moyenne.
\item Compléter le programme pour qu'il affiche \emph{Félicitations} si la moyenne est supérieure ou égale à 15, \emph{Bon travail} si elle est comprise entre 10 et 15, \emph{Doit fournir des efforts} si elle est inférieure à 10.
\end{enumerate}
\end{exo}
\begin{exo}
Devinette
\begin{enumerate}
\item Écrire un programme qui demande à l'utilisateur de penser à un nombre entre 0 et 100 puis qui devine ce nombre. Le programme posera la question \guill{Le nombre est-il ...?} tant qu'il n'a pas trouvé et attendra une des trois réponses:
\begin{itemize}
\item = si le nombre a été trouvé,
\item + si le nombre à deviner est plus grand que celui proposé,
\item - si le nombre à deviner est plus petit que celui proposé.
\end{itemize}
\item Ajouter une variable qui compte le nombre de questions posées et l'affiche.
\item Tester le programme dix fois. Noter le nombre de coups joués.
\item Quel est le nombre minimum de questions que doit poser l'ordinateur pour être certain de gagner? 
\end{enumerate}
\end{exo}
\end{Form}
\end{document}