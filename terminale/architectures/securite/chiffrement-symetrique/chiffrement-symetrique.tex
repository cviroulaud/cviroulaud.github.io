\documentclass[a4paper,11pt]{article}
\input{/home/tof/Documents/Cozy/latex-include/preambule_doc.tex}
\input{/home/tof/Documents/Cozy/latex-include/preambule_commun.tex}
\newcommand{\showprof}{show them}  % comment this line if you don't want to see todo environment
\setlength{\fboxrule}{0.8pt}
\fancyhead[L]{\fbox{\Large{\textbf{Secu 01}}}}
\fancyhead[C]{\textbf{Chiffrement symétrique}}
\newdate{madate}{10}{09}{2020}
%\fancyhead[R]{\displaydate{madate}} %\today
%\fancyhead[R]{Seconde - SNT}
%\fancyhead[R]{Première - NSI}
\fancyhead[R]{Terminale - NSI}
\fancyfoot[L]{\vspace{1mm}Christophe Viroulaud}
\AtEndDocument{\label{lastpage}}
\fancyfoot[C]{\textbf{Page \thepage/\pageref{lastpage}}}
\fancyfoot[R]{\includegraphics[width=2cm,align=t]{/home/tof/Documents/Cozy/latex-include/cc.png}}

\begin{document}
\section{Problématique}
% rappel des couches
Les paquets IP transitent sur le réseau internet en circulant de routeurs en routeurs. En théorie, rien n'interdit à un routeur d'inspecter un paquet et donc d'en connaître son contenu. Cette situation est problématique dans de nombreuses situations (transactions bancaires, documents personnels\dots).
\begin{center}
    \framebox{Comment chiffrer le contenu des communications?}
\end{center}
\section{Chiffrement symétrique}
\subsection{Principe: le code de César}
La sécurisation d'une communication se déroule en deux étapes:
\begin{itemize}
    \item La source utilise une \emph{fonction de chiffrement} pour coder un message \emph{m} avec une clé de chiffrement \emph{k}. La fonction produit en sortie un message chiffré \emph{s}.
    \begin{center}
        \textbf{\texttt{chiffrement(m, k) $\rightarrow$ s}}
    \end{center}
    \item Le destinataire utilise une \emph{fonction de déchiffrement} pour décoder le message \emph{s} avec la clé de chiffrement \emph{k}. La fonction produit en sortie le message clair \emph{m}.
    \begin{center}
        \textbf{\texttt{déchiffrement(s, k) $\rightarrow$ m}}
    \end{center}
\end{itemize}
\begin{aretenir}[]
Dans un chiffrement symétrique on utilise la même clé pour chiffrer et déchiffrer le message.
\end{aretenir}
\begin{activite}
    %ROT13; substitution monoalphabétique
Le chiffrement de César utilise un décalage alphabétique comme clé de chiffrement.
\begin{enumerate}
    \item Écrire la fonction \textbf{\texttt{chiffrement(message: str, cle: int) $\rightarrow$ str}} qui code le \emph{message}. On n'utilisera que des caractères majuscules ASCII dans le message et on supprimera les espaces.
    \item Écrire la fonction \textbf{\texttt{dechiffrement(message: str, cle: int) $\rightarrow$ str}} qui déchiffre le \emph{message}.
    \item Tester la méthode avec une clé $k=+3$ sur le message: \emph{LANSIESTFANTASTIQUE}
    \item Quelles sont les faiblesses de cette méthode?
    %26 possibilités (25) facile à tester tous les cas
\end{enumerate}
\end{activite}
\subsection{Chiffrement polyalphabétique}
\subsubsection{Principe}
Plutôt que d'opérer un simple décalage, on recopie la clé de chiffrement de façon à obtenir une chaîne de la longueur du message. Utilisons la clé \emph{NSI}.
\begin{center}
    \begin{tabular}{*{5}{c}}
        B&R&A&V&O\\
        N&S&I&N&S\\
    \end{tabular}
\end{center}
Une même lettre ne sera plus forcément codée par le même symbole.
\begin{activite}
\begin{enumerate}
    \item Remplacer chaque lettre en son équivalent ASCII.
    \item Écrire la fonction \textbf{\texttt{int\_en\_bin(nb: int) $\rightarrow$ str}} qui renvoie la représentation binaire de l'entier \emph{nb}.
    % on cache un peu sous le tapis: on est toujours sur 7 bits
    \item Convertir chaque entier en binaire.
\end{enumerate}
\end{activite}
\subsubsection{Chiffrement par \emph{ou exclusif}}
On applique la porte logique \emph{xor} entre chaque bit du message et de la clé. Une propriété intéressante de cette porte est qu'elle est réversible:
$$\mbox{Si }A\oplus B = C \mbox{ alors } A\oplus C=B \mbox{ et }B\oplus C=A$$
%vigenere, enigma = substitution polyalphabétique
%clé trop courte attention; clé taille du message = très sûr (shannon)https://fr.wikipedia.org/wiki/Cryptographie_sym%C3%A9trique
\begin{activite}
\begin{enumerate}
    \item Appliquer le ou exclusif pour chaque bit du message.
    \item Écrire la fonction \textbf{\texttt{bin\_en\_int(paquet: str) $\rightarrow$ int}} qui renvoie l'entier correspondant au paquet de bits.
    \item Utiliser la fonction pour trouver l'entier correspondant à chaque paquet de sept bits.
    \item Donner alors le message chiffré.
    %vérifier qu'on retrouve le message d'origine
\end{enumerate}
\end{activite}
\section{Avantages du chiffrement symétrique}
Les algorithmes de chiffrement symétriques utilisés dans les communications appliquent des principes similaires au chiffrement par \emph{ou exclusif}. Ils sont très sûrs: la taille minimale des clés est actuellement de 80 bits soit $2^{80}$ possibilités. Ils sont également très rapides et permettent ainsi de chiffrer, en temps réel, les données stockées sur un disque dur.
% xor est une fonction implémentée dans les processeurs

On peut citer:
\begin{itemize}
    \item \emph{AES pour Advanced Encryption Standard:} choisi par l'institut de standardisation américain NIST (National Institute of Standards and Technology) en décembre 2001.
    \item \emph{Chacha20:} date de 2008 et améliore les performances d'un autre algorithme (Salsa20)
    % 20 = étapes de mélange
\end{itemize}
\end{document}