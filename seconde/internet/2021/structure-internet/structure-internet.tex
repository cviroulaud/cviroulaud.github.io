\documentclass[a4paper,11pt]{article}
\input{/home/tof/Documents/Cozy/latex-include/preambule_lua.tex}
\newcommand{\showprof}{show them}  % comment this line if you don't want to see todo environment
\fancyhead[L]{Structure de l'internet}
\newdate{madate}{03}{09}{2020}
\fancyhead[R]{\displaydate{madate}} %\today
\fancyfoot[L]{~\\Christophe Viroulaud}
\fancyfoot[C]{\textbf{Page \thepage}}
\fancyfoot[R]{\includegraphics[width=2cm,align=t]{/home/tof/Documents/Cozy/latex-include/cc.png}}

\begin{document}
\begin{Form}
\begin{commentprof}
\textbf{MATÉRIEL: ficelles}
\end{commentprof}
\paragraph{Objectif:} Comprendre ce qu'est internet.
\begin{commentprof}
\section*{Retour historique}
\begin{itemize}
\item Licklider imagina un ensemble d’ordinateurs interconnectés au niveau mondial à travers lequel chacun pourrait accéder rapidement aux données et programmes depuis n’importe quel site. En théorie, le concept était très semblable à l’Internet d’aujourd’hui. 
\item Pendant son emploi à DARPA, il persuada ses successeurs, Ivan Sutherland, Bob Taylor et Lawrence G. Roberts, chercheur au MIT, de l’intérêt de ce concept de réseau informatique.
\item 1972 démo ARPANET (Advanced Research Projects Agency Network) par Kahn; 23 nœuds en 1971, 111 en 1974; En 1980, Arpanet se divise en deux réseaux distincts, l'un militaire et l'autre, universitaire, que les militaires abandonnent au monde civil. Opérationnel le 20 septembre 1969, Arpanet sert de banc d'essai à de nouvelles technologies de gestion de réseau, liant plusieurs universités et centres de recherches. Les deux premiers nœuds qui forment l'Arpanet sont l'université de Californie à Los Angeles (UCLA) et l'Institut de recherche de Stanford (le premier message, le simple mot « login », sera envoyé sur le réseau le 29 octobre 1969 à 22 h 30 entre ces deux institutions, à la suite d'un bug, les trois dernières lettres mettront une heure pour arriver).
\item 1990: disparition d'ARPANET (démilitarisé) pour internet (civil).
\end{itemize}
\end{commentprof}
\section{Différentes topologies de réseaux}
\subsection{Les topologies possibles}
\begin{commentprof}
vocabulaire: terminal, ordinateur, machine, hôte
\end{commentprof}
\begin{itemize}
\item linéaire
\item anneau: en pratique un réseau en anneau est souvent composé de 2 anneaux contra-rotatifs.
\item bus: vient de omnibus ("pour tous")
\item étoile: autre inconvénient: longueur des câbles à utiliser.
\item hiérarchique: ou en arbre 
\item maillée: nb de liaison croît en $n^2$: s'il y a N terminaux, le nombre de liaisons nécessaires est de $\dfrac{N×(N-1)}{2}$
\end{itemize}
\subsection{Les topologies de l'internet}
\begin{commentprof}
slide\\
"petits réseaux" peut signifier plusieurs centaines de machines\\
LAN\\
MAN Metropolitan Area Network\\
WAN Wide Area Network (réseau étendu)\\
GAN\\
internet = internetting = interconnecter les réseaux
\end{commentprof}
\end{Form}
\end{document}