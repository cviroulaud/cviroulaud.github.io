\documentclass[a4paper,11pt]{article}
\input{/home/tof/Documents/Cozy/latex-include/preambule_lua.tex}
\newcommand{\showprof}{show them}  % comment this line if you don't want to see todo environment
\fancyhead[L]{Exercices fonctions booléennes}
\newdate{madate}{10}{09}{2020}
\fancyhead[R]{Première - NSI} %\today
\fancyfoot[L]{~\\Christophe Viroulaud}
\fancyfoot[C]{\textbf{Page \thepage}}
\fancyfoot[R]{\includegraphics[width=2cm,align=t]{/home/tof/Documents/Cozy/latex-include/cc.png}}

\begin{document}
\begin{Form}
\begin{exo}
Soit P et Q deux propositions logiques. Déterminer la table de vérité de la proposition \guill{non (P) ou Q}.
\begin{table}[!h]
\begin{center}
\begin{tabular}{|c|c|c|}
\hline 
P & Q & non (P) ou Q \\ 
\hline 
0 & 0 &  \\ 
\hline 
0 & 1 & \\ 
\hline 
1 & 0 & \\
\hline 
1 & 1 & \\
\hline 
\end{tabular}
\caption{\label{ex1}non (P) ou Q}
\end{center}
\end{table} 
\end{exo}
\begin{exo}
Soit P et Q deux propositions logiques. On considère une proposition T(P,Q), construite à partir des propositions P et Q, dont la table de vérité est donnée ci-dessous. 
\begin{table}[!h]
\begin{center}
\begin{tabular}{|c|c|c|}
\hline 
P & Q & T(P,Q) \\ 
\hline 
F & F & F \\ 
\hline 
F & V & F \\ 
\hline 
V & F & V \\
\hline 
V & V & F \\
\hline 
\end{tabular}
\caption{\label{ex2}T(P,Q)}
\end{center}
\end{table} 
\\Parmi les propositions suivantes, laquelle est logiquement équivalente à T(P,Q) ?
\begin{itemize}
\item Q et non(P)
\item non(Q) ou P
\item non(Q) et P
\item Q ou non(P)
\end{itemize}
\end{exo}
\begin{exo}
Soit A, B et C trois propositions logiques. Déterminer la table de vérité de la proposition \guill{C et (A ou B)}.
\begin{table}[!h]
\begin{center}
\begin{tabular}{|*4{c|}}
\hline 
A & B & C & C et (A ou B) \\ 
\hline 
0 & 0 & 0 &  \\ 
\hline 
0 & 0 & 1 & \\ 
\hline 
0 & 1 & 0 & \\
\hline 
0 & 1 & 1 & \\
\hline 
1 & 0 & 0 &  \\ 
\hline 
1 & 0 & 1 & \\ 
\hline 
1 & 1 & 0 & \\
\hline 
1 & 1 & 1 & \\
\hline 
\end{tabular}
\caption{\label{ex3}C et (A ou B)}
\end{center}
\end{table} 
\end{exo}
\begin{exo}
Montrer l'égalité suivante:
$$(x\land y)\lor (\lnot y \land z)=(x \lor \lnot y)\land (y \lor z)$$
\begin{commentprof}
Construire les deux tables de vérités
\end{commentprof}
\end{exo}
\begin{exo}
On considère la fonction booléenne suivante:
$$f(x,y,z)=(x\land \lnot y \land \lnot z)\lor (\lnot x \land y \land \lnot z) \lor (\lnot x \land \lnot y \land z)$$
\begin{enumerate}
\item Donner sa table de vérité.
\item Que fait cette fonction?
\item En s'aidant de la table de vérité, donner une expression plus simple de cette fonction.
\end{enumerate}
\begin{commentprof}
regarder les 4 premiers puis les 4 derniers
\end{commentprof}
\end{exo}
\begin{exo}
En combinant 2 additionneurs \emph{1 bit} nous pouvons obtenir un additionneur \emph{2 bits}. Notons $e_0e_1$ et $e_2e_3$ deux nombres binaires. Le premier additionneur se charge d'additionner les bits de poids faible.
\begin{enumerate}
\item Quels bits additionnent cet additionneur?
\item Que vaut l'entrée $c_0$ de cet additionneur?
\item Où est envoyé la sortie \emph{c} de ce premier additionneur?
\item Combien de lignes y-aura-t-il dans la table de vérité de l'additionneur 2 bits?
\item Compléter la table de vérité de l'additionneur 2 bits ci-après:
\end{enumerate}
\begin{table}[!h]
\begin{center}
\begin{tabular}{|*4{c|}|*3{c|}}
\hline 
$e_0$ & $e_1$ & $e_2$ & $e_4$ & $s_0$ & $s_1$ & $c$ \\ 
\hline 
0 & 0 & 0 & 0 & 0 & 0 & 0 \\ 
\hline 
... &  &  &  &  &  &  \\ 
\hline 
\end{tabular}
\caption{\label{addi}Additionneur 2 bits}
\end{center}
\end{table} 
\end{exo}
\end{Form}
\end{document}