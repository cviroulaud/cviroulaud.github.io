\documentclass[a4paper,11pt]{article}
\input{/home/tof/Documents/Cozy/latex-include/preambule_doc.tex}
\input{/home/tof/Documents/Cozy/latex-include/preambule_commun.tex}
\newcommand{\showprof}{show them}  % comment this line if you don't want to see todo environment
\setlength{\fboxrule}{0.8pt}
\fancyhead[L]{\fbox{\Large{\textbf{Routage 01}}}}
\fancyhead[C]{\textbf{Principe du routage}}
\newdate{madate}{10}{09}{2020}
%\fancyhead[R]{\displaydate{madate}} %\today
%\fancyhead[R]{Seconde - SNT}
%\fancyhead[R]{Première - NSI}
\fancyhead[R]{Terminale - NSI}
\fancyfoot[L]{\vspace{1mm}Christophe Viroulaud}
\AtEndDocument{\label{lastpage}}
\fancyfoot[C]{\textbf{Page \thepage/\pageref{lastpage}}}
\fancyfoot[R]{\includegraphics[width=2cm,align=t]{/home/tof/Documents/Cozy/latex-include/cc.png}}

\begin{document}
\section{Problématique}
Le réseau internet permet de communiquer avec n'importe quelle machine connectée. En juin 2020 on dénombrait 1,78 milliards de sites web dans le monde.
\begin{center}
    \framebox{Comment retrouver une machine précise dans le réseau?}
\end{center}
\section{Adresse IP}
Sur un réseau chaque machine est repérée par son \emph{adresse IP}. L'\textbf{Internet Protocol} version 4 (IPv4) est peu à peu remplacée par la version 6 pour pallier la pénurie d'adresses. Une adresse IPv4 est composée de 4 octets.
\begin{center}
    Un exemple: \large{192.168.10.3}
\end{center}
Une adresse IP est accompagnée de son masque de sous-réseau. Il permet de déterminer le réseau auquel appartient la machine.
\begin{center}
    \begin{tabular}{ccccc}
        adresse IP & 192 & 168 & 10  & 3 \\
        masque     & 255 & 255 & 255 & 0 \\
    \end{tabular}
\end{center}
Pour connaître le réseau on convertit les adresses en binaire et on applique une porte logique \emph{AND}.
\begin{center}
    \begin{tabular}{ccccc}
        adresse IP & 11000000 & 10101000 & 00001010 & 00000011 \\
        masque     & 11111111 & 11111111 & 11111111 & 00000000 \\
        réseau     & 11000000 & 10101000 & 00001010 & 00000000 \\
    \end{tabular}
\end{center}

\begin{aretenir}[]
    On note une adresse IP avec son masque de sous-réseau. Le nombre après / correspond au nombre de 1 du masque (notation \emph{CIDR} - (Classless Inter-Domain Routing)).
    \begin{center}
        192.168.10.3/24
    \end{center}
    Les 24 premiers bits correspondent au réseau.
    \begin{commentprof}
        Il y a donc $2^{32-24}$ adresses disponibles dans le réseau (- adresse de réseau et adresse de broadcast). adresse du réseau: 192.168.10.0
    \end{commentprof}
\end{aretenir}

\begin{activite}
    \begin{enumerate}
        \item Donner le réseau auquel appartient l'adresse 10.2.10.103/12
        \item Combien d'adresses peut-on créer dans ce réseau?
        \item Ouvrir un terminal et taper la commande (code \ref{ip}).
        \begin{center}
            \begin{lstlisting}[language=bash]
ip -4 a
            \end{lstlisting}
            \captionof{code}{Adresse IPv4}
            \label{ip}
        \end{center}

        \item Quelle est l'adresse de la machine?
        \item Quelle est l'adresse du réseau?
    \end{enumerate}
\end{activite}
\section{Structure maillée}
\end{document}