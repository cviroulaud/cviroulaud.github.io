\documentclass[svgnames,11pt]{beamer}
\input{/home/tof/Documents/Cozy/latex-include/preambule_commun.tex}
\input{/home/tof/Documents/Cozy/latex-include/preambule_beamer.tex}
%\usepackage{pgfpages} \setbeameroption{show notes on second screen=left}
\author[]{Christophe Viroulaud}
\title{TP les fractions}
\date{}
%\logo{}
\institute{Terminale - NSI}

\begin{document}
\begin{frame}
    \titlepage
\end{frame}
\begin{frame}
    \frametitle{Fraction}

    En mathématiques une fraction est un moyen d'écrire un nombre rationnel sous la forme d'un quotient de deux entiers. La fraction $\frac{a}{b}$ désigne le quotient de a par b ($b\neq0$). Dans cette fraction, a est appelé le \emph{numérateur} et b le \emph{dénominateur}\footnote{source: wikipedia}.
\end{frame}
\begin{frame}
    \frametitle{}

    \framebox{Peut-on effectuer les opérations classiques sur les fractions?}

\end{frame}
\section{Classe \textbf{\texttt{Object}}}
\subsection{La classe \textbf{\emph{Fraction}}}
\begin{frame}
    \frametitle{La classe \textbf{\emph{Fraction}}}

    \begin{activite}
\begin{enumerate}
    \item Créer le fichier \textbf{\texttt{mod\_fraction.py}}
    \item Construire la classe \textbf{\texttt{Fraction}} et son constructeur. Cette classe possédera les attributs \textbf{\texttt{numerateur}} et \textbf{\texttt{denominateur}}, deux entiers initialisés à 1.
    \item Écrire les \emph{accesseurs} et les \emph{mutateurs}. On utilisera une assertion pour le cas de l'affectation du dénominateur nul.
\end{enumerate}
    \end{activite}

\end{frame}
\begin{frame}
    \frametitle{Avant de regarder la correction}
\begin{center}
    \centering
    \includegraphics[width=3cm]{/home/tof/Documents/Cozy/latex-include/stop.png}
    \end{center}
{\Large
    \begin{itemize}
        \item Prendre le temps de réfléchir,
        \item Analyser les messages d'erreur,
        \item Demander au professeur.
    \end{itemize}
}
\end{frame}
\begin{frame}[fragile]
    \frametitle{Correction}

\begin{center}
\begin{lstlisting}[language=Python , basicstyle=\ttfamily\small, xleftmargin=2em, xrightmargin=1em]
class Fraction:
    def __init__(self):
        self.numerateur: int = 1
        self.denominateur: int = 1

    def get_numerateur(self) -> int:
        return self.numerateur

    def get_denominateur(self) -> int:
        return self.denominateur
\end{lstlisting}
\captionof{code}{La classe et les accesseurs}
\end{center}

\end{frame}
\begin{frame}[fragile]
    \frametitle{Correction}

\begin{center}
\begin{lstlisting}[language=Python , basicstyle=\ttfamily\small, xleftmargin=2em, xrightmargin=1em]
    def set_numerateur(self, n: int) -> None:
        self.numerateur = n

    def set_denominateur(self, d: int) -> None:
        assert d != 0, "Le dénominateur ne peut pas être nul."
        self.denominateur = d
\end{lstlisting}
\captionof{code}{Les mutateurs}
\end{center}

\end{frame}
\subsection{Héritage (notion hors programme)}
\begin{frame}[fragile]
    \frametitle{Héritage (notion hors programme)}
Python possède une classe native \textbf{\texttt{Object}} qui possède ses propres méthodes. Par exemple la méthode \textbf{\texttt{\_\_str\_\_}} renvoie une chaîne de caractère représentant l'objet. Ainsi le code \ref{str} renvoie le résultat \ref{resstr}.
\begin{center}
\begin{lstlisting}[language=Python , basicstyle=\ttfamily\small, xleftmargin=2em, xrightmargin=1em]
print(f)
\end{lstlisting}
\captionof{code}{La méthode est appelée automatiquement.}
\label{str}
\end{center}
\begin{center}
    \begin{lstlisting}[language=Python , basicstyle=\ttfamily\small, xleftmargin=2em, xrightmargin=1em]
<__main__.Fraction object at 0x7f976febc0f0>
\end{lstlisting}
\captionof{code}{La méthode renvoie des informations sur l'objet (adresse mémoire\dots)}
    \label{resstr}
    \end{center}  

\end{frame}
\begin{frame}
    \frametitle{}

    Quand le programmeur construit une classe, Python crée en réalité un \emph{enfant} de la classe \textbf{\texttt{Object}}. On parle d'\textbf{héritage}.\\ Ainsi la classe \textbf{\texttt{Fraction}} possède également la méthode \textbf{\texttt{\_\_str\_\_}}. Mieux il est possible de redéfinir le rôle de la méthode. On parle alors de \textbf{surcharge}.

\end{frame}
\begin{frame}[fragile]
    \frametitle{}

    \begin{activite}
\begin{enumerate}
    \item Écrire la méthode \ref{str2} dans la classe \textbf{\texttt{Fraction}}.
\begin{center}
\begin{lstlisting}[language=Python , basicstyle=\ttfamily\small, xleftmargin=2em, xrightmargin=1em]
def __str__(self):
    return "Bonjour je suis un objet!"
\end{lstlisting}
\captionof{code}{Surcharge de la méthode}
\label{str2}
\end{center}
\item Exécuter à nouveau \textbf{\texttt{print(f)}}.
\item Modifier la méthode pour qu'elle affiche la fraction sous la forme \textbf{\texttt{a/b}}.
\end{enumerate}
    \end{activite}

\end{frame}
\begin{frame}
    \frametitle{Avant de regarder la correction}
\begin{center}
    \centering
    \includegraphics[width=3cm]{/home/tof/Documents/Cozy/latex-include/stop.png}
    \end{center}
{\Large
    \begin{itemize}
        \item Prendre le temps de réfléchir,
        \item Analyser les messages d'erreur,
        \item Demander au professeur.
    \end{itemize}
}
\end{frame}
\begin{frame}[fragile]
    \frametitle{Correction}

\begin{center}
\begin{lstlisting}[language=Python , basicstyle=\ttfamily\small, xleftmargin=2em, xrightmargin=1em]
def __str__(self):
    return "{}/{}".format(self.numerateur, self.denominateur)
\end{lstlisting}
\end{center}    
\begin{center}
    \begin{lstlisting}[language=Python , basicstyle=\ttfamily\small, xleftmargin=2em, xrightmargin=1em]
print(f)
>>> 2/3
\end{lstlisting}
    \end{center} 
\end{frame}
\begin{frame}[fragile]
    \frametitle{Correction}

    \begin{aretenir}[Remarque]
\textbf{\texttt{\_\_str\_\_}} est une méthode, elle peut donc être appelée classiquement.
\begin{center}
\begin{lstlisting}[language=Python , basicstyle=\ttfamily\small, xleftmargin=2em, xrightmargin=1em]
print(f)
\end{lstlisting}
\begin{lstlisting}[language=Python , basicstyle=\ttfamily\small, xleftmargin=2em, xrightmargin=1em]
print(f.__str__())
\end{lstlisting}
\captionof{code}{Ces appels réalisent la même chose.}
\label{CODE}
\end{center}
    \end{aretenir}

\end{frame}
\section{Opérations sur les fractions}
\subsection{Opérations mathématiques}
\begin{frame}
    \frametitle{Opérations mathématiques}

    \begin{itemize}
        \item<1-> Multiplier deux fractions:
        $$\frac{a}{b}×\frac{c}{d}=\frac{a×c}{b×d}$$
        \item<2-> Additionner deux fractions:
        $$\frac{a}{b}+\frac{c}{d}=\frac{a×d}{b×d}+\frac{c×b}{d×b}=\frac{a×d+c×b}{b×d}$$
    \end{itemize}

\end{frame}
\subsection{Opérateurs Python}
\begin{frame}[fragile]
    \frametitle{}

    Pour additionner deux nombres entiers, il suffit d'utiliser l'opérateur $+$
\begin{center}
\begin{lstlisting}[language=Python , basicstyle=\ttfamily\small, xleftmargin=2em, xrightmargin=1em]
2 + 3
>>> 5
\end{lstlisting}
\end{center}
Mais comment Python sait-il ce qu'il doit réaliser dans les cas suivants?
\begin{center}
\begin{lstlisting}[language=Python , basicstyle=\ttfamily\small, xleftmargin=2em, xrightmargin=1em]
[1, 2, 3] + [4, 5, 6]
"Hello " + "world!"
\end{lstlisting}
\captionof{code}{L'opérateur d'addition a un comportement adapté à chaque situation.}
\label{CODE}
\end{center}

\end{frame}

\begin{frame}
    \frametitle{}
    Il est possible de redéfinir le comportement des opérations de la classe \textbf{\texttt{Object}} en \emph{surchargeant} les méthodes:
    \begin{itemize}
        \item  \textbf{\texttt{\_\_add\_\_}} définit l'opérateur \textbf{\texttt{+}},
        \item  \textbf{\texttt{\_\_mul\_\_}} définit l'opérateur \textbf{\texttt{*}},
        \item \textbf{\texttt{\_\_eq\_\_}} définit l'opérateur \textbf{\texttt{==}},
        \item \textbf{\texttt{\_\_lt\_\_}} définit l'opérateur \textbf{\texttt{<}},
        \item \dots
    \end{itemize}
    On peut retrouver une liste exhaustive sur \url{https://docs.python.org/fr/3/library/operator.html}

\end{frame}
\subsection{Implémentation}
\begin{frame}[fragile]
    \frametitle{}
\begin{activite}
\begin{enumerate}
    \item Redéfinir les méthodes \textbf{\texttt{\_\_eq\_\_}} et \textbf{\texttt{\_\_lt\_\_}} qui reçoivent une deuxième fraction en paramètre et renvoient \texttt{\textbf{True}} si la première fraction est égale ou strictement inférieure à la deuxième fraction.
    \item Tester le code \ref{egal}
\begin{center}
\begin{lstlisting}[language=Python , basicstyle=\ttfamily\small, xleftmargin=2em, xrightmargin=1em]
f = Fraction()
f.set_numerateur(2)
f.set_denominateur(3)

f1 = Fraction()
f1.set_numerateur(8)
f1.set_denominateur(12)
print(f == f1)
\end{lstlisting}
\captionof{code}{Comparaison de fractions}
\label{egal}
\end{center}
\end{enumerate}
\end{activite}
\end{frame}
\begin{frame}
    \frametitle{Avant de regarder la correction}
\begin{center}
    \centering
    \includegraphics[width=3cm]{/home/tof/Documents/Cozy/latex-include/stop.png}
    \end{center}
{\Large
    \begin{itemize}
        \item Prendre le temps de réfléchir,
        \item Analyser les messages d'erreur,
        \item Demander au professeur.
    \end{itemize}
}
\end{frame}
\begin{frame}[fragile]
    \frametitle{Correction}

\begin{center}
\begin{lstlisting}[language=Python , basicstyle=\ttfamily\small, xleftmargin=2em, xrightmargin=1em]
def __eq__(self, f):
    return self.numerateur * f.denominateur == self.denominateur * f.numerateur

def __lt__(self, f):
    return self.numerateur * f.denominateur < self.denominateur * f.numerateur

\end{lstlisting}
\captionof{code}{Méthodes de comparaison}
\end{center}  

\end{frame}
\begin{frame}[fragile]
    \frametitle{Correction}

    \begin{aretenir}[Remarque]
\textbf{\texttt{\_\_eq\_\_}} est une méthode, elle peut donc être appelée classiquement.
\begin{center}
\begin{lstlisting}[language=Python , basicstyle=\ttfamily\small, xleftmargin=2em, xrightmargin=1em]
print(f == f1)
\end{lstlisting}
\begin{lstlisting}[language=Python , basicstyle=\ttfamily\small, xleftmargin=2em, xrightmargin=1em]
print(f.__eq__(f1))
\end{lstlisting}
\captionof{code}{Ces appels réalisent la même chose.}
\label{CODE}
\end{center}
    \end{aretenir}

\end{frame}
\begin{frame}
    \frametitle{}

    \begin{activite}
        \begin{enumerate}
            \item Redéfinir les méthodes \textbf{\texttt{\_\_add\_\_}} et \textbf{\texttt{\_\_mul\_\_}} qui reçoivent une deuxième fraction en paramètre et renvoient une nouvelle fraction représentant respectivement la somme et le produit des deux
        fractions.
        \item Additionner et multiplier deux fractions.
        \end{enumerate}
    \end{activite}

\end{frame}
\begin{frame}
    \frametitle{Avant de regarder la correction}
\begin{center}
    \centering
    \includegraphics[width=3cm]{/home/tof/Documents/Cozy/latex-include/stop.png}
    \end{center}
{\Large
    \begin{itemize}
        \item Prendre le temps de réfléchir,
        \item Analyser les messages d'erreur,
        \item Demander au professeur.
    \end{itemize}
}
\end{frame}
\begin{frame}[fragile]
    \frametitle{Correction}

\begin{center}
\begin{lstlisting}[language=Python , basicstyle=\ttfamily\small, xleftmargin=2em, xrightmargin=1em]
def __add__(self, f):
    res = Fraction()
    res.set_numerateur(self.numerateur * f.denominateur + f.numerateur * self.denominateur)
    res.set_denominateur(self.denominateur * f.denominateur)
    return res
\end{lstlisting}
\captionof{code}{Addition}
\label{CODE}
\end{center}   

\end{frame}
\begin{frame}[fragile]
    \frametitle{Correction}

\begin{center}
\begin{lstlisting}[language=Python , basicstyle=\ttfamily\small, xleftmargin=2em, xrightmargin=1em]
def __mul__(self, f):
    res = Fraction()
    res.set_numerateur(self.numerateur * f.numerateur)
    res.set_denominateur(self.denominateur * f.denominateur)
    return res
\end{lstlisting}
\captionof{code}{Multiplication}
\label{CODE}
\end{center}

\end{frame}
\begin{frame}[fragile]
    \frametitle{Correction}

\begin{center}
\begin{lstlisting}[language=Python , basicstyle=\ttfamily\small, xleftmargin=2em, xrightmargin=1em]
f = Fraction()
f.set_numerateur(2)
f.set_denominateur(3)

f2 = Fraction()
f2.set_numerateur(12)
f2.set_denominateur(7)
print(f + f2)
\end{lstlisting}
\captionof{code}{Addition de deux fractions}
\label{CODE}
\end{center}

\end{frame}

\begin{frame}
    \frametitle{Code complet}

    Le code complet est accessible \href{https://cviroulaud.github.io/terminale/langages/paradigmes/POO/tp-fraction/scripts/fraction-correction.zip}{ici}.

\end{frame}
\end{document}