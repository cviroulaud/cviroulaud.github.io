\documentclass[a4paper,11pt]{article}
\input{/home/tof/Documents/Cozy/latex-include/preambule_lua.tex}
\newcommand{\showprof}{show them}  % comment this line if you don't want to see todo environment
\fancyhead[L]{Une jolie page web}
\newdate{madate}{10}{09}{2020}
\fancyhead[R]{Seconde - SNT} %\today
\fancyfoot[L]{~\\Christophe Viroulaud}
\fancyfoot[C]{\textbf{Page \thepage}}
\fancyfoot[R]{\includegraphics[width=2cm,align=t]{/home/tof/Documents/Cozy/latex-include/cc.png}}

\begin{document}
\begin{Form}
\section{Problématique}
Le langage \emph{html} s'occupe du contenu de la page web. Mais l'esthétique de la page pourrait être améliorée.
\begin{center}
\shadowbox{\parbox{10cm}{\centering Comment améliorer l'esthétique d'une page web?}}
\end{center}
\section{Un autre langage: le CSS}
\subsection{Présentation}
Le \emph{Cascading Style Sheets} (feuille de style en cascades) est un langage qui gère la mise en forme du contenu html.
\begin{activite}
\begin{enumerate}
\item Créer un fichier \emph{custom.css} dans le même répertoire que le fichier \emph{html}. Pour créer un fichier depuis le logiciel \emph{Atom} il faut effectuer un clic-droit sur le projet et choisir \emph{New File}.
\item Dans le fichier html, ajouter la ligne suivante entre les balises \emph{head}:
\begin{lstlisting}[language=html]
<link rel="stylesheet" href="custom.css">
\end{lstlisting}
Cette ligne crée le lien (\emph{link}) entre les fichiers html et css.
\end{enumerate}
\end{activite}
\subsection{Cibler les balises html}
Nous pouvons modifier la mise en forme des balises html.
\begin{activite}
\begin{enumerate}
\item Dans le fichier \emph{custom.css}, ajouter le code suivant:
\begin{center}
\begin{lstlisting}
h1 {
	color: red;
}
\end{lstlisting}
\captionof{code}{Modifier les titres}
\label{h1}
\end{center}
\item Enregistrer le fichier et recharger la page web \textbf{dans le navigateur}.
\item Effectuer la même tâche avec les codes suivants:
\begin{lstlisting}
body {
  width: 800px;
  margin: auto;
  background-color: #00ff00;
}
\end{lstlisting}
\begin{lstlisting}
p {
  text-align: justify;
  font-family: "Gill Sans", sans-serif;
}
\end{lstlisting}
\captionof{code}{Modifier divers éléments}
\end{enumerate}
\end{activite}
\subsection{Cibler une classe d'éléments}
L'inconvénient du code \ref{h1} est que tous les titres \emph{h1} seront rouges. Il est possible de ne cibler qu'un groupe d'éléments.
\begin{activite}
\begin{enumerate}
\item Dans le fichier html modifier deux titres comme ci-après:
\begin{lstlisting}
<h1 class="titre-special">Mon texte à afficher</h1>
\end{lstlisting}
\captionof{code}{Ajouter un nom de classe}
\medskip
\item Dans le fichier css ajouter le code ci-après. Il faut bien remarquer le point en début de ligne.
\begin{lstlisting}
.titre-special {
	color: green;
}
\end{lstlisting}
\captionof{code}{La nouvelle classe cible les bons éléments}
\medskip
\end{enumerate}
\end{activite}
\subsection{Pour aller plus loin}
Le langage \emph{css} permet de réaliser des mises en forme très pointues. Le site du \emph{Mozilla Developer Center} offre des tutoriels détaillés:
\begin{center}
\url{https://developer.mozilla.org/fr/docs/Web/CSS}
\end{center}
La page ci-après référence toutes les propriétés \emph{css}:
\begin{center}
\url{https://developer.mozilla.org/fr/docs/Web/CSS/Reference}
\end{center}
\section{Mise en application}
\begin{activite}
Modifier le fichier \emph{custom.css} pour mettre en forme la page web et obtenir le rendu désiré.
\end{activite}
\end{Form}
\end{document}