\documentclass[a4paper,11pt]{article}
\input{/home/tof/Documents/Cozy/latex-include/preambule_lua.tex}
\newcommand{\showprof}{show them}  % comment this line if you don't want to see todo environment
\fancyhead[L]{Concessionnaire - dictionnaire}
\newdate{madate}{10}{09}{2020}
%\fancyhead[R]{\displaydate{madate}} %\today
%\fancyhead[R]{Seconde - SNT}
\fancyhead[R]{Première - NSI}
%\fancyhead[R]{Terminale - NSI}
\fancyfoot[L]{~\\Christophe Viroulaud}
\AtEndDocument{\label{lastpage}}
\fancyfoot[C]{\textbf{Page \thepage/\pageref{lastpage}}}
\fancyfoot[R]{\includegraphics[width=2cm,align=t]{/home/tof/Documents/Cozy/latex-include/cc.png}}

\begin{document}
\begin{Form}
\section{Problématique}
Un concessionnaire désire concevoir une application qui permet de stocker les informations de ses véhicules (marque, modèle, kilométrage, date de première immatriculation...).
\begin{center}
\shadowbox{\parbox{10cm}{\centering Comment structurer des données hétérogènes?}}
\end{center}
\section{Premières approches}
L'utilisation de tableaux est possible mais peu adapté.
\begin{center}
\begin{lstlisting}[language=Python]
marques = ["Renault", "Renault", "Mercedes", "Peugeot"]
modeles = ["Twingo", "Trafic", "Classe A", "1007"]
kilometrages = [23410, 100230, 45000, 34001]
immatriculation = ["AA-123-AA", "AZ-34-DE", "AA-145-JU", "BB-156-TR"]
premiere_immat = ["2010-10", "2007-05", "2019-06", "2018-01"]
dates_vidanges = ["2020-01", "2019-12", "2019-06", "2020-10"]
\end{lstlisting}
\captionof{code}{Première approche}
\label{moncode}
\end{center}
\begin{center}
\begin{lstlisting}[language=Python]
vehicule1 = ["Renault", "Twingo", 23410, "AA-123-AA", "2010-10", "2020-01"]
vehicule3 = ["Mercedes", "Classe A", 45000, "AA-145-JU", "2019-06", "2019-06"]
\end{lstlisting}
\captionof{code}{Seconde approche}
\label{moncode}
\end{center}
\begin{activite}
Déterminer les défauts des approches présentées.
\end{activite}
\section{Le dictionnaire}
\subsection{Une nouvelle structure}
Python offre une structure mieux adaptée pour organiser les données: \textbf{les n-uplets nommés ou dictionnaires}.
\begin{aretenir}[]
Un dictionnaire associe chaque \textbf{valeur} à une \textbf{clé}. Chaque clé peut être un \emph{string}, un \emph{entier} voire un \emph{tuple}, c'est à dire un type \emph{non-mutable}.
\end{aretenir}
\begin{center}
\begin{lstlisting}[language=Python]
# On crée un dictionnaire avec des accolades ou le mot-clé dict()
vehicule1 = {"marque": "Renault", 
		"modele": "Twingo", 
		"kilometrage": 23410, 
		"immatriculation": "AA-123-AA", 
		"premiere_imat": "2010-10", 
		"date_vidange": "2020-01"}
\end{lstlisting}
\captionof{code}{Créer un dictionnaire}
\label{dico}
\end{center}
\subsection{Modifier un dictionnaire}
Il est possible de construire un dictionnaire vide que l'on remplit au fur et à mesure.
\begin{center}
\begin{lstlisting}[language=Python]
vehicule2 = {}
vehicule3 = dict()
\end{lstlisting}
\captionof{code}{Créer un dictionnaire vide}
\label{moncode}
\end{center}
On crée alors les clés \emph{à la volée} en utilisant la notation des crochets.
\begin{center}
\begin{lstlisting}[language=Python]
vehicule2["marque"] = "Renault"
\end{lstlisting}
\captionof{code}{Créer un couple clé/valeur}
\label{moncode}
\end{center}
\begin{activite}
\begin{enumerate}
\item Créer le dictionnaire \emph{vehicule2} en prenant modèle sur le code \ref{dico}.
\item Afficher l'immatriculation du \emph{vehicule2} dans la sortie console.
\item Créer le dictionnaire vide \emph{vehicule3}.
\item Ajouter les uns après les autres les couples clé/valeur dans le dictionnaire \emph{vehicule3}.
\item Modifier le kilométrage du \emph{vehicule3}.
\end{enumerate}
\end{activite}
\subsection{Itérer sur un dictionnaire}
Comme un tableau, un dictionnaire est une structure \emph{itérable}.
\begin{center}
\lstinputlisting[firstline=17,lastline=18]{"scripts/concessionnaire.py"}
\captionof{code}{Itérer sur les clés}
\label{moncode}
\end{center}
\begin{center}
\lstinputlisting[firstline=20,lastline=21]{"scripts/concessionnaire.py"}
\captionof{code}{Itérer sur les valeurs}
\label{moncode}
\end{center}
\begin{center}
\lstinputlisting[firstline=23,lastline=24]{"scripts/concessionnaire.py"}
\captionof{code}{Itérer sur les couples clé/valeur}
\label{moncode}
\end{center}
\end{Form}
\end{document}