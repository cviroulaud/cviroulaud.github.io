\documentclass[a4paper,11pt]{article}
\input{/home/tof/Documents/Cozy/latex-include/preambule_lua.tex}
\newcommand{\showprof}{show them}  % comment this line if you don't want to see todo environment
\fancyhead[L]{Listes chaînées - exercices}
\newdate{madate}{10}{09}{2020}
\fancyhead[R]{Terminale - NSI} %\today
\fancyfoot[L]{~\\Christophe Viroulaud}
\fancyfoot[C]{\textbf{Page \thepage}}
\fancyfoot[R]{\includegraphics[width=2cm,align=t]{/home/tof/Documents/Cozy/latex-include/cc.png}}
\usepackage{tikz}

\begin{document}
\begin{Form}
\begin{exo}
Listes chaînées - nouvelles fonctionnalités\\
Reprendre la classe \emph{Liste} vue en cours et ajouter les méthodes ci-après:
\begin{enumerate}
\item \emph{\_\_str\_\_} s'appelle directement: \textbf{print(nom\_objet)}.
\lstinputlisting[firstline=57,lastline=67]{"scripts/liste-exo.py"}
\item Version impérative.
\lstinputlisting[firstline=31,lastline=38]{"scripts/liste-exo.py"}
\item S'appuyer sur le schéma
\lstinputlisting[firstline=40,lastline=55]{"scripts/liste-exo.py"}
\item La complexité dépend de la taille de la première liste mais pas du tout de celle de la seconde.
\end{enumerate}
\end{exo}

\pagebreak

\begin{exo}
\lstinputlisting[firstline=69,lastline=82]{"scripts/liste-exo.py"}
\end{exo}
\end{Form}
\end{document}