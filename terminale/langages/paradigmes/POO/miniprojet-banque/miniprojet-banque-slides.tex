\documentclass[svgnames,11pt]{beamer}
\input{/home/tof/Documents/Cozy/latex-include/preambule_commun.tex}
\input{/home/tof/Documents/Cozy/latex-include/preambule_beamer.tex}
\usepackage{pgfpages} \setbeameroption{show notes on second screen=left}
\author[]{Christophe Viroulaud}
\title{Mini-projet\\La banque}
\date{\framebox{\textbf{Lang 04}}}
%\logo{}
\institute{Terminale - NSI}

\begin{document}
\begin{frame}
\titlepage
\end{frame}
\begin{frame}
    \frametitle{}

    La \emph{Banque des Périgourdins (BdP)} est une institution nouvellement implantée en Dordogne. Outre un compte courant classique, elle propose également des comptes rémunérés à différents taux. À une date fixée, la banque calcule les intérêts et crédite les comptes rémunérés des clients. Elle vous demande de créer un logiciel pour administrer correctement sa clientèle.
    \note{\fcolorbox{black}{red}{{\LARGE modularite.zip sur site}}}
\end{frame}
\section{Cahier des charges}
\begin{frame}
    \frametitle{Cahier des charges}

    \begin{aretenir}[]
    Pour pouvoir modéliser le projet il faut s'appuyer sur un \emph{cahier des charges}. Celui-ci doit être le plus précis possible.
    \end{aretenir}

\end{frame}
\begin{frame}
    \frametitle{}
    \begin{itemize}
        \item<1-> Un \texttt{\textbf{compte}} possède une somme d'argent, initialisée à 0 à l'ouverture. Ce compte peut être rémunéré ou non. Ce compte peut être crédité ou débité. En cas de débit supérieur à la somme présente sur le compte, le retrait est refusé.
        \item <2-> Un \textbf{\texttt{client}} possède un ou plusieurs \textbf{\texttt{comptes}} rémunérés ou non. Il peut alimenter ou utiliser l'argent sur son compte. Il peut également décider d'ouvrir un nouveau compte en demandant un nouveau numéro à la \textbf{\texttt{banque}}.
        \item <3-> La \textbf{\texttt{banque}} possède une liste de tous ses clients. Elle peut enregistrer un nouveau \textbf{\texttt{client}}. Dans ce cas elle lui ouvre automatiquement un \textbf{\texttt{compte}} non rémunéré. Enfin elle peut également appliquer un taux de rémunération de 3\% à tous les comptes rémunérés.
    \end{itemize}
\end{frame}
\section{Programmation modulaire}
\begin{frame}
    \frametitle{Programmation modulaire}
\begin{itemize}
    \item <1-> Un projet important implique plusieurs programmeurs. On le découpe en plusieurs objets.
    \item <2-> Chaque personne n'a pas besoin de connaître l'implémentation de chaque objet. Ce dernier lui fournit par contre une \textbf{interface} qui définit ce que peut faire l'objet.
    \item <3-> La \texttt{\textbf{classe}} garantit ce que va faire l'objet (mais pas comment le programmeur va s'en servir).
\end{itemize}

\end{frame}
\begin{frame}
    \frametitle{}

    \begin{aretenir}[]
On crée une classe dans un fichier séparé. La classe, ainsi que chacune de ses méthodes est complétée d'une \textbf{\texttt{docstring}}. Elle servira d'\textbf{interface}.
    \end{aretenir}
\begin{activite}
\begin{itemize}
    \item Télécharger le dossier compressé \textbf{\texttt{modularite.zip}} et extraire les fichiers.
    \item Lancer le fichier \textbf{\texttt{modularite\_poo.py}}.
    \item Observer l'aide affichée.
\end{itemize}
\end{activite}
\end{frame}
\section{Modélisation}
\begin{frame}
    \frametitle{Modélisation}

    \begin{activite}
    Réfléchir en commun pour modéliser le concept de la banque. Définir les objets nécessaires et les actions possibles (cahier des charges page suivante).
    \end{activite}

\end{frame}
\begin{frame}
    \frametitle{}
    \begin{itemize}
        \item Un \texttt{\textbf{compte}} possède une somme d'argent, initialisée à 0 à l'ouverture. Ce compte peut être rémunéré ou non. Ce compte peut être crédité ou débité. En cas de débit supérieur à la somme présente sur le compte, le retrait est refusé.
        \item Un \textbf{\texttt{client}} possède un ou plusieurs \textbf{\texttt{comptes}} rémunérés ou non. Il peut alimenter ou utiliser l'argent sur son compte. Il peut également décider d'ouvrir un nouveau compte en demandant un nouveau numéro à la \textbf{\texttt{banque}}.
        \item La \textbf{\texttt{banque}} possède une liste de tous ses clients. Elle peut enregistrer un nouveau \textbf{\texttt{client}}. Dans ce cas elle lui ouvre automatiquement un \textbf{\texttt{compte}} non rémunéré. Enfin elle peut également appliquer un taux de rémunération de 3\% à tous les comptes rémunérés.
    \end{itemize}
\end{frame}
\section{Implémentation}
\begin{frame}
    \frametitle{Implémentation}

    \begin{activite}
    \begin{itemize}
        \item Partager les tâches à effectuer et construire les différentes classes.
        \item Construire un programme qui crée la banque et plusieurs clients.
    \end{itemize}
    \end{activite}

\end{frame}
\end{document}