\documentclass[a4paper,11pt]{article}
\input{/home/tof/Documents/Cozy/latex-include/preambule_lua.tex}
\newcommand{\showprof}{show them}  % comment this line if you don't want to see todo environment
\fancyhead[L]{Exercices nombres flottants}
\newdate{madate}{10}{09}{2020}
\fancyhead[R]{Première - NSI} %\today
\fancyfoot[L]{~\\Christophe Viroulaud}
\fancyfoot[C]{\textbf{Page \thepage}}
\fancyfoot[R]{\includegraphics[width=2cm,align=t]{/home/tof/Documents/Cozy/latex-include/cc.png}}

\begin{document}
\begin{Form}
\begin{exo}
Quel est le type Python des valeurs suivantes?
\begin{enumerate}
\item 12
\item 23.7
\item -8
\item 36.
\end{enumerate}
\end{exo}
\begin{exo}
Convertir ces nombres réels en représentation binaire sur 32 bits, en utilisant la norme IEEE~754.
\begin{enumerate}
\item 1 01111111 11110000000000000000000
\item 0 10000011 11100000000000000000000
\item 0 01011000 01000101100000000000000
\end{enumerate}
\end{exo}
\begin{exo}
Donner la représentation flottante en simple précision des nombres réels suivants:
\begin{enumerate}
\item 255
\item -32,75
\item 0,125
\end{enumerate}
\end{exo}
\end{Form}
\end{document}