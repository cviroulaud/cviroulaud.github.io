\documentclass[a4paper,11pt]{article}
\input{/home/tof/Documents/Cozy/latex-include/preambule_lua.tex}
\newcommand{\showprof}{show them}  % comment this line if you don't want to see todo environment
\fancyhead[L]{Paradigme fonctionnel - correction exercices}
\newdate{madate}{10}{09}{2020}
\fancyhead[R]{Terminale - NSI} %\today
\fancyfoot[L]{~\\Christophe Viroulaud}
\fancyfoot[C]{\textbf{Page \thepage}}
\fancyfoot[R]{\includegraphics[width=2cm,align=t]{/home/tof/Documents/Cozy/latex-include/cc.png}}

\begin{document}
\begin{Form}
\begin{exo}
\lstinputlisting[firstline=9,lastline=19]{"scripts/applique.py"}
\end{exo}
\begin{exo}
\lstinputlisting[firstline=9,lastline=25]{"scripts/repeter_delai.py"}
\end{exo}
\begin{exo}
\lstinputlisting[firstline=9,lastline=47]{"scripts/trouve.py"}
\end{exo}
\begin{exo}
\begin{enumerate}
\item Un invariant de boucle est une propriété qui est vraie avant chaque itération de la boucle et qui reste vraie après. Il permet de prouver la correction de l'algorithme.
\item Avant la première itération, \emph{i} vaut 1. Le tableau [O:1] ne contient qu'une valeur, il est donc trié.
\item Raisonnement par récurrence:
\begin{itemize}
\item \textbf{initialisation:} La propriété est vraie avant la première itération (question précédente.
\item \textbf{hérédité:} Considérons la propriété vraie au rang \emph{k}. Le tableau [0:k] est trié.
\item \textbf{conclusion:} À l'itération \emph{n-1} (dernière itération) la boucle \emph{while} insère à la bonne place le \emph{n-ème} élément dans le tableau [0:n-1] déjà trié. Le tableau [0:n] est donc trié.
\end{itemize}
\end{enumerate}
\end{exo}
\begin{exo}
\lstinputlisting[firstline=10,lastline=16]{"scripts/compose.py"}
\end{exo}
\begin{exo}
\lstinputlisting[firstline=9,lastline=25]{"scripts/calcul.py"}
\end{exo}
\end{Form}
\end{document}