\documentclass[svgnames,11pt]{beamer}
\input{/home/tof/Documents/Cozy/latex-include/preambule_commun.tex}
\input{/home/tof/Documents/Cozy/latex-include/preambule_beamer.tex}
%\usepackage{pgfpages} \setbeameroption{show notes on second screen=left}
\author[]{Christophe Viroulaud}
\title{Exercices arbre binaire\\correction}
\date{\framebox{\textbf{Algo 08}}}
%\logo{}
\institute{Terminale - NSI}

\begin{document}
\begin{frame}
    \titlepage
\end{frame}
\section{Exercice 1}
\begin{frame}
    \frametitle{Exercice 1}

    \begin{enumerate}
        \item \begin{itemize}
                  \item préfixe: × - 12 8 + 7 9
                  \item infixe: 12 - 8 × 7 + 9
                  \item postfixe: 12 8 - 7 9 + ×
              \end{itemize}
        \item 64
        \item Parcours infixe
    \end{enumerate}

\end{frame}
\section{Exercice 2}
\begin{frame}
    \frametitle{Exercice 2}

    \begin{enumerate}
        \item
              \begin{itemize}
                  \item en largeur: 1 2 3 4 5 6 7 8 9 10 11 12 13
                  \item préfixe: 1 2 4 8 5 3 6 9 10 12 13 7 11
                  \item infixe: 4 8 2 5 1 9 6 12 10 13 3 11 7
                  \item postfixe: 8 4 5 2 9 12 13 10 6 11 7 3 1

              \end{itemize}
        \item La hauteur est 4.
        \item Cet arbre est équilibré car la hauteur de chaque sous-arbre gauche diffère au plus de 1 de chaque sous-arbre droit.
        \item Cet arbre n'est pas complet car tous les niveaux ne sont pas remplis.
    \end{enumerate}

\end{frame}
\section{Exercice 3}
\begin{frame}[fragile]
    \frametitle{Exercice 3}

\begin{center}
\begin{lstlisting}[language=Python , basicstyle=\ttfamily\small, xleftmargin=.5em, xrightmargin=-1em]
class Arbre_binaire:
    def __init__(self, h: int):
        self.hauteur = h
        self.arbre = [None for i in range(2**(h+1)-1)]
        self.arbre[0] = "r"
\end{lstlisting}
\captionof{code}{Constructeur}
\label{CODE}
\end{center}    

\end{frame}
\begin{frame}[fragile]

\begin{center}
\begin{lstlisting}[language=Python , basicstyle=\ttfamily\small, xleftmargin=.5em, xrightmargin=-1em]
def get_taille(self) -> int:
    taille = 0
    for elt in self.arbre:
        # on évite les noeuds vides
        if elt is not None:
            taille += 1
    return taille
\end{lstlisting}
\end{center}    
\begin{aretenir}[Remarque]
Pour alléger le diaporama les \emph{docstring} ne sont pas présentes. Elles sont écrites dans le fichier Python. 
\end{aretenir}
\end{frame}
\begin{frame}[fragile]

\begin{center}
\begin{lstlisting}[language=Python , basicstyle=\ttfamily\small, xleftmargin=.5em, xrightmargin=-1em]
def get_indice(self, chaine: str) -> int:
    i = 0
    while self.arbre[i] != chaine:
        i = i+1
    return i
\end{lstlisting}
\end{center}    
    \begin{aretenir}[Remarque]
    Tous les nœuds sont distincts.
    \end{aretenir}
    \end{frame}
\begin{frame}[fragile]

\begin{center}
\begin{lstlisting}[language=Python , basicstyle=\ttfamily\small, xleftmargin=.5em, xrightmargin=-5.5em]
def inserer(self, pere: str, gauche: str, droit: str) -> None:
    i_pere = self.get_indice(pere)
    #assert 2*i_pere+2 < len(self.arbre)
    assert 2*i_pere+2 < 2**(self.hauteur+1)-1, "dépassement de taille"
    self.arbre[2*i_pere+1] = gauche
    self.arbre[2*i_pere+2] = droit
\end{lstlisting}
\end{center}    
    \begin{aretenir}[Remarque]
La taille du tableau est fixée.
    \end{aretenir}
    \end{frame}
\begin{frame}[fragile]

\begin{center}
\begin{lstlisting}[language=Python , basicstyle=\ttfamily\small, xleftmargin=.5em, xrightmargin=-3em]
def prefixe(self, position: int, parcours: list) -> None:
    if position < len(self.arbre) and \
                    self.arbre[position] is not None:
        parcours.append(self.arbre[position])
        self.prefixe(2*position+1, parcours)
        self.prefixe(2*position+2, parcours)
\end{lstlisting}
\end{center} 
    \end{frame}
    \begin{frame}[fragile]

\begin{center}
\begin{lstlisting}[language=Python , basicstyle=\ttfamily\small, xleftmargin=.5em, xrightmargin=-3.5em]
def infixe(self, position: int, parcours: list) -> None:
    if position < len(self.arbre) and \
                    self.arbre[position] is not None:
        self.infixe(2*position+1, parcours)
        parcours.append(self.arbre[position])
        self.infixe(2*position+2, parcours)




def postfixe(self, position: int, parcours: list) -> None:
    if position < len(self.arbre) and \
                    self.arbre[position] is not None:
        self.postfixe(2*position+1, parcours)
        self.postfixe(2*position+2, parcours)
        parcours.append(self.arbre[position])
\end{lstlisting}
\end{center} 
    \end{frame}
\begin{frame}[fragile]

\begin{center}
\begin{lstlisting}[language=Python , basicstyle=\ttfamily\small, xleftmargin=.5em, xrightmargin=-1em]
def prefixe_2(self, position: int) -> list:
    if position < len(self.arbre) and \
                    self.arbre[position] is not None:
        return [self.arbre[position]] + \
                    self.prefixe_2(2*position+1) + \
                    self.prefixe_2(2*position+2)
    else:  # cas limite
        return []
\end{lstlisting}
\end{center} 
\begin{aretenir}[Remarque]
La méthode retourne un tableau, il faut donc retourner un tableau vide dans le cas limite.
\end{aretenir}
    \end{frame}
\end{document}