\documentclass[a4paper,11pt]{article}
\input{/home/tof/Documents/Cozy/latex-include/preambule_lua.tex}
\newcommand{\showprof}{show them}  % comment this line if you don't want to see todo environment
\fancyhead[L]{Récursivité - exercices}
\newdate{madate}{10}{09}{2020}
\fancyhead[R]{Terminale - NSI} %\today
\fancyfoot[L]{~\\Christophe Viroulaud}
\fancyfoot[C]{\textbf{Page \thepage}}
\fancyfoot[R]{\includegraphics[width=2cm,align=t]{/home/tof/Documents/Cozy/latex-include/cc.png}}

\begin{document}
\begin{Form}
\begin{exo}
La somme des entiers s'écrit:
$$0+1+2+...+n$$
\begin{enumerate}
\item Donner une définition récursive de la somme des entiers.
\item Implémenter la fonction \textbf{somme(n: int)\;\rightarrow\;int}.
\end{enumerate}
\end{exo}
\begin{exo}
La fonction factorielle est définie par:
$$n!=1×2×3...×n \qquad si\qquad n>0 \qquad et \qquad 0!=1$$
\begin{enumerate}
\item Donner une définition récursive qui correspond au calcul de la fonction factorielle.
\item Implémenter la fonction \textbf{factorielle(n: int)\;\rightarrow\;int}.
\end{enumerate}
\end{exo}
\begin{exo}
Soit $u_n$ la suite d'entiers définie par $u_0>1$ et:
$$
u_{n+1} = \left\{
    \begin{array}{ll}
        u_n/2 & \mbox{si  }u_n \mbox{ est pair}\\
        3×u_n+1 & \mbox{sinon}\
    \end{array}
\right.
$$
Écrire la fonction \emph{syracuse(u:int)\;\rightarrow\;None} qui affiche les valeurs successives de la suite $u_n$ tant que $u_n>1$. L'appel de la fonction s'effectuera avec une valeur de $u_0$ quelconque.
\end{exo}
\begin{exo}
\begin{enumerate}
\item Écrire une fonction récursive \textbf{entiers(i: int, k: int)\;\rightarrow\;None} qui affiche les entiers entre i et k. Par exemple, entiers(0,3) doit afficher 0 1 2 3.
\item Écrire une fonction récursive \textbf{impairs(i: int, k: int)\;\rightarrow\;None} qui affiche les nombres impairs entre i et k.
\end{enumerate}
\end{exo}
\begin{exo}
Écrire la fonction récursive \textbf{pgcd(a: int, b: int)\;\rightarrow\;int} qui renvoie le Plus Grand Commun Diviseur de a et b. On donne comme précondition: a < b.
\begin{commentprof}
méthode d'Euclide: (20,35)\\
$35 = \overbrace{20}^{a\rightarrow b}×1 + \overbrace{15}^{b\%a\rightarrow a}$\\
20 = 15×1 + 5\\
15 = 5×3 + 0\\
pgcd = 5
\end{commentprof}
\end{exo}
\begin{exo}
Écrire une fonction récursive \textbf{nombre\_chiffres(n: int)\;\rightarrow\;int} qui renvoie le nombre de chiffres qui compose n.
\end{exo}
\begin{exo}
La formulation récursive ci-après permet de calculer les coefficients binomiaux:
$$
C(n,p) = \left\{
    \begin{array}{ll}
        1 & \mbox{si } p=0 \mbox{ ou } n=p\\
        C(n-1,p-1)+C(n-1,p) & \mbox{sinon}\\
    \end{array}
\right.
$$
\begin{enumerate}
\item Écrire une fonction récursive \textbf{C(n: int, p: int)\;\rightarrow\;int} qui renvoie la valeur de C(n,p).
\item Le triangle de Pascal est une présentation des coefficients binomiaux sous la forme d'un triangle. Dessiner le triangle de Pascal à l'aide d'une double boucle \emph{for} pour n variant de 0 à 10.
\end{enumerate}
\end{exo}
\begin{commentprof}
Pour ceux qui ont fini: palindrome
\lstinputlisting[firstline=9,lastline=18]{"scripts/palindrome.py"}
\end{commentprof}
\end{Form}
\end{document}