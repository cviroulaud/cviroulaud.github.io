\documentclass[svgnames,11pt]{beamer}
\input{/home/tof/Documents/Cozy/latex-include/preambule_commun.tex}
\input{/home/tof/Documents/Cozy/latex-include/preambule_beamer.tex}
%\usepackage{pgfpages} \setbeameroption{show notes on second screen=left}
\author[]{Christophe Viroulaud}
\title{Contourner un obstacle}
\date{}
%\logo{}
\institute{Seconde - SNT}

\begin{document}
\begin{frame}
\titlepage
\end{frame}

\section{Problématique}
\begin{frame}
    \frametitle{Problématique}

    Dans la pratique un véhicule n'est pas seul sur la route. Par exemple, il peut être freiné par un véhicule plus lent et il doit donc s'adapter aux divers obstacles qu'il peut rencontrer. 
\begin{center}
    \framebox{Comment contourner un obstacle?}
\end{center}

\end{frame}
\end{document}