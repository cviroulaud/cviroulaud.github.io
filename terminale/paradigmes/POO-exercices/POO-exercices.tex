\documentclass[a4paper,11pt]{article}
\input{/home/tof/Documents/Cozy/latex-include/preambule_lua.tex}
\newcommand{\showprof}{show them}  % comment this line if you don't want to see todo environment
\fancyhead[L]{Exercices POO}
\newdate{madate}{10}{09}{2020}
\fancyhead[R]{\displaydate{madate}} %\today
\fancyfoot[L]{~\\Christophe Viroulaud}
\fancyfoot[C]{\textbf{Page \thepage}}
\fancyfoot[R]{\includegraphics[width=2cm,align=t]{/home/tof/Documents/Cozy/latex-include/cc.png}}

\begin{document}
\begin{Form}
\begin{exo}
En Python, la fonction \emph{\_\_init\_\_} est nommée \emph{constructeur}. Elle est appelée automatiquement lors de l'instanciation de l'objet.
\begin{commentprof}
En pratique elle est réalisée par la composition de deux opérations : l'allocation et l'initialisation. L'allocation consiste à réserver un espace mémoire au nouvel objet. L'initialisation consiste à fixer l'état du nouvel objet. 
\end{commentprof}
Il existe un certain nombre d'autres fonctions particulières, chacune avec un nom prédéfini et entourée d'une paire de \_.
Tous les opérateurs classiques de Python peuvent être ainsi redéfinis pour un objet:
\begin{itemize}
\item \emph{\_\_eq\_\_} définit l'opérateur \emph{==}
\item \emph{\_\_lt\_\_} définit l'opérateur \emph{<}
\item \emph{\_\_contains\_\_} définit l'opérateur \emph{in}
\end{itemize}
\begin{commentprof}
\_\_le\_\_ <=, \_\_gt\_\_ >, \_\_ge\_\_ >=, \_\_ne\_\_ !=
\end{commentprof}
\begin{enumerate}
\item Définir une classe \emph{Loto}. Cette classe possède un attribut \emph{numeros} de type \emph{list} qui contiendra les 6 numéros du tirage du loto et un attribut \emph{complementaire} qui désignera le numéro complémentaire.
\item Définir la méthode \_\_str\_\_ qui renverra une chaîne de caractères des numéros du loto.
\item Définir la méthode \_\_contains\_\_ qui vérifiera si le numéro donné en argument est dans les 6 numéros du loto.
\item Définir la méthode \emph{est\_gagnant} qui possède deux arguments: une liste d'entiers et un entier. Elle renverra True si le tirage correspond exactement à la proposition.
\end{enumerate}
\end{exo}
\begin{exo}
On définit une classe \emph{Fraction} pour représenter un nombre rationnel. Cette classe possède deux attributs \emph{numerateur} et \emph{denominateur}. Le dénominateur doit être strictement positif.
\begin{enumerate}
\item Écrire le constructeur de cette classe. Il doit lever une ValueError si le dénominateur n'est pas strictement positif.
\item Définir la méthode \emph{\_\_str\_\_} qui renvoie une chaîne de caractère de la forme "12 / 7" ou simplement "12" si le dénominateur est égal à 1.
\item Définir les méthodes \emph{\_\_eq\_\_} et \emph{\_\_lt\_\_} qui reçoivent une deuxième fraction en argument et renvoient True si la première fraction représente respectivement un nombre égal ou strictement inférieur à la deuxième fraction.
\item Définir les méthodes \emph{\_\_add\_\_} et \emph{\_\_mul\_\_} qui reçoivent une deuxième fraction en argument et renvoient une nouvelle fraction représentant respectivement la somme et le produit des deux fractions.
\end{enumerate}
\end{exo}
\begin{exo}
On définit une classe \emph{Date} pour représenter une date avec trois attributs \emph{jour, mois} et \emph{annee}.
\begin{enumerate}
\item Écrire son constructeur.
\item Définir la méthode \emph{\_\_str\_\_} qui renvoie une chaîne de la forme "8 mai 1945".
\item Définir la méthode \emph{\_\_lt\_\_} qui permet de comparer deux dates.
\item Tester ces méthodes.
\end{enumerate}
\end{exo}
\end{Form}
\end{document}