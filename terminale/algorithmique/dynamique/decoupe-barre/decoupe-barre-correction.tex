\documentclass[a4paper,11pt]{article}
\input{/home/tof/Documents/Cozy/latex-include/preambule_doc.tex}
\input{/home/tof/Documents/Cozy/latex-include/preambule_commun.tex}
\newcommand{\showprof}{show them}  % comment this line if you don't want to see todo environment
\setlength{\fboxrule}{0.8pt}
\fancyhead[L]{\fbox{\Large{\textbf{ProgDyn 04}}}}
\fancyhead[C]{\textbf{Découpe d'une barre métallique - correction}}
\newdate{madate}{10}{09}{2020}
%\fancyhead[R]{\displaydate{madate}} %\today
%\fancyhead[R]{Seconde - SNT}
%\fancyhead[R]{Première - NSI}
\fancyhead[R]{Terminale - NSI}
\fancyfoot[L]{\vspace{1mm}Christophe Viroulaud}
\AtEndDocument{\label{lastpage}}
\fancyfoot[C]{\textbf{Page \thepage/\pageref{lastpage}}}
\fancyfoot[R]{\includegraphics[width=2cm,align=t]{/home/tof/Documents/Cozy/latex-include/cc.png}}

\begin{document}
On crée cette fois, un dictionnaire pour stocker les informations.

\lstinputlisting[firstline=25 ,lastline= 25]{"scripts/decoupe-barre.py"}

On peut s'appuyer sur le problème du rendu de monnaie. L'approche naïve effectue un nombre d'appels conséquent.

\lstinputlisting[firstline=10 ,lastline= 20]{"scripts/decoupe-barre.py"}

\lstinputlisting[firstline=29 ,lastline= 46]{"scripts/decoupe-barre.py"}

\lstinputlisting[firstline=50 ,lastline= 59]{"scripts/decoupe-barre.py"}

\end{document}