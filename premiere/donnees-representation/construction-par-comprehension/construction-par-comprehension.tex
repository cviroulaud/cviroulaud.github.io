\documentclass[a4paper,11pt]{article}
\input{/home/tof/Documents/Cozy/latex-include/preambule_lua.tex}
\newcommand{\showprof}{show them}  % comment this line if you don't want to see todo environment
\fancyhead[L]{Construction par compréhension}
\newdate{madate}{10}{09}{2020}
\fancyhead[R]{Première - NSI} %\today
\fancyfoot[L]{~\\Christophe Viroulaud}
\fancyfoot[C]{\textbf{Page \thepage}}
\fancyfoot[R]{\includegraphics[width=2cm,align=t]{/home/tof/Documents/Cozy/latex-include/cc.png}}

\begin{document}
\begin{Form}
\section{Problématique}
Construire et remplir un tableau demande plusieurs étapes.
\begin{center}
\begin{lstlisting}
tab = []
for i in range(100):
	tab.append(i)

tab2 = [0] * 100
for i in range(len(tab2)):
	tab2[i] = i
\end{lstlisting}
\captionof{code}{Construire un tableau}
\label{construction}
\end{center}

Pourtant Python est un langage de haut niveau et devrait pouvoir faciliter ces constructions.
\begin{center}
\shadowbox{\parbox{16cm}{\centering Comment construire des tableaux en utilisant les propriétés d'un langage de haut niveau?}}
\end{center}
\section{Construction par compréhension}
\subsection{Un tableau simple}
La méthode ne diffère pas énormément du code \ref{construction}. La boucle est exécutée directement pendant la construction du tableau.
\begin{code}[!h]
\begin{lstlisting}
tab = [i for i in range(100)]
\end{lstlisting}
\captionof{code}{Construction par compréhension}
\label{comprehension}
\end{code}

Ce code exécute d'abord la boucle, garde les valeurs en mémoire à chaque tour de boucle (ici \emph{i}) puis construit le tableau avec ces valeurs. Enfin il affecte ce tableau à la variable \emph{tab}.\\
Cet ordre à une réelle importance. Il est ainsi possible de construire un tuple par compréhension.
\begin{code}[!h]
\begin{lstlisting}
t = tuple(i for i in range(100))	
\end{lstlisting}
\captionof{code}{Tuple par compréhension}
\label{tuple}
\end{code}
\begin{commentprof}
mettre \textbf{tuple} sinon on construit un générateur; avec une boucle on ne peut pas remplir un tuple
\end{commentprof}
\begin{activite}
\begin{enumerate}
\item Construire par compréhension le tableau des carrés des nombres de 0 à 10.
\item Construire par compréhension le tableau des nombres impairs de 0 à 20.
\item Construire par compréhension le tableau des multiples de 3 compris entre 0 et 50.
\end{enumerate}
\end{activite}
\subsection{Tableau de tableaux}\label{tableau}
Un tableau peut stocker tous types d'objets même d'autres tableaux. Ceci peut s'avérer très utiles pour représenter une matrice mathématique ou le plateau d'un jeu.\\
Le jeu du \emph{puissance 4} comporte six lignes de sept colonnes. Une représentation en mémoire possible est:
\begin{code}[!h]
\begin{lstlisting}
grille = [[False for col in range(7)] for ligne in range(6)]
\end{lstlisting}
\captionof{code}{Puissance 4}
\label{puissance4}
\end{code}
\begin{activite}
Tester le code \ref{puissance4} sur \url{http://pythontutor.com/} et visualiser la représentation en mémoire.
\end{activite}
\subsection{Une erreur classique}
Il pourrait être tentant de coder la grille de cette manière.
\begin{code}[!h]
\begin{lstlisting}
grille = [[False] * 7] * 6
\end{lstlisting}
\captionof{code}{Puissance 4}
\label{puissance4bad}
\end{code}
\begin{activite}
Tester le code \ref{puissance4bad} sur \emph{Pythontutor} et observer la représentation en mémoire.
\end{activite}
Il faut rappeler que la variable grille ne contient pas réellement le tableau mais uniquement l'adresse vers celui-ci. Ainsi le même tableau contenant sept valeurs \emph{False} est crée une fois puis pointé six fois. D'une manière générale nous construirons les tableaux par compréhension en utilisant la méthodologie vue en \ref{tableau}.
\section{Slice (hors programme)}
Les \emph{slices} (tranches) sont des expressions qui permettent d'extraire facilement des éléments d'un tableau.
\begin{code}[!h]
\begin{lstlisting}
tab = [0, 1, 2, 3, 4, 5]
tranche = tab[1:3] # La variable tranche contient [1, 2]
\end{lstlisting}
\captionof{code}{Extraire une tranche}
\label{slice}
\end{code}

Le code \ref{slice} montre la syntaxe d'un \emph{slice}. La variable \emph{tranche} contient les valeurs entre l'indice 1 \emph{inclus} et 3 \emph{exclus}.\\
Les exemples ci-après montrent la puissance d'un langage de haut-niveau comme Python.
\begin{lstlisting}
tab = [0, 1, 2, 3, 4, 5]
debut = tab[:3] # [0, 1, 2]
fin = tab[3:] # [3, 4, 5]
\end{lstlisting}
\end{Form}
\end{document}