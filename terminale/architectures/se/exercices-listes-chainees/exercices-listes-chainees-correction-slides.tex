\documentclass[svgnames,11pt]{beamer}
\input{/home/tof/Documents/Cozy/latex-include/preambule_commun.tex}
\input{/home/tof/Documents/Cozy/latex-include/preambule_beamer.tex}
%\usepackage{pgfpages} \setbeameroption{show notes on second screen=left}
\author[]{Christophe Viroulaud}
\title{Exercices listes chaînées\\Correction}
\date{\framebox{\textbf{Archi 04}}}
%\logo{}
\institute{Terminale - NSI}

\begin{document}
\begin{frame}
\titlepage
\end{frame}
\section{Exercice 1}
\begin{frame}[fragile]
    \frametitle{Exercice 1 - 1}

\begin{lstlisting}[language=Python , basicstyle=\ttfamily\small, xleftmargin=0em, xrightmargin=-1em]
class Maillon:
    """
    Crée un maillon de la liste chaînée
    """

    def __init__(self, val: int, suiv: object) -> None:
        self.valeur = val
        self.suivant = suiv
\end{lstlisting}

\end{frame}
\begin{frame}[fragile]
    \frametitle{Exercice 1 - 2}

\begin{lstlisting}[language=Python , basicstyle=\ttfamily\small, xleftmargin=2em, xrightmargin=2em]
class Liste:
    """
    Crée une liste chaînée
    """

    def __init__(self):
        self.tete: Maillon = None
\end{lstlisting}

\end{frame}
\begin{frame}[fragile]
    \frametitle{Exercice 1 - 3}

\begin{lstlisting}[language=Python , basicstyle=\ttfamily\small, xleftmargin=2em, xrightmargin=2em]
def ajoute(self, val: int) -> None:
    self.tete = Maillon(val, self.tete)
\end{lstlisting}

\end{frame}
\begin{frame}[fragile]
    \frametitle{Exercice 1 - 4}

\begin{lstlisting}[language=Python , basicstyle=\ttfamily\small, xleftmargin=2em, xrightmargin=2em]
lst = Liste()
lst.ajoute(8)
lst.ajoute(5)
lst.ajoute(3)
lst.ajoute(9)
lst.ajoute(10)
\end{lstlisting}

\end{frame}
\begin{frame}[fragile]
    \frametitle{Exercice 1 - 5}

\begin{lstlisting}[language=Python , basicstyle=\ttfamily\small, xleftmargin=2em, xrightmargin=2em]
def dernier(self) -> int:
    # gestion d'erreur
    if self.tete is None:
        return -1

    en_cours = self.tete
    while en_cours.suivant is not None:
        en_cours = en_cours.suivant
    return en_cours.valeur
\end{lstlisting}

\end{frame}
\begin{frame}[fragile]
    \frametitle{Exercice 1 - 6}

\begin{lstlisting}[language=Python , basicstyle=\ttfamily\small, xleftmargin=2em, xrightmargin=2em]
def dernier_aux(self, m: Maillon) -> int:
    if m.suivant is None:
        return m.valeur
    else:
        return self.dernier_aux(m.suivant)

def dernier_rec(self) -> int:
    # gestion d'erreur
    if self.tete is None:
        return -1

    return self.dernier_aux(self.tete)
\end{lstlisting}

\end{frame}
\begin{frame}[fragile]
    \frametitle{Exercice 1 - 7}
\begin{center}
    

    \begin{tikzpicture}[scale=0.6]
        \node[draw,minimum height=0.5cm] (9) at (-10,2) {9};
        \node[draw,minimum height=0.5cm] (3) at (-8,2) {3};
        \node[draw,minimum height=0.5cm] (5) at (-6,2) {5};
        \node[draw,minimum height=0.5cm] (8) at (-4,2) {8};
        \node[draw,minimum height=0.5cm] (fin) at (-2,2) {fin};
        \draw[->,>=latex] (-10,0) -- (9);

        \draw[->,>=latex] (9) -- (3);
        \draw[->,>=latex] (3) -- (5);
        \draw[->,>=latex] (5) -- (8);
        \draw[->,>=latex] (8) -- (fin);

        \node[draw,minimum height=0.5cm] (lst) at (-13,5) {fin};

    \end{tikzpicture}
    \captionof{figure}{Tant que le \textbf{\texttt{Maillon}} en cours n'est pas \textbf{\texttt{None}}}
\end{center}
\end{frame}
\begin{frame}[fragile]
    \frametitle{Exercice 1 - 7}
\begin{center}
    \begin{tikzpicture}[scale=0.6]
        \node[draw,minimum height=0.5cm] (9) at (-10,2) {9};
        \node[draw,minimum height=0.5cm] (3) at (-8,2) {3};
        \node[draw,minimum height=0.5cm] (5) at (-6,2) {5};
        \node[draw,minimum height=0.5cm] (8) at (-4,2) {8};
        \node[draw,minimum height=0.5cm] (fin) at (-2,2) {fin};
        \draw[->,>=latex] (-8,0) -- (3);

        \draw[->,>=latex] (9) -- (3);
        \draw[->,>=latex] (3) -- (5);
        \draw[->,>=latex] (5) -- (8);
        \draw[->,>=latex] (8) -- (fin);

        \node[draw,minimum height=0.5cm] (lst) at (-13,6) {fin};
        \node[draw,minimum height=0.5cm] (92) at (-10,6) {9};

        \draw[->,>=latex] (92) -- (lst);
    \end{tikzpicture}
    \captionof{figure}{Créer un nouveau \textbf{\texttt{Maillon}} et l'attacher au précédent.}
\end{center}
\end{frame}
\begin{frame}[fragile]
    \frametitle{Exercice 1 - 8}

\begin{lstlisting}[language=Python , basicstyle=\ttfamily\small, xleftmargin=2em, xrightmargin=2em]
def renverser(self) -> None:
    res = None
    en_cours = self.tete
    while en_cours is not None:
        # crée maillon et l'attache au précédant
        res = Maillon(en_cours.valeur, res)
        # va voir le maillon suivant
        en_cours = en_cours.suivant
    self.tete = res
\end{lstlisting}

\end{frame}
\begin{frame}[fragile]
    \frametitle{Exercice 1 - 9}

\begin{lstlisting}[language=Python , basicstyle=\ttfamily\small, xleftmargin=2em, xrightmargin=2em]
def dupliquer(self):
    en_cours = self.tete
    while en_cours is not None:
        doublon = Maillon(en_cours.valeur, en_cours.suivant)
        en_cours.suivant = doublon
        en_cours = doublon.suivant
\end{lstlisting}

\end{frame}
\section{Exercice 2}
\begin{frame}[fragile]
    \frametitle{Exercice 2 - 1}

\begin{lstlisting}[language=Python , basicstyle=\ttfamily\small, xleftmargin=2em, xrightmargin=2em]
from liste import Liste, Maillon
\end{lstlisting}    

\end{frame}
\begin{frame}[fragile]
    \frametitle{Exercice 2 - 2}

\begin{lstlisting}[language=Python , basicstyle=\ttfamily\small, xleftmargin=2em, xrightmargin=2em]
l1 = Liste()
l1.ajoute(8)
l1.ajoute(5)
l1.ajoute(3)
l1.ajoute(9)
l1.ajoute(10)

l2 = Liste()
l2.ajoute(4)
l2.ajoute(0)
l2.ajoute(2)
\end{lstlisting}    

\end{frame}
\begin{frame}[fragile]
    \frametitle{Exercice 2 - 3}

\begin{lstlisting}[language=Python , basicstyle=\ttfamily\small, xleftmargin=1em, xrightmargin=0em]
def concatener(l1: Liste, l2: Liste) -> Liste:
    def concatener_rec(tete1: Maillon, tete2: Maillon) -> Maillon:
        """
        fonction interne pour additionner 2 listes
        """
        if tete1 is None:
            return tete2
        else:
            return Maillon(tete1.valeur, concatener_rec(tete1.suivant, tete2))

    res = Liste()
    res.tete = concatener_rec(l1.tete, l2.tete)
    return res
\end{lstlisting}    

\end{frame}
\begin{frame}
    \frametitle{Exercice 2 - 4}

La liste 2 n'est pas recopiée dans la nouvelle liste. Une modification du contenu de l'une d'elle modifie l'autre: c'est un \textbf{effet de bord}.

\end{frame}
\begin{frame}
    \frametitle{Exercice 2 - 5}

Le complexité dépend de la taille de la première liste, qui est entièrement copiée.

\end{frame}
\section{Exercice 3}
\begin{frame}[fragile]
    \frametitle{Exercice 3 - 1}

\begin{lstlisting}[language=Python , basicstyle=\ttfamily\small, xleftmargin=2em, xrightmargin=2em]
def longueur(lst: tuple) -> int:
    if len(lst) == 0:
        return 0
    else:
        return 1+longueur(lst[1])
\end{lstlisting}    

\end{frame}
\begin{frame}[fragile]
    \frametitle{Exercice 3 - 2}

\begin{lstlisting}[language=Python , basicstyle=\ttfamily\small, xleftmargin=1em, xrightmargin=1em]
def afficher(lst: tuple) -> str:
    if len(lst) == 0:
        return "fin"
    else:
        return lst[0] + " - " + afficher(lst[1])
\end{lstlisting}    

\end{frame}
\end{document}