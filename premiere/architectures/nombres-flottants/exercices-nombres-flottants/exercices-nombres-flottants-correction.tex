\documentclass[a4paper,11pt]{article}
\input{/home/tof/Documents/Cozy/latex-include/preambule_lua.tex}
\newcommand{\showprof}{show them}  % comment this line if you don't want to see todo environment
\fancyhead[L]{Exercices nombres flottants}
\newdate{madate}{10}{09}{2020}
\fancyhead[R]{Première - NSI} %\today
\fancyfoot[L]{~\\Christophe Viroulaud}
\fancyfoot[C]{\textbf{Page \thepage}}
\fancyfoot[R]{\includegraphics[width=2cm,align=t]{/home/tof/Documents/Cozy/latex-include/cc.png}}

\begin{document}
\begin{Form}
\begin{exo}
Quel est le type Python des valeurs suivantes?
\begin{enumerate}
\item 12$\;\rightarrow\;$int
\item 23.7$\;\rightarrow\;$float
\item -8$\;\rightarrow\;$int
\item 36.$\;\rightarrow\;$float
\end{enumerate}
\end{exo}
\begin{exo}
Convertir ces nombres réels en représentation binaire sur 32 bits, en utilisant la norme IEEE~754.
\begin{enumerate}
\item 1 01111110 11110000000000000000000
\begin{itemize}
\item signe: $(-1)^1=-1$
\item exposant: $(2^6+2^5+2^4+2^3+2^2+2^1)-127=126-127=-1$
\item mantisse: $1+2^{-1}+2^{-2}+2^{-3}+2^{-4} = 1,9375$
\item $-1×1,9375×2^{-1}=-0,96875$
\end{itemize}
\item 0 10000011 11100000000000000000000
\begin{itemize}
\item signe: $(-1)^0=1$
\item exposant: $(2^7+2^1+2^0)-127=131-127=4$
\item mantisse: $1+2^{-1}+2^{-2}+2^{-3} = 1,875$
\item $1×1,875×2^{4}=30$
\end{itemize}
\item 0 01011000 01000101100000000000000
\begin{itemize}
\item signe: $(-1)^0=1$
\item exposant: $(2^6+2^4+2^3)-127=88-127=-39$
\item mantisse: $1+2^{-2}+2^{-6}+2^{-8}+2^{-9} = 1,271484375$
\item $1×1,271484375×2^{-39}=2,3128166×10^{-12}$
\end{itemize}
\end{enumerate}
\end{exo}
\begin{exo}
Donner la représentation flottante en simple précision des nombres réels suivants:
\begin{enumerate}
\item $255_{10}=11111111_2=1,1111111×2^7$
\begin{itemize}
\item signe: 0
\item exposant: $7+127=134_{10}=10000110_2$
\item mantisse: $11111110000000000000000$
\item $0 10000110 11111110000000000000000$
\end{itemize}
\item $-1×32,75_{10}=-1×100000,11=-1×1,0000011×2^5$
\begin{itemize}
\item signe: 1
\item exposant: $5+127=132_{10}=10000100_2$
\item mantisse: $00000110000000000000000$
\item $1 10000100 00000110000000000000000$
\end{itemize}
\item $0,125_{10}=0,001_2=1×2^{-3}$
\begin{itemize}
\item signe: 0
\item exposant: $-3+127=124_{10}=01111100_2$
\item mantisse: $00000000000000000000000$
\item $0 01111100 00000000000000000000000$
\end{itemize}
\end{enumerate}
\end{exo}
\end{Form}
\end{document}