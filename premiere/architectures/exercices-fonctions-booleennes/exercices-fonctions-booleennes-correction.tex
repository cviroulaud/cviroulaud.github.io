\documentclass[a4paper,11pt]{article}
\input{/home/tof/Documents/Cozy/latex-include/preambule_lua.tex}
\newcommand{\showprof}{show them}  % comment this line if you don't want to see todo environment
\fancyhead[L]{Exercices fonctions booléennes}
\newdate{madate}{10}{09}{2020}
\fancyhead[R]{Première - NSI} %\today
\fancyfoot[L]{~\\Christophe Viroulaud}
\fancyfoot[C]{\textbf{Page \thepage}}
\fancyfoot[R]{\includegraphics[width=2cm,align=t]{/home/tof/Documents/Cozy/latex-include/cc.png}}

\begin{document}
\begin{Form}
\begin{exo}
\begin{table}[!h]
\begin{center}
\begin{tabular}{|c|c|c|}
\hline 
P & Q & non (P) ou Q \\ 
\hline 
0 & 0 & 1 \\ 
\hline 
0 & 1 & 1 \\ 
\hline 
1 & 0 & 0 \\
\hline 
1 & 1 & 1 \\
\hline 
\end{tabular}
\caption{\label{ex1}non (P) ou Q}
\end{center}
\end{table} 
\end{exo}
\begin{exo}
La réponse est: non(Q) et P
\end{exo}
\begin{exo}
\begin{table}[!h]
\begin{center}
\begin{tabular}{|*4{c|}}
\hline 
A & B & C & C et (A ou B) \\ 
\hline 
0 & 0 & 0 & 0 \\ 
\hline 
0 & 0 & 1 & 0 \\ 
\hline 
0 & 1 & 0 & 0 \\
\hline 
0 & 1 & 1 & 1 \\
\hline 
1 & 0 & 0 & 0 \\ 
\hline 
1 & 0 & 1 & 1 \\ 
\hline 
1 & 1 & 0 & 0 \\
\hline 
1 & 1 & 1 & 1 \\
\hline 
\end{tabular}
\caption{\label{ex3}C et (A ou B)}
\end{center}
\end{table} 
\end{exo}
\begin{exo}
Il faut établir la table de vérité des deux parties de l'égalité.
\begin{multicols}{2}
\begin{center}
\begin{tabular}{|*4{c|}}
\hline 
x & y & z & $f_1(x,y,z)$ \\ 
\hline 
0 & 0 & 0 & 0 \\ 
\hline 
0 & 0 & 1 & 1 \\ 
\hline 
0 & 1 & 0 & 0 \\
\hline 
0 & 1 & 1 & 0 \\
\hline 
1 & 0 & 0 & 0 \\ 
\hline 
1 & 0 & 1 & 1 \\ 
\hline 
1 & 1 & 0 & 1 \\
\hline 
1 & 1 & 1 & 1 \\
\hline 
\end{tabular}
\end{center}
\begin{center}
\begin{tabular}{|*4{c|}}
\hline 
x & y & z & $f_2(x,y,z)$ \\ 
\hline 
0 & 0 & 0 & 0 \\ 
\hline 
0 & 0 & 1 & 1 \\ 
\hline 
0 & 1 & 0 & 0 \\
\hline 
0 & 1 & 1 & 0 \\
\hline 
1 & 0 & 0 & 0 \\ 
\hline 
1 & 0 & 1 & 1 \\ 
\hline 
1 & 1 & 0 & 1 \\
\hline 
1 & 1 & 1 & 1 \\
\hline 
\end{tabular}
\end{center}
\end{multicols} 
\end{exo}
\begin{exo}
\begin{table}[!h]
\begin{center}
\begin{tabular}{|*4{c|}}
\hline 
x & y & z & $f(x,y,z)$ \\ 
\hline 
0 & 0 & 0 & 0 \\ 
\hline 
0 & 0 & 1 & 1 \\ 
\hline 
0 & 1 & 0 & 1 \\
\hline 
0 & 1 & 1 & 0 \\
\hline 
1 & 0 & 0 & 1 \\ 
\hline 
1 & 0 & 1 & 0 \\ 
\hline 
1 & 1 & 0 & 0 \\
\hline 
1 & 1 & 1 & 0 \\
\hline 
\end{tabular}
\caption{\label{ex5}$f(x,y,z)=(x\land \lnot y \land \lnot z)\lor (\lnot x \land y \land \lnot z) \lor (\lnot x \land \lnot y \land z)$}
\end{center}
\end{table} 
Cette fonction vérifie si une et une seule variable vaut 1 (voir tableau \ref{ex5}).\\
On reconnaît une fonction XOR sur les quatre premières lignes.
$$(x,y,z)=(\lnot x \land (y \oplus z)) \lor (x \land \lnot y \land \lnot z)$$
\end{exo}
\begin{exo} Voir tableau \ref{addi}.
\begin{table}[!h]
\begin{center}
\begin{tabular}{|*4{c|}|*3{c|}}
\hline 
$e_0$ & $e_1$ & $e_2$ & $e_4$ & $s_0$ & $s_1$ & $c$ \\ 
\hline 
0 & 0 & 0 & 0 & 0 & 0 & 0 \\ 
\hline 
0 & 0 & 0 & 1 & 0 & 1 & 0 \\ 
\hline
0 & 0 & 1 & 0 & 1 & 0 & 0 \\ 
\hline
0 & 0 & 1 & 1 & 1 & 1 & 0 \\ 
\hline
0 & 1 & 0 & 0 & 0 & 1 & 0 \\ 
\hline
0 & 1 & 0 & 1 & 1 & 0 & 0 \\ 
\hline
0 & 1 & 1 & 0 & 1 & 1 & 0 \\ 
\hline
0 & 1 & 1 & 1 & 0 & 0 & 1 \\ 
\hline
1 & 0 & 0 & 0 & 1 & 0 & 0 \\ 
\hline
1 & 0 & 0 & 1 & 1 & 1 & 0 \\ 
\hline
1 & 0 & 1 & 0 & 0 & 0 & 1 \\ 
\hline
1 & 0 & 1 & 1 & 0 & 1 & 1 \\ 
\hline
1 & 1 & 0 & 0 & 1 & 1 & 0 \\ 
\hline
1 & 1 & 0 & 1 & 0 & 0 & 1 \\ 
\hline
1 & 1 & 1 & 0 & 0 & 1 & 1 \\ 
\hline
1 & 1 & 1 & 1 & 1 & 0 & 1 \\ 
\hline
\end{tabular}
\caption{\label{addi}Additionneur 2 bits}
\end{center}
\end{table} 
\end{exo}
\end{Form}
\end{document}