\documentclass[svgnames,11pt]{beamer}
\input{/home/tof/Documents/Cozy/latex-include/preambule_commun.tex}
\input{/home/tof/Documents/Cozy/latex-include/preambule_beamer.tex}
%\usepackage{pgfpages} \setbeameroption{show notes on second screen=left}
\author[]{Christophe Viroulaud}
\title{Principe d'un\\système d'exploitation}
\date{\framebox{\textbf{ArchMat 08}}}
%\logo{}
\institute{Première - NSI}

\begin{document}
\begin{frame}
\titlepage
\end{frame}
\begin{frame}[fragile]
    \frametitle{}

    \begin{center}
        \begin{lstlisting}[language=Bash , basicstyle=\small, xleftmargin=2em, xrightmargin=2em]
MOV R0, #3
ADD R1,R0,#5
HALT
\end{lstlisting}
\captionof{code}{\centering Pour communiquer directement avec le processeur il faut utiliser un langage de bas-niveau.}
    \end{center}

\end{frame}
\begin{frame}
    \frametitle{}

    \begin{itemize}
        \item <1-> Chaque processeur accepte un langage différent.
        \item <2-> Chaque matériel (carte graphique, webcam, clavier\dots) possède des caractéristiques différentes.
    \end{itemize}

\end{frame}
\begin{frame}
    \frametitle{}

    \begin{framed}
        \centering Comment gérer les accès matériels de manière transparente?
    \end{framed}

\end{frame}
\end{document}