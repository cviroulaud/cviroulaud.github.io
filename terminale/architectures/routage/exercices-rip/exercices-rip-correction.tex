\documentclass[a4paper,11pt]{article}
\input{/home/tof/Documents/Cozy/latex-include/preambule_doc.tex}
\input{/home/tof/Documents/Cozy/latex-include/preambule_commun.tex}
\newcommand{\showprof}{show them}  % comment this line if you don't want to see todo environment
\setlength{\fboxrule}{0.8pt}
\fancyhead[L]{\fbox{\Large{\textbf{Routage 03}}}}
\fancyhead[C]{\textbf{Exercices RIP - correction}}
\newdate{madate}{10}{09}{2020}
%\fancyhead[R]{\displaydate{madate}} %\today
%\fancyhead[R]{Seconde - SNT}
%\fancyhead[R]{Première - NSI}
\fancyhead[R]{Terminale - NSI}
\fancyfoot[L]{\vspace{1mm}Christophe Viroulaud}
\AtEndDocument{\label{lastpage}}
\fancyfoot[C]{\textbf{Page \thepage/\pageref{lastpage}}}
\fancyfoot[R]{\includegraphics[width=2cm,align=t]{/home/tof/Documents/Cozy/latex-include/cc.png}}
\usepackage{tikz}

\begin{document}
%DODO pour contrôle voir Q13 P17 du cours sur le routage de Lore Petit
\begin{exo}
    \begin{aretenir}[Remarque]
        Selon la littérature les notations des interfaces et passerelles sont variables. \textbf{Pour chaque interface le routeur possède une adresse IP appartenant au réseau auquel il est connecté.}
    \end{aretenir}
    \begin{enumerate}
        \item \begin{itemize}
            \item 18.13.0.0/16
            \item 192.168.0.0/16
            \item 19.20.1.0/24
        \end{itemize}
        \item Table de R1
        \begin{center}
            \begin{tabular}{|*{4}{c|}}
                \hline
                Destination & Passerelle & Interface & Distance \\
                \hline
                192.168.0.0/16 & & 192.168.0.1 & 1\\
                \hline
                18.13.0.0/16 & 192.168.0.254 (R2) & 192.168.0.1 & 2\\
                \hline
                19.20.1.0/24 & 192.168.0.254 (R2) & 192.168.0.1 & 2\\
                \hline
            \end{tabular}
        \end{center}
    \end{enumerate}
\end{exo}
\begin{exo}
    \begin{aretenir}[Remarque]
    Dans cet exercice, la destination à atteindre n'est pas un réseau mais un routeur.
    \end{aretenir}
\begin{enumerate}
    \item Pour atteindre G on lit les tables de routage:
    \begin{itemize}
        \item table A: Vecteur (C,3),
        \item table C: Vecteur (F,2),
        \item table F: Vecteur (G,1)
    \end{itemize}
    Soit une distance de 6 pour un trajet: A $\rightarrow$ C $\rightarrow$ F $\rightarrow$ G.
    \item Table de G:
    \begin{center}
        \begin{tabular}{|*{3}{c|}}
            \hline
            Destination & Routeur suivant (Passerelle) & Distance \\
            \hline
            A & E & 3\\
            \hline
            B & E & 3\\
            \hline
            C & E & 2\\
            \hline
            D & E & 2\\
            \hline
            E & E & 1\\
            \hline
            F & F & 1\\
            \hline
        \end{tabular}
    \end{center}
    \begin{aretenir}[Remarque]
    Pour atteindre A et C il est possible de passer par F.
    \end{aretenir}
\end{enumerate}
\end{exo}
\begin{exo}
    
    \begin{enumerate}
        \item Phase d'initialisation
        \begin{center}
            \begin{tabular}{|*{4}{c|}}
                \hline
                Destination & Passerelle & Interface & Distance \\
                \hline
                C & & eth1 & 1\\
                \hline
                B & & eth0 & 1\\
                \hline
            \end{tabular}
        \end{center}
        \item Extrait de table pour atteindre G
        \begin{center}
            \begin{tabular}{|*{5}{c|}}
                \hline
                Extrait de la table & Destination & Passerelle & Interface & Distance \\
                \hline
                A & G & B & eth0 & 3\\
                \hline
                B & G & F & eth0 & 2\\
                \hline
                C & G & B & eth2 & 3\\
                \hline
                D & G & E & eth1 & 3\\
                \hline
                E & G & F & eth1 & 2\\
                \hline
                F & G &  & eth1 & 1\\
                \hline
            \end{tabular}
        \end{center}
        \item B envoie une route infinie soit le vecteur (G,16). Le maximum de sauts est 15 avec le protocole RIP.
        \item 
              \begin{itemize}
                  \item Les routeurs A et C reçoivent une route existante plus longue; cela signifie qu'un problème est apparu. Ils mettent leur table à jour: pour G ils enregistrent la route (B, 16).
                  \item Le routeur D reçoit une nouvelle route plus longue vers G: il l'ignore.
                  \item Le routeur C possède la route (B, 16) vers G. Il la remplace par (D, 4).
              \end{itemize}
        \item Le routeur C envoie le vecteur (G, 4) à A (et à B). Ces routeurs mettent leur route vers G à jour avec le vecteur (C, 5).
    \end{enumerate}
\end{exo}
\end{document}