\documentclass[a4paper,11pt]{article}
\input{/home/tof/Documents/Cozy/latex-include/preambule_doc.tex}
\input{/home/tof/Documents/Cozy/latex-include/preambule_commun.tex}
%\newcommand{\showprof}{show them}  % comment this line if you don't want to see todo environment
\setlength{\fboxrule}{0.8pt}
\fancyhead[L]{\fbox{\Large{\textbf{Secu 04}}}}
\fancyhead[C]{\textbf{Exercices sécurité}}
\newdate{madate}{10}{09}{2020}
%\fancyhead[R]{\displaydate{madate}} %\today
%\fancyhead[R]{Seconde - SNT}
%\fancyhead[R]{Première - NSI}
\fancyhead[R]{Terminale - NSI}
\fancyfoot[L]{\vspace{1mm}Christophe Viroulaud}
\AtEndDocument{\label{lastpage}}
\fancyfoot[C]{\textbf{Page \thepage/\pageref{lastpage}}}
\fancyfoot[R]{\includegraphics[width=2cm,align=t]{/home/tof/Documents/Cozy/latex-include/cc.png}}

\begin{document}
\begin{exo}
Le chiffre par \emph{ou exclusif} est réversible: on utilise la même fonction pour chiffrer et déchiffrer le message. Pour pouvoir chiffrer le message bit à bit, il est plus aisé de manipuler des chaînes d'octets (\textbf{\texttt{bytes}}). Chaque caractère est converti en une suite d'octets (voir le cours de première sur l'Unicode). On transforme une chaîne de caractères en chaîne d'octets avec la méthode \textbf{\texttt{encode}}. On utilise la méthode \textbf{\texttt{decode}} pour l'opération inverse.
\begin{commentprof}
Également si on n'utilise pas des chaînes d'octets, le chiffrage peut créer des caractères non imprimables.

UTF-8 utilise 2 octets pour coder le ë; \textbackslash x $\rightarrow$ les 2 prochains caractères sont hexa.
\begin{itemize}
    \item ë $\rightarrow$ \textbackslash xc3\textbackslash xab
    \item c3ab $\rightarrow$ 1100 0011 1010 1011
    \item 1100 0011 1010 1011 $\rightarrow$ 000 1110 1011
    \item 1110 1011 $\rightarrow$ eb
    \item ë $\rightarrow$ U+00eb
\end{itemize}
\end{commentprof}
\begin{enumerate}
    \item Encoder la chaîne de caractères, \emph{noël}, en chaîne d'octets. Remarquer que la chaîne d'octets est enveloppée dans: \emph{b' '}.
    \item Décoder la chaîne obtenue pour retrouver le message d'origine.
\end{enumerate}
En Python il est possible d'effectuer des opérations directement sur les bits. L'opérateur $\wedge$ permet d'effectuer un \emph{ou exclusif} entre deux bits.
\begin{enumerate}[resume]
    \item Dans la console tester l'opérateur et retrouver la table de vérité du \emph{ou exclusif}.
\end{enumerate}
Alice a envoyé à Bob le message chiffré suivant:
\begin{center}
    b'\textbackslash x06Sb\textbackslash x04a\textbackslash x0bjQe/A6j\_\textbackslash x81\textbackslash xe8\textbackslash xf1\textbackslash xe0-[3?Wc'
\end{center}
La clé est:
\begin{center}
    J2B
\end{center}
\begin{enumerate}[resume]
    \item Écrire la fonction \textbf{\texttt{chiffrer\_xor(message: bytes, cle: bytes) $\rightarrow$ bytes}} qui permet de coder et décoder un message par la méthode du \emph{ou exclusif}. \underline{Indication:} Une utilisation judicieuse du modulo permettra d'étendre la clé sous le message.
    \item Retrouver le message envoyé par Alice.
\end{enumerate} 
\end{exo}
\begin{exo}
Le chiffrement par \emph{ou exclusif} n'est pas très sécurisé. Si la clé est trop courte et en connaissant quelques informations, il est possible de casser le code par \emph{brute force}.\\Le message suivant a été intercepté.
\begin{commentprof}
Attention aux guillemets lors d'un copier-coller!
\end{commentprof}
\begin{center}
    b'>\textbackslash x04\textbackslash tR\textbackslash xa2\textbackslash xd3\textbackslash x1e\textbackslash xa2\textbackslash xd2\textbackslash x04\textbackslash x04\textbackslash tR\textbackslash x05\textbackslash x1fR\textbackslash x15\textbackslash x1f\textbackslash x00\textbackslash x0c\textbackslash x13\textbackslash x1c\textbackslash x00\textbackslash x16\textbackslash x17A\\
    \textbackslash t\textbackslash x1d\textbackslash x0f\textbackslash x0eR\textbackslash x15\textbackslash x08\textbackslash x1d\textbackslash x11Z\textbackslash x10\textbackslash x00\textbackslash x16\textbackslash xb1\textbackslash xc9\textbackslash x00\textbackslash x17\textbackslash x12[R5\textbackslash x15\textbackslash x14@'
\end{center}
On sait que:
\begin{itemize}
    \item les quatre dernières lettres du message en clair sont \emph{Tof!}
    \item la clé est une combinaison de trois lettres minuscules.
\end{itemize}
\begin{enumerate}
    \item Écrire la fonction \textbf{\texttt{compare\_fin(message: bytes, fin: bytes) $\rightarrow$ bool}} qui renvoie \emph{True} si la fin de \emph{message} et \emph{fin} sont des chaînes d'octets identiques. Les chaînes d'octets peuvent être considérés comme des tableaux.
    \item Écrire la fonction \textbf{\texttt{bruteforce(secret: bytes) $\rightarrow$ tuple}} qui teste pour clé toutes les combinaisons de trois lettres minuscules. Chaque combinaison est utilisée pour déchiffrer le message puis la fin du message est comparée à \emph{Tof}. La fonction renvoie le tuple (message décrypté, clé).
    \item Tester la fonction et mesurer sa durée d'exécution.
\end{enumerate}
\end{exo}
\end{document}