\documentclass[a4paper,11pt]{article}
\input{/home/tof/Documents/Cozy/latex-include/preambule_lua.tex}
\newcommand{\showprof}{show them}  % comment this line if you don't want to see todo environment
\fancyhead[L]{Représentation des entiers - correction exercices}
\newdate{madate}{10}{09}{2020}
\fancyhead[R]{Première - NSI} %\today
\fancyfoot[L]{~\\Christophe Viroulaud}
\fancyfoot[C]{\textbf{Page \thepage}}
\fancyfoot[R]{\includegraphics[width=2cm,align=t]{/home/tof/Documents/Cozy/latex-include/cc.png}}

\begin{document}
\begin{Form}
\begin{exo}
\begin{commentprof}
faire les ÷
\end{commentprof}
\begin{itemize}
\item $14_{10} \rightarrow 00001110_2$
\item $222_{10} \rightarrow 11011110_2$
\item $42_{10} \rightarrow 00101010_2$
\item $79_{10} \rightarrow 01001111_2$
\end{itemize}
\end{exo}
\begin{exo}
Donner la représentation décimale des nombres binaires (non signés) suivants:
\begin{itemize}
\item $1010_2 \rightarrow 10_{10}$
\item $111110_2 \rightarrow 62_{10}$
\item $100101001_2 \rightarrow 297_{10}$
\end{itemize}
\end{exo}
\begin{exo}
Donner la représentation hexadécimale des nombres binaires suivants:
\begin{commentprof}
penser à faire blocs de 4
\end{commentprof}
\begin{itemize}
\item $10010101_2 \rightarrow 95_{16}$
\item $11010101_2 \rightarrow D5_{16}$
\item $100010001_2 \rightarrow 111_{16}$
\item $11001101001010_2 \rightarrow 334A_{16}$
\end{itemize}
\end{exo}
\begin{exo}
$B×16^3+E×16^2+E×16^1+F×16^0=11×16^3+14×16^2+14×16^1+15×16^0=48879$
\end{exo}
\begin{exo}
\begin{itemize}
\item $10_{10}=00001010_2\;donc\;-10_{10}=11110101+1=11110110_2$
\item $128_{10}=10000000_2\;donc\;-128_{10}=01111111+1=10000000_2$ Nous remarquons qu'il s'agit de la même représentation que 128: sur 8 bits, nous ne pouvons pas représenter l'entier positif 128!!!
\item $42_{10}=00101010_2 donc -42_{10}=11010101+1=11010110_2$
\item $97_{10}=01100001_2$
\end{itemize}
\end{exo}
\begin{exo}
\begin{commentprof}
La méthode rapide: 11100111 \rightarrow on inverse les bits à partir du premier 1: 00011001 \rightarrow 25
\end{commentprof}
\begin{itemize}
\item Le complément à 2 de $11100111_2$ vaut $00011000_2$. Ensuite $00011000_2+1_2=00011001_2=25_{10}$ donc $11100111_2=-25_{10}$. Ou $11100111_2=231_{10}\;et\;231-2^8=-25$
\item Le complément à 2 de $11000001_2$ vaut $00111110_2$. Ensuite $00111110_2+1_2=00111111_2=63_{10}$ donc $11000001_2=-63_{10}$. Ou $11000001_2=193_{10}\;et\;193-2^8=-63$
\end{itemize}
\end{exo}
\begin{exo}
Réaliser le QCM d’entraînement depuis le site \url{https://cviroulaud.github.io} 
\end{exo}
\begin{exo}
Il faut lire 1\;0 en binaire et non 10 en décimal et $10_2=2_{10}$.
\end{exo}
\end{Form}
\end{document}