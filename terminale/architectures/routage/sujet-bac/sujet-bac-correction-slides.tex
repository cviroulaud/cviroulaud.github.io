\documentclass[svgnames,11pt]{beamer}
\input{/home/tof/Documents/Cozy/latex-include/preambule_commun.tex}
\input{/home/tof/Documents/Cozy/latex-include/preambule_beamer.tex}
%\usepackage{pgfpages} \setbeameroption{show notes on second screen=left}
\author[]{Christophe Viroulaud}
\title{Sujet bac\\correction}
\date{\framebox{\textbf{Archi 17}}}
%\logo{}
\institute{Terminale - NSI}

\begin{document}
\begin{frame}
\titlepage
\end{frame}
\section{Exercice 3}
\begin{frame}
    \frametitle{Exercice 3}
Partie A:
\begin{enumerate}
    \item \textbf{\texttt{/sbin/init:}} dans le dossier \textbf{\texttt{sbin}} de l'arborescence
    \item processus actif: 5440 (et 5450), les autres sont endormis
    \item C'est l'application \textbf{\texttt{Bash}}. Il s'agit d'un terminal.
    \item Il faut remonter les pères successifs:
    \begin{itemize}
        \item père du programme 1: 1912
        \item père du programme 2: 2013, qui a 1912 pour père.
    \end{itemize}
    Le programme 1 a été exécuté en premier. On peut évoquer l'ordre de \texttt{\textbf{PID (Process IDentifier)}} également.
    \item Non, on n'a aucune information sur le contenu des programmes.
\end{enumerate}
    

\end{frame}
\begin{frame}
    \frametitle{}

    Partie B:
    \begin{center}
        \begin{tabular}{|*{3}{c|}}
            \hline
            Machine&Prochain saut&Distance\\
            \hline
            A&D&3\\
            \hline
            B&C&3\\
            \hline
            C&E&2\\
            \hline
            D&E&2\\
            \hline
            E&F&1\\
            \hline
        \end{tabular}
    \end{center}

\end{frame}
\begin{frame}
    \frametitle{}
Il faut calculer les coûts
    \begin{center}
        \begin{tabular}{|*{3}{c|}}
            \hline
            Machine&Prochain saut&Distance\\
            \hline
            A&B&4\\
            \hline
            B&C&3\\
            \hline
            C&E&2\\
            \hline
            D&E&11\\
            \hline
            E&F&1\\
            \hline
        \end{tabular}
    \end{center}
Le protocole OSPF est plus performant en terme de débit, même si on traverse davantage de routeurs (en partant de A).
\end{frame}
\section{Exercice 4}
\begin{frame}
    \frametitle{Exercice 4}
Partie A:
\begin{enumerate}
    \item Il y a 3 sous-réseaux. Le switch peut être vu comme une \emph{multiprise}. Pour les transferts intra-réseau, le paquet est dirigé directement par le switch, sans passer par le routeur. 
    \item \begin{enumerate}
        \item IPv4: 4 octets (32 bits)
        \item document réponse
        \begin{center}
            \begin{tabular}{ccccc}
                adresse IP & 192      & 168      & 20       & 10        \\
                adresse IP & 11000000 & 10101000 & 00010100 & 00001010 \\
                masque     & 11111111 & 11111111 & 11111111 & 00000000 \\
                masque &255&255&255&0\\
                réseau     & 11000000 & 10101000 & 00010100 & 00000000 \\
                réseau&192&168&20&0\\
                
            \end{tabular}
        \end{center}
    \end{enumerate}
    \item réponses correctes
    \begin{itemize}
        \item 192.168.20.0
        \item 192.168.20.30
        \item 192.168.20.230
    \end{itemize}
\end{enumerate}
    

\end{frame}
\begin{frame}[fragile]
    \frametitle{}

Partie B:
\begin{center}
\begin{lstlisting}[language=Python , basicstyle=\ttfamily\small, xleftmargin=2em, xrightmargin=2em]
def dec_bin(nb: int) -> list:
    binaire = [0 for _ in range(8)]
    for i in range(7, -1, -1):
        binaire[i] = nb % 2
        nb = nb//2
    return binaire
\end{lstlisting}
\end{center}
\end{frame}
\begin{frame}[fragile]
    \frametitle{}


\begin{center}
\begin{lstlisting}[language=Python , basicstyle=\ttfamily\small, xleftmargin=2em, xrightmargin=2em]
def IP_bin(ip: list) -> list:
    return [dec_bin(ip[i]) for i in range(4)]
\end{lstlisting}
\end{center}
\end{frame}
\end{document}