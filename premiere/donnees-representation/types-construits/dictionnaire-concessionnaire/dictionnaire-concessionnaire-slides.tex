\documentclass[svgnames,11pt]{beamer}
\input{/home/tof/Documents/Cozy/latex-include/preambule_commun.tex}
\input{/home/tof/Documents/Cozy/latex-include/preambule_beamer.tex}
%\usepackage{pgfpages} \setbeameroption{show notes on second screen=left}
\author[]{Christophe Viroulaud}
\title{Le concessionnaire\\les dictionnaires}
\date{\framebox{\textbf{DonRep 08}}}
%\logo{}
\institute{Première - NSI}

\begin{document}
\begin{frame}
    \titlepage
\end{frame}
\begin{frame}
    \frametitle{}

    Un concessionnaire désire concevoir une application qui permet de stocker les informations de ses véhicules (marque, modèle, kilométrage, date de première immatriculation\dots).

\end{frame}
\begin{frame}
    \frametitle{}

    \begin{framed}
        \centering Comment structurer des données hétérogènes dans un programme?
    \end{framed}

\end{frame}
\section{Première approche}
\begin{frame}[fragile]
    \frametitle{Première approche}

    \begin{center}
        \begin{lstlisting}[language=Python , basicstyle=\ttfamily\small, xleftmargin=1em, xrightmargin=0em]
marques = ["Renault", "Mercedes", "Peugeot"]
modeles = ["Twingo", "Classe A", "1007"]
kilometrages = [23410, 45000, 34001]
immatriculation = ["AA-123-AA", "AA-145-JU", "BB-156-TR"]
premiere_immat = ["2010-10", "2019-06", "2018-01"]
dates_vidanges = ["2020-01", "2019-06", "2020-10"]
\end{lstlisting}
        \captionof{code}{Utiliser des tableaux}
        \label{CODE}
    \end{center}
    \begin{activite}
        Déterminer les défauts des approches présentées.
    \end{activite}
\end{frame}
\section{Les dictionnaires}
\subsection{Présentation}
\begin{frame}[fragile]
    \frametitle{Les dictionnaires - présentation}

    \begin{aretenir}[]
    Un dictionnaire est une structure qui associe chaque \textbf{valeur} à une \textbf{clé}. Les données peuvent être hétérogènes.
    \end{aretenir}
\begin{center}
\begin{lstlisting}[language=Python , basicstyle=\ttfamily\small, xleftmargin=2em, xrightmargin=2em]
vehicule1 = {"marque": "Renault", 
            "modele": "Twingo", 
            "kilometrage": 23410, 
            "immatriculation": "AA-123-AA", 
            "premiere_imat": "2010-10", 
            "date_vidange": "2020-01"}
\end{lstlisting}
\captionof{code}{Construire un dictionnaire en Python}
\label{CODE}
\end{center}
\end{frame}
\begin{frame}[fragile]
    \frametitle{}

\begin{center}
\begin{lstlisting}[language=Python , basicstyle=\ttfamily\small, xleftmargin=2em, xrightmargin=2em]
vehicule1 = {"marque": "Renault", 
            "modele": "Twingo", 
            "kilometrage": 23410, 
            "immatriculation": "AA-123-AA", 
            "premiere_imat": "2010-10", 
            "date_vidange": "2020-01"}
\end{lstlisting}
\captionof{code}{Construire un dictionnaire en Python}
\label{CODE}
\end{center} 
\begin{activite}
Créer le dictionnaire \textbf{\texttt{vehicule1}}.
\end{activite}
\end{frame}
\begin{frame}
    \frametitle{}

    \begin{aretenir}[]
    Les clés sont des éléments \textbf{non mutables}: entier, chaîne de caractère, tuple\dots
    \end{aretenir}

\end{frame}
\subsection{Utiliser un dictionnaire}
\begin{frame}[fragile]
    \frametitle{Utiliser un dictionnaire - lecture}

\begin{aretenir}[]
On utilise la structure à crochets pour lire ou écrire dans un dictionnaire.
\end{aretenir}

\begin{center}
\begin{lstlisting}[language=Python , basicstyle=\ttfamily\small, xleftmargin=2em, xrightmargin=2em]
>>> vehicule1["marque"]
'Renault'
\end{lstlisting}
\captionof{code}{Lire la valeur associée à une clé}
\label{CODE}
\end{center}

\begin{center}
    \begin{lstlisting}[language=Python , basicstyle=\ttfamily\small, xleftmargin=2em, xrightmargin=2em]
>>> vehicule1["puissance"]
KeyError: 'puissance'
\end{lstlisting}
    \captionof{code}{\textbf{\texttt{Traceback: }}Erreur si la clé n'existe pas.}
    \label{CODE}
    \end{center}

\end{frame}
\begin{frame}[fragile]
    \frametitle{Écriture}
    \begin{center}
    \begin{lstlisting}[language=Python , basicstyle=\ttfamily\small, xleftmargin=2em, xrightmargin=2em]
>>> vehicule1["kilometrage"] = 25000
\end{lstlisting}
    \captionof{code}{Modifier la valeur associée à une clé}
    \label{CODE}
    \end{center}
\begin{center}
\begin{lstlisting}[language=Python , basicstyle=\ttfamily\small, xleftmargin=2em, xrightmargin=2em]
vehicule1["couleur"] = "rouge"
\end{lstlisting}
\captionof{code}{Si la clé n'existe pas, le couple est crée.}
\label{CODE}
\end{center}
\begin{activite}
Ajouter la clé \textbf{\texttt{couleur}} et afficher le dictionnaire dans la console.
\end{activite}
\end{frame}
\subsection{Propriétés}
\begin{frame}[fragile]
    \frametitle{Propriétés}
\begin{aretenir}[]
La fonction \textbf{\texttt{len}} renvoie la taille d'un dictionnaire.
\end{aretenir}
\begin{lstlisting}[language=Python , basicstyle=\ttfamily\small, xleftmargin=2em, xrightmargin=2em]
>>> len(vehicule1)
7
\end{lstlisting}

\end{frame}
\section{Itérer sur un dictionnaire}
\subsection{Itérer sur les clés}
\begin{frame}[fragile]
    \frametitle{Itérer sur les clés}

\begin{lstlisting}[language=Python , basicstyle=\ttfamily\small, xleftmargin=2em, xrightmargin=2em]
for cle in vehicule1.keys():
    print(cle)
\end{lstlisting}

\begin{lstlisting}[language=Python , basicstyle=\ttfamily\small, xleftmargin=2em, xrightmargin=2em]
marque
modele
kilometrage
immatriculation
premiere_imat
date_vidange
couleur
\end{lstlisting}
\end{frame}
\subsection{Itérer sur les valeurs}
\begin{frame}[fragile]
    \frametitle{Itérer sur les valeurs}

\begin{lstlisting}[language=Python , basicstyle=\ttfamily\small, xleftmargin=2em, xrightmargin=2em]
for val in vehicule1.values():
    print(val)
\end{lstlisting}

\begin{lstlisting}[language=Python , basicstyle=\ttfamily\small, xleftmargin=2em, xrightmargin=2em]
Renault
Twingo
23410
AA-123-AA
2010-10
2020-01
rouge
\end{lstlisting}
\end{frame}
\subsection{Itérer sur les couples}
\begin{frame}[fragile]
    \frametitle{Itérer sur les couples}

\begin{lstlisting}[language=Python , basicstyle=\ttfamily\small, xleftmargin=2em, xrightmargin=2em]
for cle, val in vehicule1.items():
    print(cle, " -> ", val)
\end{lstlisting}

\begin{lstlisting}[language=Python , basicstyle=\ttfamily\small, xleftmargin=2em, xrightmargin=2em]
marque  ->  Renault
modele  ->  Twingo
kilometrage  ->  23410
immatriculation  ->  AA-123-AA
premiere_imat  ->  2010-10
date_vidange  ->  2020-01
couleur  ->  rouge
\end{lstlisting}
\end{frame}
\end{document}