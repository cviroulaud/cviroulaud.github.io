\documentclass[a4paper,11pt]{article}
\input{/home/tof/Documents/Cozy/latex-include/preambule_lua.tex}
\newcommand{\showprof}{show them}  % comment this line if you don't want to see todo environment
\fancyhead[L]{Arbre binaire de recherche}
\newdate{madate}{10}{09}{2020}
%\fancyhead[R]{\displaydate{madate}} %\today
%\fancyhead[R]{Seconde - SNT}
%\fancyhead[R]{Première - NSI}
\fancyhead[R]{Terminale - NSI}
\fancyfoot[L]{~\\Christophe Viroulaud}
\AtEndDocument{\label{lastpage}}
\fancyfoot[C]{\textbf{Page \thepage/\pageref{lastpage}}}
\fancyfoot[R]{\includegraphics[width=2cm,align=t]{/home/tof/Documents/Cozy/latex-include/cc.png}}
\usepackage{tikz}

\begin{document}
\begin{Form}
\section{Problématique}
Les arbres binaires, les tas imposent des contraintes aux structures arborescentes. Il en résulte des objets avec des propriétés très utiles. Par exemple, la complexité du tri par tas est $O(n) = n.log(n)$.
\begin{center}
\shadowbox{\parbox{12cm}{\centering Comment obtenir une méthode de recherche efficace avec les arbres?}}
\end{center}
\section{Arbre binaire de recherche}
\subsection{Définition}
Imposons une contrainte à chaque nœud d'un arbre binaire:
\begin{itemize}
\item les valeurs du sous-arbre gauche sont plus petites que celle du nœud,
\item les valeurs du sous-arbre droit sont plus grandes que celle du nœud.
\end{itemize}
\begin{center}
\begin{tikzpicture}
\node[draw] (A) at (1,0) {33};
\node[draw] (B) at (-3,-1) {25};
\node[draw] (C) at (-5,-2) {20};
\node[draw] (D) at (-1,-2) {28};
\node[draw] (E) at (-6,-3) {8};
\node[draw] (F) at (-4,-3) {21};
\node[draw] (H) at (-2,-3) {26};
\node[draw] (I) at (4,-1) {56};
\node[draw] (J) at (2,-2) {40};
\node[draw] (K) at (6,-2) {60};
\node[draw] (M) at (1,-3) {35};
\node[draw] (O) at (5,-3) {58};
\node[draw] (P) at (7,-3) {65};

\draw (A) -- (B);
\draw (C) -- (B);
\draw (C) -- (E);
\draw (C) -- (F);
\draw (D) -- (B);
\draw (D) -- (H);
\draw (A) -- (I);
\draw (J) -- (I);
\draw (I) -- (K);
\draw (J) -- (M);
\draw (K) -- (O);
\draw (K) -- (P);
\draw [white] (0,-3) -- (0,-4.5); 
\end{tikzpicture}
\captionof{figure}{Un Arbre Binaire de Recherche (\emph{ABR})}
\begin{aretenir}[Remarque]
On suppose que chaque valeur n'apparaît qu'une seule fois dans l'arbre.
\end{aretenir}
\label{arbre}
\end{center}
\begin{activite}
\begin{enumerate}
\item Placer les valeurs 23, 27, 55, 59 dans l'ABR.
\item Où se trouve la plus grande valeur? La plus petite?
\item Effectuer un parcours infixe de l'arbre. Que remarque-t-on?
\end{enumerate}
\end{activite}
\subsection{Propriété}
\section{Implémentation}
\end{Form}
\end{document}