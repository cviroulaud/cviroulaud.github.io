\documentclass[a4paper,11pt]{article}
\input{/home/tof/Documents/Cozy/latex-include/preambule_doc.tex}
\input{/home/tof/Documents/Cozy/latex-include/preambule_commun.tex}
%\newcommand{\showprof}{show them}  % comment this line if you don't want to see todo environment
\setlength{\fboxrule}{0.8pt}
\fancyhead[L]{\fbox{\Large{\textbf{Secu 04}}}}
\fancyhead[C]{\textbf{Exercices sécurité - correction}}
\newdate{madate}{10}{09}{2020}
%\fancyhead[R]{\displaydate{madate}} %\today
%\fancyhead[R]{Seconde - SNT}
%\fancyhead[R]{Première - NSI}
\fancyhead[R]{Terminale - NSI}
\fancyfoot[L]{\vspace{1mm}Christophe Viroulaud}
\AtEndDocument{\label{lastpage}}
\fancyfoot[C]{\textbf{Page \thepage/\pageref{lastpage}}}
\fancyfoot[R]{\includegraphics[width=2cm,align=t]{/home/tof/Documents/Cozy/latex-include/cc.png}}

\begin{document}
\begin{exo}
\begin{enumerate}
    \item "noël".encode() $\rightarrow$ b'no\textbackslash xc3\textbackslash xabl'
    \item b'no\textbackslash xc3\textbackslash xabl'.decode() $\rightarrow$ "noël"
    \item Table de vérité
    \begin{center}
    \begin{lstlisting}[language=Python]
0 ^ 0 # affiche 0
0 ^ 1 # affiche 1
1 ^ 0 # affiche 1
1 ^ 1 # affiche 0
\end{lstlisting}
    \end{center}
    \item Chiffrer
    \lstinputlisting[firstline=10 ,lastline=12, xleftmargin=1em,xrightmargin=1em ]{"scripts/xor.py"}
    \item Message en clair
    \lstinputlisting[firstline=19 ,lastline=21 ]{"scripts/xor.py"}
\end{enumerate} 
\end{exo}
\begin{exo}
    \lstinputlisting[firstline=13 ,lastline=29 ]{"scripts/bruteforce.py"}
    \lstinputlisting[firstline=32 ,lastline=47 ]{"scripts/bruteforce.py"}
    \lstinputlisting[firstline=51 ,lastline=55 ]{"scripts/bruteforce.py"}
    On obtient un durée de 0,15 secondes pour décrypter le message. À noter qu'avec une clé de quatre caractères on monte à 3,5 secondes.
\end{exo}
\end{document}