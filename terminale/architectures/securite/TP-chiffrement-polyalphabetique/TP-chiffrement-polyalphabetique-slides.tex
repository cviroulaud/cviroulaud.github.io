\documentclass[svgnames,11pt]{beamer}
\input{/home/tof/Documents/Cozy/latex-include/preambule_commun.tex}
\input{/home/tof/Documents/Cozy/latex-include/preambule_beamer.tex}
%\usepackage{pgfpages} \setbeameroption{show notes on second screen=left}
\author[]{Christophe Viroulaud}
\title{TP\\chiffrement polyalphabétique}
\date{\framebox{\textbf{Algo 21}}}
%\logo{}
\institute{Terminale - NSI}

\begin{document}
\begin{frame}
    \titlepage
\end{frame}
\begin{frame}
    \frametitle{}

    Un chiffrement polyalphabétique consiste à:
    \begin{itemize}
        \item utiliser une clé de chiffrement composée de plusieurs lettres,
        \item recopier la clé de façon à obtenir une chaîne de la longueur du message.
    \end{itemize}
    \begin{center}
        \begin{tabular}{*{5}{c}}
            B & R & A & V & O \\
            N & S & I & N & S \\
        \end{tabular}
    \end{center}

\end{frame}
\begin{frame}
    \frametitle{}

    \begin{framed}
        \centering Construire un algorithme de chiffrement polyalphabétique.
    \end{framed}

\end{frame}
\begin{frame}
    \frametitle{}

    \begin{aretenir}[Remarque]
        Dans un souci de simplification, nous nous limiterons à des lettres majuscules. Chaque caractère sera donc:
        \begin{itemize}
            \item équivalent à une valeur ASCII comprise entre 65 (A) et 90 (Z),
            \item codé sur 7 bits.
        \end{itemize}
    \end{aretenir}

\end{frame}
\section{Conversions}
\begin{frame}
    \frametitle{Conversions}

    La première étape consiste à convertir une chaîne de caractère en sa correspondance binaire.
    \begin{center}
        \begin{tabular}{*{5}{c}}
            B       & R       & A       & V       & O       \\

            66      & 82      & 65      & 86      & 79      \\
            1000010 & 1010010 & 1000001 & 1010110 & 1001111 \\
            N       & S       & I       & N       & S       \\

            78      & 83      & 73      & 78      & 83      \\
            1001110 & 1010011 & 1001001 & 1001110 & 1010011 \\
        \end{tabular}
        \captionof{table}{Conversion en binaire}
    \end{center}
\end{frame}
\begin{frame}
    \frametitle{}

    \begin{activite}
        \begin{enumerate}
            \item Télécharger et extraire le dossier compressé \textbf{\texttt{TP-annexe-chiffrement.zip}} sur le site \url{https://cviroulaud.github.io}
            \item Écrire la fonction \textbf{\texttt{renverser(tab: list) $\rightarrow$ list}} qui renverse le contenu de \textbf{\texttt{tab}}
            \item Écrire la fonction \textbf{\texttt{int\_en\_bin(nb: int) $\rightarrow$ list}} qui renvoie le tableau des bits de l'entier \textbf{\texttt{nb}}
            \item Écrire la fonction \textbf{\texttt{bin\_en\_int(paquet: list) $\rightarrow$ int}} qui renvoie l'entier correspondant à la représentation binaire \textbf{\texttt{paquet}}.
            \item Exécuter le fichier \textbf{\texttt{test\_conversion.py}} pour tester les trois fonctions.
        \end{enumerate}
    \end{activite}

\end{frame}
\begin{frame}
    \frametitle{Avant de regarder la correction}
\begin{center}
    \centering
    \includegraphics[width=3cm]{/home/tof/Documents/Cozy/latex-include/stop.png}
    \end{center}
{\Large
    \begin{itemize}
        \item Prendre le temps de réfléchir,
        \item Analyser les messages d'erreur,
        \item Demander au professeur.
    \end{itemize}
}
\end{frame}
\begin{frame}[fragile]
    \frametitle{Correction}

\begin{center}
\begin{lstlisting}[language=Python , basicstyle=\ttfamily\small, xleftmargin=0.2em, xrightmargin=0em]
def renverser(tab: list) -> list:
    """
    renverse un tableau
    """
    l = len(tab)
    res = [0 for _ in range(l)]
    for i in range(l):
        res[l-1-i] = tab[i]
    return res
\end{lstlisting}
\end{center}

\end{frame}
\begin{frame}[fragile]
    \frametitle{}


\begin{center}
\begin{lstlisting}[language=Python , basicstyle=\ttfamily\small, xleftmargin=0.2em, xrightmargin=0em]
def int_en_bin(nb: int) -> list:
    """
    Convertit un entier en sa représentation binaire
    """
    q = nb
    r = []
    while q > 0:
        r.append(q % 2)
        q = q//2

    return renverser(r)
\end{lstlisting}
\end{center}

\end{frame}

\begin{frame}[fragile]
    \frametitle{}


\begin{center}
\begin{lstlisting}[language=Python , basicstyle=\ttfamily\small, xleftmargin=0.2em, xrightmargin=-0.5em]
def bin_en_int(paquet: list) -> int:
    """
    Convertit un paquet de bits en entier
    """
    entier = 0
    for i in range(len(paquet)):
        entier += int(paquet[i])*2**(len(paquet)-1-i)
    return entier
\end{lstlisting}
\end{center}

\end{frame}
\section{Convertir la clé}
\begin{frame}
    \frametitle{Convertir la clé}

    Afin de pouvoir chiffrer le message lettre par lettre, il faut allonger la clé de la taille du message:
    \begin{center}
        \begin{tabular}{*{5}{c}}
            B&R&A&V&O\\
            N&S&I&N&S\\
        \end{tabular}
    \end{center}
\end{frame}
\begin{frame}[fragile]
    \frametitle{}

    \begin{activite}
    Écrire la fonctionne \textbf{\texttt{creer\_cle\_bin(cle: str, taille: int) $\rightarrow$ list}} qui construit dans un tableau la version \underline{binaire} de la clé, à la \textbf{\texttt{taiile}} du message.
    \end{activite}
    \begin{center}
        \begin{tabular}{*{5}{c}}
            N&S&I&N&S\\
    
            78&83&73&78&83\\
            1001110&1010011&1001001&1001110&1010011\\
        \end{tabular}
\begin{lstlisting}[language=Python , basicstyle=\ttfamily\small, xleftmargin=0.2em, xrightmargin=0em]
>>> creer_cle_bin("NSI", 5)
[[1, 0, 0, 1, 1, 1, 0], 
 [1, 0, 1, 0, 0, 1, 1], 
 [1, 0, 0, 1, 0, 0, 1], 
 [1, 0, 0, 1, 1, 1, 0], 
 [1, 0, 1, 0, 0, 1, 1]]
\end{lstlisting}
\captionof{code}{Conversion en binaire}
\label{CODE}
\end{center}
\end{frame}
\begin{frame}
    \frametitle{Avant de regarder la correction}
\begin{center}
    \centering
    \includegraphics[width=3cm]{/home/tof/Documents/Cozy/latex-include/stop.png}
    \end{center}
{\Large
    \begin{itemize}
        \item Prendre le temps de réfléchir,
        \item Analyser les messages d'erreur,
        \item Demander au professeur.
    \end{itemize}
}
\end{frame}
\begin{frame}[fragile]
    \frametitle{Correction}

\begin{center}
\begin{lstlisting}[language=Python , basicstyle=\ttfamily\small, xleftmargin=0.2em, xrightmargin=0em]
def creer_cle_bin(cle: str, taille: int) -> list:
    """
    crée la version binaire de la clé, de la taille
    du message à chiffrer
    """
    cle_bin = []
    for i in range(taille):
        # si dépasse taille de la clé, revient à 0
        lettre = cle[i % len(cle)]
        cle_bin.append(int_en_bin(ord(lettre)))
    return cle_bin
\end{lstlisting}
\end{center}   

\end{frame}
\section{Appliquer le chiffrement XOR}
\begin{frame}
    \frametitle{Appliquer le chiffrement XOR}

La porte \textbf{\texttt{xor}} n'existe pas par défaut en Python. Il faut donc la créer.
\begin{activite}
\begin{enumerate}
    \item Écrire la fonction \textbf{\texttt{xor(x: int, y: int) $\rightarrow$ int}}
    \item Écrire alors la table de vérité de la porte logique \textbf{\texttt{XOR}}.
\end{enumerate}
\end{activite}

\end{frame}
\begin{frame}
    \frametitle{Avant de regarder la correction}
\begin{center}
    \centering
    \includegraphics[width=3cm]{/home/tof/Documents/Cozy/latex-include/stop.png}
    \end{center}
{\Large
    \begin{itemize}
        \item Prendre le temps de réfléchir,
        \item Analyser les messages d'erreur,
        \item Demander au professeur.
    \end{itemize}
}
\end{frame}
\begin{frame}[fragile]
    \frametitle{Correction}
\begin{center}
\begin{lstlisting}[language=Python , basicstyle=\ttfamily\small, xleftmargin=2em, xrightmargin=2em]
def xor(x: int, y: int) -> int:
    if x == 0:
        if y == 0:
            return 0
        else:
            return 1
    else:
        if y == 0:
            return 1
        else:
            return 0
\end{lstlisting}
\begin{lstlisting}[language=Python , basicstyle=\ttfamily\small, xleftmargin=2em, xrightmargin=2em]
print(f"0 xor 0 -> {xor(0, 0)}")
print(f"0 xor 1 -> {xor(0, 1)}")
print(f"1 xor 0 -> {xor(1, 0)}")
print(f"1 xor 1 -> {xor(1, 1)}")
\end{lstlisting}
\end{center}
    

\end{frame}
\begin{frame}
    \frametitle{}

    La fonction créée permet de chiffrer la version binaire du message.
    \begin{center}
        \begin{tabular}{*{6}{c}}
            Message&1000010&1010010&1000001&1010110&1001111\\
            $\oplus$&1001110&1010011&1001001&1001110&1010011\\
            \hline
            Chiffré&0001100&0000001&0001000&0011000&0011100\\
        \end{tabular}
        \captionof{table}{Application de la porte XOR}
    \end{center}

\end{frame}
\begin{frame}
    \frametitle{}

    \begin{activite}
    Écrire la fonction \textbf{\texttt{appliquer\_xor(entree\_bin: list, cle\_bin: list) $\rightarrow$ list}} qui renvoie un tableau de tableau représentant la version binaire chiffrée du message.
    \end{activite}

\end{frame}
\begin{frame}
    \frametitle{Avant de regarder la correction}
\begin{center}
    \centering
    \includegraphics[width=3cm]{/home/tof/Documents/Cozy/latex-include/stop.png}
    \end{center}
{\Large
    \begin{itemize}
        \item Prendre le temps de réfléchir,
        \item Analyser les messages d'erreur,
        \item Demander au professeur.
    \end{itemize}
}
\end{frame}
\begin{frame}[fragile]
    \frametitle{Correction}

\begin{center}
\begin{lstlisting}[language=Python , basicstyle=\ttfamily\small, xleftmargin=0.2em, xrightmargin=-4em]
def appliquer_xor(entree_bin: list, cle_bin: list) -> list:
    # préparation du tableau de tableau de sortie
    sortie_bin = [[0 for _ in range(len(entree_bin[0]))]
                  for _ in range(len(entree_bin))]
    
    # application de xor sur chaque bit
    for i in range(len(entree_bin)):
        for j in range(len(entree_bin[0])):
            sortie_bin[i][j] = xor(entree_bin[i][j], 
                                        cle_bin[i][j])
    
    return sortie_bin
\end{lstlisting}
\end{center}

\end{frame}
\section{Chiffrement, déchiffrement}
\begin{frame}
    \frametitle{Chiffrement, déchiffrement}

L'algorithme de chiffrement s'écrit alors:
\begin{itemize}
    \item Convertir le message en binaire.
    \item Convertir la clé en binaire.
    \item Chiffrer le message en binaire.
\end{itemize}
\begin{aretenir}[Remarque]
Il est inutile de convertir le message en caractères. C'est la version binaire qui transite jusqu'au destinataire.
\end{aretenir}
\end{frame}
\begin{frame}
    \frametitle{}

    \begin{activite}
    Écrire la fonction \textbf{\texttt{chiffrement(message: str, cle: str) $\rightarrow$ list}} qui applique l'algorithme précédent.
    \end{activite}

\end{frame}
\begin{frame}
    \frametitle{Avant de regarder la correction}
\begin{center}
    \centering
    \includegraphics[width=3cm]{/home/tof/Documents/Cozy/latex-include/stop.png}
    \end{center}
{\Large
    \begin{itemize}
        \item Prendre le temps de réfléchir,
        \item Analyser les messages d'erreur,
        \item Demander au professeur.
    \end{itemize}
}
\end{frame}
\begin{frame}[fragile]
    \frametitle{Correction}

\begin{center}
\begin{lstlisting}[language=Python , basicstyle=\ttfamily\small, xleftmargin=0.2em, xrightmargin=0em]
def chiffrement(message: str, cle: str) -> list:
    # convertit le message en binaire
    message_bin = []
    for lettre in message:
        message_bin.append(int_en_bin(ord(lettre)))

    # convertit la clé en binaire
    cle_bin = creer_cle_bin(cle, len(message))

    # chiffre le message
    secret_bin = appliquer_xor(message_bin, cle_bin)
    return secret_bin
\end{lstlisting}
\end{center}

\end{frame}
\begin{frame}
    \frametitle{}

    Pour le déchiffrement, le destinataire applique l'algorithme suivant:
    \begin{itemize}
        \item Convertir la clé en binaire.
        \item Déchiffrer le message en binaire.
        \item Convertir le message en caractères.
    \end{itemize}
\begin{activite}
    Écrire la fonction \textbf{\texttt{dechiffrement(secret\_bin: list, cle: str) $\rightarrow$ str}} qui applique l'algorithme précédent.
\end{activite}
\end{frame}
\begin{frame}
    \frametitle{Avant de regarder la correction}
\begin{center}
    \centering
    \includegraphics[width=3cm]{/home/tof/Documents/Cozy/latex-include/stop.png}
    \end{center}
{\Large
    \begin{itemize}
        \item Prendre le temps de réfléchir,
        \item Analyser les messages d'erreur,
        \item Demander au professeur.
    \end{itemize}
}
\end{frame}
\begin{frame}[fragile]
    \frametitle{Correction}

\begin{center}
\begin{lstlisting}[language=Python , basicstyle=\ttfamily\small, xleftmargin=0.2em, xrightmargin=-1em]
def dechiffrement(secret_bin: list, cle: str) -> str:
    # convertit la clé en binaire
    cle_bin = creer_cle_bin(cle, len(secret_bin))

    # déchiffre le secret binaire
    message_bin = appliquer_xor(secret_bin, cle_bin)

    # convertit le message binaire en caractères
    message = ""
    for lettre_bin in message_bin:
        message = message+chr(bin_en_int(lettre_bin))
    return message
\end{lstlisting}
\begin{lstlisting}[language=Python , basicstyle=\ttfamily\small, xleftmargin=0.2em, xrightmargin=-1em]
cle = "NSI"
s = chiffrement("BRAVO", cle)
print(dechiffrement(s, cle))
\end{lstlisting}
\end{center}

\end{frame}
\section{Opérateurs binaires}
\begin{frame}[fragile]
    \frametitle{Opérateurs binaires}

En Python, il est possible d'effectuer des opérations directement sur les bits des entiers.
\begin{center}
    \begin{tabular}{*{6}{c}}
        &$10_{10}=$&1&0&1&0\\
        \&&$8_{10}=$&1&0&0&0\\
        \hline
        &$8_{10}=$&1&0&0&0\\
    \end{tabular}
\end{center}
\begin{center}
\begin{lstlisting}[language=Python , basicstyle=\ttfamily\small, xleftmargin=2em, xrightmargin=2em]
>>> 10 & 8
8
\end{lstlisting}
\captionof{code}{\textbf{\texttt{AND}} binaire}
\end{center}
\end{frame}
\begin{frame}[fragile]
    \frametitle{}

\begin{center}
    \begin{tabular}{*{6}{c}}
        &$10_{10}=$&1&0&1&0\\
        |&$8_{10}=$&1&0&0&0\\
        \hline
        &$10_{10}=$&1&0&1&0\\
    \end{tabular}
\end{center}
\begin{center}
\begin{lstlisting}[language=Python , basicstyle=\ttfamily\small, xleftmargin=2em, xrightmargin=2em]
>>> 10 | 8
10
\end{lstlisting}
\captionof{code}{\textbf{\texttt{OR}} binaire}
\end{center}
\begin{center}
    \begin{tabular}{*{6}{c}}
        &$10_{10}=$&1&0&1&0\\
        \^&$8_{10}=$&1&0&0&0\\
        \hline
        &$2_{10}=$&0&0&1&0\\
    \end{tabular}
\end{center}
\begin{center}
\begin{lstlisting}[language=Python , basicstyle=\ttfamily\small, xleftmargin=2em, xrightmargin=2em]
>>> 10 ^ 8
2
\end{lstlisting}
\captionof{code}{\textbf{\texttt{XOR}} binaire}
\end{center}
\end{frame}
\begin{frame}
    \frametitle{}
    \begin{center}
        \begin{tabular}{*{5}{c}}
            B&R&A&V&O\\
            66&82&65&86&79\\
            N&S&I&N&S\\
            78&83&73&78&83\\
        \end{tabular}
        \captionof{table}{Conversion en ASCII}
    \end{center}
    \begin{activite}
    Écrire la fonction \textbf{\texttt{str\_en\_int(texte: str, taille: int) $\rightarrow$ list}} qui renvoie le tableau des valeurs ASCII d'un \textbf{\texttt{texte}}. La \textbf{\texttt{taille}} correspond à la longueur de la chaîne \textbf{\texttt{texte}}.
    \end{activite}

\end{frame}
\begin{frame}
    \frametitle{Avant de regarder la correction}
\begin{center}
    \centering
    \includegraphics[width=3cm]{/home/tof/Documents/Cozy/latex-include/stop.png}
    \end{center}
{\Large
    \begin{itemize}
        \item Prendre le temps de réfléchir,
        \item Analyser les messages d'erreur,
        \item Demander au professeur.
    \end{itemize}
}
\end{frame}
\begin{frame}[fragile]
    \frametitle{Correction}

\begin{center}
\begin{lstlisting}[language=Python , basicstyle=\ttfamily\small, xleftmargin=0.2em, xrightmargin=0em]
def str_en_int(texte: str, taille: int) -> list:
    """
    renvoie le tableau des valeurs ASCII d'un texte
    """
    texte_ascii = [0 for _ in range(taille)]
    for i in range(taille):
        texte_ascii[i] = ord(texte[i % len(texte)])
    return texte_ascii
\end{lstlisting}
\begin{lstlisting}[language=Python , basicstyle=\ttfamily\small, xleftmargin=0.2em, xrightmargin=0em]
>>> str_en_int("BRAVO",5)
[66, 82, 65, 86, 79]
>>> str_en_int("NSI", 5)
[78, 83, 73, 78, 83]
\end{lstlisting}
\end{center}

\end{frame}
\begin{frame}
    \frametitle{}

    L'algorithme de chiffrement s'écrit alors:
    \begin{itemize}
        \item Convertir le message en entiers ASCII.
        \item Convertir la clé en entiers ASCII, de la même taille que le message.
        \item Chiffre les entiers ASCII en appliquant un \textbf{\texttt{OU EXCLUSIF}} binaire.
    \end{itemize}
\begin{activite}
Écrire la fonction \textbf{\texttt{chiffrement(message: str, cle: str) $\rightarrow$ list}} qui applique l'algorithme précédent.
\end{activite}
\end{frame}
\begin{frame}
    \frametitle{Avant de regarder la correction}
\begin{center}
    \centering
    \includegraphics[width=3cm]{/home/tof/Documents/Cozy/latex-include/stop.png}
    \end{center}
{\Large
    \begin{itemize}
        \item Prendre le temps de réfléchir,
        \item Analyser les messages d'erreur,
        \item Demander au professeur.
    \end{itemize}
}
\end{frame}
\begin{frame}[fragile]
    \frametitle{Correction}
\begin{center}
\begin{lstlisting}[language=Python , basicstyle=\ttfamily\small, xleftmargin=0.2em, xrightmargin=-2.5em]
def chiffrement(message: str, cle: str) -> list:
    # convertit le message en valeurs sASCII
    message_ascii = str_en_int(message, len(message))

    # convertit la clé en valeurs ASCII
    cle_ascii = str_en_int(cle, len(message))

    # chiffre les valeurs ASCII
    secret_ascii = [0 for _ in range(len(message))]
    for i in range(len(message_ascii)):
        # opérateur binaire xor
        secret_ascii[i] = message_ascii[i] ^ cle_ascii[i]
    return secret_ascii
\end{lstlisting}
\end{center}
    

\end{frame}
\begin{frame}
    \frametitle{}

\begin{activite}
\begin{enumerate}
    \item Écrire l'algorithme de déchiffrement.
    \item Écrire la fonction \textbf{\texttt{dechiffrement(secret\_ascii: list, cle: str) $\rightarrow$ str}} qui applique l'algorithme précédent.
\end{enumerate}
\end{activite}
\end{frame}
\begin{frame}
    \frametitle{Avant de regarder la correction}
\begin{center}
    \centering
    \includegraphics[width=3cm]{/home/tof/Documents/Cozy/latex-include/stop.png}
    \end{center}
{\Large
    \begin{itemize}
        \item Prendre le temps de réfléchir,
        \item Analyser les messages d'erreur,
        \item Demander au professeur.
    \end{itemize}
}
\end{frame}
\begin{frame}
    \frametitle{}
\begin{itemize}
    \item Convertir la clé en entier ASCII, de la taille du message.
    \item Déchiffrer les entiers ASCII du message secret.
    \item Convertir les entiers ASCII en lettres
\end{itemize}
    

\end{frame}
\begin{frame}[fragile]
\begin{center}
\begin{lstlisting}[language=Python , basicstyle=\ttfamily\small, xleftmargin=0.2em, xrightmargin=-2.5em]
def dechiffrement(secret_ascii: list, cle: str) -> str:
    # convertit la clé en valeurs ASCII
    cle_ascii = str_en_int(cle, len(secret_ascii))

    # déchiffre les valeurs ASCII
    message_ascii = [0 for _ in range(len(secret_ascii))]
    for i in range(len(secret_ascii)):
        # opérateur binaire xor
        message_ascii[i] = secret_ascii[i] ^ cle_ascii[i]

    # convertit les valeurs ASCII en lettre
    message = ""
    for lettre_ascii in message_ascii:
        message = message+chr(lettre_ascii)
    return message
\end{lstlisting}
\begin{lstlisting}[language=Python , basicstyle=\ttfamily\small, xleftmargin=0.2em, xrightmargin=-2.5em]
cle = "NSI"
s = chiffrement("BRAVO", cle)
print(dechiffrement(s, cle))
\end{lstlisting}
\end{center}
    

\end{frame}
\begin{frame}

\begin{aretenir}[Remarque]
    Le chiffrement par \textbf{\texttt{OU EXCLUSIF}} est faible si la clé n'est pas suffisamment longue et si, de plus, nous disposons de quelques informations sur le contenu du message.
\end{aretenir}

\end{frame}
\end{document}