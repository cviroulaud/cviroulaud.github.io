\documentclass[a4paper,11pt]{article}
\input{/home/tof/Documents/Cozy/latex-include/preambule_doc.tex}
\input{/home/tof/Documents/Cozy/latex-include/preambule_commun.tex}
\newcommand{\showprof}{show them}  % comment this line if you don't want to see todo environment
\setlength{\fboxrule}{0.8pt}
\fancyhead[L]{\fbox{\Large{\textbf{ProgDyn 04}}}}
\fancyhead[C]{\textbf{Découpe d'une barre métallique}}
\newdate{madate}{10}{09}{2020}
%\fancyhead[R]{\displaydate{madate}} %\today
%\fancyhead[R]{Seconde - SNT}
%\fancyhead[R]{Première - NSI}
\fancyhead[R]{Terminale - NSI}
\fancyfoot[L]{\vspace{1mm}Christophe Viroulaud}
\AtEndDocument{\label{lastpage}}
\fancyfoot[C]{\textbf{Page \thepage/\pageref{lastpage}}}
\fancyfoot[R]{\includegraphics[width=2cm,align=t]{/home/tof/Documents/Cozy/latex-include/cc.png}}

\begin{document}
\section{Problématique}
Un ferrailleur découpe des barres métalliques pour la revente. Les barres, plus petites, obtenues sont utilisées dans diverses constructions (garde-corps, châssis, fenêtre\dots). Le prix de revente des barres n'est donc pas proportionnel à leur dimension.
\begin{center}
    \begin{tabular}{|*{9}{c|}}
        \hline
        Longueur & 1 & 2 & 3 & 4 & 5 & 6 & 8 & 10 \\
        \hline
        Prix & 2 & 5 & 8 & 10 & 11 & 14 & 17 & 20 \\
        \hline
    \end{tabular}
\end{center}
\begin{center}
    \framebox{Comment maximiser le prix de vente d'une barre?}
    
\end{center}
\section{Approche naïve}
Une formalisation mathématique du débitage d'une barre de longueur \emph{longueur} permet d'écrire pour une première découpe de taille \emph{taille}:
$$prix\_max(longueur) = prix(taille) + prix\_max(longueur-taille)$$
Si on effectue une première découpe \emph{taille}, il reste une barre \emph{longueur-taille}. L'objectif est de trouver le prix pour chaque découpe possible et ne garder que le prix maximum.
\begin{activite}
\begin{enumerate}
    \item Déterminer \emph{à la main} le prix maximal pour une barre de longueur 10.
    \item Construire \underline{le début} de l'arbre des possibilités de découpe permettant de calculer les prix de vente.
    \item Construire un dictionnaire associant la longueur de barre à son prix.
    \item Déterminer une fonction \emph{récursive} permettant de calculer le prix maximal obtenu pour la découpe d'une barre.   
\end{enumerate}
\end{activite}
\section{Approche dynamique}
En calculant une fois pour toute, le prix maximum pour chaque longueur de barre on économise des calculs.
\begin{activite}
\begin{enumerate}
    \item Modifier la fonction précédente pour qu'elle évite de calculer plusieurs fois le même prix (approche top-down).
    \item Reprendre le problème avec une approche bottom-up.
\end{enumerate}
\end{activite}
\end{document}