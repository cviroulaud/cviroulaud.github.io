\documentclass[a4paper,11pt]{article}
\input{/home/tof/Documents/Cozy/latex-include/preambule_doc.tex}
\input{/home/tof/Documents/Cozy/latex-include/preambule_commun.tex}
\newcommand{\showprof}{show them}  % comment this line if you don't want to see todo environment
\setlength{\fboxrule}{0.8pt}
\fancyhead[L]{\fbox{\Large{\textbf{Tab 03}}}}
\fancyhead[C]{\textbf{Exercices - données en table}}
\newdate{madate}{10}{09}{2020}
%\fancyhead[R]{\displaydate{madate}} %\today
\fancyhead[R]{Première - NSI}
\fancyfoot[L]{\vspace{1mm}Christophe Viroulaud}
\AtEndDocument{\label{lastpage}}
\fancyfoot[C]{\textbf{Page \thepage/\pageref{lastpage}}}
\fancyfoot[R]{\includegraphics[width=2cm,align=t]{/home/tof/Documents/Cozy/latex-include/cc.png}}

\begin{document}
\begin{exo}
    Repérer les erreurs de construction du fichier \textbf{\texttt{csv}} suivant:\\
    titre,auteur,Présentation,Parution\\
    Le cas Nelson Kerr, John Grisham,Le jour où Bruce Cable,2022\\
    Lucia,Bernard Minier,À l'université de Salamanque, un groupe d'étudiants,2022\\
    Impact,Olivier Norek,Face au mal qui se propage,2021,\\
    "Labyrinthes,Franck Thilliez,Suivez le fil infernal, une scène de pure folie dans un chalet,2022"
\end{exo}
\begin{exo}
    Un institut de sondage a étudié la corrélation entre l'âge, la profession et le salaire des individus. Les données sont rassemblées dans le fichier \textbf{\texttt{sondage.csv}}
    \begin{enumerate}
        \item Télécharger le fichier compressé \textbf{\texttt{exercices-donnees.zip}} sur le site \url{https://cviroulaud.github.io}
        \item À l'aide du bloc-notes, ouvrir le fichier et observer les données.
        \item Dans un programme Python, importer les données sous la forme d'un tableau de dictionnaire. Il faudra veiller à formater correctement les données en typant les données chiffrées.
        \item Écrire la fonction \textbf{\texttt{salaire\_max(donnees: list) $\rightarrow$ int}} qui renvoie le salaire le plus élevé.
        \item Écrire la fonction \textbf{\texttt{age\_moyen(donnees:list) $\rightarrow$ float}} qui calcule la moyenne d'âge des individus.
        \item Écrire la fonction \textbf{\texttt{dernier(donnees: list) $\rightarrow$ str}} qui renvoie le dernier nom dans l'ordre alphabétique.
    \end{enumerate}
\end{exo}
\begin{exo}
    Le Géocube (\url{https://geobs.fr}) est un outil de mesures météorologiques conçu par l'IGN. Grâce à partenariat avec l'organisme, le lycée Jay de Beaufort a la chance de posséder un de ces appareils (sur le toit de l'établissement). Voici un exemple des données récoltées:
    \begin{itemize}
        \item Jour, Heure, Temperature
        \item 2021-12-03, 23:09:43, 9.1
        \item 2021-12-03, 23:29:43, 9.4
        \item 2021-12-03, 23:49:43, 9.8
        \item 2021-12-04, 00:09:43, 10.2
    \end{itemize}
    Pour écrire dans un fichier \textbf{\texttt{csv}}, il faut:
    \begin{itemize}
        \item Ouvrir un fichier en écriture; s'il n'existe pas le fichier et crée.
        \item Écrire les entêtes.
        \item Écrire les lignes.
        \item Fermer le fichier
    \end{itemize}
    \begin{enumerate}
        \item En s'aidant de la documentation \url{https://tinyurl.com/pytdic}, écrire un programme qui crée le fichier \textbf{\texttt{csv}} des données précédentes.
        \item Cette implémentation pose l'inconvénient d'écraser les données déjà présentes dans le fichier \textbf{\texttt{csv}} à chaque exécution du fichier. Que faut-il modifier dans la ligne de l'ouverture du fichier pour pouvoir ajouter de nouvelles lignes dans un fichier \texttt{csv} existant?
    \end{enumerate}
\end{exo}
\end{document}