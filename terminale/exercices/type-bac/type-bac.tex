\documentclass[a4paper,11pt]{article}
\input{/home/tof/Documents/Cozy/latex-include/preambule_doc.tex}
\input{/home/tof/Documents/Cozy/latex-include/preambule_commun.tex}
\newcommand{\showprof}{show them}  % comment this line if you don't want to see todo environment
\setlength{\fboxrule}{0.8pt}
\fancyhead[L]{\fbox{\Large{\textbf{TypeBAC}}}}
\fancyhead[C]{\textbf{Sujet bac 07}}
\newdate{madate}{10}{09}{2020}
%\fancyhead[R]{\displaydate{madate}} %\today
%\fancyhead[R]{Seconde - SNT}
%\fancyhead[R]{Première - NSI}
\fancyhead[R]{Terminale - NSI}
\fancyfoot[L]{\vspace{1mm}Christophe Viroulaud}
\AtEndDocument{\label{lastpage}}
\fancyfoot[C]{\textbf{Page \thepage/\pageref{lastpage}}}
\fancyfoot[R]{\includegraphics[width=2cm,align=t]{/home/tof/Documents/Cozy/latex-include/cc.png}}

\begin{document}
\begin{exo}
    On s’intéresse à la suite d’entiers définie par
    $U_ 1 = 1$, $U_2 = 1$ et, pour tout entier naturel n, par $U_{n+2} = U_{n+1} + U_n$ .

    Elle s’appelle la suite de Fibonnaci.
    Écrire la fonction fibonacci qui prend un entier n > 0 et qui renvoie l’élément d’indice n de cette suite.

    On utilisera une programmation dynamique (pas de récursivité).

    \begin{center}
    \begin{lstlisting}[language=Python]
>>> fibonacci(1)
1
>>> fibonacci(2)
1
>>> fibonacci(25)
75025
>>> fibonacci(45)
1134903170   
    \end{lstlisting}
    \captionof{code}{Exemples}
    \label{CODE}
    \end{center}
    
\end{exo}
\begin{exo}
    Les variables \emph{liste\_eleves} et \emph{liste\_notes} ayant été préalablement définies et étant
    de même longueur, la fonction \emph{meilleures\_notes} renvoie la note maximale qui a été
    attribuée, le nombre d’élèves ayant obtenu cette note et la liste des noms de ces élèves.
    Compléter le code Python de la fonction \emph{meilleures\_notes} ci-dessous.
\lstinputlisting[firstline=1 ,lastline=19 ]{"scripts/exo-07.py"}
\begin{center}
\begin{lstlisting}[language=Python]
>>> meilleures_notes()
(80, 3, ['c', 'f', 'h'])
\end{lstlisting}
\captionof{code}{Une fois complété, le code donne}
\label{CODE}
\end{center}
\end{exo}
\end{document}