\documentclass[svgnames,11pt]{beamer}
\input{/home/tof/Documents/Cozy/latex-include/preambule_commun.tex}
\input{/home/tof/Documents/Cozy/latex-include/preambule_beamer.tex}
%\usepackage{pgfpages} \setbeameroption{show notes on second screen=left}
\author[]{Christophe Viroulaud}
\title{Exercice type bac BDD\\correction}
\date{\framebox{\textbf{BDD 08}}}
%\logo{}
\institute{Terminale - NSI}

\begin{document}
\begin{frame}
    \titlepage
\end{frame}
\begin{frame}[fragile]
    \frametitle{1.a}

    \begin{center}
        \begin{lstlisting}[language=SQL , basicstyle=\ttfamily\small, xleftmargin=2em, xrightmargin=2em]
SELECT salle, marque_ordi FROM Ordinateur;
\end{lstlisting}
    \end{center}
    \begin{center}
        \begin{tabular}{|c|c|}
            \hline
            salle & marque\_ordi \\
            \hline
            012   & HP           \\
            114   & Lenovo       \\
            223   & Dell         \\
            223   & Dell         \\
            223   & Dell         \\
            \hline
        \end{tabular}
    \end{center}

\end{frame}
\begin{frame}[fragile]
    \frametitle{1.b}

    \begin{center}
        \begin{lstlisting}[language=SQL , basicstyle=\ttfamily\small, xleftmargin=2em, xrightmargin=2em]
SELECT nom_ordi, salle FROM Ordinateur WHERE video = true;
\end{lstlisting}
    \end{center}
    \begin{center}
        \begin{tabular}{|c|c|}
            \hline
            salle & marque\_ordi \\
            \hline
            Gen-24&012          \\
            Tech-62&114       \\
            Gen-132&223        \\
            \hline
        \end{tabular}
    \end{center}

\end{frame}
\begin{frame}[fragile]
    \frametitle{2}

    \begin{center}
        \begin{lstlisting}[language=SQL , basicstyle=\ttfamily\small, xleftmargin=2em, xrightmargin=0em]
SELECT * FROM Ordinateur WHERE annee >= 2017 ORDER BY annee;
\end{lstlisting}
    \end{center}
\begin{aretenir}[Remarque]
Les 3 lignes peuvent sortir dans un ordre aléatoire ici car l'année est 2019 pour chacune. On peut ajouter un deuxième attribut de tri.
\end{aretenir}
\end{frame}
\begin{frame}
    \frametitle{3.a}

    L'attribut \textbf{\texttt{salle}} ne respecte pas \underline{la contrainte d'entité}: plusieurs entités (n-uplets) possèdent le même numéro de salle.

\end{frame}
\begin{frame}
    \frametitle{3.b}

    Imprimante(\underline{nom\_imprimante: \textbf{\texttt{String}}, \dashuline{nom\_ordi: \textbf{\texttt{String}}}}, marque\_imp: \textbf{\texttt{String}}, modele\_imp: \textbf{\texttt{String}}, salle: \textbf{\texttt{Int}})
\begin{aretenir}[]
Une clé étrangère référence une clé \underline{primaire} d'une autre table.
\end{aretenir}
\end{frame}
\begin{frame}[fragile]
    \frametitle{4.a}

    \begin{center}
        \begin{lstlisting}[language=SQL , basicstyle=\ttfamily\small, xleftmargin=2em, xrightmargin=0em]
INSERT INTO Videoprojecteur VALUES
(315, "NEC", "ME402X", false);
\end{lstlisting}
    \end{center}
\begin{aretenir}[Remarque]
Le type \textbf{\texttt{boolean}} ne fait pas partie de la spécification du langage SQL. Tous les SGBD ne l'implémentent pas nécessairement.
\end{aretenir}
\end{frame}
\begin{frame}[fragile]
    \frametitle{4.b}

    \begin{center}
        \begin{lstlisting}[language=SQL , basicstyle=\ttfamily\small, xleftmargin=.5em, xrightmargin=0em]
SELECT nom_ordi, marque_video, Videoprojecteur.salle 
FROM Ordinateur
JOIN Videoprojecteur ON Ordinateur.salle = Videoprojecteur.salle
WHERE tni = true;
\end{lstlisting}
\captionof{code}{Avec jointure}
    \end{center}
\begin{aretenir}[Remarques]
\begin{itemize}
    \item Il faut préciser la table pour les attributs dont le nom se retrouve dans les 2 relations.
    \item On récupère des informations des 2 tables: il faut faire une jointure.
\end{itemize}
\end{aretenir}
\end{frame}
\end{document}