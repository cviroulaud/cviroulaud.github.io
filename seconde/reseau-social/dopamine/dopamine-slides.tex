\documentclass[svgnames,11pt]{beamer}
\input{/home/tof/Documents/Cozy/latex-include/preambule_commun.tex}
\input{/home/tof/Documents/Cozy/latex-include/preambule_beamer.tex}
%\usepackage{pgfpages} \setbeameroption{show notes on second screen=left}
\author[]{Christophe Viroulaud}
\title{Dopamine}
\date{\framebox{\textbf{ResSoc 02}}}
%\logo{}
\institute{Seconde - SNT}

\begin{document}
\begin{frame}
\titlepage
\end{frame}
\begin{frame}
    \frametitle{}

    Quand on est présent sur un réseau social, il est parfois difficile de s’obliger à s’en détacher. Les
créateurs de ces réseaux ont pour objectif de garder le plus longtemps possible leurs clients sur la
plate-forme.
\begin{framed}
    \centering Quels mécanismes nous incitent à rester connecté en permanence ?
\end{framed}

\end{frame}
\begin{frame}
    \frametitle{}

    \begin{activite}
    \begin{enumerate}
        \item Se rendre sur le site \url{https://www.arte.tv} et effectuer la recherche \textbf{\texttt{dopamine}}.
        \item Regarder -au moins- les vidéos:
        \begin{itemize}
            \item Facebook
            \item Snapchat
            \item Youtube
            \item Instagram
        \end{itemize}
        \item Pour deux des réseaux, établir une liste de:
        \begin{itemize}
            \item 3 arguments négatifs,
            \item 3 arguments positifs.
        \end{itemize}
    \end{enumerate}
    \end{activite}

\end{frame}
\begin{frame}
    \frametitle{}

    \begin{activite}
    Avec le groupe sélectionné et pour le thème désigné, préparer l'argumentation pour le débat.\\
    Le débat est organisé comme tel:
    \begin{itemize}
        \item 2 groupes face à face,
        \item 1 groupe \emph{jury} à convaincre,
        \item le public qui évalue,
        \item 10 minutes de débat maximum.
    \end{itemize}
    Les critères d'évaluation prendront en compte (voir fiche détaillée):
    \begin{itemize}
        \item la qualité de l'argumentation,
        \item la capacité à réagir à l'argumentation du groupe adverse,
        \item la répartition du temps de parole dans le groupe,
        \item le respect du temps de parole de l'adversaire.
    \end{itemize}
    \end{activite}

\end{frame}
\end{document}