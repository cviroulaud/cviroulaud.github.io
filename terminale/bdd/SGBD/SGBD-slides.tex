\documentclass[svgnames,11pt]{beamer}
\input{/home/tof/Documents/Cozy/latex-include/preambule_commun.tex}
\input{/home/tof/Documents/Cozy/latex-include/preambule_beamer.tex}
%\usepackage{pgfpages} \setbeameroption{show notes on second screen=left}
\author[]{Christophe Viroulaud}
\title{Système de Gestion de Base de Données}
\date{\framebox{\textbf{BDD 03}}}
%\logo{}
\institute{Terminale - NSI}

\begin{document}
\begin{frame}
\titlepage
\end{frame}
\begin{frame}
    \frametitle{}

    Le modèle relationnel présenté dans le cours précédent est un modèle mathématique qu'il faut maintenant concrétiser sur machine.
\begin{framed}\centering 
    Quels sont les outils permettant de construire une base de données?
\end{framed}

\end{frame}
\section{Organisation}
\subsection{Un logiciel}
\begin{frame}
    \frametitle{Organisation - un logiciel}

    Un \textbf{Système de Gestion de Base de Données (SGBD)} est un logiciel permettant de manipuler les données d'une base de données.    

\end{frame}
\begin{frame}
    \frametitle{}

    Ils sont la plupart du temps basés sur un modèle client-serveur:
\begin{itemize}
\item la base de données se trouve sur un \emph{serveur},
\item un \emph{logiciel client} va interroger le serveur et transmettre la réponse que ce-dernier lui aura donné.
\end{itemize}
\begin{aretenir}[Remarque]
    Un SGBD qui implémente le modèle relationnel est noté SGBDR. 
\end{aretenir}
\end{frame}
\begin{frame}
    \frametitle{}

    \begin{center}
        \begin{tabular}{ccccc}
            \includegraphics[width=0.15\textwidth]{ressources/mysql.png} &
            \includegraphics[width=0.15\textwidth]{ressources/oracle.png} &
            \includegraphics[width=0.15\textwidth]{ressources/postgresql.png}
            &
            \includegraphics[width=0.15\textwidth]{ressources/maria.png}
            &
            \includegraphics[width=0.15\textwidth]{ressources/sqlite.png} \\
            MySQL &
            Oracle &
            PostgreSQL &
            MariaDB &
            SQLite
            \end{tabular}
            \captionof{figure}{Principaux systèmes }
    \end{center}

\end{frame}
\subsection{Retour historique}
\begin{frame}
    \frametitle{Retour historique}

    

\end{frame}
\end{document}