\documentclass[a4paper,11pt]{article}
\input{/home/tof/Documents/Cozy/latex-include/preambule_lua.tex}
\newcommand{\showprof}{show them}  % comment this line if you don't want to see todo environment
\fancyhead[L]{Diviser pour régner - exercices}
\newdate{madate}{10}{09}{2020}
\fancyhead[R]{Terminale - NSI} %\today
\fancyfoot[L]{~\\Christophe Viroulaud}
\fancyfoot[C]{\textbf{Page \thepage}}
\fancyfoot[R]{\includegraphics[width=2cm,align=t]{/home/tof/Documents/Cozy/latex-include/cc.png}}

\begin{document}
\begin{Form}
\begin{exo}Dichotomie
\begin{enumerate}
\item Rappeler le programme impératif de la recherche dichotomique.
\item Écrire une version récursive de cet algorithme. Il sera peut-être nécessaire d'ajouter des arguments.
\item Créer une liste \emph{l} de cinquante éléments triés.
\item Tester les deux fonctions sur la liste \emph{l}.
\item Rappeler la complexité de l'algorithme de recherche dichotomique.
\end{enumerate}
\end{exo}
\begin{exo}
Le tri rapide est un autre exemple d'algorithme utilisant la méthode \emph{diviser pour régner}. C'est un algorithme naturellement récursif que nous pouvons décrire ainsi:
\begin{itemize}
\item Choisir un élément pivot.
\item Sélectionner tous les éléments inférieurs au pivot.
\item Sélectionner tous les éléments supérieurs ou égaux au pivot.
\item Placer récursivement à gauche du pivot les éléments inférieurs à ce-dernier et à droite les éléments supérieurs.
\end{itemize}
\begin{commentprof}
\begin{itemize}
\item ÷: partionner en 3 (gauche, pivot, droite)
\item régner: les 2 sous-tableaux sont traités récursivement
\item combiner: pas de recombinaison: les tableaux ont été triés en place
\end{itemize}
\end{commentprof}
\begin{enumerate}
\item Écrire une fonction \textbf{tri\_rapide(tab: list)\;\rightarrow\;list} qui implémente l'algorithme du tri rapide. Nous choisirons le premier élément du tableau comme pivot.
\item Construire en compréhension une liste \emph{l} de vingt éléments compris entre 0 et 100.
\item Tester la fonction de tri sur la liste \emph{l}.
\end{enumerate}
\textbf{Remarque:} Le tri rapide a une complexité en $O(n×log_2(n))$.
\end{exo}
\begin{exo}
La fonction \emph{mystere} implémente un algorithme du type \emph{diviser pour régner}.
\lstinputlisting[firstline=18,lastline=31]{"scripts/maxi.py"}
Soit la liste:
\begin{lstlisting}
tab = [5, 71, 23, 45, 28, 89, 63, 39]
\end{lstlisting}
\begin{enumerate}
\item Dessiner l'arbre des séparations engendré par la fonction sur la liste \emph{tab}.
\item Dessiner l'arbre des recombinaisons. Quelle valeur renvoie l'appel \textbf{mystere(tab)}?
\item Que fait cette fonction?
\end{enumerate}
\end{exo}
\end{Form}
\end{document}