\documentclass[svgnames,11pt]{beamer}
\input{/home/tof/Documents/Cozy/latex-include/preambule_commun.tex}
\input{/home/tof/Documents/Cozy/latex-include/preambule_beamer.tex}
%\usepackage{pgfpages} \setbeameroption{show notes on second screen=left}
\author[]{Christophe Viroulaud}
\title{Bibliothèque \textbf{\texttt{deque}}\\exercice commenté}
\date{\framebox{\textbf{Archi 07}}}
%\logo{}
\institute{Terminale - NSI}

\begin{document}
\begin{frame}
\titlepage
\end{frame}
\begin{frame}[fragile]
    \frametitle{}
En utilisant la pile construite à partir d'un tableau, mesurer la durée d'exécution pour:
\begin{itemize}
    \item empiler 100000 éléments,
    \item dépiler 100000 éléments.
\end{itemize}
    
\begin{lstlisting}[language=Python , basicstyle=\ttfamily\small, xleftmargin=2em, xrightmargin=2em]
class Pile:
    def __init__(self):
        self.donnees = []

    def est_vide(self) -> bool:
        return self.donnees == []

    def empiler(self, e: int) -> None:
        self.donnees.append(e)

    def depiler(self) -> int:
        if not self.est_vide():
            return self.donnees.pop()
\end{lstlisting}
\end{frame}
\begin{frame}[fragile]
    \frametitle{}

\begin{lstlisting}[language=Python , basicstyle=\ttfamily\small, xleftmargin=2em, xrightmargin=2em]
p = Pile()
deb = time()
for i in range(NB):
    p.empiler(i)
fin = time()
print("empiler ", fin-deb)

deb = time()
for i in range(NB):
    p.depiler()
fin = time()
print("dépiler ", fin-deb)
\end{lstlisting}
\begin{lstlisting}[language=Python , basicstyle=\ttfamily\small, xleftmargin=2em, xrightmargin=2em]
empiler  0.04126310348510742
dépiler  0.06682586669921875    
\end{lstlisting}
\end{frame}
\begin{frame}[fragile]
    \frametitle{}

Effectuer les mêmes mesures pour la file:
\begin{lstlisting}[language=Python , basicstyle=\ttfamily\small, xleftmargin=2em, xrightmargin=2em]
class File:
    def __init__(self):
        self.donnees = []

    def est_vide(self) -> bool:
        return self.donnees == []

    def enfiler(self, e: int) -> None:
        self.donnees.append(e)

    def defiler(self) -> int:
        if not self.donnees == []:
            return self.donnees.pop(0)
\end{lstlisting}
\end{frame}
\begin{frame}[fragile]
    \frametitle{}

\begin{lstlisting}[language=Python , basicstyle=\ttfamily\small, xleftmargin=2em, xrightmargin=2em]
f = File()
deb = time()
for i in range(NB):
    f.enfiler(i)
fin = time()
print("enfiler ", fin-deb)

deb = time()
for i in range(NB):
    f.defiler()
fin = time()
print("défiler ", fin-deb)    
\end{lstlisting}
\begin{lstlisting}[language=Python , basicstyle=\ttfamily\small, xleftmargin=2em, xrightmargin=2em]
enfiler  0.041974782943725586
défiler  2.0046255588531494   
\end{lstlisting}
\end{frame}
\begin{frame}
    \frametitle{}

    \begin{aretenir}[]
Python propose des outils optimisés dans la bibliothèque \textbf{\texttt{collections}}: 
\begin{center}
    \url{https://docs.python.org/fr/3/library/collections.html}
\end{center}
    \end{aretenir}

\end{frame}
\begin{frame}
    \frametitle{}

    Écrire une nouvelle classe \textbf{\texttt{File}} en utilisant une \textbf{\texttt{deque}}.

\end{frame}
\begin{frame}[fragile]
    \frametitle{}

    \begin{lstlisting}[language=Python , basicstyle=\ttfamily\small, xleftmargin=2em, xrightmargin=2em]
class File2:
    def __init__(self):
        self.donnees = deque()

    def est_vide(self) -> bool:
        return len(self.donnees) == 0

    def enfiler(self, e: int) -> None:
        self.donnees.append(e)

    def defiler(self) -> int:
        if not self.est_vide():
            return self.donnees.popleft()    
\end{lstlisting}
\begin{aretenir}[Observation]
L'\textbf{interface} reste identique pour l'utilisateur de la classe.
\end{aretenir}
\end{frame}
\begin{frame}[fragile]
    \frametitle{}

\begin{lstlisting}[language=Python , basicstyle=\ttfamily\small, xleftmargin=2em, xrightmargin=2em]
f = File2()
deb = time()
for i in range(NB):
    f.enfiler(i)
fin = time()
print("enfiler ", fin-deb)

deb = time()
for i in range(NB):
    f.defiler()
fin = time()
print("défiler ", fin-deb)
\end{lstlisting}
\begin{lstlisting}[language=Python , basicstyle=\ttfamily\small, xleftmargin=2em, xrightmargin=2em]
enfiler  0.02210521697998047
défiler  0.038460731506347656
\end{lstlisting}
\end{frame}
\end{document}