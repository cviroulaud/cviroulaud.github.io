\documentclass[a4paper,11pt]{article}
\input{/home/tof/Documents/Cozy/latex-include/preambule_lua.tex}
\newcommand{\showprof}{show them}  % comment this line if you don't want to see todo environment
\fancyhead[L]{Correction exercices - fonctions}
\newdate{madate}{10}{09}{2020}
%\fancyhead[R]{\displaydate{madate}} %\today
%\fancyhead[R]{Seconde - SNT}
\fancyhead[R]{Première - NSI}
%\fancyhead[R]{Terminale - NSI}
\fancyfoot[L]{~\\Christophe Viroulaud}
\AtEndDocument{\label{lastpage}}
\fancyfoot[C]{\textbf{Page \thepage/\pageref{lastpage}}}
\fancyfoot[R]{\includegraphics[width=2cm,align=t]{/home/tof/Documents/Cozy/latex-include/cc.png}}

\begin{document}
\begin{Form}
\begin{aretenir}[Commentaires]
Chaque fonction est accompagnée d'un commentaire appelé \textbf{docstring}. Sa syntaxe varie selon les programmeurs mais une bonne pratique consiste à:
\begin{itemize}
\item donner une brève description,
\item préciser le type et le rôle des paramètres,
\item préciser ce que renvoie la fonction.
\end{itemize}
Cette \emph{docstring} est accessible depuis la console en écrivant \textbf{help(nom\_fonction)}.\\
De plus une autre bonne pratique consiste à préciser le \emph{type} des arguments de la fonction. La syntaxe
\lstinputlisting[firstline=10,lastline=10]{"scripts/est_pair.py"}
précise que le paramètre \emph{x} est un entier et que la fonction renvoie un \emph{booléen}.
\end{aretenir}
\begin{exo}
\lstinputlisting[firstline=10,lastline=22]{"scripts/est_pair.py"}
\end{exo}
\begin{exo}
\lstinputlisting[firstline=10,lastline=17]{"scripts/valeur_absolue.py"}
\end{exo}
\begin{exo}
\lstinputlisting[firstline=10,lastline=14]{"scripts/surface.py"}
\end{exo}
\begin{exo}
\lstinputlisting[firstline=10,lastline=23]{"scripts/est_majeur.py"}
\end{exo}
\begin{exo}
\lstinputlisting[firstline=10,lastline=28]{"scripts/puissance.py"}
\end{exo}
\begin{exo}
\lstinputlisting[firstline=10,lastline=24]{"scripts/pythagore.py"}
\end{exo}
\begin{exo}
\lstinputlisting[firstline=10,lastline=40]{"scripts/bissextile.py"}
\end{exo}
\begin{exo}
\lstinputlisting[firstline=10,lastline=14]{"scripts/nombres_pairs.py"}
\end{exo}
\begin{exo}
\lstinputlisting[firstline=10,lastline=22]{"scripts/diviseur.py"}
\end{exo}
\begin{exo}
\lstinputlisting[firstline=10,lastline=22]{"scripts/est_premier.py"}
\end{exo}
\begin{exo}
\lstinputlisting[firstline=10,lastline=39]{"scripts/aleatoire_100.py"}
\end{exo}
\begin{exo}
\lstinputlisting[firstline=10,lastline=19]{"scripts/nb_voyelles.py"}
\end{exo}
\begin{exo}
\lstinputlisting[firstline=10,lastline=25]{"scripts/sapin.py"}
\end{exo}
\end{Form}
\end{document}