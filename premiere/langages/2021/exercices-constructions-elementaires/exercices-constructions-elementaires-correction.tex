\documentclass[a4paper,11pt]{article}
\input{/home/tof/Documents/Cozy/latex-include/preambule_lua.tex}
\newcommand{\showprof}{show them}  % comment this line if you don't want to see todo environment
\fancyhead[L]{Exercices constructions élémentaires - correction}
\newdate{madate}{10}{09}{2020}
\fancyhead[R]{Première - NSI} %\today
\fancyfoot[L]{~\\Christophe Viroulaud}
\fancyfoot[C]{\textbf{Page \thepage}}
\fancyfoot[R]{\includegraphics[width=2cm,align=t]{/home/tof/Documents/Cozy/latex-include/cc.png}}

\begin{document}
\begin{Form}
\begin{exo}
Répondre aux questions sans ordinateur.
\begin{enumerate}
\item 6
\item 64
\item La première instruction affiche \emph{i+} la seconde affiche un message d'erreur.
\item La séquence inverse les valeurs de a et b. En Python il est possible de faire cela sans valeur intermédiaire:
\begin{lstlisting}
a,b = b,a
\end{lstlisting}
\end{enumerate}
\end{exo}
\begin{exo}
\lstinputlisting[firstline=10,lastline=12]{"scripts/rectangle.py"}
\end{exo}
\begin{exo}
\lstinputlisting[firstline=10,lastline=18]{"scripts/age.py"}
\end{exo}
\begin{exo}
\lstinputlisting[firstline=10,lastline=22]{"scripts/cinema.py"}
\end{exo}
\begin{exo}
\begin{enumerate}
\item Différentes divisions sont possibles:
\begin{itemize}
\item 20/3 renvoie le résultat de la division. Nous reviendrons plus tard sur le \emph{type} de ce résultat.
\item 20//3 renvoie la partie entière de la division. C'est un \emph{entier}.
\item 20\%3 renvoie le reste de la division. C'est un \emph{entier}. On appelle cette opération le \emph{modulo}.
\end{itemize}
\item Le programme répond directement à la question 3
\lstinputlisting[firstline=10,lastline=26]{"scripts/seconde.py"}
\end{enumerate}
\end{exo}
\begin{exo}
\lstinputlisting[firstline=10,lastline=18]{"scripts/multiplication.py"}
\end{exo}
\begin{exo}
\lstinputlisting[firstline=10,lastline=14]{"scripts/rebours.py"}
\end{exo}
\begin{exo}
\lstinputlisting[firstline=10,lastline=15]{"scripts/pair.py"}
\end{exo}
\begin{exo}
\lstinputlisting[firstline=10,lastline=31]{"scripts/moyenne.py"}
\end{exo}
\begin{exo}
\lstinputlisting[firstline=9,lastline=31]{"scripts/devinette.py"}
\end{exo}
\end{Form}
\end{document}