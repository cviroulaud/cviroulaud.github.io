\documentclass[a4paper,11pt]{article}
\input{/home/tof/Documents/Cozy/latex-include/preambule_doc.tex}
\input{/home/tof/Documents/Cozy/latex-include/preambule_commun.tex}
\newcommand{\showprof}{show them}  % comment this line if you don't want to see todo environment
\setlength{\fboxrule}{0.8pt}
\fancyhead[L]{\fbox{\Large{\textbf{DonRep 02}}}}
\fancyhead[C]{\textbf{Représentation des entiers relatifs}}
\newdate{madate}{10}{09}{2020}
%\fancyhead[R]{\displaydate{madate}} %\today
\fancyhead[R]{Première - NSI}
\fancyfoot[L]{\vspace{1mm}Christophe Viroulaud}
\AtEndDocument{\label{lastpage}}
\fancyfoot[C]{\textbf{Page \thepage/\pageref{lastpage}}}
\fancyfoot[R]{\includegraphics[width=2cm,align=t]{/home/tof/Documents/Cozy/latex-include/cc.png}}

\begin{document}
\begin{center}
    \framebox{Comment sont représentés les entiers négatifs en mémoire?}
\end{center}
\section{Addition de deux nombres binaires}
Une addition en base 2 applique les mêmes principes qu'en base 10:
    \begin{itemize}
        \item $0 + 0 = 0$
        \item $1 + 0 = 1$
        \item $1 + 1 = 0$ et une retenue de 1
        \item $1 + 1 + 1 = 1$ et une retenue de 1
    \end{itemize}
    \begin{center}
        \begin{tabular}{*{9}{c}}
              &   &   & {\small 1} & {\small 1} &   &   &   &   \\
              & 0 & 0 & 0          & 1          & 1 & 0 & 0 & 1 \\
            + & 0 & 0 & 0          & 0          & 1 & 1 & 0 & 0 \\
            \hline
              & 0 & 0 & 1          & 0          & 0 & 1 & 0 & 1 \\
        \end{tabular}
    \end{center}
    \section{Une représentation naïve des entiers négatifs}
    \subsection{Bit de poids fort}
    Une première idée serait d'utiliser ce bit comme marqueur de signe:
    \begin{itemize}
        \item 0 pour un entier positif,
        \item 1 pour un entier négatif.
    \end{itemize}
    Ainsi l'entier $-25$ serait encodé dans un mot mémoire de 1 octet:
    $$-25_{10} = 10011001_2$$
    \subsection{Inconvénients de la représentation}
\begin{itemize}
    \item Il y a 2 zéros.
    \item L'algorithme d'addition ne fonctionne pas.
\end{itemize}
\section{Le complément à 2 puissance \emph{n}}
\subsection{Définition}
La valeur $2^n - \vert x\vert$ représente l'entier négatif \emph{x}.
\begin{center}
    \begin{tabular}{|cccccccccc|c|}
        \hline
        1 & 1   & 1 & 1 & 1 & 1 & 1 & 1 & = & $-1$   & $2^8 - \vert -1\vert = 255$   \\
        \hline
        1 & 1   & 1 & 1 & 1 & 1 & 1 & 0 & = & $-2$   & $2^8 - \vert -2\vert = 254$   \\
        \hline
        1 & ... &   &   &   &   &   &   & = & ...    &                               \\
        \hline
        1 & 0   & 0 & 0 & 0 & 0 & 0 & 1 & = & $-127$ & $2^8 - \vert -127\vert = 129$ \\
        \hline
        1 & 0   & 0 & 0 & 0 & 0 & 0 & 0 & = & $-128$ & $2^8 - \vert -128\vert=128$   \\
        \hline
        0 & 1   & 1 & 1 & 1 & 1 & 1 & 1 & = & 127    &                               \\
        \hline
        0 & ... &   &   &   &   &   &   & = & ...    &                               \\
        \hline
        0 & 0   & 0 & 0 & 0 & 0 & 1 & 0 & = & 2      &                               \\
        \hline
        0 & 0   & 0 & 0 & 0 & 0 & 0 & 1 & = & 1      &                               \\
        \hline
        0 & 0   & 0 & 0 & 0 & 0 & 0 & 0 & = & 0      &                               \\
        \hline
    \end{tabular}
\end{center}
\subsection{Calculer le complément à 2}
Pour coder (−20):
\begin{enumerate}
    \item \begin{itemize}
        \item Prendre le nombre positif 20 : 00010100
        \item Inverser les bits : 11101011
        \item Ajouter 1 : 11101100
        \item −20 : 11101100
    \end{itemize}
    \item \begin{itemize}
        \item Prendre le nombre positif 20 : 00010100
        \item Garder la partie à droite telle quelle : 00010\underline{100}
        \item Inverser la partie de gauche après le premier un : \underline{11101}100
        \item −20 : 11101100
    \end{itemize}
\end{enumerate}
\subsection{Intérêt de la méthode}
\begin{itemize}
    \item Il n'y a qu'un seul zéro.
    \item On peut appliquer l'algorithme d'addition.
\end{itemize}
    
\end{document}