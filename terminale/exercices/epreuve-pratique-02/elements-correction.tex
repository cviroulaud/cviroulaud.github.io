\documentclass[a4paper,11pt]{article}
\input{/home/tof/Documents/Cozy/latex-include/preambule_doc.tex}
\input{/home/tof/Documents/Cozy/latex-include/preambule_commun.tex}
\newcommand{\showprof}{show them}  % comment this line if you don't want to see todo environment
\setlength{\fboxrule}{0.8pt}
\fancyhead[L]{\fbox{\Large{\textbf{BAC 02}}}}
\fancyhead[C]{\textbf{Épreuves pratiques}}
\newdate{madate}{10}{09}{2020}
%\fancyhead[R]{\displaydate{madate}} %\today
\fancyhead[R]{Terminale - NSI}
\fancyfoot[L]{\vspace{1mm}Christophe Viroulaud}
\AtEndDocument{\label{lastpage}}
\fancyfoot[C]{\textbf{Page \thepage/\pageref{lastpage}}}
\fancyfoot[R]{\includegraphics[width=2cm,align=t]{/home/tof/Documents/Cozy/latex-include/cc.png}}

\begin{document}
\section{Sujet 04}
\begin{itemize}
    \item recherche dans un tableau
    \item récursivité
\end{itemize}
\section{Sujet 17}
\begin{itemize}
    \item parcours de chaîne de caractère; il ne faut pas utiliser les méthodes \emph{buit-in} (\textbf{\texttt{split}}).
    \item arbre binaire de recherche; \underline{Observation: } dans \textbf{\texttt{parcours}}, mettre une valeur mutable en valeur par défaut est risqué.
\end{itemize}
\section{Sujet 27}
\begin{itemize}
    \item arbre binaire + récursivité
    \item tri itératif (= par sélection); \underline{Observations:} \begin{itemize}
              \item erreur dans le résultat du test. Il faut lire \textbf{\texttt{[6, 12, 18, 21, 25, 41, 55]}} et pas \textbf{\texttt{[6, 18, 12, 21, 25, 41, 55]}}
              \item la fonction effectue un tri en place mais renvoie tout de même le tableau. Il faut rappeler qu'une donnée mutable passée en paramètre d'une fonction est \textbf{une référence et non une copie}.
          \end{itemize}
\end{itemize}
\section{Sujet 30}
\begin{itemize}
    \item fusion tableaux triés
    \item conversion romain/décimal; \underline{Observation:} utilisation du \emph{slice} (hors programme)
\end{itemize}
\section{Sujet 32}
\begin{itemize}
    \item recherche dans tableau
    \item adresse IP (POO)
\end{itemize}
\section{Sujet 33}
\begin{itemize}
    \item conversion binaire
    \item tri par insertion; \underline{Observation:} la fonction effectue un tri en place mais renvoie tout de même le tableau. Il faut rappeler qu'une donnée mutable passée en paramètre d'une fonction est \textbf{une référence et non une copie}.
\end{itemize}
\end{document}