\documentclass[svgnames,11pt]{beamer}
\input{/home/tof/Documents/Cozy/latex-include/preambule_commun.tex}
\input{/home/tof/Documents/Cozy/latex-include/preambule_beamer.tex}
\usepackage{pgfpages} \setbeameroption{show notes on second screen=left}
\author[]{Christophe Viroulaud}
\title{Ordonnancement des processus}
\date{\framebox{\textbf{Archi 02}}}
%\logo{}
\institute{Terminale - NSI}
%DODO détailler les processus
\begin{document}
\begin{frame}
\titlepage
\end{frame}
\begin{frame}
    \frametitle{}

    Un processeur ne peut exécuter qu'une seule instruction à la fois. Pourtant sur un ordinateur, il est possible d'écouter de la musique tout en surfant sur le web.

\end{frame}
\begin{frame}
    \frametitle{}

    \begin{framed}
        \centering Comment exécuter plusieurs activités en même temps sur une machine?
    \end{framed}

\end{frame}
\section{Les processus}
\subsubsection{Définition}
\begin{frame}
    \frametitle{Les processus - définition}

    \begin{aretenir}[]
        Un \emph{programme} est un fichier en mémoire qui ne fait rien. Un \emph{processus} est l'exécution d'un programme. 
    \end{aretenir}

\end{frame}
\begin{frame}[fragile]
    \frametitle{}

    \begin{activite}
    \begin{enumerate}
        \item Ouvrir un terminal.
        \item Écrire la commande
        \begin{center}
        \begin{lstlisting}[language=Bash , basicstyle=\ttfamily\small, xleftmargin=2em, xrightmargin=2em]
top
\end{lstlisting}
        \captionof{code}{Visualiser les processus en cours}
        \label{CODE}
        \end{center}
        \item Dans Debian, ouvrir le logiciel \emph{Firefox} et observer l'apparition de nouveaux processus.
        \item Depuis le terminal, utiliser la combinaison de touche \textbf{\texttt{Ctrl+c}} pour stopper la surveillance des processus.
    \end{enumerate}
    \end{activite}

\end{frame}
\subsection{Création d'un processus}
\begin{frame}
    \frametitle{Création d'un processus}

    \begin{aretenir}[]
        Chaque processus possède un identifiant unique, le \textbf{\texttt{PID}} (\textbf{P}rocess \textbf{ID}entifier). Au démarrage de la machine un premier processus spécial  (\emph{init}) est lancé. Ce processus crée d'autres \emph{processus fils}. Ainsi chaque processus possède un (et un seul) parent, le \textbf{\texttt{PPID}} (\textbf{P}arent \textbf{P}rocess \textbf{ID}entifier).
    \end{aretenir}

\end{frame}
\begin{frame}[fragile]
    \frametitle{}

    \begin{activite}
    \begin{enumerate}
        \item Afficher la liste de tous les processus:
        \begin{center}
        \begin{lstlisting}[language=Bash , basicstyle=\ttfamily\small, xleftmargin=2em, xrightmargin=2em]
ps all
\end{lstlisting}
        \end{center}
        \item Retrouver le  PID du processus de Firefox. Tuer le processus avec l'instruction:
        \begin{center}
            \begin{lstlisting}[language=Bash , basicstyle=\ttfamily\small, xleftmargin=2em, xrightmargin=2em]
kill numéro_PID
\end{lstlisting}
            \end{center}
    \end{enumerate}
    \end{activite}
\note{
    d'autres infos: nom de l'utilisateur qui a crée, utilisation CPU, état (STAT):
\begin{itemize}
\item R en cours d'exécution.
\item T processus stoppé.
\item I processus endormi (>20s).
\item S processus endormi (<20s).
\item Z processus zombie.
\item D processus non interruptible.
\item W processus swappé (échangé) sur disque. 
\end{itemize}
}
\end{frame}
\section{Ordonnancement}
\subsection{Le chef d'orchestre}
\begin{frame}
    \frametitle{Le chef d'orchestre}

    
    Dans le système plusieurs processus sont en cours simultanément, mais le processeur ne peut exécuter qu’une seule instruction à la fois. Le processeur travaille donc \emph{en temps partagé}. Il bascule constamment d’un processus à l’autre.

    \note[item]{chaque processus considère qu'il a le processeur pour lui tout seul. multiprogrammation}
    \note[item]{scheduleur = 1 module du système d'exploitation}
\end{frame}
\begin{frame}
    \frametitle{}

    \begin{aretenir}[]
        \textbf{L'ordonnanceur (scheduleur)} sélectionne le prochain processus prêt (\emph{Ready}) qui sera exécuté par le processeur. L'objectif est d'obtenir un \emph{temps de traitement moyen} le plus court possible.
    \end{aretenir}

\end{frame}
\subsection{Le scheduling}
\begin{frame}
    \frametitle{Le scheduling}

    Les algorithmes d’ordonnancement peuvent être classés en deux catégories:
    \begin{itemize}
    \item<1-> \textbf{Non pré~emptif:} Sélectionne un processus, puis le laisse s’exécuter jusqu’à ce qu’il bloque (soit sur une E/S, soit en attente d’un autre processus) où qu’il libère volontairement le processeur.
    \item<2-> \textbf{Pré~emptif:} Sélectionne un processus et le laisse s’exécuter pendant un délai déterminé.
    \end{itemize}

\end{frame}
\subsection{Quelques algorithmes d'ordonnancement}
\begin{frame}
    \frametitle{Quelques algorithmes d'ordonnancement}

  \begin{itemize}
      \item \textbf{First Come First Served: }Une fois que le CPU a été allouée à un processus, celui-ci le garde jusqu’à ce qu’il décide de le libérer. 
      \item \textbf{Shortest Job First:} Quand le CPU est disponible, elle est assignée au processus qui possède le prochain cycle le plus petit. 
      \item \textbf{Round Robin:} Chaque processus a une petite unité de temps appelée \emph{quantum} (en général de 10 à 100 ms). L'ordonnanceur parcourt la file d’attente des processus prêts et alloue le CPU à chaque processus pendant un quantum.
  \end{itemize}  
\note[item]{\textbf{First Come: }Cet algorithme est particulièrement incommode pour le temps partagé où il est important que chaque utilisateur obtienne le CPU à des intervalles réguliers.}
\note[item]{\textbf{Shortest:} La difficulté est pouvoir connaître la longueur de la prochaine requête du CPU.}
\note[item]{La performance du \emph{tourniquet} dépend fortement du choix du quantum de base. }
\note[item]{Linux propose ces 3 politiques d'ordonnancement. Un système de priorité est également mis en place. Par défaut, un processus est associé à la politique de temps partagé. Seul root peut associer un processus à une des classes d’ordonnancement en temps réel.}
\end{frame}
\end{document}
