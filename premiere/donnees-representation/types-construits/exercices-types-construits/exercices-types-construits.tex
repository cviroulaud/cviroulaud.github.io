\documentclass[a4paper,11pt]{article}
\input{/home/tof/Documents/Cozy/latex-include/preambule_doc.tex}
\input{/home/tof/Documents/Cozy/latex-include/preambule_commun.tex}
\newcommand{\showprof}{show them}  % comment this line if you don't want to see todo environment
\setlength{\fboxrule}{0.8pt}
\fancyhead[L]{\fbox{\Large{\textbf{DonRep 09}}}}
\fancyhead[C]{\textbf{Exercices types construits}}
\newdate{madate}{10}{09}{2020}
%\fancyhead[R]{\displaydate{madate}} %\today
\fancyhead[R]{Première - NSI}
\fancyfoot[L]{\vspace{1mm}Christophe Viroulaud}
\AtEndDocument{\label{lastpage}}
\fancyfoot[C]{\textbf{Page \thepage/\pageref{lastpage}}}
\fancyfoot[R]{\includegraphics[width=2cm,align=t]{/home/tof/Documents/Cozy/latex-include/cc.png}}

\begin{document}
\begin{exo}
\begin{enumerate}
    \item Ouvrir le qcm \url{https://tinyurl.com/qcmtyp} et répondre aux questions. L'identifiant sera donné par le professeur.
    \item Vérifier les réponses dans le corrigé \url{https://tinyurl.com/qcmtypcor}
\end{enumerate}
\end{exo}
\begin{exo}
    On considère le tableau:
    \begin{lstlisting}[language=Python  , xleftmargin=2em, xrightmargin=2em]
tab = [0, 1, 2, 3, 4, 5]
\end{lstlisting}
    Écrire un programme qui remplace chaque entier du tableau par son carré.
\end{exo}
\begin{exo}
    On considère le tuple qui ne peut contenir que des entiers inférieurs ou égaux à 100:
    \begin{lstlisting}[language=Python  , xleftmargin=2em, xrightmargin=2em]
tup = (17, 32, 8, 4, 35, 13)
\end{lstlisting}
    Écrire la fonction \textbf{\texttt{mini(t: tuple) $\rightarrow$ int}} qui renvoie l'entier minimum du tuple passé en paramètre.
\end{exo}
\begin{exo}
    En mathématiques, la médiane est la valeur qui sépare la moitié inférieure de la moitié supérieure d'un ensemble ordonné. Deux cas existent:
    \begin{itemize}
        \item la taille de l'ensemble est impair: la médiane est l'élément du milieu,
        \item la taille de l'ensemble est pair: la médiane est la moyenne des deux éléments du milieu.
    \end{itemize}
    On considère le tuple:
    \begin{lstlisting}[language=Python  , xleftmargin=2em, xrightmargin=2em]
salaire = (800, 830, 830, 950, 1002, 1100, 1100, 1103, 1340, 1530, 1600)
salaire1 = (900, 950, 950, 960, 1050, 1060, 1100, 1160, 1370, 1555)
\end{lstlisting}
    \begin{enumerate}
        \item Quelle est la médiane des deux tuples?
        \item Écrire la fonction \textbf{\texttt{mediane(t: tuple) $\rightarrow$ int}} qui renvoie la médiane du tuple passé en paramètre.
    \end{enumerate}
\end{exo}
\begin{exo}
On considère le tableau ordonné de mesures:
\begin{lstlisting}[language=Python  , xleftmargin=2em, xrightmargin=2em]
mesures = [10, 15, 16, 23, 25, 38, 41, 43]
\end{lstlisting}
Écrire la fonction \textbf{\texttt{ecart\_max(t: list) $\rightarrow$ int}} qui renvoie l'écart maximum entre deux mesures consécutives du tableau. Par exemple:
\begin{lstlisting}[language=Python  , xleftmargin=2em, xrightmargin=2em]
>>> ecart_max(mesures)
13 # écart entre 38 et 25
\end{lstlisting}
\end{exo}
\begin{exo}
    On considère les tableaux de taille 5:
    \begin{lstlisting}[language=Python  , xleftmargin=2em, xrightmargin=2em]
t1 = [12, 17, 8, 10, 13]
t2 = [4, 18, 9, 11, 23]
\end{lstlisting}
    Écrire la fonction \textbf{\texttt{somme(t1: list, t2: list) $\rightarrow$ list}} qui:
    \begin{itemize}
        \item fait la somme des éléments de même indice des deux tableaux,
        \item stocke ces sommes dans un nouveau tableau,
        \item renvoie ce tableau.
    \end{itemize}
On considérera que les tableaux passés en paramètres ont obligatoirement une taille 5.
\end{exo}
\begin{exo}Un livre peut être caractérisé par son titre, son auteur, son éditeur, son prix.
    \begin{enumerate}
        \item Construire un dictionnaire qui contient les informations du livre: \emph{Il était deux fois} de Franck Thilliez aux éditions \emph{Poche} à 8,70€.
        \item Construire un dictionnaire pour \emph{Fahrenheit 451} de Ray Bradbury aux éditions \emph{Folio} à 6,30€.
        \item Construire un tableau \textbf{\texttt{bibliotheque}} contenant les deux dictionnaires.
        \item Écrire une boucle qui parcourt le tableau et affiche l'auteur de chaque livre.
    \end{enumerate}
\end{exo}
\begin{exo}
    Le groupe d'élèves de NSI est composé de:
    \begin{itemize}
        \item Alice Durant,
        \item Bob Bois,
        \item John Doe,
        \item Jules Dupont,
        \item Alan Turing.
    \end{itemize}
    Au cours du semestre les notes obtenues sont:
    \begin{center}
        \begin{tabular}{|c|*{4}{c}|}
            \hline
            Alice & 12   & 8  & 10   & 9.5  \\
            \hline
            Bob   & 15   & 17 & 18   & 14   \\
            \hline
            John  & 10.5 & 8  & 16   & 13.5 \\
            \hline
            Jules & 12   & 9  & 17.5 & 10   \\
            \hline
            Alan  & 14   & 18 & 16   & 19   \\
            \hline
        \end{tabular}
    \end{center}
    \begin{enumerate}
        \item Créer un dictionnaire contenant les informations de chaque élève et stocker tous ces dictionnaires dans un tableau.
        \item Écrire la fonction \textbf{moyenne(eleve: dict) $\rightarrow$ float} qui renvoie la moyenne de l'élève.
        \item Calculer la moyenne générale du groupe.
    \end{enumerate}
\end{exo}
\end{document}