\documentclass[svgnames,11pt]{beamer}
\input{/home/tof/Documents/Cozy/latex-include/preambule_commun.tex}
\input{/home/tof/Documents/Cozy/latex-include/preambule_beamer.tex}
%\usepackage{pgfpages} \setbeameroption{show notes on second screen=left}
\author[]{Christophe Viroulaud}
\title{Exercices - données en table\\Correction}
\date{\framebox{\textbf{Tab 03}}}
%\logo{}
\institute{Première - NSI}

\begin{document}
\begin{frame}
    \titlepage
\end{frame}
\section{Exercice 1}
\begin{frame}
    \frametitle{Exercice 1}

    \begin{itemize}
        \item \underline{ligne 3: }\emph{À l'université de Salamanque, un groupe d'étudiants:} la virgule est interprétée comme un séparateur. On peut entourer ce texte avec des guillemets.
        \item \underline{ligne 4: } La virgule finale est en trop.
        \item \underline{ligne 5: } Les guillemets sont mal placés.
    \end{itemize}

\end{frame}
\section{Exercice 2}
\begin{frame}[fragile]
    \frametitle{Exercice 2}

    \begin{center}
        \begin{lstlisting}[language=Python , basicstyle=\ttfamily\small, xleftmargin=2em, xrightmargin=2em]
f = open("sondage.csv", "r", encoding="utf8")
reader = csv.DictReader(f)
donnees = []
for ligne in reader:
    individu = {}
    # formatage des données
    for cle, val in ligne.items():
        if cle == "age" or cle == "salaire":
            val = int(val)
        individu[cle] = val
    # ajout dans le tableau
    donnees.append(individu)
f.close()
\end{lstlisting}
    \end{center}

\end{frame}
\begin{frame}[fragile]

    \begin{center}
        \begin{lstlisting}[language=Python , basicstyle=\ttfamily\small, xleftmargin=2em, xrightmargin=2em]
def salaire_max(donnees: list) -> int:
    maxi = 0
    for individu in donnees:
        # individu est un dictionnaire
        if individu["salaire"] > maxi:
            maxi = individu["salaire"]
    return maxi
\end{lstlisting}
    \end{center}

\end{frame}
\begin{frame}[fragile]

\begin{center}
    \begin{lstlisting}[language=Python , basicstyle=\ttfamily\small, xleftmargin=2em, xrightmargin=2em]
def age_moyen(donnees: list) -> float:
    somme = 0
    for individu in donnees:
        somme = somme+individu["age"]
    return somme/len(donnees)
\end{lstlisting}
\end{center}

\end{frame}
\begin{frame}[fragile]

    \begin{center}
        \begin{lstlisting}[language=Python , basicstyle=\ttfamily\small, xleftmargin=2em, xrightmargin=2em]
def dernier(donnees: list) -> str:
    nom = ""
    for individu in donnees:
        # on peut comparer des chaînes de caractères
        if individu["nom"] > nom:
            nom = individu["nom"]
    return nom
\end{lstlisting}
    \end{center}

\end{frame}
\section{Exercice 3}
\begin{frame}[fragile]
    \frametitle{Exercice 3}

    \begin{center}
        \begin{lstlisting}[language=Python , basicstyle=\ttfamily\small, xleftmargin=1em, xrightmargin=0em]
import csv

f = open("geocube.csv", "w")
writer = csv.DictWriter(f, ["Jour", "Heure", "Temperature"])
writer.writeheader()
writer.writerow(
    {"Jour": "2021-12-03", 
    "Heure": "23:09:43", 
    "Temperature": 9.1})

writer.writerow(
    {"Jour": "2021-12-03", 
    "Heure": "23:29:43", 
    "Temperature": 9.4})

f.close()    
\end{lstlisting}
    \end{center}

\end{frame}
\begin{frame}[fragile]
    \frametitle{}

    \begin{center}
        \begin{lstlisting}[language=Python , basicstyle=\ttfamily\small, xleftmargin=2em, xrightmargin=2em]
# a pour append
f = open("geocube.csv", "a")
\end{lstlisting}
        \captionof{code}{Ajouter des données en fin de fichier.}
        \label{CODE}
    \end{center}

\end{frame}
\end{document}