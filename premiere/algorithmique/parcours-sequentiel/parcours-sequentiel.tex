\documentclass[a4paper,11pt]{article}
\input{/home/tof/Documents/Cozy/latex-include/preambule_doc.tex}
\input{/home/tof/Documents/Cozy/latex-include/preambule_commun.tex}
\newcommand{\showprof}{show them}  % comment this line if you don't want to see todo environment
\setlength{\fboxrule}{0.8pt}
\fancyhead[L]{\fbox{\Large{\textbf{ParSeq 01}}}}
\fancyhead[C]{\textbf{Parcours séquentiel}}
\newdate{madate}{10}{09}{2020}
%\fancyhead[R]{\displaydate{madate}} %\today
%\fancyhead[R]{Seconde - SNT}
\fancyhead[R]{Première - NSI}
%\fancyhead[R]{Terminale - NSI}
\fancyfoot[L]{\vspace{1mm}Christophe Viroulaud}
\AtEndDocument{\label{lastpage}}
\fancyfoot[C]{\textbf{Page \thepage/\pageref{lastpage}}}
\fancyfoot[R]{\includegraphics[width=2cm,align=t]{/home/tof/Documents/Cozy/latex-include/cc.png}}

\begin{document}
\section{Recherche d'extremum}
\begin{aretenir}[Remarques]
\begin{itemize}
    \item Les fonctions ne devront pas utiliser les méthodes natives fournies par le langage Python, telles que \emph{max(), min()}.
    \item On attend une \emph{docstring} pour chaque fonction.
\end{itemize}
\end{aretenir}
\begin{enumerate}
    \item Construire un tableau de trente notes aléatoires comprises entre 0 et 20.
    \item Écrire la fonction \textbf{maxi(tab: list) $\rightarrow$ int} qui renvoie la note maximale du tableau \emph{tab}.
    \item Écrire la fonction \textbf{extrema(tab: list) $\rightarrow$ tuple} qui renvoie les notes minimale et maximale du tableau \emph{tab}, sous forme de tuple.
    \item Écrire la fonction \textbf{maxi\_position(tab: list) $\rightarrow$ tuple} qui renvoie la note maximale du tableau \emph{tab} et l'indice de la première apparition de cette note, sous forme de tuple.
    \item Écrire la fonction \textbf{maxi\_position\_dernier(tab: list) $\rightarrow$ tuple} qui renvoie la note maximale du tableau \emph{tab} et l'indice de la \underline{dernière} apparition de cette note, sous forme de tuple.
    \item Écrire la fonction \textbf{maxi\_nb(tab: list) $\rightarrow$ int} qui renvoie le nombre d'apparitions de la note maximale du tableau \emph{tab}.

\end{enumerate}
\section{Efficacité}
\begin{enumerate}
    \item \begin{enumerate}
              \item Écrire la fonction \textbf{est\_present(tab: list, note: int) $\rightarrow$ bool} qui renvoie \emph{True} si \emph{note} est présent dans \emph{tab}.
              \item De quel paramètre dépend la durée d'exécution de la fonction? Peut-on faire mieux?
          \end{enumerate}
    \item \begin{enumerate}
              \item Écrire la fonction \textbf{moyenne(tab: int) $\rightarrow$ float} qui renvoie la moyenne des valeurs du tableau.
              \item De quel paramètre dépend la durée d'exécution de la fonction? Peut-on faire mieux?
          \end{enumerate}
\end{enumerate}
\section{Parcours d'un dictionnaire}
\begin{enumerate}
    \item Construire un dictionnaire \textbf{voyelles} qui associe chaque voyelle à l'entier 0.
    \item Écrire la fonction \textbf{compter\_voyelles(mot: str) $\rightarrow$ dict} qui renvoie un dictionnaire du décompte des voyelles. On pourra construire \emph{dans la fonction} un dictionnaire sur le modèle de la question précédente.
    \item Écrire la boucle qui affiche dans la console chaque voyelle suivie de son nombre d'occurrences. Par exemple, on pourra afficher:
    \begin{itemize}
        \item A: 3
        \item E: 5
        \item \dots
    \end{itemize}
    \item Écrire la fonction \textbf{max\_voyelles(voyelles: dict) $\rightarrow$ list} qui renvoie les voyelles qui comptent le plus d'occurrences dans le dictionnaire renvoyé par la fonction précédente. Par exemple, pour le mot "orangeade" le tableau renvoyé sera ['a', 'e'].
\end{enumerate}
\end{document}