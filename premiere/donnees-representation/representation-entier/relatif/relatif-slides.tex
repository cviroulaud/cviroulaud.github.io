\documentclass[svgnames,11pt]{beamer}
\input{/home/tof/Documents/Cozy/latex-include/preambule_commun.tex}
\input{/home/tof/Documents/Cozy/latex-include/preambule_beamer.tex}
\usepackage{pgfpages} \setbeameroption{show notes on second screen=left}
\author[]{Christophe Viroulaud}
\title{Représentation des entiers relatifs}
\date{\framebox{\textbf{DonRep 02}}}
%\logo{}
\institute{Première - NSI}

\begin{document}
\begin{frame}
    \titlepage
\end{frame}
\begin{frame}[fragile]
    \frametitle{}

    Un système \emph{64 bits} peut représenter $2^{64}$ entiers.
    \begin{center}
        \begin{lstlisting}[language=Python , basicstyle=\ttfamily\small, xleftmargin=2em, xrightmargin=2em]
>>> import sys
>>> sys.maxsize 
9223372036854775807
\end{lstlisting}
        \captionof{code}{Cette valeur correspond à $2^{63}-1$.}
        \label{CODE}
    \end{center}
    \begin{aretenir}[Observation]
        Un des bits ne semble pas utilisé.
    \end{aretenir}
\end{frame}
\begin{frame}
    \frametitle{}

    \begin{center}
        \framebox{Comment sont représentés les entiers négatifs en mémoire?}
    \end{center}

\end{frame}
\section{Addition de deux nombres binaires}
\begin{frame}
    \frametitle{Addition de deux nombres binaires}

    Une addition en base 2 applique les mêmes principes qu'en base 10:
    \begin{itemize}
        \item $0 + 0 = 0$
        \item $1 + 0 = 1$
        \item $1 + 1 = 0$ et une retenue de 1
        \item $1 + 1 + 1 = 1$ et une retenue de 1
    \end{itemize}

\end{frame}
\begin{frame}
    \frametitle{}

    Dans un mot mémoire de 1 octet:
    $$25_{10} = 00011001_2$$
    \note{D'après l'intro dans 8 bits on peut stocker des nombres allant jusqu'à $$2^7-1 = 127_{10} = 01111111_2$$.}
\end{frame}
\begin{frame}
    \frametitle{}

    \begin{activite}
        \begin{enumerate}
            \item Convertir 25 et 12 en base 2.
            \item Effectuer l'addition binaire de ces nombres.
            \item Convertir le résultat en base 10. Le résultat est-il correct?
        \end{enumerate}
    \end{activite}

\end{frame}
\begin{frame}
    \frametitle{Correction}

    \begin{center}
        \begin{tabular}{*{9}{c}}
              &   &   & {\small 1} & {\small 1} &   &   &   &   \\
              & 0 & 0 & 0          & 1          & 1 & 0 & 0 & 1 \\
            + & 0 & 0 & 0          & 0          & 1 & 1 & 0 & 0 \\
            \hline
              & 0 & 0 & 1          & 0          & 0 & 1 & 0 & 1 \\
        \end{tabular}
    \end{center}

\end{frame}
\begin{frame}
    \frametitle{Correction}

    $$0×2^7+0×2^6+1×2^5+0×2^4+0×2^3+1×2^2+0×2^1+1×2^0=37$$

\end{frame}
\section{Une représentation naïve des entiers négatifs}
\subsection{Bit de poids fort}
\begin{frame}
    \frametitle{Bit de poids fort}
    Le bit le plus à gauche de la représentation n'est pour l'instant pas utilisé. C'est le \textbf{bit de poids fort}.


\begin{aretenir}[]
Pour représenter un nombre entier, il faut connaître la taille du mot mémoire.
\end{aretenir}
\end{frame}
\begin{frame}
    \frametitle{}

    Une première idée serait d'utiliser ce bit comme marqueur de signe:
    \begin{itemize}
        \item 0 pour un entier positif,
        \item 1 pour un entier négatif.
    \end{itemize}
    Ainsi l'entier $-25$ serait encodé dans un mot mémoire de 1 octet:
    $$-25_{10} = 10011001_2$$

\end{frame}
\subsection{Inconvénients de la représentation}
\begin{frame}
    \frametitle{Le zéro}

    Dans un système \emph{8 bits} le zéro est représenté par $00000000_2$. Cependant $10000000_2$ se traduit par~$-0$. Il y a donc deux représentations pour zéro.

\end{frame}
\begin{frame}
    \frametitle{Erreur d'addition}
    $$-25+12$$
    \begin{center}
        \begin{tabular}{*{9}{c}}
              &   &   & {\small 1} & {\small 1} &   &   &   &   \\
              & 1 & 0 & 0          & 1          & 1 & 0 & 0 & 1 \\
            + & 0 & 0 & 0          & 0          & 1 & 1 & 0 & 0 \\
            \hline
              & 1 & 0 & 1          & 0          & 0 & 1 & 0 & 1 \\
        \end{tabular}
    \end{center}
    Avec cette représentation:
    $$2^5+2^2+2^0=37$$
    $$10100101_2 = -37_{10}$$
    $$-25+12\neq-37$$
    \begin{aretenir}[]
        Cette représentation est erronée.
    \end{aretenir}
\end{frame}
\section{Le complément à 2 puissance \emph{n}}
\subsection{Définition}
\begin{frame}
    \frametitle{Définition}

    Le complément à 2 puissance \emph{n} est une représentation qui ne change rien pour les entiers positifs. Ainsi sur 8 bits:
    \begin{center}
        \begin{tabular}{|cccccccccc|}
            \hline
            0 & 1   & 1 & 1 & 1 & 1 & 1 & 1 & = & 127 \\
            \hline
            0 & ... &   &   &   &   &   &   & = & ... \\
            \hline
            0 & 0   & 0 & 0 & 0 & 0 & 1 & 0 & = & 2   \\
            \hline
            0 & 0   & 0 & 0 & 0 & 0 & 0 & 1 & = & 1   \\
            \hline
            0 & 0   & 0 & 0 & 0 & 0 & 0 & 0 & = & 0   \\
            \hline
        \end{tabular}
    \end{center}

\end{frame}
\begin{frame}
    \frametitle{}

    Par contre la valeur $2^n - \vert x\vert$ représente l'entier négatif \emph{x}. Ainsi sur 8 bits, $-1$ s'écrit $$2^8-1 = 256 - 1 = 255_{10} = 11111111_2$$
    \begin{center}
        \begin{tabular}{|cccccccccc|c|}
            \hline
            1 & 1   & 1 & 1 & 1 & 1 & 1 & 1 & = & $-1$   & $2^8 - \vert -1\vert = 255$   \\
            \hline
            1 & 1   & 1 & 1 & 1 & 1 & 1 & 0 & = & $-2$   & $2^8 - \vert -2\vert = 254$   \\
            \hline
            1 & ... &   &   &   &   &   &   & = & ...    &                               \\
            \hline
            1 & 0   & 0 & 0 & 0 & 0 & 0 & 1 & = & $-127$ & $2^8 - \vert -127\vert = 129$ \\
            \hline
            1 & 0   & 0 & 0 & 0 & 0 & 0 & 0 & = & $-128$ & $2^8 - \vert -128\vert=128$   \\
            \hline
            0 & 1   & 1 & 1 & 1 & 1 & 1 & 1 & = & 127    &                               \\
            \hline
            0 & ... &   &   &   &   &   &   & = & ...    &                               \\
            \hline
            0 & 0   & 0 & 0 & 0 & 0 & 1 & 0 & = & 2      &                               \\
            \hline
            0 & 0   & 0 & 0 & 0 & 0 & 0 & 1 & = & 1      &                               \\
            \hline
            0 & 0   & 0 & 0 & 0 & 0 & 0 & 0 & = & 0      &                               \\
            \hline
        \end{tabular}
    \end{center}

\end{frame}
\subsection{Calculer le complément à 2}
\begin{frame}
    \frametitle{Calculer le complément à 2}

    Pour coder (−20):
    \begin{itemize}
        \item Prendre le nombre positif 20 : 00010100
        \item Inverser les bits : 11101011
        \item Ajouter 1 : 11101100
        \item −20 : 11101100
    \end{itemize}

\end{frame}
\begin{frame}
    \frametitle{Seconde méthode}

    Garder tous les chiffres depuis la droite jusqu'au premier 1 (compris) puis d'inverser tous les suivants.
    \begin{itemize}
        \item Prendre le nombre positif 20 : 00010100
        \item Garder la partie à droite telle quelle : 00010\underline{100}
        \item Inverser la partie de gauche après le premier un : \underline{11101}100
        \item −20 : 11101100
    \end{itemize}
\end{frame}
\begin{frame}
    \frametitle{}

    \begin{activite}
        Calculer le complément à 2 (sur 1 octet) de $-25$.
    \end{activite}

\end{frame}
\begin{frame}
    \frametitle{Correction}
    \begin{itemize}
        \item $25_{10} = 00011001_2$
        \item $0001100\underline{1}$
        \item $-25_{10}=11100111$
    \end{itemize}


\end{frame}
\subsection{Intérêt de la méthode}
\begin{frame}
    \frametitle{Intérêt de la méthode}
\begin{center}
        Il n'y a qu'un seul zéro.
    
\end{center}

\end{frame}
\begin{frame}
    \frametitle{Addition}

    $-25+12= 11100111+00001100 = 11110011 = 243$ et $243-2^8=243-256=-13$

    $$-25+12$$
    \begin{center}
        \begin{tabular}{*{9}{c}}
              &   &   &  & {\small 1} &  {\small 1} &   &   &   \\
              & 1 & 1 & 1          & 0          & 0 & 1 & 1 & 1 \\
            + & 0 & 0 & 0          & 0          & 1 & 1 & 0 & 0 \\
            \hline
              & 1 & 1 & 1          & 1          & 0 & 0 & 1 & 1 \\
        \end{tabular}
    \end{center}
    Avec cette représentation:
    $$2^7+2^6+2^5+2^4+2^1+2^0=243$$
    $$243 -2^8 = 243-256=-13$$
    
\end{frame}
\begin{frame}
    \frametitle{}

    \begin{aretenir}[]
        Les nombres entiers négatifs sont représentés par le complément à 2.
            \end{aretenir}

\end{frame}
\end{document}