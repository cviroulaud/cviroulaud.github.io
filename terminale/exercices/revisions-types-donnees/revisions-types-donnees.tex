\documentclass[a4paper,11pt]{article}
\input{/home/tof/Documents/Cozy/latex-include/preambule_doc.tex}
\input{/home/tof/Documents/Cozy/latex-include/preambule_commun.tex}
\newcommand{\showprof}{show them}  % comment this line if you don't want to see todo environment
\setlength{\fboxrule}{0.8pt}
\fancyhead[L]{\fbox{\Large{\textbf{Rev 01}}}}
\fancyhead[C]{\textbf{Révisions types de données}}
\newdate{madate}{10}{09}{2020}
%\fancyhead[R]{\displaydate{madate}} %\today
\fancyhead[R]{Terminale - NSI}
\fancyfoot[L]{\vspace{1mm}Christophe Viroulaud}
\AtEndDocument{\label{lastpage}}
\fancyfoot[C]{\textbf{Page \thepage/\pageref{lastpage}}}
\fancyfoot[R]{\includegraphics[width=2cm,align=t]{/home/tof/Documents/Cozy/latex-include/cc.png}}

\begin{document}
\section{Les différents types de données}
\subsection{Tableau (list)}
Un tableau contient des éléments de même type (entiers, booléens\dots) repérés par leur indice (position) dans le tableau. Les indices commencent à zéro.
\begin{lstlisting}[language=Python  , xleftmargin=2em, xrightmargin=2em]
tab = [3, 5, 9, 8]
# lire un élément
tab[2] # renvoie l'entier 9

# écrire un élément
tab[2] = 10 # le 9 est remplacé par 10
\end{lstlisting}
En Python, les tableaux se nomment des \textbf{\texttt{list}}.
\subsection{Dictionnaire}
Un dictionnaire contient des éléments repérés par une clé. Une clé est un élément non mutable: entier, chaîne de caractères, tuple.
\begin{lstlisting}[language=Python  , xleftmargin=2em, xrightmargin=2em]
dico = {"prems": 18, "deuz": 13, "troiz": 9}
# lire un élément
dico["deuz"] # renvoie l'entier 13

# écrire un élément
tab["deuz"] = 10 # le 13 est remplacé par 10
\end{lstlisting}
\subsection{Tuple}
Un tuple est un n-uplet non mutable.
\begin{lstlisting}[language=Python  , xleftmargin=2em, xrightmargin=2em]
tup = (3, 5, 9, 8)
# lire un élément
tup[2] # renvoie l'entier 9

# Il n'est pas possible de modifier le contenu d'un tuple.
\end{lstlisting}
\subsection{Construction par compréhension}
En Python il est possible de construire une structure de données de manière rapide et efficace.
\begin{center}
    \begin{lstlisting}[language=Python  , xleftmargin=2em, xrightmargin=2em]
tab = [0 for i in range(5)]
# tab = [0, 0, 0, 0, 0]

dico = {i: 0 for i in range(5)}
# dico = {0:0, 1:0, 2:0, 3:0, 4:0}

tup = (0 for i in range(5))
# tup = (0, 0, 0, 0, 0)
\end{lstlisting}
    \captionof{code}{Construction par compréhension}
    \label{CODE}
\end{center}
\section{Applications}
\subsection{Construction par compréhension}
\begin{exo}
    Construire par compréhension le tableau \textbf{\texttt{[1, 1, 1, 1, 1]}}.
\end{exo}
\begin{exo}
    Construire par compréhension le tableau \textbf{\texttt{[0, 1, 2, 3, 4]}}.
\end{exo}
\begin{exo}
    Construire par compréhension le tuple \textbf{\texttt{[4, 3, 2, 1, 0]}}.
\end{exo}
\begin{exo}
    Construire par compréhension le tuple \textbf{\texttt{[0, 2, 4, 6, 8]}}.
\end{exo}
\begin{exo}
    Construire par compréhension le dictionnaire \textbf{\texttt{[0:1, 1:1, 2:1, 3:1, 4:1]}}.
\end{exo}
\begin{exo}
    Construire par compréhension le dictionnaire \textbf{\texttt{["A":0, "B":1, "C":2, "D":3, "E":4]}}. La fonction native \textbf{\texttt{chr()}} renvoie le caractère correspondant au code ASCII donné.
    \begin{lstlisting}[language=Python  , xleftmargin=2em, xrightmargin=2em]
chr(65) # renvoie A
chr(66) # renvoie B
\end{lstlisting}
\end{exo}
\begin{exo}
    Construire par compréhension un tableau de dix entiers aléatoires compris entre 0 et 100. Il sera nécessaire d'utiliser la bibliothèque \textbf{\texttt{random}}.
\end{exo}
\subsection{Utilisation de structure de données}
\begin{exo}
    \begin{enumerate}
        \item Construire par compréhension un tableau de dix entiers aléatoires compris entre 0 et 100.
        \item Écrire la fonction \textbf{\texttt{maxi(tab: list) $\rightarrow$ int}} qui renvoie le plus grand élément du tableau.
    \end{enumerate}
\end{exo}
\begin{exo}
    \begin{enumerate}
        \item Construire par compréhension un tuple de dix entiers aléatoires compris entre 0 et 1000.
        \item Écrire la fonction \textbf{\texttt{somme(tup: tuple) $\rightarrow$ int}} qui renvoie la somme de tous les entiers du tuple.
    \end{enumerate}
\end{exo}

\begin{exo}
    \begin{enumerate}
    \item Construire un tableau de 6 mots.
    \item Demander deux indices i et j à l'utilisateur.
    \item Échanger les mots aux indices i et j.
    \end{enumerate}
    \end{exo}
\begin{exo}Un livre peut être caractérisé par son titre, son auteur, son éditeur, son prix.
    \begin{enumerate}
        \item Construire un dictionnaire qui contient les informations du livre: \emph{Il était deux fois} de Franck Thilliez aux éditions \emph{Poche} à 8,70€.
        \item Construire un dictionnaire pour \emph{Fahrenheit 451} de Ray Bradbury aux éditions \emph{Folio} à 6,30€.
        \item Construire un tableau contenant les deux dictionnaires. Ajouter au moins un autre livre.
        \item Écrire une boucle qui parcourt le tableau et affiche l'auteur de chaque livre.
    \end{enumerate}
\end{exo}
\begin{exo}
    Écrire la fonction \texttt{\textbf{lettres(mot: str) $\rightarrow$ dict}} qui renvoie un dictionnaire contenant le nombre d'occurrences de chaque lettre de \emph{mot}. Par exemple:
        \begin{lstlisting}[language=Python]
>>> lettres("bonjour")
>>> {"b": 1, "o": 2, "n": 1, "j": 1, "u": 1, "r": 1}
\end{lstlisting}
\end{exo}
\end{document}