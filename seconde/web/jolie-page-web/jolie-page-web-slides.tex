\documentclass[svgnames,11pt]{beamer}
\input{/home/tof/Documents/Cozy/latex-include/preambule_commun.tex}
\input{/home/tof/Documents/Cozy/latex-include/preambule_beamer.tex}
%\usepackage{pgfpages} \setbeameroption{show notes on second screen=left}
\author[]{Christophe Viroulaud}
\title{Jolie page web}
\date{\framebox{\textbf{Web 02}}}
%\logo{}
\institute{Seconde - SNT}

\begin{document}
\begin{frame}
\titlepage
\end{frame}
\begin{frame}
    \frametitle{}

    Le rôle du langage \textbf{\texttt{HTML}} est d'afficher un contenu. Mais l'esthétique de la page pourrait être amélioré.

\end{frame}
\begin{frame}
    \frametitle{}

    \begin{framed}
        \centering Comment gérer l'esthétique d'une page web?
    \end{framed}

\end{frame}
\section{Langage CSS - présentation}
\begin{frame}[fragile]
    \frametitle{Langage CSS - Présentation}

    \begin{aretenir}[]
        Le \textbf{Cascading Style Sheets} (feuille de style en cascades) est un langage qui gère la mise en forme du contenu \emph{html}.
    \end{aretenir}
\begin{activite}
\begin{enumerate}
    \item Dans le même dossier que le fichier \emph{html}, créer le fichier \textbf{\texttt{style.css}}
    \item Ajouter la ligne suivante dans le bloc \textbf{\texttt{head}}:
\begin{lstlisting}[language=html , basicstyle=\ttfamily\small, xleftmargin=0.2em, xrightmargin=-1em]
<head>
    <link rel="stylesheet" href="style.css">
</head>
\end{lstlisting}
\end{enumerate}
\end{activite}
\end{frame}
\section{Styliser le contenu des balises}
\begin{frame}[fragile]
    \frametitle{Styliser le contenu des balises}

    Les feuilles \emph{css} appliquent des règles de mise en forme aux blocs \textbf{\texttt{html}} ciblés. Par exemple, le code \ref{titre1} change la couleur des titres \textbf{\texttt{h1}} en rouge.

\begin{center}
\begin{lstlisting}[language=HTML , basicstyle=\ttfamily\small, xleftmargin=2em, xrightmargin=2em]
h1 {
    color: red;
}
\end{lstlisting}
\captionof{code}{Styliser les titres \texttt{\textbf{h1}}}
\label{titre1}
\end{center}

\end{frame}
\begin{frame}[fragile]
    \frametitle{}

    \begin{activite}
    \begin{enumerate}
        \item Observer et comprendre les règles du code \ref{body}.
        \begin{center}
            \begin{lstlisting}[language=HTML , basicstyle=\ttfamily\small, xleftmargin=2em, xrightmargin=2em]
body {
    width: 800px;
    margin: auto;
    background-color: #00ff00;
}
\end{lstlisting}
\captionof{code}{Style de \textbf{\texttt{body}}}
\label{body}
    \end{center}
\item Ajouter ce code dans le fichier \textbf{\texttt{style.css}} puis afficher la page \textbf{\texttt{html}} dans un navigateur.
\item Ajouter une règle de style pour afficher le contenu des \emph{paragraphes} en bleu.
    \end{enumerate}
    \end{activite}

\end{frame}
\begin{frame}[fragile]
    \frametitle{Correction}

    \begin{center}
\begin{lstlisting}[language=HTML , basicstyle=\ttfamily\small, xleftmargin=2em, xrightmargin=2em]
p {
    color: blue;
}
\end{lstlisting}
\captionof{code}{Styliser les paragraphes}
\end{center}  

\end{frame}
\section{Nombreuses modifications}
\begin{frame}
    \frametitle{Nombreuses modifications}

Il est possible de styliser les pages web très finement. Il existe une grande quantité de mots-clés en \textbf{\texttt{css}}.
\begin{activite}
\begin{enumerate}
    \item Ouvrir la page \url{https://tinyurl.com/reglecss}
    \item Observer le rôle des mots: \textbf{\texttt{text-align, font-family, background-color, background-image, text-shadow, list-style}}
    \item Mettre en application certaines règles.
\end{enumerate}
\end{activite}

\end{frame}
\section{Cibler des blocs précis}
\begin{frame}
    \frametitle{Cibler des blocs précis}

    \begin{aretenir}[]
        Il est possible de styliser différemment plusieurs blocs d'une même catégorie. Il faut pour cela identifier les blocs dans la page \textbf{\texttt{html}}.
    \end{aretenir}
\end{frame}
\begin{frame}[fragile]
    \frametitle{}

    
\begin{activite}
    \begin{enumerate}
        \item \underline{Dans le fichier html,} ajouter un attribut à un paragraphe (code \ref{pid}).
        \begin{center}
            \begin{lstlisting}[language=HTML , basicstyle=\ttfamily\small, xleftmargin=1em, xrightmargin=0em]
<p id="monidentifiant" >blablablabl...</p>
\end{lstlisting}
\captionof{code}{Attribut \textbf{\texttt{id}}}
\label{pid}
\end{center}
\item \underline{Dans le fichier css,} ajouter une règle (code \ref{cssid}).
\begin{center}
    \begin{lstlisting}[language=HTML , basicstyle=\ttfamily\small, xleftmargin=1em, xrightmargin=0em]
#monidentifiant {
    color: green;
}
\end{lstlisting}
\captionof{code}{Style pour un bloc identifié \textbf{\texttt{monidentifiant}}}
\label{cssid}
\end{center}
    \end{enumerate}
    \end{activite}

\end{frame}
\section{Paramétrer une image}
\begin{frame}[fragile]
    \frametitle{Paramétrer une image}

    Pour centrer une image il faut utiliser la règle \textbf{\texttt{display}}:

\begin{center}
    \begin{lstlisting}[language=HTML , basicstyle=\ttfamily\small, xleftmargin=1em, xrightmargin=0em]
<img id="monimage" src="logo.jpg">
\end{lstlisting}
\captionof{code}{Dans le fichier \textbf{\texttt{html}}}
\end{center}
\begin{center}
    \begin{lstlisting}[language=HTML , basicstyle=\ttfamily\small, xleftmargin=1em, xrightmargin=0em]
#monimage{
    display: block;
    margin: auto;
    height: 200px;
}
\end{lstlisting}
\captionof{code}{Dans le fichier \textbf{\texttt{css}}}
\end{center}
\end{frame}
\section{Mise en application}
\begin{frame}
    \frametitle{Mise en application}

\begin{activite}
Styliser la page web sur l'orientation. Il faudra ajouter les règles suivantes:
\begin{itemize}
    \item colorer le fond de la page,
    \item centrer et styliser le titre de la page,
    \item styliser les listes,
    \item styliser différemment au moins deux paragraphes,
    \item ajouter un lien sur une image,
\end{itemize}
\end{activite}

\end{frame}
\end{document}