\documentclass[a4paper,11pt]{article}
\input{/home/tof/Documents/Cozy/latex-include/preambule_lua.tex}
\newcommand{\showprof}{show them}  % comment this line if you don't want to see todo environment
\fancyhead[L]{Paradigme fonctionnel - exercices}
\newdate{madate}{10}{09}{2020}
\fancyhead[R]{Terminale - NSI} %\today
\fancyfoot[L]{~\\Christophe Viroulaud}
\fancyfoot[C]{\textbf{Page \thepage}}
\fancyfoot[R]{\includegraphics[width=2cm,align=t]{/home/tof/Documents/Cozy/latex-include/cc.png}}

\begin{document}
\begin{Form}
\begin{exo}
\begin{enumerate}
\item Écrire une fonction \textbf{applique(f, t: tuple)\;\rightarrow\;tuple} qui applique la fonction \emph{f} sur tous les éléments du tuple \emph{t} et renvoie le nouvel ensemble.
\item Écrire une fonction \textbf{double(x: int)\;\rightarrow\;int} qui renvoie le double de \emph{x}.
\item Tester \emph{applique} avec la fonction \emph{double} et un tuple d'entiers.
\end{enumerate}
\end{exo}
\begin{exo}
\begin{enumerate}
\item Écrire une fonction \textbf{repeter\_delai(f, n: int, t: int)\;\rightarrow\;None} qui effectue n appels \emph{f(0,...,f(n-1)} où chaque appel est suivi d'une pause de \emph{t} secondes. La documentation Python ci-après peut être utile pour effectuer la pause: \url{https://docs.python.org/fr/3/library/time.html}
\item Écrire une fonction \textbf{dessine(n: int)\;\rightarrow\;None} qui trace un trait de longueur $n*50$ avec la bibliothèque \emph{turtle} puis qui tourne à gauche d'un angle de 90°.
\item Tester \emph{repeter\_delai} avec la fonction \emph{dessine}.
\end{enumerate}
\end{exo}
\begin{exo}
\begin{enumerate}
\item Écrire une fonction \textbf{trouve(p, t: tuple)\;\rightarrow\;object} qui reçoit une fonction \emph{p} et un tuple \emph{t} et renvoie le premier élément \emph{x} de \emph{t} tel que $p(x)=True$. Si aucun élément ne satisfait p alors la fonction renvoie \emph{None}.
\item Écrire une fonction \textbf{est\_positif(x: int)\;\rightarrow\;bool} qui renvoie True si \emph{x} est positif. \item Utiliser la fonction \emph{trouve} sur un tuple d'entiers et la fonction \emph{est\_positif}.
\item Écrire une fonction \textbf{est\_premier(n: int)\;\rightarrow\;bool} qui renvoie True si \emph{n} est un nombre premier.
\item Utiliser la fonction \emph{trouve} sur un tuple d'entiers et la fonction \emph{est\_premier}.
\end{enumerate}
\end{exo}
\begin{exo}
Notion d'invariant
\begin{lstlisting}
def tri_insertion(tab):
    taille = len(tab)
    for i in range(1,taille):
        en_cours = tab[i]
        j = i-1
        while j >= 0 and tab[j] > en_cours:
            tab[j+1] = tab[j]
            j -= 1
        tab[j+1] = en_cours
    return tab
\end{lstlisting}
L'invariant du tri par insertion peut être: \guill{Le tableau tab[0:i] est trié.}
\begin{enumerate}
\item Rappeler la définition d'un invariant. À quoi sert-il?
\item Quelle est la valeur de \emph{i} avant la première itération du tri? L'invariant est-il vérifié?
\item Appliquer un raisonnement par récurrence pour montrer que l'invariant est vrai pour tout \emph{i}. Conclure.
\end{enumerate}
\end{exo}
\begin{exo}
L'expression Python \textbf{lambda} crée une fonction anonyme
(\url{https://docs.python.org/fr/3/reference/expressions.html?#lambda}).
\begin{enumerate}
\item Dans la console, tester le code ci-après:
\begin{lstlisting}
f = lambda x: x*2
f(5)
\end{lstlisting}
\item Écrire une fonction \emph{g} telle que pour tout x, $g(x) = 2x+3$.
\item Écrire une fonction \textbf{h(f,g)} qui reçoit deux fonctions \emph{f} et \emph{g} en arguments et renvoie leur composition, c'est à dire une fonction \emph{h} telle que pour tout x, $h(x)=f(g(x))$.
\item Tester \emph{h(5)}.
\end{enumerate}
\end{exo}
\begin{exo}
\begin{enumerate}
\item Écrire une fonction \textbf{calcul(operation, l: tuple)\;\rightarrow\;int} qui effectue le calcul suivant:
$$t[0]\;operation\;t[1]\;operation\;...\;operation\;t[n-1]$$
\item Écrire une fonction anonyme \textbf{addition} qui additionne deux entiers \emph{x} et \emph{y}.
\item Utiliser la fonction \emph{calcul} avec un tuple d'entiers et la fonction \emph{addition}.
\item Tester la fonction \emph{calcul} avec d'autres opérations.
\end{enumerate}
\end{exo}
\end{Form}
\end{document}