\documentclass[a4paper,11pt]{article}
\input{/home/tof/Documents/Cozy/latex-include/preambule_doc.tex}
\input{/home/tof/Documents/Cozy/latex-include/preambule_commun.tex}
\newcommand{\showprof}{show them}  % comment this line if you don't want to see todo environment
\setlength{\fboxrule}{0.8pt}
\fancyhead[L]{\fbox{\Large{\textbf{ResSoc 02}}}}
\fancyhead[C]{\textbf{Dopamine}}
\newdate{madate}{10}{09}{2020}
%\fancyhead[R]{\displaydate{madate}} %\today
\fancyhead[R]{Seconde - SNT}
%\fancyhead[R]{Première - NSI}
%\fancyhead[R]{Terminale - NSI}
\fancyfoot[L]{\vspace{1mm}Christophe Viroulaud}
\AtEndDocument{\label{lastpage}}
\fancyfoot[C]{\textbf{Page \thepage/\pageref{lastpage}}}
\fancyfoot[R]{\includegraphics[width=2cm,align=t]{/home/tof/Documents/Cozy/latex-include/cc.png}}

\begin{document}
\section{Problématique}
Quand on est présent sur un réseau social, il est parfois difficile de s'obliger à s'en détacher. Les créateurs de ces réseaux ont pour objectif de garder le plus longtemps possible leurs clients sur la plate-forme.
\begin{center}
    \framebox{Quels mécanismes nous incitent à rester connecté en permanence?}
\end{center}
\section{Débat}
Des journalistes ont observé les principes de construction d'un réseau social et leurs influences sur le cerveau. La série de courtes vidéos \emph{Dopamine} présente leurs analyses.
\begin{activite}
\begin{enumerate}
    \item Se rendre sur la page \url{https://www.arte.tv/fr/videos/RC-017841/dopamine/}
    \item Regarder -au moins- les vidéos:
    \begin{itemize}
        \item Youtube,
        \item Facebook,
        \item Snapchat,
        \item Twitter,
        \item Instagram.
    \end{itemize}
    \item Pour trois réseaux parmi les cinq vus, établir une liste \textbf{d'arguments négatifs} du réseau.
    \item Pour trois réseaux parmi les cinq vus (pas forcément les mêmes que ceux de la question précédente), établir une liste \textbf{d'arguments positifs} du réseau.
\end{enumerate}
Ces arguments doivent être présentés sur une feuille ou un document texte et envoyés par mail au professeur. Il est possible de trouver des idées autres que celles présentées dans les vidéos. Ils seront ensuite utilisés lors d'un débat organisé en classe.
\end{activite}
\end{document}