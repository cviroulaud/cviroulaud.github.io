\documentclass[a4paper,11pt]{article}
\input{/home/tof/Documents/Cozy/latex-include/preambule_lua.tex}
\newcommand{\showprof}{show them}  % comment this line if you don't want to see todo environment
\fancyhead[L]{Système d'exploitation - principe}
\newdate{madate}{10}{09}{2020}
\fancyhead[R]{Première - NSI} %\today
\fancyfoot[L]{~\\Christophe Viroulaud}
\fancyfoot[C]{\textbf{Page \thepage}}
\fancyfoot[R]{\includegraphics[width=2cm,align=t]{/home/tof/Documents/Cozy/latex-include/cc.png}}

\begin{document}
\begin{Form}
\section{Problématique}
Travailler en langage machine (\emph{assembleur}) devient rapidement fastidieux. Gérer les appels mémoires, les registres est une tâche qui peut engendrer de nombreuses erreurs de programmation. De plus le langage machine dépend du processeur utilisé. Il faudrait donc créer une version d'un même programme pour chaque microprocesseur.
\begin{center}
\shadowbox{\parbox{12cm}{\centering Comment gérer les accès matériels de manière transparente?}}
\end{center}
\begin{commentprof}
\section*{Contexte historique}

\end{commentprof}
\section{Principe du système d'exploitation}
\begin{activite}
Regarder la vidéo à l'adresse suivante puis répondre aux questions
\begin{center}
\url{https://www.youtube.com/watch?v=SpCP2oaCx8A}
\end{center}
\begin{enumerate}
\item
\end{enumerate}
\end{activite}
\section{Le système UNIX}
\begin{activite}
Regarder la frise chronologique à l'adresse suivante puis répondre aux questions
\begin{center}
\url{https://cdn.knightlab.com/libs/timeline3/latest/embed/index.html?source=1aroW_4ry_iAT3EiVA8aiR6DhcdzIwMU2uymPzkb7UAs&font=Default&lang=fr&initial_zoom=2&height=650}
\end{center}
\begin{enumerate}
\item
\end{enumerate}
\end{activite}
\end{Form}
\end{document}