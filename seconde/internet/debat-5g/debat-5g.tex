\documentclass[a4paper,11pt]{article}
\input{/home/tof/Documents/Cozy/latex-include/preambule_lua.tex}
\newcommand{\showprof}{show them}  % comment this line if you don't want to see todo environment
\fancyhead[L]{La 5G}
\newdate{madate}{10}{09}{2020}
\fancyhead[R]{Seconde - SNT} %\today
\fancyfoot[L]{~\\Christophe Viroulaud}
\fancyfoot[C]{\textbf{Page \thepage}}
\fancyfoot[R]{\includegraphics[width=2cm,align=t]{/home/tof/Documents/Cozy/latex-include/cc.png}}

\begin{document}
\begin{Form}
\paragraph{Objectif:} Réaliser un débat entre les pro et anti 5G.\\
\section{Modalités}
Trois rôles seront répartis:
\begin{itemize}
\item \textbf{Le jury} présentera la 5G de manière neutre (\emph{maximum 3 minutes}) puis écoutera les arguments des deux groupes. Il votera pour celui qui lui a paru le plus convaincant et argumentera son choix.
\item \textbf{les pro-5G} sont pour ce nouveau réseau et défendront leur point de vue. 
\item \textbf{les anti-5G} sont contre ce nouveau réseau et défendront leur point de vue.
\end{itemize}
Chaque groupe pourra préparer et utiliser des documents (audios, vidéos, images...) pendant le débat.
\section{Évaluation}
La grille d'évaluation est présentée page suivante. La note donnée pour chaque item respectera le codage ci-après:
\begin{itemize}
\item 1: le groupe ne répond pas du tout au critère.
\item 2: le groupe ne répond que très partiellement au critère et a encore une marge de progression importante.
\item 3: le groupe répond au critère mais certaines améliorations sont possibles
\item 4: le groupe répond complètement au critère et dépasse même les attendus.
\end{itemize}
Chaque élève qui ne participe pas au débat remplira une grille en fonction de son ressenti mais également de l'avis du jury.
\renewcommand{\arraystretch}{1.2}

\begin{commentprof}
\section{Évaluation du groupe jury}
\begin{center}
\begin{tabular}{|c|p{10cm}|c|}
\hline 
groupe jury: &  & notes \\ 
\hline 
\multicolumn{3}{|c|}{PRÉSENTATION}   \\ 
\hline 
\multicolumn{2}{|c|}{Tout le groupe a participé.} &  \\
\hline 
\multicolumn{2}{|c|}{La prise de parole est maîtrisée: facilité de communiquer, vocabulaire adapté.} &  \\
\hline 
\multicolumn{3}{|c|}{CONTENU}   \\ 
\hline 
\multicolumn{2}{|c|}{La présentation ne prend pas parti.} &  \\
\hline 
\multicolumn{2}{|c|}{L’argumentation du choix est justifiée.} &  \\ 
\hline 
\end{tabular}
\end{center}
\begin{center}

\begin{tabular}{|c|p{10cm}|c|}
\hline 
groupe jury: &  & notes \\ 
\hline 
\multicolumn{3}{|c|}{PRÉSENTATION}   \\ 
\hline 
\multicolumn{2}{|c|}{Tout le groupe a participé.} &  \\
\hline 
\multicolumn{2}{|c|}{La prise de parole est maîtrisée: facilité de communiquer, vocabulaire adapté.} &  \\
\hline 
\multicolumn{3}{|c|}{CONTENU}   \\ 
\hline 
\multicolumn{2}{|c|}{La présentation ne prend pas parti.} &  \\
\hline 
\multicolumn{2}{|c|}{L’argumentation du choix est justifiée.} &  \\ 
\hline 
\end{tabular}
\end{center}
\end{commentprof}

\pagebreak
\renewcommand{\arraystretch}{1.4}

\section*{Débat 1}
\begin{center}
\begin{tabular}{|c|p{10cm}|c|}
\hline 
groupe pro-5G: &  & notes \\ 
\hline 
\multicolumn{3}{|c|}{PRÉSENTATION}   \\ 
\hline 
\multicolumn{2}{|c|}{Tout le groupe a participé.} &  \\
\hline 
\multicolumn{2}{|c|}{La prise de parole est maîtrisée: facilité de communiquer, vocabulaire adapté.} &  \\
\hline 
\multicolumn{3}{|c|}{CONTENU}   \\ 
\hline 
\multicolumn{2}{|c|}{Les arguments sont valables et vérifiés.} &  \\
\hline 
\multicolumn{2}{|c|}{L’argumentation est riche (vocabulaire spécifique, explication précise).} &  \\ 
\hline 
\multicolumn{2}{|c|}{Le groupe est capable de répondre aux arguments des adversaires.} &  \\ 
\hline 
\end{tabular}
\end{center}

\begin{center}
\begin{tabular}{|c|p{10cm}|c|}
\hline 
groupe anti-5G: &  & notes \\ 
\hline 
\multicolumn{3}{|c|}{PRÉSENTATION}   \\ 
\hline 
\multicolumn{2}{|c|}{Tout le groupe a participé.} &  \\
\hline 
\multicolumn{2}{|c|}{La prise de parole est maîtrisée: facilité de communiquer, vocabulaire adapté.} &  \\
\hline 
\multicolumn{3}{|c|}{CONTENU}   \\ 
\hline 
\multicolumn{2}{|c|}{Les arguments sont valables et vérifiés.} &  \\
\hline 
\multicolumn{2}{|c|}{L’argumentation est riche (vocabulaire spécifique, explication précise).} &  \\ 
\hline 
\multicolumn{2}{|c|}{Le groupe est capable de répondre aux arguments des adversaires.} &  \\ 
\hline 
\end{tabular}
\end{center}

\section*{Débat 2}
\begin{center}
\begin{tabular}{|c|p{10cm}|c|}
\hline 
groupe pro-5G: &  & notes \\ 
\hline 
\multicolumn{3}{|c|}{PRÉSENTATION}   \\ 
\hline 
\multicolumn{2}{|c|}{Tout le groupe a participé.} &  \\
\hline 
\multicolumn{2}{|c|}{La prise de parole est maîtrisée: facilité de communiquer, vocabulaire adapté.} &  \\
\hline 
\multicolumn{3}{|c|}{CONTENU}   \\ 
\hline 
\multicolumn{2}{|c|}{Les arguments sont valables et vérifiés.} &  \\
\hline 
\multicolumn{2}{|c|}{L’argumentation est riche (vocabulaire spécifique, explication précise).} &  \\ 
\hline 
\multicolumn{2}{|c|}{Le groupe est capable de répondre aux arguments des adversaires.} &  \\ 
\hline 
\end{tabular}
\end{center}

\begin{center}
\begin{tabular}{|c|p{10cm}|c|}
\hline 
groupe anti-5G: &  & notes \\ 
\hline 
\multicolumn{3}{|c|}{PRÉSENTATION}   \\ 
\hline 
\multicolumn{2}{|c|}{Tout le groupe a participé.} &  \\
\hline 
\multicolumn{2}{|c|}{La prise de parole est maîtrisée: facilité de communiquer, vocabulaire adapté.} &  \\
\hline 
\multicolumn{3}{|c|}{CONTENU}   \\ 
\hline 
\multicolumn{2}{|c|}{Les arguments sont valables et vérifiés.} &  \\
\hline 
\multicolumn{2}{|c|}{L’argumentation est riche (vocabulaire spécifique, explication précise).} &  \\ 
\hline 
\multicolumn{2}{|c|}{Le groupe est capable de répondre aux arguments des adversaires.} &  \\ 
\hline 
\end{tabular}
\end{center}
\end{Form}
\end{document}