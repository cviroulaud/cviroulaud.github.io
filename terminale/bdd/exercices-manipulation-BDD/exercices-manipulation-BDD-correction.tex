\documentclass[a4paper,11pt]{article}
\input{/home/tof/Documents/Cozy/latex-include/preambule_lua.tex}
\newcommand{\showprof}{show them}  % comment this line if you don't want to see todo environment
\fancyhead[L]{Correction exercices - manipulation de base de données}
\newdate{madate}{10}{09}{2020}
%\fancyhead[R]{\displaydate{madate}} %\today
%\fancyhead[R]{Seconde - SNT}
%\fancyhead[R]{Première - NSI}
\fancyhead[R]{Terminale - NSI}
\fancyfoot[L]{~\\Christophe Viroulaud}
\AtEndDocument{\label{lastpage}}
\fancyfoot[C]{\textbf{Page \thepage/\pageref{lastpage}}}
\fancyfoot[R]{\includegraphics[width=2cm,align=t]{/home/tof/Documents/Cozy/latex-include/cc.png}}

\begin{document}
\begin{Form}
\begin{exo}
\begin{itemize}
\item colonne, column, attribut
\item entité, ligne, row
\item domaine, type
\item relation, table
\item schéma (description d'une relation)
\item base de données (ensemble des relations)
\end{itemize}
\end{exo}
\begin{exo}
\begin{enumerate}
\item \begin{itemize}
\item Especes(\underline{id Integer}, nom String)
\item Animaux(\underline{id Integer}, nom String, age Integer, \dashuline{id\_espece Integer})
\item Soins(\underline{id Integer}, \dashuline{id\_animal Integer}, soin String)
\end{itemize}
\item Dans un souci de lisibilité il est judicieux d'écrire un attribut par ligne.
\begin{center}
\begin{lstlisting}[language=SQL]
CREATE TABLE Especes (
		id Integer PRIMARY KEY AUTOINCREMENT, 
		nom String);

CREATE TABLE Animaux (
		id Integer PRIMARY KEY AUTOINCREMENT, 
		nom String, 
		age Integer, 
		id_espece Integer, 
		FOREIGN KEY (id_espece) REFERENCES Especes(id));

CREATE TABLE Soins (
	id Integer PRIMARY KEY AUTOINCREMENT, 
	id_animal Integer, 
	soin String,
	FOREIGN KEY (id_animal) REFERENCES Animaux(id));
\end{lstlisting}
\captionof{code}{Création des trois tables}
\label{moncode}
\end{center}
\end{enumerate}
\end{exo}
\begin{exo}
\begin{enumerate}
\item Dans le dossier compressé.
\item \url{https://fr.wikipedia.org/wiki/Soundex}.
\item Les codes:
\begin{center}
\begin{lstlisting}[language=SQL]
SELECT * FROM Departements WHERE departement_code = 24;

SELECT * FROM Departements WHERE departement_nom_soundex = 'M200';

SELECT departement_nom FROM Departements WHERE departement_code < 10;

SELECT departement_code, departement_nom FROM Departements WHERE departement_code > 20 AND departement_code < 30;

-- SQLite LIKE operator is case-insensitive. It means "A" LIKE "a" is true.
-- However, for Unicode characters that are not in the ASCII ranges, the LIKE operator is case sensitive e.g., "Ä" LIKE "ä" is false
SELECT departement_nom FROM Departements WHERE departement_nom LIKE '%haut%';

SELECT departement_nom FROM Departements WHERE departement_nom NOT LIKE '%-%' AND departement_nom NOT LIKE '% %';
\end{lstlisting}
\captionof{code}{Sélections des départements}
\label{moncode}
\end{center}
\begin{aretenir}[Remarque]
Il est possible de comparer des \emph{String} comme des \emph{Integer}. Le SQL est très permissif: departement\_code est de type \emph{String}, pourtant il accepte la comparaison avec un \emph{Integer}.
\end{aretenir}
\end{enumerate}.
\end{exo}
\begin{exo}
\begin{center}
\begin{lstlisting}[language=SQL]
SELECT * FROM employees WHERE name = 'GARFIELD';

SELECT name FROM employees WHERE designation = 'MANAGER';

SELECT name FROM employees WHERE name LIKE 'H%';

SELECT name FROM employees WHERE hired_on > '1997-01-01';

SELECT name, salary FROM employees WHERE salary > 25000 AND salary < 55000;

SELECT name, salary FROM employees WHERE salary > 25000 AND salary < 55000 AND commission IS NOT NULL;
-- ou bien; NULL est différent de 0; commission = 0 ne renverrait rien ici
SELECT name, salary FROM employees WHERE salary > 25000 AND salary < 55000 AND commission > 0;

INSERT INTO employees (name, designation, manager, hired_on, salary, dept) VALUES ('DURAN', 'TECH', 6, '1999-01-13', 35000, 4);

UPDATE employees SET salary = 60000 WHERE name = 'FILLMORE';
\end{lstlisting}
\captionof{code}{Sélections des employés}
\label{moncode}
\end{center}
\end{exo}
\end{Form}
\end{document}