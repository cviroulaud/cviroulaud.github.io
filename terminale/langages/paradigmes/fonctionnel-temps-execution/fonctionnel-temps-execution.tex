\documentclass[a4paper,11pt]{article}
\input{/home/tof/Documents/Cozy/latex-include/preambule_lua.tex}
\newcommand{\showprof}{show them}  % comment this line if you don't want to see todo environment
\fancyhead[L]{Paradigme fonctionnel}
\newdate{madate}{10}{09}{2020}
\fancyhead[R]{Terminale - NSI} %\today
\fancyfoot[L]{~\\Christophe Viroulaud}
\fancyfoot[C]{\textbf{Page \thepage}}
\fancyfoot[R]{\includegraphics[width=2cm,align=t]{/home/tof/Documents/Cozy/latex-include/cc.png}}

\begin{document}
\begin{Form}
\begin{commentprof}
Donner page 1 seule d'abord
\end{commentprof}
\paragraph{Objectif:}Découvrir un nouveau paradigme de programmation.
\section{Problématique}
Le \emph{tri} est une fonctionnalité très utilisée en programmation. Il est donc primordial qu'il soit le plus efficace possible.
\begin{center}
\shadowbox{\parbox{12cm}{\centering Peut-on concevoir un outil de comparaison des algorithmes de tri?}}
\end{center}
\section{Algorithmes de tri}
\begin{activite}
\begin{enumerate}
\item Choisir un des trois algorithmes de tri ci-après et implémenter une fonction qui accepte une liste et renvoie cette dernière triée.
\item Construire (en compréhension) une liste \textbf{tab} de 1000 entiers compris entre 0 et 100.
\item Par groupe de trois ayant choisis un algorithme différent, partager puis présenter les codes.
\end{enumerate}
\end{activite}
\begin{commentprof}
\begin{itemize}
\item utiliser jupyter?
\item docstring (+doctest?)
\item amélioration tri à bulles possible: Si lors du parcours de la liste une permutation au moins a été effectuée, recommencer un nouveau parcours.
\item invariant? en exercice
\end{itemize}
\end{commentprof}
\subsection{Tri par sélection}
En partant du début du tableau et pour chaque élément de rang \emph{n}:
\begin{itemize}
\item Rechercher la plus petite valeur et la permuter avec l'élément de rang \emph{n}.
\end{itemize}
\subsection{Tri par insertion}
En partant du début du tableau et pour chaque élément de rang \emph{n}:
\begin{itemize}
\item Mémoriser l'élément de rang \emph{n}.
\item En partant de l'élément \emph{n-1}, décaler vers la droite les éléments qui sont plus grands que celui mémorisé.
\item Placer dans le trou l'élément mémorisé.
\end{itemize}
\subsection{Tri à bulles}
En partant du début du tableau et pour chaque élément de rang \emph{n}:
\begin{itemize}
\item Parcourir le tableau en comparant chaque élément avec son successeur.
\item Si ce dernier est le plus petit des deux, les permuter.
\end{itemize}
\pagebreak
\section{Notion de complexité}
\label{complexite}
Il est légitime de se demander si certains algorithmes sont plus rapides que d'autres. La première question à se poser est de déterminer quels sont les paramètres qui influencent la durée d'exécution.\\
Le tri par sélection ci-après est composé de deux boucles. Chaque boucle dépend de la taille (notée \emph{n}) du tableau à trier. La durée d'exécution dépend donc du facteur $n×n$.
\begin{center}
\shadowbox{\parbox{10cm}{\centering Le tri par sélection a une \textbf{complexité en temps} en $O(n^2).$}}
\end{center}
\begin{lstlisting}
def tri_selection(tab):
    taille = len(tab)
    for i in range(taille):
        rang_mini = i
        for j in range (i+1,taille):
            if tab[j] < tab[rang_mini]:
                rang_mini = j
        tab[i],tab[rang_mini] = tab[rang_mini],tab[i]
    return tab
\end{lstlisting}
\begin{commentprof}
\begin{itemize}
\item Évoquer complexité en espace
\item tri bulles/insertion: meilleur des cas = (n-1) comparaisons, linéaire. Pire des cas: tableau trié à l'envers, quadratique $\dfrac{n.(n-1)}{2}$ comparaisons
\item évoquer stabilité (insertion et bulles) $\leftarrow$ on reverra en exo
\end{itemize}
\end{commentprof}
\begin{activite}
Déterminer la complexité en temps des algorithmes de tri à bulles et tri par insertion.
\end{activite}
\section{Comparaison des temps d'exécution}
\subsection{Contexte}
Nous désirons comparer la durée d'exécution des trois fonctions de tri précédentes. Nous pouvons déjà remarquer que ces fonctions implémentent le même schéma, à savoir elles:
\begin{itemize}
\item acceptent une liste en entrée,
\item renvoient la liste triée en sortie.
\end{itemize}
\medskip
La bibliothèque \emph{time} propose la fonction:
\begin{framed}
time.time() → float
\\Renvoie le temps en secondes depuis \emph{epoch} sous forme de nombre à virgule flottante. 
\end{framed}
\begin{commentprof}
Pour Unix, epoch est le 1er janvier 1970 à 00:00:00 (UTC: Universel Temps Coordonné). 
\end{commentprof}
\subsection{Paradigme fonctionnel: une fonction est une donnée comme une autre}
\begin{commentprof}
\section*{Contexte historique}
$\lambda-calcul$ de Church / machine de Turing = principe équivalent qui permet de montrer calculabilité (cherche à identifier la classe des fonctions qui peuvent être calculées à l'aide d'un algorithme)\\
Quand von Neumann décrit son architecture d'un ordinateur, un des plus grands bouleversements tient au fait que données et programmes sont stockés dans la mémoire: un programme peut être considéré comme une donnée.\\
Ocaml, Scheme. langages sans effet de bord = plus facile à maintenir.\\
LISP pour LISt Processing
\end{commentprof}
Un des premiers principes du \emph{paradigme fonctionnel} est de considérer une fonction comme une donnée. Elle peut donc être passée en variable à une autre fonction.
\begin{activite}
\begin{enumerate}
\item Implémenter une fonction \textbf{duree\_tri(fonction, tab: list) $\rightarrow$ float} qui mesure la durée que met \emph{fonction} pour trier la liste \emph{tab} et renvoie cette durée.
\item Tester la fonction \textbf{duree\_tri} avec une fonction de tri et la liste \emph{tab}. 
\end{enumerate}
\end{activite}
\subsection{Paradigme fonctionnel: données immuables}
\subsubsection{Effet de bord}
Une fonction est dite à \emph{effet de bord} si elle modifie un état en dehors de son environnement local.
\begin{activite}
Un enseignant a obtenu des résultats décevants au dernier devoir donné. Il veut tester différentes majorations.
\begin{lstlisting}
notes = [7,12,8,5,19,10,7,6,1,15,13,8]

def majoration(bonus: int):
    for i in range(len(notes)):
        if notes[i] <= 8:
            notes[i] += bonus
\end{lstlisting}
Que devient la liste \emph{notes} après les deux appels suivants?
\begin{lstlisting}
majoration(2)
majoration(3)
\end{lstlisting}
\end{activite}
\subsubsection{Tuple}
Afin de s'affranchir des effets de bord, la programmation fonctionnel n'utilise que des données non mutables. En Python les \emph{tuples} possèdent ces caractéristiques.
\begin{activite}
\begin{enumerate}
\item Construire (en compréhension) un tuple \textbf{tab\_immuable} de 1000 entiers compris entre 0 et 100.
\item Implémenter une fonction \textbf{duree\_tri\_fonctionnel(fonction, tab: tuple) $\rightarrow$ float} qui mesure la durée que met \emph{fonction} pour trier \emph{tab} et renvoie cette durée.
\end{enumerate}
\end{activite}
\begin{commentprof}
\begin{itemize}
\item pour tuple en compréhension, voir slides: il faut ajouter \emph{tuple} sinon on crée un générateur
\item list(t) convertit tuple t en liste
\end{itemize}
\end{commentprof}
\subsection{Évolution du temps d'exécution}
\begin{commentprof}
À faire en devoir?
\end{commentprof}
Le module \emph{pyplot} de la bibliothèque \emph{matplotlib} permet de réaliser des représentations graphiques facilement. Il est commun de l'importer avec un alias:
\begin{lstlisting}
import matplotlib.pyplot as plt
\end{lstlisting}
Le site ci-après présente un rapide tutoriel permettant de tracer la courbe représentative d'une fonction:
\begin{center}
\begin{commentprof}
\url{https://zestedesavoir.com/tutoriels/469/introduction-aux-graphiques-en-python-avec-matplotlib-pyplot/}
\end{commentprof}
\url{https://vu.fr/Pqyi}
\end{center}
\begin{activite}
\begin{enumerate}
\item Réaliser une représentation graphique de la durée d'exécution d'un des tris étudiés, pour des tailles de listes variant de 10 à 5000 items. Il est conseillé d'utiliser au moins 10 listes pour avoir un résultat significatif.
\item La courbe obtenue confirme-t-elle les résultats du chapitre \ref{complexite}?
\end{enumerate}
\end{activite}
\end{Form}
\end{document}