\documentclass[svgnames,11pt]{beamer}
\input{/home/tof/Documents/Cozy/latex-include/preambule_commun.tex}
\input{/home/tof/Documents/Cozy/latex-include/preambule_beamer.tex}
%\usepackage{pgfpages} \setbeameroption{show notes on second screen=left}
\author[]{Christophe Viroulaud}
\title{Exercices requêtes avancées\\correction}
\date{\framebox{\textbf{BDD 07}}}
%\logo{}
\institute{Terminale - NSI}

\begin{document}
\begin{frame}
\titlepage
\end{frame}
\begin{frame}[fragile]
    \frametitle{}

\begin{center}
\begin{lstlisting}[language=SQL , basicstyle=\ttfamily\small, xleftmargin=1em, xrightmargin=0em]
INSERT INTO Especes(nom) VALUES ("poisson");
\end{lstlisting}
\captionof{code}{Insérer l'espèce \emph{poisson}}
\label{CODE}
\end{center}

\end{frame}
\begin{frame}[fragile]
    \frametitle{}

\begin{center}
\begin{lstlisting}[language=SQL , basicstyle=\ttfamily\small, xleftmargin=1em, xrightmargin=0em]
INSERT INTO Soins(id_animal,soin) VALUES (3,"patte cassée");
\end{lstlisting}
\captionof{code}{Insérer le soin \emph{patte cassée}}
\label{CODE}
\end{center}

\end{frame}
\begin{frame}[fragile]
    \frametitle{}

\begin{center}
\begin{lstlisting}[language=SQL , basicstyle=\ttfamily\small, xleftmargin=1em, xrightmargin=0em]
SELECT * FROM Animaux WHERE age >= 10 ORDER BY nom;
\end{lstlisting}
\captionof{code}{Sélectionner les animaux de plus de 10 ans}
\label{CODE}
\end{center}

\end{frame}

\begin{frame}[fragile]
    \frametitle{}

\begin{center}
\begin{lstlisting}[language=SQL , basicstyle=\ttfamily\small, xleftmargin=1em, xrightmargin=0em]
SELECT DISTINCT(soin) FROM Soins;
\end{lstlisting}
\captionof{code}{Sélectionner les soins distincts}
\label{CODE}
\end{center}

\end{frame} 
\begin{frame}[fragile]
    \frametitle{}

\begin{center}
\begin{lstlisting}[language=SQL , basicstyle=\ttfamily\small, xleftmargin=1em, xrightmargin=0em]
SELECT COUNT(soin) FROM Soins;
\end{lstlisting}
\captionof{code}{Compter les soins}
\label{CODE}
\end{center}

\end{frame} 
\begin{frame}[fragile]
    \frametitle{}

\begin{center}
\begin{lstlisting}[language=SQL , basicstyle=\ttfamily\small, xleftmargin=1em, xrightmargin=0em]
SELECT COUNT(DISTINCT(soin)) FROM Soins;
\end{lstlisting}
\captionof{code}{Compter tous les soins distincts}
\label{CODE}
\end{center}

\end{frame} 
\begin{frame}[fragile]
    \frametitle{}

\begin{center}
\begin{lstlisting}[language=SQL , basicstyle=\ttfamily\small, xleftmargin=1em, xrightmargin=0em]
SELECT COUNT(soin) FROM Soins WHERE soin = "stérilisation";
\end{lstlisting}
\captionof{code}{Compter les stérilisations}
\label{CODE}
\end{center}

\end{frame} 
\begin{frame}[fragile]
    \frametitle{}

\begin{center}
\begin{lstlisting}[language=SQL , basicstyle=\ttfamily\small, xleftmargin=.5em, xrightmargin=0em]
UPDATE Animaux SET nom="charlie" WHERE nom="charly";
\end{lstlisting}
\captionof{code}{Renommer Charly}
\label{CODE}
\end{center}

\end{frame} 
\begin{frame}[fragile]
    \frametitle{}

\begin{center}
\begin{lstlisting}[language=SQL , basicstyle=\ttfamily\small, xleftmargin=.5em, xrightmargin=0em]
SELECT Especes.nom FROM Especes
JOIN Animaux ON Especes.id = Animaux.id_espece
WHERE Animaux.nom="zappy";
\end{lstlisting}
\captionof{code}{Espèce de Zappy}
\label{CODE}
\end{center}

\end{frame} 
\begin{frame}[fragile]
    \frametitle{}

\begin{center}
\begin{lstlisting}[language=SQL , basicstyle=\ttfamily\small, xleftmargin=.5em, xrightmargin=0em]
SELECT Animaux.nom FROM Animaux
JOIN Especes ON Especes.id = Animaux.id_espece
WHERE Especes.nom="chat"
ORDER BY Animaux.nom;
\end{lstlisting}
\captionof{code}{Noms des chats}
\label{CODE}
\end{center}

\end{frame}
\begin{frame}[fragile]
    \frametitle{}

\begin{center}
\begin{lstlisting}[language=SQL , basicstyle=\ttfamily\small, xleftmargin=.5em, xrightmargin=0em]
SELECT COUNT(Animaux.nom) FROM Animaux
JOIN Especes ON Especes.id = Animaux.id_espece
WHERE Especes.nom="chien";
\end{lstlisting}
\captionof{code}{Nombre de chiens}
\label{CODE}
\end{center}

\end{frame}
\begin{frame}[fragile]
    \frametitle{}

\begin{center}
\begin{lstlisting}[language=SQL , basicstyle=\ttfamily\small, xleftmargin=.5em, xrightmargin=0em]
SELECT COUNT(Soins.soin) FROM Soins
JOIN Animaux ON Animaux.id = Soins.id_animal
WHERE Animaux.nom="chouchou";
\end{lstlisting}
\captionof{code}{Soins de \emph{chouchou}}
\label{CODE}
\end{center}

\end{frame}
\begin{frame}[fragile]
    \frametitle{}

\begin{center}
\begin{lstlisting}[language=SQL , basicstyle=\ttfamily\small, xleftmargin=.5em, xrightmargin=0em]
SELECT Animaux.nom AS n FROM Animaux
JOIN Soins ON Animaux.id = Soins.id_animal
JOIN Especes ON Animaux.id_espece = Especes.id
WHERE Especes.nom="chat" AND Soins.soin="stérilisation"
ORDER BY n;
\end{lstlisting}
\captionof{code}{Chats stérilisés}
\label{CODE}
\end{center}

\end{frame}
\begin{frame}[fragile]
    \frametitle{}

\begin{center}
\begin{lstlisting}[language=SQL , basicstyle=\ttfamily\small, xleftmargin=.5em, xrightmargin=0em]
INSERT INTO Animaux(nom,age,id_espece) VALUES ("bubulle",3,(SELECT id FROM Especes WHERE nom="poisson")); 
\end{lstlisting}
\captionof{code}{Insertion \emph{Bubulle}}
\label{CODE}
\end{center}

\end{frame}
\begin{frame}[fragile]
    \frametitle{}

\begin{center}
\begin{lstlisting}[language=SQL , basicstyle=\ttfamily\small, xleftmargin=.5em, xrightmargin=0em]
INSERT INTO Soins(id_animal, soin) VALUES ((SELECT id FROM Animaux WHERE nom="cartman"),"stérilisation");
\end{lstlisting}
\captionof{code}{Soins \emph{Cartman}}
\label{CODE}
\end{center}

\end{frame}
\begin{frame}
    \frametitle{}

Créer une table \emph{Types\_soins} qui référence les différentes possibilités de soins. Il faut alors remplacer l'attribut\emph{soin} de la table \emph{Soins} par une clé étrangère sur l'\emph{id} de \emph{Types\_soins}.

\end{frame}
\begin{frame}
    \frametitle{}

    
\begin{itemize}
    \item Especes(\underline{id Integer}, nom String)
    \item Animaux(\underline{id Integer}, nom String, age Integer, \dashuline{id\_espece Integer})
    \item Soins(\underline{id Integer}, \dashuline{id\_animal Integer}, \dashuline{id\_soin Integer})
    \item Types\_soins(\underline{id Integer}, soin String)
    \end{itemize}

\end{frame}
\end{document}