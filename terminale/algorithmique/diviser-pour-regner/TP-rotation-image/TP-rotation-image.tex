\documentclass[a4paper,11pt]{article}
\input{/home/tof/Documents/Cozy/latex-include/preambule_lua.tex}
\newcommand{\showprof}{show them}  % comment this line if you don't want to see todo environment
\fancyhead[L]{TP rotation image}
\newdate{madate}{10}{09}{2020}
\fancyhead[R]{Termianle - NSI} %\today
\fancyfoot[L]{~\\Christophe Viroulaud}
\fancyfoot[C]{\textbf{Page \thepage}}
\fancyfoot[R]{\includegraphics[width=2cm,align=t]{/home/tof/Documents/Cozy/latex-include/cc.png}}

\begin{document}
\begin{Form}
\section{Problématique}
La rotation d'une image est ne fonctionnalité proposée par n'importe quel logiciel de retouche tel \emph{Gimp}. L'opération n'est cependant pas triviale et peut demander une durée non négligeable.
\begin{center}
\shadowbox{\parbox{14cm}{\centering Construire un algorithme de rotation d'une image en appliquant le principe de \guill{diviser pour régner}.}}
\end{center}
\section{Principe}
Un algorithme de type \guill{diviser pour régner} se décompose en trois parties:
\begin{itemize}
\item \emph{diviser:} Le problème est partagé en plusieurs petits problèmes identiques.
\item \emph{traitement:} Chaque petit problème est résolu.
\item \emph{recombinaison:} Les petits problèmes résolus sont assemblés pour remonter au problème principal.
\end{itemize}
\begin{activite}
\textbf{Réflexion commune:} Considérons une image aux dimensions connues. Quelles étapes pourrions-nous imaginer pour répondre à notre problématique?
\end{activite}
\section{Algorithme de rotation}
\subsection{Chargement de l'image}
\emph{PIL (Python Image Library) -anciennement pillow-} est une bibliothèque de traitement d'image. Le code ci-après charge et affiche une image:
\begin{lstlisting}
from PIL import Image

im = Image.open("image.png")
im.show()
\end{lstlisting}
Afin de travailler sur l'image nous pouvons récupérer des informations:
\begin{lstlisting}
largeur, hauteur = im.size
px = im.load()
\end{lstlisting}
La variable \emph{px} est une matrice représentative des pixels de l'image. La couleur du pixel de coordonnées \emph{(x,y)} est donnée par l'instruction \textbf{px[x,y]}. Il est également possible d'affecter une nouvelle couleur \emph{c} à un pixel: \textbf{px[x,y] = c}.
\begin{activite}
Charger et afficher une image carrée. Il est possible de récupérer cette image sur \url{https://www.freepng.fr/}.
\end{activite}
\subsection{Application de l'algorithme \emph{Diviser pour régner}}
\begin{activite}
\begin{enumerate}
\item Écrire l'affectation qui permet de faire tourner dans le sens anti-horaire, les quatre pixels de coordonnées (0,0) (0,1) (1,0) (1,1).
\item Généraliser cette affectation pour les blocs de pixels de coordonnées (x,y) (x,y+t) (x+t,y) (x+t,y+t).
\item Écrire une fonction \textbf{rotation\_auxiliaire(px: object, x: int, y: int, t: int)\;\rightarrow\;None} qui applique l'algorithme récursif exprimé dans l'activité 1. Inclure la boucle de la question précédente dans cette fonction.
\item Écrire la fonction \textbf{rotation(px: object, taille: int)->None} qui effectuera l'appel principal de la fonction précédente.
\end{enumerate}
\begin{commentprof}
Plutôt qu'une seconde fonction nous pourrions donner des valeurs par défaut à x et y.\\
im.save("fichier.png") pour sauvegarder le fichier.
\end{commentprof}
\end{activite}
\section{Créer une bibliothèque}
Afin de pouvoir réutiliser la fonctionnalité de rotation développée, il peut être intéressant de l'intégrer dans une classe.
\begin{activite}
\begin{enumerate}
\item Créer une classe \textbf{Image\_lib} et son constructeur qui demande un argument: le chemin du fichier source.
\item Créer les méthodes \textbf{montrer()$\;\rightarrow\;$None} et \textbf{sauvegarder(nom: str)$\;\rightarrow\;$None}.
\item Adapter les fonctions \emph{rotation} et \emph{rotation\_auxiliaire} pour en faire des méthodes. La fonction \emph{rotation\_auxiliaire} peut être vue comme une fonctionnalité interne à la classe, invisible pour l'utilisateur.
\item Adapter la fonction pour pouvoir proposer une rotation horaire ou anti-horaire.
\end{enumerate}
\begin{commentprof}
l'argument \emph{taille} dans \emph{rotation} sera passé automatiquement à \emph{rotation\_auxiliaire}
fonction \textbf{filtrer} pour les plus avancés
\end{commentprof}
\end{activite}
\end{Form}
\end{document}