\documentclass[a4paper,11pt]{article}
\input{/home/tof/Documents/Cozy/latex-include/preambule_lua.tex}
\newcommand{\showprof}{show them}  % comment this line if you don't want to see todo environment
\fancyhead[L]{Types de données}
\newdate{madate}{10}{09}{2020}
\fancyhead[R]{Première - NSI} %\today
\fancyfoot[L]{~\\Christophe Viroulaud}
\fancyfoot[C]{\textbf{Page \thepage}}
\fancyfoot[R]{\includegraphics[width=2cm,align=t]{/home/tof/Documents/Cozy/latex-include/cc.png}}

\begin{document}
\begin{Form}
\section{Typage}
Les données utilisées dans un programme peuvent être de différentes natures. Une \emph{chaîne de caractère} permettra l'affichage d'un message à l'écran alors qu'un \emph{entier} sera plutôt utilisé dans un calcul.
\section{Typage dynamique}
Dans un programme, une variable possède un type déterminé par son contenu. La fonction Python \emph{type()} donne le type d'une variable.
\begin{lstlisting}
a = 42
type(a)
\end{lstlisting}
Ce code retourne \emph{int} pour \emph{integer (entier)}.\\
De nombreux langages de programmation imposent de définir le \emph{type} de chaque variable. Les raisons sont multiples:
\begin{itemize}
\item allouer un espace mémoire adéquat,
\item éviter les erreurs de programmation.
\end{itemize}
Ce n'est pas le cas en Python:
\begin{lstlisting}
a = 42
type(a)
a = "test"
type(a)
\end{lstlisting}
Ce langage de \emph{haut-niveau} veut simplifier la tâche de l'utilisateur en \emph{typant dynamiquement} chaque variable. De prime abord, cela peut paraître appréciable, mais engendre parfois des erreurs de codage dans un programme complexe.
\begin{lstlisting}
a = 42
b = 5
a = "test"
a+b
\end{lstlisting}
\section{Un code clair}
En Python, il est possible d'indiquer le type d'une variable.
\begin{lstlisting}
a: int = 42
\end{lstlisting}
Cette information n'est qu'indicative. Python ne lèvera pas d'erreur dans le cas ci-après:
\begin{lstlisting}
a: str = 42
\end{lstlisting}
\section{Les types de base}
\begin{itemize}
\item \textbf{int:} \emph{integer}; nombre entier (42)
\item \textbf{str:} \emph{string}; chaîne de caractère ("test")
\item \textbf{bool:} \emph{boolean}; valeur booléenne (True ou False)
\item \textbf{float:} \emph{flottant}; nombre à virgule flottante (4.2)
\end{itemize}

\end{Form}
\end{document}