\documentclass[svgnames,11pt]{beamer}
\input{/home/tof/Documents/Cozy/latex-include/preambule_commun.tex}
\input{/home/tof/Documents/Cozy/latex-include/preambule_beamer.tex}
%\usepackage{pgfpages} \setbeameroption{show notes on second screen=left}
\author[]{Christophe Viroulaud}
\title{Exercices fonctions booléennes\\Correction}
\date{\framebox{\textbf{ArchMat 07}}}
%\logo{}
\institute{Première - NSI}

\begin{document}
\begin{frame}
    \titlepage
\end{frame}
\section{Exercice 1}
\begin{frame}
    \frametitle{Exercice 1}

    \begin{center}
        \begin{tabular}{|c|c|c|}
            \hline
            P & Q & non (P) ou Q \\
            \hline
            0 & 0 & 1            \\
            \hline
            0 & 1 & 1            \\
            \hline
            1 & 0 & 0            \\
            \hline
            1 & 1 & 1            \\
            \hline
        \end{tabular}
    \end{center}

\end{frame}
\section{Exercice 2}
\begin{frame}
    \frametitle{Exercice 2}

    La réponse est: non(Q) et P. Pour s'en convaincre il faut dresser les tables de vérité de chaque proposition.


\end{frame}
\section{Exercice 3}
\begin{frame}
    \frametitle{Exercice 3}

    \begin{center}
        \begin{tabular}{|*4{c|}}
            \hline
            A & B & C & C et (A ou B) \\
            \hline
            0 & 0 & 0 & 0             \\
            \hline
            0 & 0 & 1 & 0             \\
            \hline
            0 & 1 & 0 & 0             \\
            \hline
            0 & 1 & 1 & 1             \\
            \hline
            1 & 0 & 0 & 0             \\
            \hline
            1 & 0 & 1 & 1             \\
            \hline
            1 & 1 & 0 & 0             \\
            \hline
            1 & 1 & 1 & 1             \\
            \hline
        \end{tabular}
    \end{center}

\end{frame}
\section{Exercice 4}
\begin{frame}
    \frametitle{Exercice 4}
    Il faut établir la table de vérité des deux parties de l'égalité.

    \begin{itemize}
        \item $f_1(x,y,z)=(x\land y)\lor (\lnot y \land z)$
        \item $f_2(x,y,z)=(x \lor \lnot y)\land (y \lor z)$
    \end{itemize}
\end{frame}
\begin{frame}
    \frametitle{}

    \begin{center}
        \begin{tabular}{|*6{c|}}
            \hline
            x & y & z & $x\land y$ & $\lnot y \land z$ & $f_1(x,y,z)$ \\
            \hline
            0 & 0 & 0 & 0          & 0                 & 0            \\
            \hline
            0 & 0 & 1 & 0          & 1                 & 1            \\
            \hline
            0 & 1 & 0 & 0          & 0                 & 0            \\
            \hline
            0 & 1 & 1 & 0          & 0                 & 0            \\
            \hline
            1 & 0 & 0 & 0          & 0                 & 0            \\
            \hline
            1 & 0 & 1 & 0          & 1                 & 1            \\
            \hline
            1 & 1 & 0 & 1          & 0                 & 1            \\
            \hline
            1 & 1 & 1 & 1          & 0                 & 1            \\
            \hline
        \end{tabular}
    \end{center}

\end{frame}
\begin{frame}
    \frametitle{}

    \begin{center}
        \begin{tabular}{|*6{c|}}
            \hline
            x & y & z & $x \lor \lnot y$ & $y \lor z$ & $f_2(x,y,z)$ \\
            \hline
            0 & 0 & 0 & 1                & 0          & 0            \\
            \hline
            0 & 0 & 1 & 1                & 1          & 1            \\
            \hline
            0 & 1 & 0 & 0                & 1          & 0            \\
            \hline
            0 & 1 & 1 & 0                & 1          & 0            \\
            \hline
            1 & 0 & 0 & 1                & 0          & 0            \\
            \hline
            1 & 0 & 1 & 1                & 1          & 1            \\
            \hline
            1 & 1 & 0 & 1                & 1          & 1            \\
            \hline
            1 & 1 & 1 & 1                & 1          & 1            \\
            \hline
        \end{tabular}
    \end{center}

\end{frame}
\begin{frame}
    \frametitle{}

    \begin{multicols}{2}
        \begin{center}
            \begin{tabular}{|*4{c|}}
                \hline
                x & y & z & $f_1(x,y,z)$ \\
                \hline
                0 & 0 & 0 & 0            \\
                \hline
                0 & 0 & 1 & 1            \\
                \hline
                0 & 1 & 0 & 0            \\
                \hline
                0 & 1 & 1 & 0            \\
                \hline
                1 & 0 & 0 & 0            \\
                \hline
                1 & 0 & 1 & 1            \\
                \hline
                1 & 1 & 0 & 1            \\
                \hline
                1 & 1 & 1 & 1            \\
                \hline
            \end{tabular}
        \end{center}
        \begin{center}
            \begin{tabular}{|*4{c|}}
                \hline
                x & y & z & $f_2(x,y,z)$ \\
                \hline
                0 & 0 & 0 & 0            \\
                \hline
                0 & 0 & 1 & 1            \\
                \hline
                0 & 1 & 0 & 0            \\
                \hline
                0 & 1 & 1 & 0            \\
                \hline
                1 & 0 & 0 & 0            \\
                \hline
                1 & 0 & 1 & 1            \\
                \hline
                1 & 1 & 0 & 1            \\
                \hline
                1 & 1 & 1 & 1            \\
                \hline
            \end{tabular}
        \end{center}
    \end{multicols}

\end{frame}
\section{Exercice 5}
\begin{frame}
    \frametitle{Exercice 5}

    \begin{center}
        \begin{tabular}{|*4{c|}}
            \hline
            x & y & z & $f(x,y,z)$ \\
            \hline
            0 & 0 & 0 & 0          \\
            \hline
            0 & 0 & 1 & 1          \\
            \hline
            0 & 1 & 0 & 1          \\
            \hline
            0 & 1 & 1 & 0          \\
            \hline
            1 & 0 & 0 & 1          \\
            \hline
            1 & 0 & 1 & 0          \\
            \hline
            1 & 1 & 0 & 0          \\
            \hline
            1 & 1 & 1 & 0          \\
            \hline
        \end{tabular}
    \end{center}
    Cette fonction vérifie si une et une seule variable vaut 1
\end{frame}
\section{Exercice 6}
\begin{frame}
    \frametitle{Exercice 6}
{\Large
    \begin{center}
        \begin{tabular}{ccc}
            &$r$&\\
              & $e_0$ & $e_1$ \\
            + & $e_2$ & $e_3$ \\
        \end{tabular}
    \end{center}
}
Le premier additionneur ajoute $e_1$ et $e_3$. L'entrée $c_0$ vaut 0. La retenue du premier additionneur est envoyée dans le second ($r$).
\end{frame}
\begin{frame}
    \frametitle{}

    \begin{center}
        \begin{tabular}{|*4{c|}|*3{c|}}
        \hline 
        $e_0$ & $e_1$ & $e_2$ & $e_3$ & $s_0$ & $s_1$ & $c$ \\ 
        \hline 
        0 & 0 & 0 & 0 & 0 & 0 & 0 \\ 
        \hline 
        0 & 0 & 0 & 1 & 0 & 1 & 0 \\ 
        \hline
        0 & 0 & 1 & 0 & 1 & 0 & 0 \\ 
        \hline
        0 & 0 & 1 & 1 & 1 & 1 & 0 \\ 
        \hline
        0 & 1 & 0 & 0 & 0 & 1 & 0 \\ 
        \hline
        0 & 1 & 0 & 1 & 1 & 0 & 0 \\ 
        \hline
        0 & 1 & 1 & 0 & 1 & 1 & 0 \\ 
        \hline
        0 & 1 & 1 & 1 & 0 & 0 & 1 \\ 
        \hline
        1 & 0 & 0 & 0 & 1 & 0 & 0 \\ 
        \hline
        1 & 0 & 0 & 1 & 1 & 1 & 0 \\ 
        \hline
        1 & 0 & 1 & 0 & 0 & 0 & 1 \\ 
        \hline
        1 & 0 & 1 & 1 & 0 & 1 & 1 \\ 
        \hline
        1 & 1 & 0 & 0 & 1 & 1 & 0 \\ 
        \hline
        1 & 1 & 0 & 1 & 0 & 0 & 1 \\ 
        \hline
        1 & 1 & 1 & 0 & 0 & 1 & 1 \\ 
        \hline
        1 & 1 & 1 & 1 & 1 & 0 & 1 \\ 
        \hline
        \end{tabular}
        \end{center}

\end{frame}
\section{Exercice 7}
\begin{frame}[fragile]
    \frametitle{Exercice 7}
    \begin{lstlisting}[language=Python  , xleftmargin=2em, xrightmargin=2em]
if x >= 0 and x < 100:
    print("salut")
else:
    print("bonjour")
\end{lstlisting}
    \begin{center}
        \begin{tikzpicture}
            \draw (0,0) -- (8,0);
            \draw (2,-1) -- (2,1);
            \draw (2,-1) -- (2.5,-1);
            \draw (2,1) -- (2.5,1);
            \node at(2.2,-0.2){0};

            \draw (7,-1) -- (7,1);
            \draw (7,-1) -- (7.5,-1);
            \draw (7,1) -- (7.5,1);
            \node at(7.4,-0.2){100};

            \node at(4.5,1){salut};
            \node at(0,1){bonjour};
            \node at(8.5,1){bonjour};
        \end{tikzpicture}
    \end{center}

\end{frame}
\begin{frame}[fragile]
    \begin{lstlisting}[language=Python  , xleftmargin=2em, xrightmargin=2em]
if x >= 0 or y < 0:
    print("salut")
else:
    print("bonjour")
\end{lstlisting}
    \begin{center}
        \begin{tikzpicture}
            \draw[->] (-3,0) -- (3,0);
            \draw[->] (0,-3) -- (0,3);
            \node at(3,0.3){x};
            \node at(0.3,3){y};
            \node at(1.5,1.5){salut};
            \node at(1.5,-1.5){salut};
            \node at(-1.5,-1.5){salut};
            \node at(-1.5,1.5){bonjour};
        \end{tikzpicture}
    \end{center}

\end{frame}
\begin{frame}[fragile]
    \frametitle{}

\begin{center}
\begin{lstlisting}[language=Python , basicstyle=\ttfamily\small, xleftmargin=2em, xrightmargin=2em]
def xor(x: bool, y: bool) -> bool:
    return (x and not y) or (not x and y)

print(xor(False, False)) # False
print(xor(False, True)) # True
print(xor(True, False)) # True
print(xor(True, True)) # False

print(xor(2>1, 0==1)) # True
\end{lstlisting}
\captionof{code}{Fonction xor}
\label{CODE}
\end{center}

\end{frame}
\end{document}