\documentclass[a4paper,11pt]{article}
\input{/home/tof/Documents/Cozy/latex-include/preambule_lua.tex}
\newcommand{\showprof}{show them}  % comment this line if you don't want to see todo environment
\fancyhead[L]{Atelier informatique}
\newdate{madate}{10}{09}{2020}
\fancyhead[R]{\today}
\fancyfoot[L]{~\\Christophe Viroulaud}
\fancyfoot[C]{\textbf{Page \thepage}}
\fancyfoot[R]{\includegraphics[width=2cm,align=t]{/home/tof/Documents/Cozy/latex-include/cc.png}}

\begin{document}
\begin{Form}
\part*{Atelier informatique}
\section{Objectifs}
\begin{itemize}
\item Développer des compétences numériques variées afin de réaliser un projet en autonomie.
\item Développer des compétences d'initiative, de créativité et de coopération.
\item Réaliser un projet numérique parmi:
\begin{itemize}
\item Création d'un mini-jeu en ligne,
\item Création d'une présentation dynamique de l'établissement en ligne,
\item Programmation de robots numériques.
\end{itemize}
\end{itemize}
\section{Organisation}
\begin{itemize}
\item L'atelier se déroulerait le mardi entre 12h et 13h ou 13h et 14h en fonction des inscriptions. La gestion des inscriptions sera mise en place par Mme Gay et moi-même.
\item Ouvert à tous les élèves. Les projets seront adaptés en fonction des niveaux. 
\item Une incitation particulière sera réalisée auprès des secondes afin de leur faire découvrir le numérique.
\item Un ou plusieurs projets pourront être menés en fonction des demandes.
\item La salle (Agora, 430, 147) sera choisie en fonction des disponibilités.
\end{itemize}
\section{Coût}
\begin{itemize}
\item La programmation web ne demande aucun financement: utilisation de logiciels gratuits et open-source.
\item Le projet \emph{présentation} ne demande pas de financement non plus: la réalisation éventuelle de vidéos se fera via les smartphones.
\item Le projet \emph{robot numérique} utilisera les machines déjà présentes dans l'établissement.
\end{itemize}
\end{Form}
\end{document}