\documentclass[svgnames,11pt]{beamer}
\input{/home/tof/Documents/Cozy/latex-include/preambule_commun.tex}
\input{/home/tof/Documents/Cozy/latex-include/preambule_beamer.tex}
%\usepackage{pgfpages} \setbeameroption{show notes on second screen=left}
\author[]{Christophe Viroulaud}
\title{Exercices POO\\Correction}
\date{\framebox{\textbf{Lang 02}}}
%\logo{}
\institute{Terminale - NSI}

\begin{document}
\begin{frame}
\titlepage
\end{frame}
\begin{frame}
    \frametitle{Code}

    L'ensemble des programmes se trouvent \href{https://cviroulaud.github.io/terminale/langages/paradigmes/POO/exercices/scripts/poo-correction.zip}{ici}.

\end{frame}
\section{Exercice 1}
\begin{frame}
    \frametitle{Exercice 1}
\lstinputlisting[firstline=10 ,lastline=18 , basicstyle=\ttfamily\small, xleftmargin=1em, xrightmargin=1em]{"scripts/livre.py"}

\end{frame}
\begin{frame}
\lstinputlisting[firstline=26 ,lastline=28 , basicstyle=\ttfamily\small, xleftmargin=2em, xrightmargin=2em]{"scripts/livre.py"}

\end{frame}
\begin{frame}
\lstinputlisting[firstline=31 ,lastline=32 , basicstyle=\ttfamily\small, xleftmargin=2em, xrightmargin=2em]{"scripts/livre.py"}

\end{frame}
\section{Exercice 2}
\begin{frame}
    \frametitle{Exercice 2}
\lstinputlisting[firstline=10 ,lastline=14 , basicstyle=\ttfamily\small, xleftmargin=2em, xrightmargin=2em]{"scripts/rectangle.py"}

\end{frame}
\begin{frame}
\lstinputlisting[firstline=19 ,lastline=23 , basicstyle=\ttfamily\small, xleftmargin=2em, xrightmargin=2em]{"scripts/rectangle.py"}

\end{frame}
\section{Exercice 3}
\begin{frame}
    \frametitle{Exercice 3}
\lstinputlisting[firstline=10 ,lastline=17 , basicstyle=\ttfamily\small, xleftmargin=2em, xrightmargin=2em]{"scripts/complexe.py"}

\end{frame}
\section{Exercice 4}
\begin{frame}[fragile]
    \frametitle{Exercice 4}
\begin{center}
\begin{lstlisting}[language=Python , basicstyle=\ttfamily\small, xleftmargin=1em, xrightmargin=1em]
class Date:
    def __init__(self, j: int, m: int, a: int):
        self.jour = j
        self.mois = m
        self.annee = a
\end{lstlisting}
\end{center}
\end{frame}
\begin{frame}[fragile]
\begin{center}
\begin{lstlisting}[language=Python , basicstyle=\ttfamily\small, xleftmargin=2em, xrightmargin=2em]
def est_avant(self, d) -> bool:
    # Le \ permet d'écrire sur plusieurs lignes
    # and est prioritaire devant or
    return self.annee < d.annee or \
        self.annee == d.annee and \
            (self.mois < d.mois or \
            self.mois == d.mois and \
                self.jour < d.jour)
\end{lstlisting}
\end{center}

\end{frame}
\begin{frame}[fragile]
    \frametitle{}

\begin{center}
\begin{lstlisting}[language=Python , basicstyle=\ttfamily\small, xleftmargin=2em, xrightmargin=2em]
def afficher(self) -> str:
    nom_mois = ["janvier", "février", "mars", "avril", "mai", "juin", "juillet", "août", "septembre", "octobre", "novembre", "décembre"]
    return f"{self.jour} / {nom_mois[self.mois - 1]} / {self.annee}"

\end{lstlisting}
\end{center}

\end{frame}
\section{Exercice 5}
\begin{frame}
    \frametitle{Exercice 5}
\lstinputlisting[firstline=12 ,lastline=16 , basicstyle=\ttfamily\small, xleftmargin=2em, xrightmargin=2em]{"scripts/loto.py"}

\end{frame}
\begin{frame}
\lstinputlisting[firstline=25 ,lastline=31 , basicstyle=\ttfamily\small, xleftmargin=2em, xrightmargin=2em]{"scripts/loto.py"}

\end{frame}
\begin{frame}
\lstinputlisting[firstline=46 ,lastline=60 , basicstyle=\ttfamily\small, xleftmargin=2em, xrightmargin=2em]{"scripts/loto.py"}

\end{frame}
\end{document}