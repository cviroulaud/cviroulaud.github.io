\documentclass[a4paper,11pt]{article}
\input{/home/tof/Documents/Cozy/latex-include/preambule_lua.tex}
\newcommand{\showprof}{show them}  % comment this line if you don't want to see todo environment
\fancyhead[L]{Structure de l'internet}
\newdate{madate}{10}{09}{2020}
\fancyhead[R]{\displaydate{madate}} %\today
\fancyfoot[L]{~\\Christophe Viroulaud}
\fancyfoot[C]{\textbf{Page \thepage}}
\fancyfoot[R]{\includegraphics[width=2cm,align=t]{/home/tof/Documents/Cozy/latex-include/cc.png}}

\begin{document}
\begin{Form}
\setcounter{section}{1}
\section{Repérer une machine}
\subsection{Nombre d'adresses}
\begin{framed}
\noindent\textbf{Extrait de la page \url{https://fr.wikipedia.org/wiki/IPv4}:} IPv4 (Internet Protocol version~4) est la première version d'Internet Protocol (IP) à avoir été largement déployée, et qui forme encore en 2020 la base de la majorité des communications sur Internet. Elle est décrite dans la RFC 791 de septembre 1981, remplaçant la RFC 760, définie en janvier 1980.\\Une adresse IPv4 est représentée sous la forme de quatre nombres entiers séparés par des points comme 193.43.55.67. Chacun des nombres représente un octet. La plage d'attribution s'étend de 0.0.0.0 à 255.255.255.255, sachant qu'il existe des contraintes empêchant l'utilisation de certaines adresses (réservée, masque, broadcast, etc.).
\end{framed}
\begin{commentprof}
requests for comments (RFC), littéralement « demande de commentaires » sont une série numérotée de documents officiels décrivant les aspects et spécifications techniques d'Internet, ou de différents matériels informatiques.
\end{commentprof}
\begin{enumerate}
\item L'adresse 197.32.267.19 est-elle une adresse valide? Justifier.
\item Combien d'adresses peut-on distribuer avec le protocole IPv4? Effectuer le calcul.
\item En effectuant une recherche web, définir le sigle RFC.
\end{enumerate}
\subsection{Pénurie des adresses}
Lire l'article sur la page web suivante:
\begin{center}
\url{https://www.zdnet.fr/actualites/ipv4-la-penurie-d-adresses-disponibles-c-est-pour-novembre-2019-39891627.htm}
\end{center}
ainsi que cet extrait:
\begin{framed}
\noindent\textbf{Extrait de la page \url{https://fr.wikipedia.org/wiki/IPv6}:} IPv6 est l'aboutissement des travaux menés au sein de l'IETF au cours des années 1990 pour succéder à IPv4 et ses spécifications ont été finalisées dans la RFC 2460 en décembre 1998. IPv6 a été standardisé dans la RFC 8200 en juillet 2017.
Grâce à des adresses de 128 bits au lieu de 32 bits, IPv6 dispose d'un espace d'adressage bien plus important qu'IPv4 (plus de $340.10^{36}$). Cette quantité d'adresses considérable permet une plus grande flexibilité dans l'attribution des adresses et une meilleure agrégation des routes dans la table de routage d'Internet. La traduction d'adresse, qui a été rendue populaire par le manque d'adresses IPv4, n'est plus nécessaire. 
\end{framed}
\begin{enumerate}
\item Quel problème de l'internet est évoqué dans ces documents?
\item Quelle(s) solution(s) est(sont) mise(s) en place pour y remédier?
\item En estimant la surface de la Terre à 510 millions de km², combien disposera-t-on d'adresses par mm²?
\item A-t-on besoin d'autant d'adresses?
\item En effectuant une recherche web, définir \emph{l'ARCEP} et \emph{l'IETF}.
\end{enumerate}
\begin{commentprof}
\begin{itemize}
\item $6,7.10^8$ adresses par mm²
\item ARCEP: Autorité de régulation des communications électroniques, des postes et de la distribution de la presse est une autorité administrative indépendante chargée de réguler les communications électroniques et postales et la distribution de la presse en France.
\item IETF: Internet Engineering Task Force, élabore et promeut des standards Internet, en particulier les standards qui composent la suite de protocoles Internet. 
\end{itemize}
\end{commentprof}
\subsection{Trouver mon adresse}
Effectuer les manipulations suivantes sur un des quatre ordinateurs portables en réseau.
\begin{enumerate}
\item Ouvrir une session avec l'identifiant \emph{test} et le mot de passe \emph{test}.
\item Ouvrir un terminal avec la combinaison \emph{Ctrl+Alt+t}.
\item Taper l'instruction \emph{ip address}. Retrouver l'adresse de la machine; elle est de la forme \emph{192.X.X.X}
\item Demander à un autre élève, l'adresse de sa machine. Taper alors l'instruction \emph{ping -c4 sonadresse}.
\item Interpréter le résultat obtenu. Il n'est pas nécessaire d'expliquer chaque information mais plutôt d'en comprendre le sens général.
\end{enumerate}
\begin{commentprof}
essayer avec \emph{ifconfig} et \emph{ipconfig, ipconfig/all} sur Windows
\end{commentprof}
\begin{commentprof}
\section*{Bilan}
slide
\end{commentprof}
\end{Form}
\end{document}