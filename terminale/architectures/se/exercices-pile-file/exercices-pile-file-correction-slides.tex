\documentclass[svgnames,11pt]{beamer}
\input{/home/tof/Documents/Cozy/latex-include/preambule_commun.tex}
\input{/home/tof/Documents/Cozy/latex-include/preambule_beamer.tex}
%\usepackage{pgfpages} \setbeameroption{show notes on second screen=left}
\author[]{Christophe Viroulaud}
\title{Exercices pile - file\\correction}
\date{\framebox{\textbf{Archi 06}}}
%\logo{}
\institute{Terminale - NSI}

\begin{document}
\begin{frame}
    \titlepage
\end{frame}
\section{Exercice 1}
\begin{frame}[fragile]
    \frametitle{Exercice 1}

    \begin{center}
        \begin{lstlisting}[language=Python , basicstyle=\ttfamily\small, xleftmargin=2em, xrightmargin=2em]
def est_vide(p: list) -> bool:
    return len(p) == 0

def empiler(p: list, e: int) -> None:
    p.append(e)

def depiler(p: list) -> int:
    return p.pop()

p = []
\end{lstlisting}
        \captionof{code}{pile}
        \label{CODE}
    \end{center}

\end{frame}
\begin{frame}[fragile]
    \frametitle{}

    \begin{center}
        \begin{lstlisting}[language=Python , basicstyle=\ttfamily\small, xleftmargin=2em, xrightmargin=2em]
def est_vide(f: list) -> bool:
    return len(f) == 0

def enfiler(f: list, e: int) -> None:
    f.insert(0, e)

def defiler(f: list) -> int:
    return f.pop()

f = []
\end{lstlisting}
        \captionof{code}{file}
        \label{CODE}
    \end{center}

\end{frame}
\begin{frame}
    \frametitle{}

    La modification de la taille d'un tableau a un coup qui peut être linéaire.

\end{frame}
\section{Exercice 2}
\begin{frame}
    \frametitle{Exercice 2}

    \begin{center}
        \begin{tikzpicture}
            \draw (0,0) grid (1,3);
            \draw[dashed] (0,3) grid (1,4);
            \foreach \x/\y in {0/2,1/4,2/7}
                {
                    \node at(.5,.5+\x){\y};
                }
            \node at(.5,-1){gauche};
            \node (emp) at(-1,5) {5};
            \draw[->,>=latex] (emp) edge[bend left] (.5,3.5);
            \draw (2,0) grid (3,.2);
            \node at(2.5,-1){droite};

        \end{tikzpicture}
        \captionof{figure}{enfiler}
    \end{center}
    \begin{center}
        Le premier entré est 2.
    \end{center}
\end{frame}
\begin{frame}
    \begin{center}
        \begin{tikzpicture}
            \draw (0,0) grid (1,4);
            \foreach \x/\y in {0/2,1/4,2/7,3/5}
                {
                    \node at(.5,.5+\x){\y};
                }
            \node at(.5,-1){gauche};
            \draw[->,>=latex] (1,3.5) edge[bend left] (2.5,.5);
            \draw (2,0) grid (3,.2);
            \node at(2.5,-1){droite};

            \draw (5,0) grid (6,.2);
            \foreach \x/\y in {0/5,1/7,2/4,3/2}
                {
                    \node at(7.5,.5+\x){\y};
                }
            \node at(5.5,-1){gauche};
            \draw (7,0) grid (8,4);
            \node at(7.5,-1){droite};
            \draw[->,>=latex] (8,3.5) edge[bend left] (9,2.5);

        \end{tikzpicture}
        \captionof{figure}{défiler - cas 1}
    \end{center}
    \begin{center}
        La pile droite est vide, on commence par dépiler celle de gauche.
    \end{center}
\end{frame}
\begin{frame}
    \begin{center}
        \begin{tikzpicture}
            \draw (0,0) grid (1,.2);
            \foreach \x/\y in {0/5,1/7,2/4}
                {
                    \node at(2.5,.5+\x){\y};
                }
            \node at(.5,-1){gauche};
            \draw (2,0) grid (3,3);
            \node at(2.5,-1){droite};
            \draw[->,>=latex] (3,2.5) edge[bend left] (4,1.5);

        \end{tikzpicture}
        \captionof{figure}{défiler - cas 2}
    \end{center}
    \begin{center}
        La pile droite n'est pas vide. On dépile normalement.
    \end{center}
\end{frame}
\begin{frame}[fragile]
    \frametitle{}

    \begin{center}
        \begin{lstlisting}[language=Python , basicstyle=\ttfamily\small, xleftmargin=2em, xrightmargin=2em]
def est_vide(f: list) -> bool:
    return len(f) == 0

def enfiler(f: list, e: int) -> None:
    f.insert(0, e)

def defiler(f: list) -> int:
    return f.pop()

f = []
\end{lstlisting}
        \captionof{code}{file}
        \label{CODE}
    \end{center}

\end{frame}
\end{document}