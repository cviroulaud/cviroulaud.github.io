\documentclass[svgnames,11pt]{beamer}
\input{/home/tof/Documents/Cozy/latex-include/preambule_commun.tex}
\input{/home/tof/Documents/Cozy/latex-include/preambule_beamer.tex}
%\usepackage{pgfpages} \setbeameroption{show notes on second screen=left}
\author[]{Christophe Viroulaud}
\title{Révisions types de données\\Correction exercices}
\date{\framebox{\textbf{Rév 01}}}
%\logo{}
\institute{Terminale - NSI}

\begin{document}
\begin{frame}
\titlepage
\end{frame}
\begin{frame}[fragile]
    \frametitle{Exercice 1}

\begin{exo}
\begin{lstlisting}[language=Python,basicstyle=\ttfamily\small  , xleftmargin=2em, xrightmargin=2em]
tab = [1 for i in range(5)]
\end{lstlisting}
\end{exo}    

\end{frame}
\begin{frame}[fragile]
    \frametitle{Exercice 2}

\begin{exo}
\begin{lstlisting}[language=Python,basicstyle=\ttfamily\small  , xleftmargin=2em, xrightmargin=2em]
tab = [i for i in range(5)]
\end{lstlisting}
\end{exo}    

\end{frame}

\begin{frame}[fragile]
    \frametitle{Exercice 3}

\begin{exo}
\begin{lstlisting}[language=Python,basicstyle=\ttfamily\small  , xleftmargin=2em, xrightmargin=2em]
tup = tuple(i for i in range(4, -1, -1))
\end{lstlisting}
\end{exo}   

\end{frame}

\begin{frame}[fragile]
    \frametitle{Exerice 4}

\begin{exo}
\begin{lstlisting}[language=Python ,basicstyle=\ttfamily\small , xleftmargin=2em, xrightmargin=2em]
tup = tuple(i for i in range(0, 9, 2))

# seconde méthode
tup = tuple(i for i in range(9) if i%2 == 0)
\end{lstlisting}
\end{exo}    

\end{frame}

\begin{frame}[fragile]
    \frametitle{Exercice 5}

\begin{exo}
\begin{lstlisting}[language=Python,basicstyle=\ttfamily\small  , xleftmargin=2em, xrightmargin=2em]
dico = {i: 1 for i in range(5)}
\end{lstlisting}
\end{exo}
\end{frame}

\begin{frame}[fragile]
    \frametitle{Exercice 6}

\begin{exo}
\begin{lstlisting}[language=Python ,basicstyle=\ttfamily\small , xleftmargin=2em, xrightmargin=2em]
dico = {chr(65+i): i for i in range(5)}
\end{lstlisting}
\end{exo}
\end{frame}

\begin{frame}[fragile]
    \frametitle{Exercice 7}

\begin{exo}
\begin{lstlisting}[language=Python,basicstyle=\ttfamily\small  , xleftmargin=2em, xrightmargin=2em]
from random import randint

tab = [randint(0, 100) for i in range(10)]
\end{lstlisting}
\end{exo}
\end{frame}

\begin{frame}[fragile]
    \frametitle{Exercice 8}

\begin{exo}
\begin{lstlisting}[language=Python,basicstyle=\ttfamily\small  , xleftmargin=2em, xrightmargin=2em]
from random import randint

tab = [randint(0, 100) for i in range(10)]

def maxi(tab: list) -> int:
    maximum = 0
    for element in tab:
        if element > maximum:
            maximum = element
    return maximum

print(tab)
print(maxi(tab))
\end{lstlisting}
\end{exo}
\end{frame}

\begin{frame}[fragile]
    \frametitle{Exercice 9}

\begin{exo}
\begin{lstlisting}[language=Python,basicstyle=\ttfamily\small  , xleftmargin=2em, xrightmargin=2em]
from random import randint

tup = tuple(randint(0, 100) for i in range(10))

def somme(tup: tuple) -> int:
    resultat = 0
    for element in tup:
        resultat += element
    return resultat

print(tup)
print(somme(tup))
\end{lstlisting}
\end{exo}
\end{frame}

\begin{frame}[fragile]
    \frametitle{Exercice 10}

\begin{exo}
\begin{lstlisting}[language=Python,basicstyle=\ttfamily\small  , xleftmargin=2em, xrightmargin=2em]
tab = ["qui", "que", "quoi", "dont", "où", "comment"]

i1 = int(input("indice 1: "))
i2 = int(input("indice 2: "))

temp = tab[i1]
tab[i1] = tab[i2]
tab[i2] = temp

print(tab)
\end{lstlisting}
\end{exo}    

\end{frame}

\begin{frame}[fragile]
    \frametitle{Exercice 11}

\begin{exo}
\begin{lstlisting}[language=Python,basicstyle=\ttfamily\small  , xleftmargin=2em, xrightmargin=2em]
bibliotheque = [
    {"titre": "Il était deux fois", 
        "auteur": "Franck Thilliez",
        "editeur": "Poche",
        "prix": 8.70},
    {"titre": "Fahrenheit 451", 
        "auteur": "Ray Bradbury",
        "editeur": "Folio",
        "prix": 6.30},
    {"titre": "Le guide du voyageur galactique", 
        "auteur": "Douglas Adams",
        "editeur": "Folio",
        "prix": 8.10}
]

for livre in bibliotheque:
    print(livre["auteur"])
\end{lstlisting}
\end{exo}    

\end{frame}
\begin{frame}[fragile]
    \frametitle{Exercice 12}

\begin{exo}
\begin{lstlisting}[language=Python,basicstyle=\ttfamily\small , xleftmargin=2em, xrightmargin=2em]
def lettres(mot: str)->dict:
    """
    compte le nombre d'occurrences
    de chaque lettre du mot
    """
    occurrences = {}
    for l in mot:
        # si la lettre est déjà référencée
        if l in occurrences:
            occurrences[l] += 1
        else:
            occurrences[l] = 1
    return occurrences

print(lettres("bonjour"))
\end{lstlisting}
\end{exo}    

\end{frame}
\end{document}