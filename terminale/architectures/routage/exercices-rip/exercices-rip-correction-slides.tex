\documentclass[svgnames,11pt]{beamer}
\input{/home/tof/Documents/Cozy/latex-include/preambule_commun.tex}
\input{/home/tof/Documents/Cozy/latex-include/preambule_beamer.tex}
%\usepackage{pgfpages} \setbeameroption{show notes on second screen=left}
\author[]{Christophe Viroulaud}
\title{Exercices RIP\\Correction}
\date{\framebox{\textbf{Archi 12}}}
%\logo{}
\institute{Terminale - NSI}

\begin{document}
\begin{frame}
\titlepage
\end{frame}
\section{Exercice 1}
\begin{frame}
    \frametitle{Exercice 1}
    \begin{aretenir}[Remarque]
        Selon la littérature les notations des interfaces et passerelles sont variables. \textbf{Pour chaque interface le routeur possède une adresse IP appartenant au réseau auquel il est connecté.}
    \end{aretenir}
    

\end{frame}
\begin{frame}
    \frametitle{}

    \begin{itemize}
        \item 18.13.0.0/16
        \item 192.168.0.0/16
        \item 19.20.1.0/24
    \end{itemize}

\end{frame}
\begin{frame}
    \frametitle{}

        \begin{center}
            \begin{tabular}{|*{4}{c|}}
                \hline
                Destination & Passerelle & Interface & Distance \\
                \hline
                192.168.0.0/16 & & 192.168.0.1 & 1\\
                \hline
                18.13.0.0/16 & 192.168.0.254 (R2) & 192.168.0.1 & 2\\
                \hline
                19.20.1.0/24 & 192.168.0.254 (R2) & 192.168.0.1 & 2\\
                \hline
            \end{tabular}
            \captionof{table}{Table de routage de R1}
        \end{center}

\end{frame}
\section{Exercice 2}
\begin{frame}
    \frametitle{Exercice 2}

    \begin{aretenir}[Remarque]
        Dans cet exercice, la destination à atteindre n'est pas un réseau mais un routeur.
        \end{aretenir}

\end{frame}
\begin{frame}
    \frametitle{}

    Pour atteindre G on lit les tables de routage:
    \begin{itemize}
        \item table A: Vecteur (C,3),
        \item table C: Vecteur (F,2),
        \item table F: Vecteur (G,1)
    \end{itemize}
    Soit une distance de 6 pour un trajet: A $\rightarrow$ C $\rightarrow$ F $\rightarrow$ G.

\end{frame}
\begin{frame}
    \frametitle{}

    
    \begin{center}
        \begin{tabular}{|*{3}{c|}}
            \hline
            Destination & Routeur suivant (Passerelle) & Distance \\
            \hline
            A & E & 3\\
            \hline
            B & E & 3\\
            \hline
            C & E & 2\\
            \hline
            D & E & 2\\
            \hline
            E & E & 1\\
            \hline
            F & F & 1\\
            \hline
        \end{tabular}
        \captionof{table}{Table de routage de G}
    \end{center}
    \begin{aretenir}[Remarque]
    Pour atteindre A et C il est possible de passer par F.
    \end{aretenir}
\end{frame}
\section{Exercice 3}
\begin{frame}
    \frametitle{Exercice 3}

    Phase d'initialisation
        \begin{center}
            \begin{tabular}{|*{4}{c|}}
                \hline
                Destination & Passerelle & Interface & Distance \\
                \hline
                C & & eth1 & 1\\
                \hline
                B & & eth0 & 1\\
                \hline
            \end{tabular}
        \end{center}

\end{frame}
\begin{frame}
    \frametitle{}

    Extrait de table pour atteindre G
        \begin{center}
            \begin{tabular}{|*{5}{c|}}
                \hline
                Extrait de la table & Destination & Passerelle & Interface & Distance \\
                \hline
                A & G & B & eth0 & 3\\
                \hline
                B & G & F & eth0 & 2\\
                \hline
                C & G & B & eth2 & 3\\
                \hline
                D & G & E & eth1 & 3\\
                \hline
                E & G & F & eth1 & 2\\
                \hline
                F & G &  & eth1 & 1\\
                \hline
            \end{tabular}
        \end{center}

\end{frame}
\begin{frame}
    \frametitle{}

    B envoie une route infinie soit le vecteur (G,16). Le maximum de sauts est 15 avec le protocole RIP.

\end{frame}
\begin{frame}
    \frametitle{}

    \begin{itemize}
        \item Les routeurs A et C reçoivent une route existante plus longue; cela signifie qu'un problème est apparu. Ils mettent leur table à jour: pour G ils enregistrent la route (B, 16).
        \item Le routeur D reçoit une nouvelle route plus longue vers G: il l'ignore.
        \item Le routeur C possède la route (B, 16) vers G. Il la remplace par (D, 4).
    \end{itemize}

\end{frame}
\begin{frame}
    \frametitle{}

    Le routeur C envoie le vecteur (G, 4) à A (et à B). Ces routeurs mettent leur route vers G à jour avec le vecteur (C, 5).

\end{frame}
\end{document}