\documentclass[svgnames,11pt]{beamer}
\input{/home/tof/Documents/Cozy/latex-include/preambule_commun.tex}
\input{/home/tof/Documents/Cozy/latex-include/preambule_beamer.tex}
%\usepackage{pgfpages} \setbeameroption{show notes on second screen=left}
\author[]{Christophe Viroulaud}
\title{Exercices supplémentaires\\fonctions}
\date{\framebox{\textbf{Lang 07}}}
%\logo{}
\institute{Première - NSI}

\begin{document}
\begin{frame}
\titlepage
\end{frame}
\section{Exercice 1}
\begin{frame}
    \frametitle{Exercice 1}

    Écrire une fonction \texttt{\textbf{volume(long: int, larg: int, haut: int) $\rightarrow$ int}} qui calcule le volume d'un pavé.

\end{frame}
\begin{frame}
    \frametitle{Avant de regarder la correction}
\begin{center}
    \centering
    \includegraphics[width=3cm]{/home/tof/Documents/Cozy/latex-include/stop.png}
    \end{center}
{\Large
    \begin{itemize}
        \item Prendre le temps de réfléchir,
        \item Analyser les messages d'erreur,
        \item Demander au professeur.
    \end{itemize}
}
\end{frame}
\begin{frame}[fragile]
    \frametitle{Correction}

\begin{lstlisting}[language=Python , basicstyle=\ttfamily\small, xleftmargin=1em, xrightmargin=0em]
def volume(long: int, larg: int, haut: int) -> int:
    """
    renvoie le volume d'un pavé

    Args:
        long (int): longueur
        larg (int): largeur
        haut (int): hauteur

    Returns:
        int: volume
    """
    return long*larg*haut
\end{lstlisting}

\end{frame}
\section{Exercice 2}
\begin{frame}
    \frametitle{Exercice 2}

    Un entier naturel est dit parfait lorsqu’il est égal à la somme de tous ses diviseurs propres autres que lui-même. Par exemple 6 est parfait car $6 = 1 + 2 + 3$.\\
    Écrire la fonction \textbf{\texttt{est\_parfait(n: int) $\rightarrow$ bool}} qui vérifie si \textbf{\texttt{n}} est un nombre parfait.

\end{frame}
\begin{frame}
    \frametitle{Avant de regarder la correction}
\begin{center}
    \centering
    \includegraphics[width=3cm]{/home/tof/Documents/Cozy/latex-include/stop.png}
    \end{center}
{\Large
    \begin{itemize}
        \item Prendre le temps de réfléchir,
        \item Analyser les messages d'erreur,
        \item Demander au professeur.
    \end{itemize}
}
\end{frame}
\begin{frame}[fragile]
    \frametitle{Correction}

\begin{lstlisting}[language=Python , basicstyle=\ttfamily\small, xleftmargin=2em, xrightmargin=2em]
def est_parfait(n: int) -> bool:
    """
    vérifie si n est parfait
    """
    somme = 0
    for i in range(1, n):
        if n % i == 0:  # i est multiple de n
            somme += i
    # la somme des multiples == n
    return somme == n
\end{lstlisting}

\end{frame}
\section{Exercice 3}
\begin{frame}
    \frametitle{Exercice 3}

    La conjecture de COLLATZ est un énoncé mathématique dont on ne sait pas encore s’il est vrai. Prendre un nombre entier positif, et lui appliquer lui le traitement suivant :
    \begin{itemize}
        \item s’il est pair, le diviser par 2        
        \item sinon, le multiplier par 3 et ajouter 1
    \end{itemize}    
    On obtient alors un nouveau nombre, sur lequel la procédure est répétée. \begin{center}
        La conjecture s’énonce ainsi : \textbf{quel que soit l’entier choisi au départ, on finira par obtenir 1.}    
    \end{center}
    Écrire une fonction \texttt{\textbf{collatz(n: int) $\rightarrow$ bool}} qui renvoie \texttt{\textbf{True}} si le processus s'achève, c'est à dire si on finit par atteindre 1.

\end{frame}
\begin{frame}
    \frametitle{Avant de regarder la correction}
\begin{center}
    \centering
    \includegraphics[width=3cm]{/home/tof/Documents/Cozy/latex-include/stop.png}
    \end{center}
{\Large
    \begin{itemize}
        \item Prendre le temps de réfléchir,
        \item Analyser les messages d'erreur,
        \item Demander au professeur.
    \end{itemize}
}
\end{frame}
\begin{frame}[fragile]
    \frametitle{Correction}

\begin{lstlisting}[language=Python , basicstyle=\ttfamily\small, xleftmargin=2em, xrightmargin=2em]
def collatz(n: int) -> bool:
    """
    renvoie True si la procédure termine
    Si la boucle ne se termine jamais, la conjecture
    est fausse.
    """
    while n != 1:
        if n % 2 == 0:  # pair
            n = n//2
        else:
            n = 3*n+1
    return True
\end{lstlisting}

\end{frame}
\end{document}