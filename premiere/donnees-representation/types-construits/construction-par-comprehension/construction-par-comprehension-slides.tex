\documentclass[svgnames,11pt]{beamer}
\input{/home/tof/Documents/Cozy/latex-include/preambule_commun.tex}
\input{/home/tof/Documents/Cozy/latex-include/preambule_beamer.tex}
%\usepackage{pgfpages} \setbeameroption{show notes on second screen=left}
\author[]{Christophe Viroulaud}
\title{Construction par compréhension}
\date{\framebox{\textbf{DonRep 11}}}
%\logo{}
\institute{Première - NSI}

\begin{document}
\begin{frame}
\titlepage
\end{frame}
\begin{frame}[fragile]
    \frametitle{}

\begin{center}
\begin{lstlisting}[language=Python , basicstyle=\ttfamily\small, xleftmargin=2em, xrightmargin=2em]
tab = []
for i in range(100):
    tab.append(i)
\end{lstlisting}
\captionof{code}{Construire un tableau}
\label{CODE}
\end{center}   

\begin{framed}\centering
    Comment construire une structure de données efficacement?
\end{framed}
\end{frame}
\section{Tableau par compréhension}
\begin{frame}[fragile]
    \frametitle{Tableau par compréhension}

\begin{center}
\begin{lstlisting}[language=Python , basicstyle=\ttfamily\small, xleftmargin=2em, xrightmargin=2em]
tab = [i for i in range(100)]
\end{lstlisting}
\captionof{code}{Construire un tableau de 100 entiers}
\label{CODE}
\end{center}

\end{frame}
\begin{frame}
    \frametitle{}

    \begin{activite}
        Par compréhension, construire un tableau:
    \begin{enumerate}
        \item de dix zéros.
        \item de dix entiers décroissants de 9 à 0.
        \item des carrés des entiers de 0 à 9.
        \item de dix entiers aléatoires entre 1 et 10.
    \end{enumerate}
    \end{activite}

\end{frame}
\begin{frame}[fragile]
    \frametitle{}

    
\begin{lstlisting}[language=Python , basicstyle=\ttfamily\small, xleftmargin=2em, xrightmargin=2em]
>>> tab1 = [0 for i in range(10)]
>>> tab1
[0, 0, 0, 0, 0, 0, 0, 0, 0, 0]
\end{lstlisting}

\begin{lstlisting}[language=Python , basicstyle=\ttfamily\small, xleftmargin=2em, xrightmargin=2em]
>>> tab2 = [i for i in range(9, -1, -1)]
>>> tab2
[9, 8, 7, 6, 5, 4, 3, 2, 1, 0]
\end{lstlisting}
\end{frame}
\begin{frame}[fragile]
    \frametitle{}

    
\begin{lstlisting}[language=Python , basicstyle=\ttfamily\small, xleftmargin=2em, xrightmargin=0em]
>>> tab3 = [i**2 for i in range(10)]
>>> tab3
[0, 1, 4, 9, 16, 25, 36, 49, 64, 81]
\end{lstlisting}

\begin{lstlisting}[language=Python , basicstyle=\ttfamily\small, xleftmargin=2em, xrightmargin=0em]
>>> tab4 = [randint(1, 100) for i in range(10)]
>>> tab4
[32, 1, 18, 19, 87, 13, 43, 10, 98, 66]
\end{lstlisting}
\end{frame}
\section{Tuple par compréhension}
\begin{frame}[fragile]
    \frametitle{Tuple par compréhension}

    \begin{aretenir}[]
    Un tuple est non mutable. Il est cependant possible de construire un tuple par compréhension.
    \end{aretenir}
\begin{center}
\begin{lstlisting}[language=Python , basicstyle=\ttfamily\small, xleftmargin=2em, xrightmargin=2em]
t = tuple(i for i in range(10))
\end{lstlisting}
\captionof{code}{Tuple de dix entiers}
\label{CODE}
\end{center}
\end{frame}
\section{Dictionnaire par compréhension}
\begin{frame}[fragile]
    \frametitle{Dictionnaire par compréhension}

    \begin{center}
    \begin{lstlisting}[language=Python , basicstyle=\ttfamily\small, xleftmargin=2em, xrightmargin=2em]
from random import randint

dico = {lettre: randint(1, 10) for lettre in ["a", "b", "c", "d", "e"]}
\end{lstlisting}
    \captionof{code}{Création d'un dictionnaire}
    \label{dico}
    \end{center}
\begin{activite}
Que fait le code \ref{dico}?
\end{activite}
\end{frame}
\begin{frame}[fragile]
    \frametitle{Correction}

\begin{lstlisting}[language=Python , basicstyle=\ttfamily\small, xleftmargin=2em, xrightmargin=2em]
>>> dico
{'a': 8, 'b': 9, 'c': 1, 'd': 8, 'e': 1}
\end{lstlisting}

\end{frame}
\end{document}