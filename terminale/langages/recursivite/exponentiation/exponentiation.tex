\documentclass[a4paper,11pt]{article}
\input{/home/tof/Documents/Cozy/latex-include/preambule_doc.tex}
\input{/home/tof/Documents/Cozy/latex-include/preambule_commun.tex}
\newcommand{\showprof}{show them}  % comment this line if you don't want to see todo environment
\setlength{\fboxrule}{0.8pt}
\fancyhead[L]{\fbox{\Large{\textbf{Rec 01}}}}
\fancyhead[C]{\textbf{Exponentiation rapide}}
\newdate{madate}{10}{09}{2020}
%\fancyhead[R]{\displaydate{madate}} %\today
\fancyhead[R]{Terminale - NSI}
\fancyfoot[L]{\vspace{1mm}Christophe Viroulaud}
\AtEndDocument{\label{lastpage}}
\fancyfoot[C]{\textbf{Page \thepage/\pageref{lastpage}}}
\fancyfoot[R]{\includegraphics[width=2cm,align=t]{/home/tof/Documents/Cozy/latex-include/cc.png}}

\begin{document}
\begin{center}
    \framebox{Comment calculer la puissance d'un nombre de manière optimisée?}
\end{center}
\section{Étude de la fonction native}
\subsection{Fonctions Python "buit-in"}
\begin{center}
    \begin{lstlisting}[language=Python , basicstyle=\ttfamily, xleftmargin=2em, xrightmargin=2em]
    def puissance_star(x:int,n:int)->int:
        return x**n
    
    def puissance_builtin(x:int,n:int)->int:
        return pow(x,n)
    \end{lstlisting}
    \captionof{code}{Fonctions natives}
    \label{builtin}
    \end{center}
\subsection{Tester un programme}
\begin{aretenir}[]
    La programmation \emph{défensive} consiste à anticiper les problèmes éventuels. On peut utiliser des \textbf{assertions}.
\end{aretenir}
Il existe plusieurs modules permettant d'automatiser les tests: \textbf{\texttt{doctest}} est un exemple.
\section{Implémenter la fonction puissance}
\subsection{S'appuyer sur la définition mathématique}
$$a^n=\underbrace{a ×....× a}_{n fois}\qquad et \qquad a^0=1$$
\begin{center}
    \begin{lstlisting}[language=Python , basicstyle=\ttfamily, xleftmargin=2em, xrightmargin=2em]
def puissance_perso(x:int,n:int)->int:
    """
    >>> puissance_perso(2,8)
    256
    >>> puissance_perso(2,9)
    512
    """
    res = 1
    for i in range(n):
        res*=x
    return res
\end{lstlisting}
    \end{center}
\subsection{Correction de l'algorithme}
\begin{aretenir}[]
    Un \textbf{invariant de boucle} est une propriété qui si elle est vraie avant l’exécution d’une itération le demeure après l’exécution de l’itération.
\end{aretenir}
\section{Formulations récursives}
\subsection{Notation mathématique}
$$
    puissance(x,n) = \left\{
        \begin{array}{ll}
            1 & \mbox{si } n=0 \\
            x.puissance(x,n-1) & \mbox{si } n>0
        \end{array}
    \right.
    $$
    \begin{aretenir}[]
        Une fonction récursive:
        \begin{itemize}
            \item s'appelle elle-même,
            \item possède un \textbf{cas limite} pour stopper les appels.
        \end{itemize}
        \end{aretenir}
\begin{center}
    \begin{lstlisting}[language=Python , basicstyle=\ttfamily, xleftmargin=2em, xrightmargin=1em]
def puissance_recursif(x: int, n: int) -> int:
    if n == 0: # cas limite
        return 1
    else: # appel récursif
        return x*puissance_recursif(x, n-1)
\end{lstlisting}
    \captionof{code}{Traduction de la formule mathématique}
    \label{CODE}
    \end{center}
    \begin{aretenir}[]
        La \textbf{pile d'appels} stocke les appels successifs de la fonction récursive.
        \end{aretenir}
\subsection{Nouvelle formulation mathématique}
$$
puissance(x,n) = \left\{
    \begin{array}{ll}
        1 & \mbox{si } n=0 \\
        puissance(x*x,n/2) & \mbox{si } n>0 \mbox{ et n pair}\\
        x.puissance(x*x,(n-1)/2) & \mbox{si } n>0 \mbox{ et n impair}\
    \end{array}
\right.$$
\begin{center}
    \begin{lstlisting}[language=Python , basicstyle=\ttfamily\small, xleftmargin=0.5em, xrightmargin=0.5em]
def puissance_recursif_rapide(x: int, n: int) -> int:
    if n == 0: # cas limite
        return 1
    elif n % 2 == 0: # pair
        return puissance_recursif_rapide(x*x, n//2)
    else: # impair
        return x*puissance_recursif_rapide(x*x, n//2)
\end{lstlisting}
    \captionof{code}{Exponentiation rapide}
    \label{CODE}
    \end{center}
\end{document}