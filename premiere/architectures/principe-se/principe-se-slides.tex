\documentclass[svgnames,11pt]{beamer}
\input{/home/tof/Documents/Cozy/latex-include/preambule_commun.tex}
\input{/home/tof/Documents/Cozy/latex-include/preambule_beamer.tex}
%\usepackage{pgfpages} \setbeameroption{show notes on second screen=left}
\author[]{Christophe Viroulaud}
\title{Principe d'un\\système d'exploitation}
\date{\framebox{\textbf{ArchMat 08}}}
%\logo{}
\institute{Première - NSI}

\begin{document}
\begin{frame}
\titlepage
\end{frame}
\begin{frame}[fragile]
    \frametitle{}

    \begin{center}
        \begin{lstlisting}[language=Bash , basicstyle=\small, xleftmargin=2em, xrightmargin=2em]
MOV R0, #3
ADD R1,R0,#5
HALT
\end{lstlisting}
\captionof{code}{\centering Pour communiquer directement avec le processeur il faut utiliser un langage de bas-niveau.}
    \end{center}

\end{frame}
\begin{frame}
    \frametitle{}

    \begin{itemize}
        \item <1-> Chaque processeur accepte un langage différent.
        \item <2-> Chaque matériel (carte graphique, webcam, clavier\dots) possède des caractéristiques différentes.
    \end{itemize}

\end{frame}
\begin{frame}
    \frametitle{}

    \begin{aretenir}[]
    Pour réaliser un programme exécutable sur différentes machines et avec différents matériels, il faut créer une version pour chaque combinaison de périphériques.
    \\
    C'est une tâche énorme au vu de la multitude des matériels existants.
    \end{aretenir}

\end{frame}
\begin{frame}
    \frametitle{}

    \begin{framed}
        \centering Comment gérer les accès matériels de manière transparente?
    \end{framed}

\end{frame}
\section{Le système d'exploitation}
\begin{frame}
    \frametitle{Le système d'exploitation}

    \begin{activite}
        Regarder la vidéo à l'adresse suivante puis répondre aux questions
        \begin{center}
        \url{https://www.youtube.com/watch?v=SpCP2oaCx8A}
        \end{center}
        \begin{enumerate}
        \item Où situer le système d'exploitation par rapport au modèle de von Neumann?
        \item Qu'est-ce que la mémoire virtuelle?
        \item Quels sont les rôles du système d'exploitation? Donner une réponse détaillée.
        \item Citer plusieurs systèmes d'exploitation.
        \end{enumerate}
        \end{activite}
    

\end{frame}
\section{Le système UNIX}
\begin{frame}
    \frametitle{Le système UNIX}

    \begin{activite}
        Regarder la frise chronologique à l'adresse suivante puis répondre aux questions. Il peut être utile d'effectuer des recherches supplémentaires.
        \begin{center}
        \href{https://cdn.knightlab.com/libs/timeline3/latest/embed/index.html?source=1aroW_4ry_iAT3EiVA8aiR6DhcdzIwMU2uymPzkb7UAs&font=Default&lang=fr&initial_zoom=2&height=650}
        {https://vu.fr/vgxU}\end{center}
        \begin{enumerate}
        \item Grace Hopper a connu le premier bug. Détailler cette histoire.
        \item Qu'est-ce-qu'un logiciel open-source?
        \item Quel système d'exploitation a été crée à partir du MS-DOS de Bill Gates?
        \item Qui est Linus Torvalds?
        \item Est-il correct de dire que Debian est un système d'exploitation de type Linux?
        \end{enumerate}
        \end{activite}

\end{frame}
\end{document}