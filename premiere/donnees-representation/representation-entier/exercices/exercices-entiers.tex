\documentclass[a4paper,11pt]{article}
\input{/home/tof/Documents/Cozy/latex-include/preambule_doc.tex}
\input{/home/tof/Documents/Cozy/latex-include/preambule_commun.tex}
\newcommand{\showprof}{show them}  % comment this line if you don't want to see todo environment
\setlength{\fboxrule}{0.8pt}
\fancyhead[L]{\fbox{\Large{\textbf{DonRep 03}}}}
\fancyhead[C]{\textbf{Représentation des entiers - exercices}}
\newdate{madate}{10}{09}{2020}
%\fancyhead[R]{\displaydate{madate}} %\today
\fancyhead[R]{Première - NSI}
\fancyfoot[L]{\vspace{1mm}Christophe Viroulaud}
\AtEndDocument{\label{lastpage}}
\fancyfoot[C]{\textbf{Page \thepage/\pageref{lastpage}}}
\fancyfoot[R]{\includegraphics[width=2cm,align=t]{/home/tof/Documents/Cozy/latex-include/cc.png}}

\begin{document}
\begin{aretenir}[Humour]
    \begin{center}
        {\Large Le monde est partagé en 10 catégories: les informaticiens et les autres.}
    \end{center}
\end{aretenir}
\begin{exo}
    \begin{enumerate}
        \item Écrire en base 2 les entiers de 0 à 10.
        \item Donner la représentation en base de 2 et sur 8 bits des entiers 14, 222, 42, 79.
    \end{enumerate}
\end{exo}
\begin{exo}
Écrire un programme qui convertit un entier positif donné en sa représentation binaire.
\end{exo}
\begin{exo}
    Donner la représentation décimale des nombres binaires (non signés) suivants:
    \begin{itemize}
        \item 1010
        \item 111110
        \item 100101001
    \end{itemize}
\end{exo}
\begin{exo}
    Donner la représentation hexadécimale des nombres binaires suivants:
    \begin{itemize}
        \item 10010101
        \item 11010101
        \item 100010001
        \item 11001101001010
    \end{itemize}
\end{exo}
\begin{exo}
    \begin{enumerate}
        \item Donner la représentation binaire des nombres hexadécimaux: AA, BB8
        \item Quelle est la valeur en base 10 de l'entier qui s'écrit BEEF en base 16?
    \end{enumerate}
\end{exo}
\begin{exo}
    Donner, si c'est possible, la représentation en complément à 2 sur 8 bits des entiers suivants: -10, -128, -42, 97.
\end{exo}
\begin{exo}
    Donner en base décimale la valeur des octets signés suivants:
    \begin{itemize}
        \item 11100111
        \item 11000001
    \end{itemize}
\end{exo}
\begin{exo}
    Effectuer les additions suivantes en base 2:
    \begin{enumerate}
        \item $39+110$
        \item $-53+35$
        \item $119-8$
        \item $19-93$
    \end{enumerate}
\end{exo}
\begin{exo}
    Comme pour les unités de masse, de longueur, il est pratique de convertir la capacité mémoire d'un ordinateur.
    \begin{center}
    1 kilooctet = 1000 octets
    \end{center}
    Bob achète un disque dur de 500Go. Il le branche sur son ordinateur et vérifie sa capacité. Le système d'exploitation \emph{Windows} lui annonce une capacité de 465Go. Comment expliquer cet affichage?
    \\La page \url{https://tinyurl.com/octet-conversion} peut fournir des indications.
\end{exo}
\end{document}