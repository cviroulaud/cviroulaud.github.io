\documentclass[a4paper,11pt]{article}
\input{/home/tof/Documents/Cozy/latex-include/preambule_lua.tex}
\newcommand{\showprof}{show them}  % comment this line if you don't want to see todo environment
\fancyhead[L]{Correction exercices modèle relationnel}
\newdate{madate}{10}{09}{2020}
%\fancyhead[R]{\displaydate{madate}} %\today
%\fancyhead[R]{Seconde - SNT}
%\fancyhead[R]{Première - NSI}
\fancyhead[R]{Terminale - NSI}
\fancyfoot[L]{~\\Christophe Viroulaud}
\AtEndDocument{\label{lastpage}}
\fancyfoot[C]{\textbf{Page \thepage/\pageref{lastpage}}}
\fancyfoot[R]{\includegraphics[width=2cm,align=t]{/home/tof/Documents/Cozy/latex-include/cc.png}}

\begin{document}
\begin{Form}
\begin{exo}
\begin{enumerate}
\item Annuaires(nom \emph{String}, prenom \emph{String}, \underline{telephone \emph{String}})\\
Il peut être plus judicieux de définir le numéro de téléphone en String car cela permettra de prendre en compte ceux qui commencent par un +.
\item
\begin{enumerate}
\item Oui, l'ensemble vide est une relation valide.
\item Oui.
\item Non, car les deux numéros de téléphone sont identiques.
\item Oui, car les noms et prénoms ne forment pas une clé primaire.
\item Non, car la seconde entité est mal formée.
\item Non, car le téléphone est un entier et non une chaîne de caractère.
\end{enumerate}
\end{enumerate}
\end{exo}
\begin{exo}
\begin{enumerate}
\item La clé primaire \emph{nom-prenom} ne permet pas de garantir l'unicité. On peut utiliser un identifiant supplémentaire (numéro d'étudiant). La remarque est valable également dans la relation \emph{Notes}. 
\begin{itemize}
\item Eleves(nom \emph{String}, prenom \emph{String}, \underline{num\_etudiant \emph{String}})
\item Matieres(intitule \emph{String}, \underline{id \emph{Integer}})
\item Notes(\underline{\dashuline{num\_etudiant} \emph{String}, \dashuline{id\_matiere} \emph{Integer}}, notes \emph{Integer})
\end{itemize}
Les attributs \emph{num\_etudiant} et \emph{id\_matiere} sont des clés étrangères qui référencent respectivement le numéro étudiant de l'élève et l'identifiant de la matière. Il ne peut y avoir qu'une note par matière existante et par élève existant.
\item 
\begin{itemize}
\item $\{("Dupont", "Jean", "A2309"), ("Durand", "Jacques", "ER450")\}$
\item $\{("NSI", 1), ("EPS", 2)\}$
\item $\{("A2309", 1, 18), ("A2309", 2, 15)\}$
\end{itemize}
\end{enumerate}
\end{exo}
\begin{exo}
\begin{enumerate}
\item \begin{itemize}
\item Stations(nom \emph{String}, \underline{id \emph{Integer}})
\item Lignes(\underline{numero \emph{Integer}}, couleur \emph{String})
\end{itemize}
La réflexion initiale du schéma est importante. A priori, définir le domaine du \emph{numéro} de la ligne peu paraître correct. Cependant si une ligne annexe se construit et que l'on nommerait 13-A et 13-B, le modèle ne tient plus.
\item Horaires(\underline{\dashuline{numero} \emph{Integer}, \dashuline{id\_station} \emph{Integer}, heure \emph{Integer}})\\
Nous pourrions également décider de définir un \emph{id} pour clé primaire.
\end{enumerate}

\end{exo}
\end{Form}
\end{document}