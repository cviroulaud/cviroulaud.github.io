\documentclass[svgnames,11pt]{beamer}
\input{/home/tof/Documents/Cozy/latex-include/preambule_commun.tex}
\input{/home/tof/Documents/Cozy/latex-include/preambule_beamer.tex}
%\usepackage{pgfpages} \setbeameroption{show notes on second screen=left}
\author[]{Christophe Viroulaud}
\title{TP les fractions}
\date{}
%\logo{}
\institute{Terminale - NSI}

\begin{document}
\begin{frame}
    \titlepage
\end{frame}
\begin{frame}
    \frametitle{}

    En Python, la méthode \texttt{\textbf{\_\_init\_\_ }}est nommée \emph{constructeur}. Elle est appelée automatiquement lors de l'instanciation de l'objet.
    Il existe un certain nombre d'autres méthodes particulières, chacune avec un nom prédéfini et entourée d'une paire de \_.
    Tous les opérateurs classiques de Python peuvent être ainsi redéfinis pour un objet:
    \begin{itemize}
        \item \emph{\_\_eq\_\_} définit l'opérateur \emph{==}
        \item \emph{\_\_lt\_\_} définit l'opérateur \emph{<}
        \item \emph{\_\_contains\_\_} définit l'opérateur \emph{in}
    \end{itemize}
    On peut retrouver une liste exhaustive sur \url{https://docs.python.org/fr/3/library/operator.html}

\end{frame}
\begin{frame}
    \frametitle{}

    On définit une classe \emph{Fraction} pour représenter un nombre rationnel. Cette classe possède deux attributs \emph{numerateur} et \emph{denominateur}. Le dénominateur doit être strictement positif.
    \begin{enumerate}
        \item Écrire le constructeur de cette classe. Il doit lever une ValueError si le dénominateur n'est pas strictement positif.
        \item Définir la méthode \emph{\_\_str\_\_} qui renvoie une chaîne de caractère de la forme "12 / 7" ou simplement "12" si le dénominateur est égal à 1.
        \item Définir les méthodes \emph{\_\_eq\_\_} et \emph{\_\_lt\_\_} qui reçoivent une deuxième fraction en argument et renvoient True si la première fraction représente respectivement un nombre égal ou strictement inférieur à la deuxième fraction.
        \item Définir les méthodes \emph{\_\_add\_\_} et \emph{\_\_mul\_\_} qui reçoivent une deuxième fraction en argument et renvoient une nouvelle fraction représentant respectivement la somme et le produit des deux fractions.
    \end{enumerate}

\end{frame}
\end{document}