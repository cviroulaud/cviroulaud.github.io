\documentclass[svgnames,11pt]{beamer}
\input{/home/tof/Documents/Cozy/latex-include/preambule_commun.tex}
\input{/home/tof/Documents/Cozy/latex-include/preambule_beamer.tex}
%\usepackage{pgfpages} \setbeameroption{show notes on second screen=left}
\author[]{Christophe Viroulaud}
\title{Exercices récursivité\\Correction}
\date{}
%\logo{}
\institute{Terminale - NSI}

\begin{document}
\begin{frame}
\titlepage
\end{frame}
\section{Exercice 1}
\begin{frame}[fragile]
    \frametitle{Exercice 1}

\begin{center}
\begin{lstlisting}[language=Python , basicstyle=\ttfamily\small, xleftmargin=2em, xrightmargin=2em]
COMPTEUR = 0

def puissance_perso(x: int, n: int) -> int:
    global COMPTEUR
    res = 1
    for i in range(n):
        COMPTEUR += 1
        res *= x
    return res

puissance_perso(2701, 19406)
print("perso: ", COMPTEUR)
\end{lstlisting}
\captionof{code}{La fonction effectue 19406 itérations.}
\end{center}

\end{frame}
\begin{frame}[fragile]
    \frametitle{}

\begin{center}
\begin{lstlisting}[language=Python , basicstyle=\ttfamily\small, xleftmargin=2em, xrightmargin=1em]
def puissance_recursif(x: int, n: int) -> int:
    global COMPTEUR
    if n == 0:
        return 1
    else:
        COMPTEUR += 1
        return x*puissance_recursif(x, n-1)
\end{lstlisting}
\captionof{code}{La fonction effectue  également 19406 itérations.}
\label{CODE}
\end{center}
\begin{aretenir}[Remarque]
Ne pas oublier de remettre \textbf{\texttt{COMPTEUR}} à 0.
\end{aretenir}
\end{frame}
\begin{frame}[fragile]
    \frametitle{}

\begin{center}
\begin{lstlisting}[language=Python , basicstyle=\ttfamily\small, xleftmargin=1em, xrightmargin=1em]
def puissance_recursif_rapide(x, n):
    global COMPTEUR
    if n == 0:
        return 1
    elif n % 2 == 0:
        COMPTEUR += 1
        return puissance_recursif_rapide(x*x, n//2)
    else:
        COMPTEUR += 1
        return x*puissance_recursif_rapide(x*x, n//2)
\end{lstlisting}
\captionof{code}{Il n'y a que 15 itérations.}
\label{CODE}
\end{center}

\end{frame}
\begin{frame}[fragile]
    \frametitle{}

\begin{center}
\begin{lstlisting}[language=Python , basicstyle=\ttfamily\small, xleftmargin=1em, xrightmargin=1em]
def puissance_iteratif_rapide(x, n):
    global COMPTEUR
    res = 1
    while n > 0:
        COMPTEUR += 1
        if n % 2 != 0: #impair
            res = res * x
        x = x*x
        n = n//2
    return res
\end{lstlisting}
\captionof{code}{Version itérative}
\label{CODE}
\end{center}

\end{frame}
\section{Exercice 2}
\begin{frame}
    \frametitle{Exercice 2}
    $$
    somme(n) = \left\{
        \begin{array}{ll}
            0 & \mbox{si  }n=0\\
            \mbox{n + somme(n-1)} & \mbox{si }n>0\
        \end{array}
    \right.
    $$
   

\end{frame}
\begin{frame}[fragile]
    \frametitle{}

\begin{center}
    \begin{lstlisting}[language=Python , basicstyle=\ttfamily\small, xleftmargin=1em, xrightmargin=1em]
def somme(n: int) -> int:
    if n == 0:
        return 0
    else:
        return n + somme(n-1)  
\end{lstlisting}
    \captionof{code}{Somme des entiers}
    \label{CODE}
    \end{center} 

\end{frame}
\begin{frame}[fragile]
    \frametitle{}

\begin{aretenir}[Hors programme]
    Une fonction à récursivité terminale est une fonction où l'appel récursif est la dernière instruction à être évaluée.
\end{aretenir}
\begin{center}
\begin{lstlisting}[language=Python , basicstyle=\ttfamily\small, xleftmargin=1em, xrightmargin=1em]
def somme_terminale(n: int, res: int) -> int:
    if n == 0:
        return res
    else:
        return somme_terminale(n-1, res+n)
\end{lstlisting}
\captionof{code}{Version \emph{terminale}}
\label{CODE}
\end{center}
\note[item]{Dans les langages fonctionnels, le compilateur ou l'interpréteur reconnait et optimise une récursivité terminale (en transformant en itération).}
\end{frame}
\section{Exercice 3}
\begin{frame}
    \frametitle{Exercice 3}
    $$
    n! = \left\{
        \begin{array}{ll}
            1 & \mbox{si  }n=0\\
            \mbox{n × (n-1)!} & \mbox{si }n>0\
        \end{array}
    \right.
    $$
    

\end{frame}
\begin{frame}[fragile]
    \frametitle{}

\begin{center}
\begin{lstlisting}[language=Python , basicstyle=\ttfamily\small, xleftmargin=1em, xrightmargin=1em]
def factorielle(n: int)->int:
    if n == 0:
        return 1
    else:
        return n * factorielle(n-1)
\end{lstlisting}
\end{center}

\end{frame}
\section{Exercice 4}
\begin{frame}[fragile]
    \frametitle{Exercice 4}

\begin{center}
\begin{lstlisting}[language=Python , basicstyle=\ttfamily\small, xleftmargin=2em, xrightmargin=2em]
def syracuse(u: int) -> None:
    print(u, end=" ")
    if u > 1:
        if u % 2 == 0:
            syracuse(u // 2)
        else:
            syracuse(3 * u + 1)

print(syracuse(5))
\end{lstlisting}
\end{center}

\end{frame}
\begin{frame}[fragile]
    \frametitle{}

\begin{center}
\begin{lstlisting}[language=Python , basicstyle=\ttfamily\small, xleftmargin=2em, xrightmargin=2em]
def syracuse2(u: int, l: list) -> list:
    l.append(u)
    if u > 1:
        if u % 2 == 0:
            syracuse2(u // 2, l)
        else:
            syracuse2(3 * u + 1, l)
    return l

print(syracuse2(5, []))
\end{lstlisting}
\captionof{code}{Version avec renvoi des valeurs dans un tableau.}
\label{CODE}
\end{center}
\note{quelle que soit la valeur de $u_0$ il existe un \emph{n} tel que $u_n=1$. Toujours pas prouvée à ce jour.}
\end{frame}
\section{Exercice 5}
\begin{frame}[fragile]
    \frametitle{Exercice 5}

\begin{center}
\begin{lstlisting}[language=Python , basicstyle=\ttfamily\small, xleftmargin=2em, xrightmargin=2em]
def entiers(i: int, k: int) -> None:
    if i <= k:
        print(i, end=" ")
        entiers(i+1, k)
\end{lstlisting}
\end{center}  

\end{frame}
\begin{frame}[fragile]
    \frametitle{}

\begin{center}
\begin{lstlisting}[language=Python , basicstyle=\ttfamily\small, xleftmargin=2em, xrightmargin=2em]
def impairs(i: int, k: int) -> None:
    if i <= k:
        if i % 2 == 1:
            print(i, end=" ")
        impairs(i+2, k)
\end{lstlisting}
\end{center}

\end{frame}
\section{Exercice 6}
\begin{frame}[fragile]
    \frametitle{Exercice 6}

\begin{center}
\begin{lstlisting}[language=Python , basicstyle=\ttfamily\small, xleftmargin=2em, xrightmargin=2em]
from random import randint

t = [randint(1, 100) for _ in range(10)]
\end{lstlisting}
\end{center}

\end{frame}
\begin{frame}[fragile]
    \frametitle{}

\begin{center}
\begin{lstlisting}[language=Python , basicstyle=\ttfamily\small, xleftmargin=2em, xrightmargin=2em]
def somme(tab: list) -> int:
    s = 0
    for i in range(len(tab)):
        s += tab[i]
    return s

somme(t)
\end{lstlisting}
\end{center}

\end{frame}
\begin{frame}[fragile]
    \frametitle{}

\begin{center}
\begin{lstlisting}[language=Python , basicstyle=\ttfamily\small, xleftmargin=1em, xrightmargin=0.5em]
def somme_rec(tab: list, deb: int, s: int) -> int:
    """
    calcule la somme des éléments du tableau

    Args:
        tab (list): le tableau
        deb (int): indice de l'élément en cours
        s (int): somme
    """
    if deb == len(tab):
        return s
    else:
        return somme_rec(tab, deb+1, s+tab[deb])

somme_rec(t, 0, 0)

\end{lstlisting}
\captionof{code}{Version récursive}
\label{CODE}
\end{center}

\end{frame}
\begin{frame}[fragile]
    \frametitle{}

\begin{center}
\begin{lstlisting}[language=Python , basicstyle=\ttfamily\small, xleftmargin=2em, xrightmargin=2em]
def somme_rec(tab: list, s: int) -> int:
    if len(tab) == 0:
        return s
    else:
        tete = tab[0]
        queue = tab[1:] # slice
        return somme_rec(queue, s+tete)

somme_rec(t, 0)
\end{lstlisting}
\captionof{code}{Autre version récursive avec slice}
\end{center}
\begin{aretenir}[Remarque]
Le \emph{slice} est hors programme.
\end{aretenir}
\end{frame}
\section{Exercice 7}
\begin{frame}[fragile]
    \frametitle{Exercice 7}

\begin{center}
\begin{lstlisting}[language=Python , basicstyle=\ttfamily\small, xleftmargin=2em, xrightmargin=2em]
t = [randint(1, 1000) for _ in range(30)]
\end{lstlisting}
\end{center}

\end{frame}
\begin{frame}[fragile]
    \frametitle{}

\begin{center}
\begin{lstlisting}[language=Python , basicstyle=\ttfamily\small, xleftmargin=2em, xrightmargin=2em]
def mini(tab: list) -> int:
    m = float("inf")
    for i in range(len(tab)):
        if tab[i] < m:
            m = tab[i]
    return m

mini(t)
\end{lstlisting}
\end{center}

\end{frame}
\begin{frame}[fragile]
    \frametitle{}

\begin{center}
\begin{lstlisting}[language=Python , basicstyle=\ttfamily\small, xleftmargin=1em, xrightmargin=1em]
def mini_rec(tab: list, deb: int, m: int) -> int:
    """
    cherche le plus petit élément du tableau

    Args:
        tab (list): le tableau
        deb (int): indice de l'élément en cours
        m (int): l'élément mini
    """
    if deb == len(tab):
        return m
    else:
        if tab[deb] < m:
            m = tab[deb]
        return mini_rec(tab, deb+1, m)

mini_rec(t, 0, float("inf"))
\end{lstlisting}
\end{center}

\end{frame}
\begin{frame}[fragile]
    \frametitle{}

\begin{center}
\begin{lstlisting}[language=Python , basicstyle=\ttfamily\small, xleftmargin=2em, xrightmargin=2em]
def mini_rec2(tab: list, m: int) -> int:
    """
    avec slice
    """
    if len(tab) == 0:
        return m
    else:
        if tab[0] < m:
            m = tab[0]
        return mini_rec2(tab[1:], m)

mini_rec2(t, float("inf"))
\end{lstlisting}
\captionof{code}{Avec slice}
\label{CODE}
\end{center}

\end{frame}
\section{Exercice 8}
\begin{frame}
    \frametitle{Exercice 8}
    \begin{center}
        \begin{tabular}{l}
    $a=20, b=35$\\
            $35 = \overbrace{20}^{a\rightarrow b}×1 + \overbrace{15}^{b\%a\rightarrow a}$\\
    20 = 15×1 + 5\\
    15 = 5×3 + 0\\
    pgcd = 5
        \end{tabular}
    \end{center}


\end{frame}
\begin{frame}[fragile]
    \frametitle{}

    \begin{center}
        \begin{lstlisting}[language=Python , basicstyle=\ttfamily\small, xleftmargin=2em, xrightmargin=2em]
def pgcd(a: int, b: int) -> int:
    while a != 0:
        a, b = b % a, a
    return b
\end{lstlisting}
        \captionof{code}{Version itérative}
        \label{CODE}
        \end{center}

\end{frame}
\begin{frame}[fragile]
    \frametitle{}

    \begin{center}
        \begin{lstlisting}[language=Python , basicstyle=\ttfamily\small, xleftmargin=2em, xrightmargin=2em]
def pgcd_rec(a: int, b: int) -> int:
    if a == 0:
        return b
    else:
        return pgcd_rec(b % a, a)
\end{lstlisting}
        \captionof{code}{Version récursive}
        \label{CODE}
        \end{center}

\end{frame}
\section{Exercice 9}
\begin{frame}[fragile]
    \frametitle{Exercice 9}

\begin{center}
\begin{lstlisting}[language=Python , basicstyle=\ttfamily\small, xleftmargin=2em, xrightmargin=2em]
def nombre_chiffres(n: int) -> int:
    if n < 10:
        return 1
    else:
        return 1 + nombre_chiffres(n//10)
\end{lstlisting}
\captionof{code}{}
\label{CODE}
\end{center}

\end{frame}
\begin{frame}[fragile]

\begin{center}
\begin{lstlisting}[language=Python , basicstyle=\ttfamily\small, xleftmargin=2em, xrightmargin=2em]
def nombre_chiffres_terminal(n: int, acc: int) -> int:
    if n < 10:
        return acc
    else:
        return nombre_chiffres_terminal(n//10, acc+1)
\end{lstlisting}
\captionof{code}{Version \emph{terminale}}
\label{CODE}
\end{center}

\end{frame}
\section{Exercice 10}
\begin{frame}[fragile]
    \frametitle{Exercice 10}

\begin{center}
\begin{lstlisting}[language=Python , basicstyle=\ttfamily\small, xleftmargin=2em, xrightmargin=2em]
def C(n: int, p: int) -> int:
    if p == 0 or n == p:
        return 1
    else:
        return C(n-1, p-1) + C(n-1, p)
\end{lstlisting}
\end{center}

\end{frame}
\begin{frame}[fragile]
    \frametitle{Exercice 10}

\begin{center}
\begin{lstlisting}[language=Python , basicstyle=\ttfamily\small, xleftmargin=2em, xrightmargin=2em]
for n in range(10): # chaque ligne
    for p in range(n+1): # chaque élément de la ligne
        print(C(n, p), end=" ")
    print() # saut de ligne
\end{lstlisting}
\end{center}

\end{frame}
\end{document}