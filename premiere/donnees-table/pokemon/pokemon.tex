\documentclass[a4paper,11pt]{article}
\input{/home/tof/Documents/Cozy/latex-include/preambule_doc.tex}
\input{/home/tof/Documents/Cozy/latex-include/preambule_commun.tex}
\newcommand{\showprof}{show them}  % comment this line if you don't want to see todo environment
\setlength{\fboxrule}{0.8pt}
\fancyhead[L]{\fbox{\Large{\textbf{TraitDon 02}}}}
\fancyhead[C]{\textbf{Pokémon Go}}
\newdate{madate}{10}{09}{2020}
%\fancyhead[R]{\displaydate{madate}} %\today
%\fancyhead[R]{Seconde - SNT}
\fancyhead[R]{Première - NSI}
%\fancyhead[R]{Terminale - NSI}
\fancyfoot[L]{\vspace{1mm}Christophe Viroulaud}
\AtEndDocument{\label{lastpage}}
\fancyfoot[C]{\textbf{Page \thepage/\pageref{lastpage}}}
\fancyfoot[R]{\includegraphics[width=2cm,align=t]{/home/tof/Documents/Cozy/latex-include/cc.png}}

\begin{document}
% fournir pokedex.csv et dossier photo
\section{Problématique}
Le jeu pour smartphone \emph{Pokemon Go} reprend l'univers du manga éponyme. Il utilise la réalité augmentée pour donner une expérience utilisateur nouvelle. Devant le succès du jeu des communautés se créent et tentent d'établir des stratégies pour optimiser leurs résultats.

\begin{center}
    \framebox{On se propose de construire un programme pour aider les joueurs dans leurs quêtes.}
\end{center}
\section{Informations disponibles}
Quand on joue à \emph{Pokémon Go} on trouve des Pokémon sur notre chemin, mais également des œufs. Il faut parcourir une certaine distance pour faire éclore un œuf. Enfin, il est possible de faire évoluer un Pokémon à l'aide de friandises.

Un fichier de données (\emph{pokedex.csv}) recense l'ensemble des Pokémons utilisables dans le jeu. 
%détail dans slide
\begin{itemize}
    \item num: Number of the Pokémon in the official Pokédex
    \item name: Pokémon name
    \item img: URL to an image of this Pokémon
    \item type: Pokémon type
    \item height: Pokémon height
    \item weight: Pokémon weight
    \item candy: type of candy used to evolve Pokémon or given when transfered
    \item candy\_count: amount of candies required to evolve
    \item egg: Number of kilometers to travel to hatch the egg
    \item weakness: Types of Pokémon this Pokémon is weak to
    \item next\_evolution: Number of evolution of Pokémon
\end{itemize}
\begin{activite}
\begin{enumerate}
    \item Ouvrir le fichier \emph{pokedex.csv} pour observer les données fournies.
    \item Créer un programme \emph{pokemon.py}.
    \item Importer les données du pokedex dans un tableau de dictionnaires \textbf{\texttt{pokedex}}.
\end{enumerate}
\end{activite}
\section{Interface graphique}
% L'interface permet à l'utilisateur d'interagir avec les données.
\subsection{Bibliothèque tkinter}
La bibliothèque tkinter fournit des \emph{composants (widgets)} pour construire une interface graphique simple.
\begin{center}
\begin{lstlisting}[language=Python]
import tkinter
from tkinter import ttk

fenetre = tkinter.Tk()
fenetre.title("Pokemon Go")

# dernière ligne du programme: met à jour les variables 
fenetre.mainloop()
\end{lstlisting}
\captionof{code}{Créer une fenêtre d'interface}
\label{interface}
\end{center}
\subsection{Ajouter un composant}
L'ajout d'un composant se déroule en trois étapes:
\begin{itemize}
    \item Créer le composant.
    \item Placer le composant dans l'interface.
    \item Remplir le composant.
\end{itemize}
\begin{center}
\begin{lstlisting}[language=Python]
etiquette = tkinter.Label(fenetre)
etiquette.pack()
etiquette["text"] = "Bonjour"
\end{lstlisting}
\captionof{code}{Placer un \emph{label}} dans la fenêtre
\label{label}
\end{center}
\begin{aretenir}[Commentaire]
La méthode \textbf{\texttt{pack}} place les éléments les uns en dessous des autres. Il existe la méthode \textbf{\texttt{grid}} qui utilise un système de coordonnées.
\end{aretenir}
\end{document}