\documentclass[svgnames,11pt]{beamer}
\input{/home/tof/Documents/Cozy/latex-include/preambule_commun.tex}
\input{/home/tof/Documents/Cozy/latex-include/preambule_beamer.tex}
%\usepackage{pgfpages} \setbeameroption{show notes on second screen=left}
\author[]{Christophe Viroulaud}
\title{Exercices SGBD manipulation\\correction}
\date{\framebox{\textbf{BDD 05}}}
%\logo{}
\institute{Terminale - NSI}

\begin{document}
\begin{frame}
    \titlepage
\end{frame}
\section{Exercice 1}
\begin{frame}
    \frametitle{Exercice 1}

    \begin{itemize}
        \item colonne, column, attribut
        \item entité, ligne, row
        \item domaine, type
        \item relation, table
        \item schéma (description d'une relation)
        \item base de données (ensemble des relations)
    \end{itemize}

\end{frame}
\section{Exercice 2}
\begin{frame}
    \frametitle{Exercice 2}

    \begin{itemize}
        \item Especes(\underline{id Integer}, nom String)
        \item Animaux(\underline{id Integer}, nom String, age Integer, \dashuline{id\_espece Integer})
        \item Soins(\underline{id Integer}, \dashuline{id\_animal Integer}, soin String)
    \end{itemize}

\end{frame}
\begin{frame}[fragile]
    \frametitle{}

    \begin{center}
    \begin{lstlisting}[language=SQL , basicstyle=\ttfamily\small, xleftmargin=.5em, xrightmargin=0em]
CREATE TABLE Especes (
        id Integer PRIMARY KEY AUTOINCREMENT, 
        nom String);

CREATE TABLE Animaux (
        id Integer PRIMARY KEY AUTOINCREMENT, 
        nom String, 
        age Integer, 
        id_espece Integer, 
        FOREIGN KEY (id_espece) REFERENCES Especes(id));

CREATE TABLE Soins (
    id Integer PRIMARY KEY AUTOINCREMENT, 
    id_animal Integer, 
    soin String,
    FOREIGN KEY (id_animal) REFERENCES Animaux(id));
\end{lstlisting}
    \captionof{code}{Création des 3 tables}
    \label{CODE}
    \end{center}

\end{frame}
\begin{frame}[fragile]
    \frametitle{}

\begin{center}
\begin{lstlisting}[language=SQL , basicstyle=\ttfamily\small, xleftmargin=1em, xrightmargin=0em]
INSERT INTO Especes (nom) VALUES 
("chien"),
("chat"),
("poisson");
\end{lstlisting}
\captionof{code}{Insertion espèces}
\label{CODE}
\end{center}

\end{frame}
\begin{frame}[fragile]
    \frametitle{}

\begin{center}
\begin{lstlisting}[language=SQL , basicstyle=\ttfamily\small, xleftmargin=1em, xrightmargin=0em]
INSERT INTO Animaux (nom, age, id_espece) VALUES 
("Minou", 15, 2),
("Tex", 8, 1),
("Rrrrr", 2, 1);
\end{lstlisting}
\captionof{code}{Insertion animaux}
\label{CODE}
\end{center}
\begin{aretenir}[Remarque]
Les identifiants des espèces peuvent varier.
\end{aretenir}
\end{frame}
\begin{frame}[fragile]
    \frametitle{}

\begin{center}
\begin{lstlisting}[language=SQL , basicstyle=\ttfamily\small, xleftmargin=1em, xrightmargin=0em]
INSERT INTO Soins (id_animal, soin) VALUES 
(2, "patte cassée - plâtre"),
(1, "fièvre - antibiotiques");
\end{lstlisting}
\captionof{code}{Insertion soins}
\label{CODE}
\end{center}

\end{frame}
\section{Exercice 3}
\begin{frame}
    \frametitle{Exercice 3}

    \begin{center}
        Soundex: algorithme phonétique d'indexation\\
        corriger les erreurs orthographiques\\
        \url{https://fr.wikipedia.org/wiki/Soundex}
    \end{center}

\end{frame}
\begin{frame}[fragile]
    \frametitle{}

\begin{center}
\begin{lstlisting}[language=SQL , basicstyle=\ttfamily\small, xleftmargin=1em, xrightmargin=0em]
SELECT * FROM Departements WHERE 
departement_code = 24;

SELECT * FROM Departements WHERE departement_nom_soundex = 'M200';

SELECT departement_nom FROM Departements WHERE departement_code < 10;

SELECT departement_code, departement_nom FROM Departements WHERE 
departement_code > 20 AND 
departement_code < 30;
\end{lstlisting}
\end{center}    

\end{frame}
\begin{frame}
    \frametitle{}

    \begin{aretenir}[Remarque]
        \begin{itemize}
            \item SQLite LIKE operator is case-insensitive. It means "A" LIKE "a" is true.
            \item However, for Unicode characters that are not in the ASCII ranges, the LIKE operator is case sensitive e.g., "Ä" LIKE "ä" is false
        \end{itemize}
    \end{aretenir}

\end{frame}
\begin{frame}[fragile]
    \frametitle{}

\begin{center}
\begin{lstlisting}[language=SQL , basicstyle=\ttfamily\small, xleftmargin=1em, xrightmargin=0em]
SELECT departement_nom FROM Departements WHERE
departement_nom LIKE '%haut%';

SELECT departement_nom FROM Departements WHERE
departement_nom NOT LIKE '%-%' AND 
departement_nom NOT LIKE '% %';
\end{lstlisting}
\end{center}    
\begin{aretenir}[Remarque]
    Il est possible de comparer des \emph{String} comme des \emph{Integer}. Le SQL est très permissif: departement\_code est de type \emph{String}, pourtant il accepte la comparaison avec un \emph{Integer}.
    \end{aretenir}
\end{frame}
\section{Exercice 4}
\begin{frame}[fragile]
    \frametitle{Exercice 4}

\begin{center}
\begin{lstlisting}[language=SQL , basicstyle=\ttfamily\small, xleftmargin=1em, xrightmargin=0em]
SELECT * FROM employees WHERE name = 'GARFIELD';

SELECT name FROM employees WHERE designation = 'TECH';

SELECT name FROM employees WHERE name LIKE 'H%';

SELECT name FROM employees WHERE hired_on > '1997-01-01';

SELECT name, salary FROM employees WHERE 
salary > 25000 AND 
salary < 55000;
\end{lstlisting}
\end{center}    

\end{frame}
\begin{frame}[fragile]

\begin{center}
\begin{lstlisting}[language=SQL , basicstyle=\ttfamily\small, xleftmargin=1em, xrightmargin=0em]
SELECT name, salary FROM employees WHERE 
salary > 25000 AND 
salary < 55000 AND 
commission IS NOT NULL;

--ou bien
SELECT name, salary FROM employees WHERE 
salary > 25000 AND 
salary < 55000 AND 
commission > 0;
\end{lstlisting}
\end{center}    
\begin{aretenir}[Remarque]
    NULL est différent de 0; commission = 0 ne renverrait rien ici
\end{aretenir}
\end{frame}
\begin{frame}[fragile]
    \frametitle{}

\begin{center}
\begin{lstlisting}[language=SQL , basicstyle=\ttfamily\small, xleftmargin=1em, xrightmargin=0em]
INSERT INTO employees (name, designation, manager, hired_on, salary, dept) VALUES ('DURAN', 'TECH', 6, '1999-01-13', 35000, 4);

UPDATE employees SET salary = 60000 WHERE 
name = 'FILLMORE';
\end{lstlisting}
\end{center}

\end{frame}
\end{document}