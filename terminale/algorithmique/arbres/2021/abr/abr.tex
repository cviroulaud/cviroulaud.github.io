\documentclass[a4paper,11pt]{article}
\input{/home/tof/Documents/Cozy/latex-include/preambule_lua.tex}
\newcommand{\showprof}{show them}  % comment this line if you don't want to see todo environment
\fancyhead[L]{Arbre binaire de recherche}
\newdate{madate}{10}{09}{2020}
%\fancyhead[R]{\displaydate{madate}} %\today
%\fancyhead[R]{Seconde - SNT}
%\fancyhead[R]{Première - NSI}
\fancyhead[R]{Terminale - NSI}
\fancyfoot[L]{\vspace{1mm}Christophe Viroulaud}
\AtEndDocument{\label{lastpage}}
\fancyfoot[C]{\textbf{Page \thepage/\pageref{lastpage}}}
\fancyfoot[R]{\includegraphics[width=2cm,align=t]{/home/tof/Documents/Cozy/latex-include/cc.png}}
\usepackage{tikz}

\begin{document}
\section{Problématique}
Les arbres binaires, les tas imposent des contraintes aux structures arborescentes. Il en résulte des objets avec des propriétés très utiles. Par exemple, la complexité du tri par tas est $O(n) = n.log(n)$.
\begin{center}
    \shadowbox{\parbox{12cm}{\centering Comment obtenir une méthode de recherche efficace avec les arbres?}}
\end{center}
\section{Arbre binaire de recherche}
\subsection{Définition}
Imposons une contrainte à chaque nœud d'un arbre binaire:
\begin{itemize}
    \item les valeurs du sous-arbre gauche sont plus petites que celle du nœud,
    \item les valeurs du sous-arbre droit sont plus grandes que celle du nœud.
\end{itemize}
\begin{center}
    \begin{tikzpicture}
        \node[draw] (A) at (1,0) {33};
        \node[draw] (B) at (-3,-1) {25};
        \node[draw] (C) at (-5,-2) {20};
        \node[draw] (D) at (-1,-2) {28};
        \node[draw] (E) at (-6,-3) {8};
        \node[draw] (F) at (-4,-3) {21};
        \node[draw] (H) at (-2,-3) {26};
        \node[draw] (I) at (4,-1) {56};
        \node[draw] (J) at (2,-2) {40};
        \node[draw] (K) at (6,-2) {60};
        \node[draw] (M) at (1,-3) {35};
        \node[draw] (O) at (5,-3) {58};
        \node[draw] (P) at (7,-3) {65};

        \draw (A) -- (B);
        \draw (C) -- (B);
        \draw (C) -- (E);
        \draw (C) -- (F);
        \draw (D) -- (B);
        \draw (D) -- (H);
        \draw (A) -- (I);
        \draw (J) -- (I);
        \draw (I) -- (K);
        \draw (J) -- (M);
        \draw (K) -- (O);
        \draw (K) -- (P);
        \draw [white] (0,-3) -- (0,-4.5);
    \end{tikzpicture}
    \captionof{figure}{Un Arbre Binaire de Recherche (\emph{ABR})}
    \label{arbre}
\end{center}
\begin{aretenir}[Remarque]
    On suppose que chaque valeur n'apparaît qu'une seule fois dans l'arbre.
\end{aretenir}
\begin{activite}
    \begin{enumerate}
        \item Placer les valeurs 23, 27, 54, 55 dans l'ABR.
        \item Où se trouve la plus grande valeur ? La plus petite?
        \item Effectuer un parcours infixe de l'arbre. Que remarque-t-on ?
    \end{enumerate}
\end{activite}
\subsection{Propriété}
Soit \emph{n} la taille de l'arbre et \emph{h} sa hauteur. Ces grandeurs sont liées par la relation :
$$h+1 \leq n \leq 2^{h+1}-1$$
Cette propriété peut s'écrire:
$$ \lfloor \log_2 n  \rfloor \leq h$$
Considérons le tableau \ref{tab} et la construction de l'arbre binaire de recherche qui en découle (figure \ref{arbre}).
\begin{center}
    \begin{lstlisting}[language=Python]
tab = [33, 25, 56, 20, 28, 40, 60, 8, 21, 26, 35, 58, 65]
\end{lstlisting}
    \captionof{code}{Tableau de données}
    \label{tab}
\end{center}
%DODO ABR mettre en avant sur un exemple la rapidité de la recherche en faisant un test
\begin{activite}
    \begin{enumerate}
        \item Quelle la complexité temporelle dans le pire des cas de la recherche d'un élément dans le tableau?
        \item Que devient cette complexité pour l'ABR?
    \end{enumerate}
\end{activite}
\begin{aretenir}[Remarque]
    Pour que la recherche dans l'arbre soit efficace il faut qu'il soit \emph{équilibré}.
\end{aretenir}
\section{Implémentation}
La programmation objet peut nous permettre d'implémenter un ABR facilement.
\begin{activite}
    \begin{enumerate}
        \item Créer la classe \emph{Noeud} qui possède trois attributs:
              \begin{itemize}
                  \item \emph{valeur} un entier,
                  \item \emph{gauche} un Noeud,
                  \item \emph{droit} un Noeud.
              \end{itemize}
              Le constructeur acceptera trois paramètres \emph{(v: int, g = None, d = None)}.
        \item Écrire la méthode \textbf{inserer(self, v: int)$\;\rightarrow\;$None} qui crée récursivement le nœud contenant la valeur \emph{v} dans le sous-arbre gauche ou droit du nœud.
        \item Écrire la méthode \textbf{rechercher(self, v: int)$\;\rightarrow\;$bool} qui renvoie \emph{True} si la valeur \emph{v} est dans le nœud ou dans un de ses sous-arbres.
        \item Créer la classe ABR et son constructeur qui possédera un attribut \emph{racine} initialisé à \emph{None}.
        \item Écrire la méthode \textbf{est\_vide(self)$\;\rightarrow\;$bool} qui renvoie \emph{True} si l'arbre est vide.
        \item Écrire la méthode \textbf{inserer(self, v: int)$\;\rightarrow\;$None} qui:
              \begin{itemize}
                  \item crée un nœud contenant \emph{v} si l'arbre est vide,
                  \item appelle la méthode \emph{inserer} du nœud racine sinon.
              \end{itemize}
        \item Écrire la méthode \textbf{rechercher(self, v: int)$\;\rightarrow\;$None} qui renvoie \emph{True} si \emph{v} est présent dans l'arbre.
    \end{enumerate}
\end{activite}
\end{document}