\documentclass[a4paper,11pt]{article}
\input{/home/tof/Documents/Cozy/latex-include/preambule_lua.tex}
\newcommand{\showprof}{show them}  % comment this line if you don't want to see todo environment
\fancyhead[L]{Interblocage - correction exercices}
\newdate{madate}{10}{09}{2020}
\fancyhead[R]{Terminale - NSI} %\today
\fancyfoot[L]{~\\Christophe Viroulaud}
\fancyfoot[C]{\textbf{Page \thepage}}
\fancyfoot[R]{\includegraphics[width=2cm,align=t]{/home/tof/Documents/Cozy/latex-include/cc.png}}

\begin{document}
\begin{Form}
\noindent Dans le dîner des philosophes, les quatre conditions de Coffman sont satisfaites:
\begin{itemize}
\item Les ressources sont en accès exclusif: deux philosophes ne peuvent pas utiliser la même fourchette en même temps.
\item Rétention et attente: chaque philosophe prend la fourchette de gauche et attend celle de droite;
\item Non préemption: un philosophe ne peut pas prendre une fourchette dans les mains d'un autre;
\item Attente cyclique: chaque philosophe attend la fourchette (occupée) à sa droite.
\end{itemize}
\bigskip
Dans le problème de circulation, les quatre conditions de Coffman sont satisfaites:
\begin{itemize}
\item Les ressources sont en accès exclusif: deux voitures ne peuvent pas se trouver sur la même portion de route en même temps.
\item Rétention et attente: chaque voiture détient une portion de route  et attend celle qui est devant.
\item Non préemption: une voiture ne peut pas forcer le passage.
\item Attente cyclique: chaque voiture attend la portion (occupée) de route devant elle.
\end{itemize}
\end{Form}
\end{document}