\documentclass[svgnames,11pt]{beamer}
\input{/home/tof/Documents/Cozy/latex-include/preambule_commun.tex}
\input{/home/tof/Documents/Cozy/latex-include/preambule_beamer.tex}
%\usepackage{pgfpages} \setbeameroption{show notes on second screen=left}
\author[]{Christophe Viroulaud}
\title{Exercices\\Construction par compréhension\\correction}
\date{\framebox{\textbf{DonRep 13}}}
%\logo{}
\institute{Première - NSI}

\begin{document}
\begin{frame}
\titlepage
\end{frame}
\begin{frame}[fragile]
    \frametitle{}

\begin{center}
\begin{lstlisting}[language=Python , basicstyle=\ttfamily\small, xleftmargin=2em, xrightmargin=2em]
>>> [1 for i in range(5)]
[1, 1, 1, 1, 1]
\end{lstlisting}
\end{center}

\end{frame}
\begin{frame}[fragile]
    \frametitle{}

\begin{center}
\begin{lstlisting}[language=Python , basicstyle=\ttfamily\small, xleftmargin=2em, xrightmargin=2em]
>>> [i for i in range(5, 10)]
[5, 6, 7, 8, 9]
\end{lstlisting}
\end{center}

\end{frame}
\begin{frame}[fragile]
    \frametitle{}

\begin{center}
\begin{lstlisting}[language=Python , basicstyle=\ttfamily\small, xleftmargin=2em, xrightmargin=2em]
>>> [i for i in range(1, 10, 2)]
[1, 3, 5, 7, 9]
\end{lstlisting}
\end{center}
\begin{aretenir}[Remarque]
Le troisème paramètre de \textbf{\texttt{range}} est le pas.
\end{aretenir}
\end{frame}
\begin{frame}[fragile]
    \frametitle{}

\begin{center}
\begin{lstlisting}[language=Python , basicstyle=\ttfamily\small, xleftmargin=2em, xrightmargin=2em]
>>> {i: i**2 for i in range(6)}
{0: 0, 1: 1, 2: 4, 3: 9, 4: 16, 5: 25}
\end{lstlisting}
\end{center}
\begin{aretenir}[Remarque]
Chaque clé est associée à son carré; exemple: pour une clé de 4 on associe 16.
\end{aretenir}
\end{frame}
\begin{frame}[fragile]
    \frametitle{}

\begin{center}
\begin{lstlisting}[language=Python , basicstyle=\ttfamily\small, xleftmargin=2em, xrightmargin=2em]
>>> [[0 for i in range(3)] for j in range(3)]
[[0, 0, 0], [0, 0, 0], [0, 0, 0]]
\end{lstlisting}
\end{center}
\begin{aretenir}[Remarque]
Le tableau \emph{externe} contient des tableaux. Les tableaux \emph{internes} contiennent des entiers.
\end{aretenir}
\end{frame}
\begin{frame}[fragile]
    \frametitle{}

\begin{center}
\begin{lstlisting}[language=Python , basicstyle=\ttfamily\small, xleftmargin=2em, xrightmargin=2em]
>>> [[i for i in range(3)] for j in range(3)]
[[0, 1, 2], [0, 1, 2], [0, 1, 2]]
\end{lstlisting}
\end{center}
\begin{aretenir}[Remarque]
Le tableau \emph{externe} contient des tableaux. Les tableaux \emph{internes} contiennent des entiers.
\end{aretenir}
\end{frame}
\begin{frame}[fragile]
    \frametitle{}

\begin{center}
\begin{lstlisting}[language=Python , basicstyle=\ttfamily\small, xleftmargin=2em, xrightmargin=2em]
>>> [[j for i in range(3)] for j in range(3)]
[[0, 0, 0], [1, 1, 1], [2, 2, 2]]
\end{lstlisting}
\end{center}
\begin{aretenir}[Remarque]
À chaque tour de boucle externe (compteur \textbf{\texttt{j}}) on crée un tableau de taille 3, qu'on remplit avec la valeur de \textbf{\texttt{j}}.
\end{aretenir}
\end{frame}
\begin{frame}[fragile]
    \frametitle{}

\begin{center}
\begin{lstlisting}[language=Python , basicstyle=\ttfamily\small, xleftmargin=2em, xrightmargin=2em]
tab = ["Bonjour", "ça", "va", "?"]
for mot in tab :
    print(mot)
\end{lstlisting}
\end{center}
\end{frame}
\begin{frame}[fragile]
    \frametitle{}

\begin{center}
\begin{lstlisting}[language=Python , basicstyle=\ttfamily\small, xleftmargin=2em, xrightmargin=2em]
tab = [[0, 1, 2], [3, 4, 5], [6, 7, 8]]
for t in tab: # t est un tableau interne
    for elt in t: # elt est un entier
        print(elt)
\end{lstlisting}
\end{center}
\end{frame}
\end{document}