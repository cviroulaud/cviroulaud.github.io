\documentclass[a4paper,11pt]{article}
\input{/home/tof/Documents/Cozy/latex-include/preambule_doc.tex}
\input{/home/tof/Documents/Cozy/latex-include/preambule_commun.tex}
\newcommand{\showprof}{show them}  % comment this line if you don't want to see todo environment
\setlength{\fboxrule}{0.8pt}
\fancyhead[L]{\fbox{\Large{\textbf{Encodage 02}}}}
\fancyhead[C]{\textbf{Exercices encodage caractère}}
\newdate{madate}{10}{09}{2020}
%\fancyhead[R]{\displaydate{madate}} %\today
%\fancyhead[R]{Seconde - SNT}
\fancyhead[R]{Première - NSI}
%\fancyhead[R]{Terminale - NSI}
\fancyfoot[L]{\vspace{1mm}Christophe Viroulaud}
\AtEndDocument{\label{lastpage}}
\fancyfoot[C]{\textbf{Page \thepage/\pageref{lastpage}}}
\fancyfoot[R]{\includegraphics[width=2cm,align=t]{/home/tof/Documents/Cozy/latex-include/cc.png}}

\begin{document}
\begin{exo}
Voici un message codé en ASCII:
\begin{center}
    56 49 56 45 20 4C 45 53 20 56 41 43 41 4E 43 45 53 0A
\end{center}
\begin{enumerate}
    \item Les caractères sont-ils notés en décimal ou hexadécimal?
    \item Convertir le caractère 20 en binaire.
    \item Le convertir en décimal.
    \item Décoder le message.
\end{enumerate}
\end{exo}
\begin{exo}
\begin{enumerate}
    \item Le caractère @ a le point de code \textbf{U+0040}. Convertir le point de code en binaire.
    \item Combien d'octets sont nécessaires pour encoder ce caractère en UTF8?
    \item Mêmes questions pour le caractère Ê.
\end{enumerate}
\end{exo}
\end{document}