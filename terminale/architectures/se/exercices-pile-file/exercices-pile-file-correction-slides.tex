\documentclass[svgnames,11pt]{beamer}
\input{/home/tof/Documents/Cozy/latex-include/preambule_commun.tex}
\input{/home/tof/Documents/Cozy/latex-include/preambule_beamer.tex}
%\usepackage{pgfpages} \setbeameroption{show notes on second screen=left}
\author[]{Christophe Viroulaud}
\title{Exercices pile - file\\correction}
\date{\framebox{\textbf{Archi 06}}}
%\logo{}
\institute{Terminale - NSI}

\begin{document}
\begin{frame}
    \titlepage
\end{frame}
\section{Exercice 1}
\begin{frame}[fragile]
    \frametitle{Exercice 1}

    \begin{center}
        \begin{lstlisting}[language=Python , basicstyle=\ttfamily\small, xleftmargin=2em, xrightmargin=2em]
def creer_pile() -> list:
    return []

def est_vide(p: list) -> bool:
    return len(p) == 0

def empiler(p: list, e: int) -> None:
    p.append(e)

def depiler(p: list) -> int:
    if not est_vide(p):
        return p.pop()

p = creer_pile()
\end{lstlisting}
        \captionof{code}{pile}
        \label{CODE}
    \end{center}

\end{frame}
\begin{frame}[fragile]
    \frametitle{}

    \begin{center}
        \begin{lstlisting}[language=Python , basicstyle=\ttfamily\small, xleftmargin=2em, xrightmargin=2em]
def creer_file() -> list:
    return []

def est_vide(f: list) -> bool:
    return len(f) == 0

def enfiler(f: list, e: int) -> None:
    f.insert(0, e)

def defiler(f: list) -> int:
    if not est_vide(f):
        return f.pop()

f = creer_file()
\end{lstlisting}
        \captionof{code}{file}
        \label{CODE}
    \end{center}

\end{frame}
\begin{frame}
    \frametitle{}

    La modification de la taille d'un tableau a un coup qui peut être linéaire.

\end{frame}
\section{Exercice 2}
\begin{frame}
    \frametitle{Exercice 2}

    \begin{center}
        \begin{tikzpicture}
            \draw (0,0) grid (1,3);
            \draw[dashed] (0,3) grid (1,4);
            \foreach \x/\y in {0/2,1/4,2/7}
                {
                    \node at(.5,.5+\x){\y};
                }
            \node at(.5,-1){gauche};
            \node (emp) at(-1,5) {5};
            \draw[->,>=latex] (emp) edge[bend left] (.5,3.5);
            \draw (2,0) grid (3,.2);
            \node at(2.5,-1){droite};

        \end{tikzpicture}
        \captionof{figure}{enfiler}
    \end{center}
    \begin{center}
        Le premier entré est 2.
    \end{center}
\end{frame}
\begin{frame}
    \begin{center}
        \begin{tikzpicture}
            \draw (0,0) grid (1,4);
            \foreach \x/\y in {0/2,1/4,2/7,3/5}
                {
                    \node at(.5,.5+\x){\y};
                }
            \node at(.5,-1){gauche};
            \draw[->,>=latex] (1,3.5) edge[bend left] (2.5,.5);
            \draw (2,0) grid (3,.2);
            \node at(2.5,-1){droite};

            \draw (5,0) grid (6,.2);
            \foreach \x/\y in {0/5,1/7,2/4,3/2}
                {
                    \node at(7.5,.5+\x){\y};
                }
            \node at(5.5,-1){gauche};
            \draw (7,0) grid (8,4);
            \node at(7.5,-1){droite};
            \draw[->,>=latex] (8,3.5) edge[bend left] (9,2.5);

        \end{tikzpicture}
        \captionof{figure}{défiler - cas 1}
    \end{center}
    \begin{center}
        La pile droite est vide, on commence par dépiler celle de gauche.
    \end{center}
\end{frame}
\begin{frame}
    \begin{center}
        \begin{tikzpicture}
            \draw (0,0) grid (1,.2);
            \foreach \x/\y in {0/5,1/7,2/4}
                {
                    \node at(2.5,.5+\x){\y};
                }
            \node at(.5,-1){gauche};
            \draw (2,0) grid (3,3);
            \node at(2.5,-1){droite};
            \draw[->,>=latex] (3,2.5) edge[bend left] (4,1.5);

        \end{tikzpicture}
        \captionof{figure}{défiler - cas 2}
    \end{center}
    \begin{center}
        La pile droite n'est pas vide. On dépile normalement.
    \end{center}
\end{frame}
\begin{frame}[fragile]
    \frametitle{}

    \begin{lstlisting}[language=Python , basicstyle=\ttfamily\small, xleftmargin=.5em, xrightmargin=0em]
class File2:
    def __init__(self):
        self.gauche = Pile()
        self.droite = Pile()

    def est_vide(self) -> bool:
        return self.gauche.est_vide() and self.droite.est_vide()        
\end{lstlisting}

\end{frame}
\begin{frame}[fragile]
    \frametitle{}

    \begin{lstlisting}[language=Python , basicstyle=\ttfamily\small, xleftmargin=.5em, xrightmargin=-1em]
def enfiler(self, e: int) -> None:
    self.gauche.empiler(e)

def defiler(self) -> int:
    if self.droite.est_vide():
        while not self.gauche.est_vide():
            self.droite.empiler(self.gauche.depiler())

    return self.droite.depiler()       
\end{lstlisting}

\end{frame}
\section{Exercice 3}
\begin{frame}[fragile]
    \frametitle{Exercice 3}

\begin{lstlisting}[language=Python , basicstyle=\ttfamily\small, xleftmargin=2em, xrightmargin=2em]
# Création du cercle
soldats = File()

for i in range(1, 42):
    soldats.enfiler(i)

# Élimination tous les 3
while not(soldats.est_vide()):
    # les non-éliminés reviennent dans la file
    for _ in range(2):
        soldats.enfiler(soldats.defiler())

    # soldat éliminé
    elimine = soldats.defiler()

# dernier éliminé
print(elimine)
\end{lstlisting}

\end{frame}
\section{Exercice 4}
\begin{frame}[fragile]
    \frametitle{Exercice 4}

\begin{lstlisting}[language=Python , basicstyle=\ttfamily\small, xleftmargin=.5em, xrightmargin=-1em]
def bien_parenthesee(code: str) -> bool:
    parentheses = Pile()

    for car in code:
        if car == "(":
            # empile une "("
            parentheses.empiler("(")
        elif car == ")":
            # dépile une "(" quand on trouve une ")"
            if parentheses.est_vide():
                # pile vide = manque une "("
                return False
            else:
                parentheses.depiler()

    # si la pile n'est pas vide: il reste des (
    return parentheses.est_vide()
\end{lstlisting}

\end{frame}
\begin{frame}[fragile]

\begin{lstlisting}[language=Python , basicstyle=\ttfamily\small, xleftmargin=0em, xrightmargin=-5em]
def bien_parenthesee_rec(code: str, i: int, p: Pile) -> bool:
    if i == len(code):
        # si la pile n'est pas vide: il reste des (
        return p.est_vide()
    else:
        if code[i] == "(":
            p.empiler("(")
        elif code[i] == ")":
            if p.est_vide():
                # pile vide = manque une "("
                return False
            else:
                p.depiler()
        return bien_parenthesee_rec(code, i+1, p)
\end{lstlisting}

\end{frame}
\section{Exercice 5}
\begin{frame}[fragile]
    \frametitle{Exercice 5}

\begin{lstlisting}[language=Python , basicstyle=\ttfamily\small, xleftmargin=0em, xrightmargin=0em]
def polonaise(chaine: str) -> int:
    p = Pile()
    for e in chaine.split():
        if e == "+" or e == "*":
            val1 = p.depiler()
            val2 = p.depiler()
            if e == "+":
                p.empiler(val1+val2)
            else:
                p.empiler(val1*val2)
        else:
            p.empiler(int(e))

    return p.depiler()
\end{lstlisting}

\end{frame}
\end{document}