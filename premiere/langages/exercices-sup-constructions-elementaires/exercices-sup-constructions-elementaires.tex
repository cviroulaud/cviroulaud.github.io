\documentclass[a4paper,11pt]{article}
\input{/home/tof/Documents/Cozy/latex-include/preambule_lua.tex}
\newcommand{\showprof}{show them}  % comment this line if you don't want to see todo environment
\fancyhead[L]{Exercices supplémentaires - constructions élémentaires}
\newdate{madate}{10}{09}{2020}
%\fancyhead[R]{\displaydate{madate}} %\today
%\fancyhead[R]{Seconde - SNT}
\fancyhead[R]{Première - NSI}
%\fancyhead[R]{Terminale - NSI}
\fancyfoot[L]{~\\Christophe Viroulaud}
\AtEndDocument{\label{lastpage}}
\fancyfoot[C]{\textbf{Page \thepage/\pageref{lastpage}}}
\fancyfoot[R]{\includegraphics[width=2cm,align=t]{/home/tof/Documents/Cozy/latex-include/cc.png}}

\begin{document}
\begin{Form}
\begin{exo}
Écrire un programme qui, à partir de la saisie d'un rayon et d'une hauteur, calcule le volume d'un cône droit.
\end{exo}
\begin{exo}
L'utilisateur donne un entier positif \emph{n} et le programme affiche PAIR s'il est divisible par 2, IMPAIR sinon.
\end{exo}
\begin{exo}
L'utilisateur donne un entier positif et le programme annonce combien de fois de suite cet entier est divisible par 2.
\end{exo}
\begin{exo}
Écrire un programme qui demande à l’utilisateur de saisir 3 nombre x, y et z et de lui afficher leur maximum
\end{exo}
\begin{exo}
Écrire un programme pour calculer:
\begin{itemize}
\item $1+2+3+....+100$
\item $1+3+5+....+99$
\end{itemize}
\end{exo}
\begin{exo}
Écrire un programme qui demande à l’utilisateur de saisir deux nombres entiers a et b et de lui afficher le quotient et le reste de la division euclidienne de a par b.
\end{exo}
\begin{exo}
On dispose d'une feuille de papier d'épaisseur 0,1 mm.
Combien de fois doit-on la plier au minimum pour que l'épaisseur dépasse la hauteur de la tour Eiffel 324 m?
Écrire un programme pour résoudre ce problème. 
\end{exo}
\begin{exo}
Écrire un programme qui demande à l’utilisateur de saisir un nombre entier n et de lui afficher tous les diviseurs de ce nombre.
\end{exo}
\begin{exo}
Écrire un programme qui demande à l’utilisateur de saisir un nombre entier n et de lui afficher si ce nombre est premier ou non.
\end{exo}
\begin{exo}
L'ordinateur cache une bombe dont les coordonnées (x,y) sont des entiers compris entre 0 et 100 inclus.
Le joueur propose un point. Si la distance entre la bombe et le point proposé est inférieure ou égale à 10, le programme affiche "Bravo", sinon le joueur doit proposer à nouveau un point.\\
\textbf{Rappel:} La distance euclidienne entre les points A et B se calcule:
$$AB = \sqrt{(y_B-y_A)^2+(x_B-x_a)^2}$$
\end{exo}
\end{Form}
\end{document}