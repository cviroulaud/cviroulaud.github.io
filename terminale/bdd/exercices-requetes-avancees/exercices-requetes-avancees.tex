\documentclass[a4paper,11pt]{article}
\input{/home/tof/Documents/Cozy/latex-include/preambule_lua.tex}
\newcommand{\showprof}{show them}  % comment this line if you don't want to see todo environment
\fancyhead[L]{Exercices requêtes avancées}
\newdate{madate}{10}{09}{2020}
%\fancyhead[R]{\displaydate{madate}} %\today
%\fancyhead[R]{Seconde - SNT}
%\fancyhead[R]{Première - NSI}
\fancyhead[R]{Terminale - NSI}
\fancyfoot[L]{~\\Christophe Viroulaud}
\AtEndDocument{\label{lastpage}}
\fancyfoot[C]{\textbf{Page \thepage/\pageref{lastpage}}}
\fancyfoot[R]{\includegraphics[width=2cm,align=t]{/home/tof/Documents/Cozy/latex-include/cc.png}}

\begin{document}
\begin{Form}
\begin{exo}
La clinique vétérinaire utilise la base de données construite précédemment.
\begin{enumerate}
\item Ouvrir la base \emph{animaux.db} .
\item Écrire les requêtes simples pour:
\begin{enumerate}
\item insérer l'espèce \emph{poisson} dans la table \emph{Especes},
\item insérer le soin \emph{patte cassée} à l'animal d'identifiant 3 dans la table \emph{Soins},
\item sélectionner les animaux de 10 ans et plus, ordonnés par nom.
\item sélectionner tous les soins distincts réalisés,
\item compter tous les soins réalisés,
\item compter tous les soins distincts réalisés,
\item compter le nombre de stérilisations réalisé,
\item renommer l'animal \emph{charly} en \emph{charlie}.
\end{enumerate}
\item Écrire les requêtes pour:
\begin{enumerate}
\item trouver l'espèce de \emph{zappy},
\item donner les noms des chats dans l'ordre alphabétique,
\item compter le nombre de chiens,
\item compter le nombre de soins pour \emph{chouchou},
\item donner le nom (par ordre alphabétique) de tous les chats qui ont subi une \emph{stérilisation},
\item insérer le poisson \emph{bubulle}, âgé de 8 ans, dans la table \emph{Animaux},
\item insérer le soin \emph{stérilisation} pour \emph{cartman}.
\end{enumerate}
\item Les soins prodigués par le vétérinaire sont souvent les mêmes. Quelle modification pourrait-on mettre en place pour améliorer le schéma de la base?
\item Écrire le modèle relationnel avec la modification apportée puis écrire les requêtes permettant de reconstruire la base.
\end{enumerate}
\end{exo}
\begin{exo}
Se rendre sur la page \url{https://tinyurl.com/y5dvjce6} et s'entraîner en fonction du niveau et des besoins.
\end{exo}
\end{Form}
\end{document}