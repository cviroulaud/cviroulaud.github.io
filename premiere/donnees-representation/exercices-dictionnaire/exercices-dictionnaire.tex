\documentclass[a4paper,11pt]{article}
\input{/home/tof/Documents/Cozy/latex-include/preambule_doc.tex}
\input{/home/tof/Documents/Cozy/latex-include/preambule_commun.tex}
\newcommand{\showprof}{show them}  % comment this line if you don't want to see todo environment
\setlength{\fboxrule}{0.8pt}
\fancyhead[L]{\fbox{\Large{\textbf{Dico 02}}}}
\fancyhead[C]{\textbf{Exercices dictionnaire}}
\newdate{madate}{10}{09}{2020}
%\fancyhead[R]{\displaydate{madate}} %\today
%\fancyhead[R]{Seconde - SNT}
\fancyhead[R]{Première - NSI}
%\fancyhead[R]{Terminale - NSI}
\fancyfoot[L]{\vspace{1mm}Christophe Viroulaud}
\AtEndDocument{\label{lastpage}}
\fancyfoot[C]{\textbf{Page \thepage/\pageref{lastpage}}}
\fancyfoot[R]{\includegraphics[width=2cm,align=t]{/home/tof/Documents/Cozy/latex-include/cc.png}}

\begin{document}
\begin{exo}Un livre peut être caractérisé par son titre, son auteur, son éditeur, son prix.
\begin{enumerate}
    \item Construire un dictionnaire qui contient les informations du livre: \emph{Il était deux fois} de Franck Thilliez aux éditions \emph{Poche} à 8,70€.
    \item Construire un dictionnaire pour \emph{Fahrenheit 451} de Ray Bradbury aux éditions \emph{Folio} à 6,30€.
    \item Construire un tableau contenant les deux dictionnaires. Ajouter au moins un autre livre.
    \item Écrire une boucle qui parcourt le tableau et affiche l'auteur de chaque livre.
\end{enumerate}
\end{exo}
\begin{exo}
Le groupe d'élèves de NSI est composé de:
\begin{itemize}
    \item Alice Durant,
    \item Bob Bois,
    \item John Doe,
    \item Jules Dupont,
    \item Alan Turing.
\end{itemize}
Au cours du semestre les notes obtenues sont:
\begin{center}
    \begin{tabular}{|c|*{4}{c}|}
        \hline
        Alice & 12 & 8 & 10 & 9.5 \\
        \hline
        Bob & 15 & 17 & 18 & 14  \\
        \hline
        John & 10.5 & 8 & 16 & 13.5  \\
        \hline
        Jules & 12 & 9 & 17.5 & 10  \\
        \hline
        Alan & 14 & 18 & 16 & 19  \\
        \hline
    \end{tabular}
\end{center}
\begin{enumerate}
    \item Organiser ces données dans un tableau et en créant un dictionnaire pour chaque élève.
    \item Écrire la fonction \textbf{moyenne(eleve: dict) $\rightarrow$ float} qui renvoie la moyenne de l'élève.
    \item Calculer la moyenne générale du groupe.
\end{enumerate}
\end{exo}
\begin{exo}
Écrire la fonction \textbf{lettres(mot: str) $\rightarrow$ dict} qui renvoie un dictionnaire le nombre d'occurrences de chaque lettre de \emph{mot}. Par exemple:
\begin{center}
\begin{lstlisting}[language=Python]
>>>lettres("bonjour")
>>>{"b": 1, "o": 2, "n": 1, "j": 1, "u": 1, "r": 1}
\end{lstlisting}
\captionof{code}{Renvoi de l'appel de la fonction}
\label{ip}
\end{center}
\end{exo}
\end{document}