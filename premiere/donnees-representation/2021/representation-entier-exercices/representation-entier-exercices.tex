\documentclass[a4paper,11pt]{article}
\input{/home/tof/Documents/Cozy/latex-include/preambule_lua.tex}
\newcommand{\showprof}{show them}  % comment this line if you don't want to see todo environment
\fancyhead[L]{Représentation des entiers - exercices}
\newdate{madate}{10}{09}{2020}
\fancyhead[R]{Première - NSI} %\today
\fancyfoot[L]{~\\Christophe Viroulaud}
\fancyfoot[C]{\textbf{Page \thepage}}
\fancyfoot[R]{\includegraphics[width=2cm,align=t]{/home/tof/Documents/Cozy/latex-include/cc.png}}

\begin{document}
\begin{Form}
\begin{commentprof}
Vérifier avec \emph{bin, hex} en Python mais montrer les démarches sur papier!!\\
bin() donne version non signée:
\begin{itemize}
\item bin(6) = 0b110
\item bin(-6) = -0b110
\end{itemize}
\end{commentprof}
\begin{exo}
Donner la représentation en base de 2 et sur 8 bits des entiers 14, 222, 42, 79.
\end{exo}
\begin{exo}
Donner la représentation décimale des nombres binaires (non signés) suivants:
\begin{itemize}
\item 1010
\item 111110
\item 100101001
\end{itemize}
\end{exo}
\begin{exo}
Donner la représentation hexadécimale des nombres binaires suivants:
\begin{itemize}
\item 10010101
\item 11010101
\item 100010001
\item 11001101001010
\end{itemize}
\end{exo}
\begin{exo}
Quelle est la valeur en base 10 de l'entier qui s'écrit BEEF en base 16?
\end{exo}
\begin{exo}
Donner la représentation en complément à 2 sur 8 bits des entiers suivants: -10, -128, -42, 97.
\end{exo}
\begin{exo}
Donner en base décimale la valeur des octets signés suivants:
\begin{itemize}
\item 11100111
\item 11000001
\end{itemize}
\end{exo}
\begin{exo}
Réaliser le QCM d’entraînement depuis le site \url{https://cviroulaud.github.io} 
\end{exo}
\begin{exo}
Effectuer les additions suivantes en base 2:
\begin{enumerate}
\item $39+110$
\item $-53+35$
\item $119-8$
\item $19-93$
\end{enumerate}
\end{exo}
\begin{exo}
Blague de geek:\guill{Le monde est partagé en 10 catégories: les informaticiens et les autres.}\\
Comment comprendre cette phrase?
\end{exo}
\end{Form}
\end{document}