\documentclass[svgnames,11pt]{beamer}
\input{/home/tof/Documents/Cozy/latex-include/preambule_commun.tex}
\input{/home/tof/Documents/Cozy/latex-include/preambule_beamer.tex}
%\usepackage{pgfpages} \setbeameroption{show notes on second screen=left}
\author[]{Christophe Viroulaud}
\title{Exercices parcours séquentiel\\Correction}
\date{\framebox{\textbf{Algo 02}}}
%\logo{}
\institute{Première - NSI}

\begin{document}
\begin{frame}
\titlepage
\end{frame}
\begin{frame}[fragile]
    \frametitle{}

\begin{center}
\begin{lstlisting}[language=Python , basicstyle=\ttfamily\small, xleftmargin=2em, xrightmargin=2em]
les_notes = [randint(0, 20) for _ in range(30)]
\end{lstlisting}
\captionof{code}{Construction par compréhension}
\label{CODE}
\end{center}

\end{frame}
\section{Exercice 1}
\begin{frame}[fragile]
    \frametitle{Exercice 1}

\begin{center}
\begin{lstlisting}[language=Python , basicstyle=\ttfamily\small, xleftmargin=2em, xrightmargin=2em]
def extrema(tab: list) -> tuple:
    """
    Renvoie le mini et le maxi de tab

    Args:
        tab (list): 
    Returns:
        tuple: (mini, maxi)
    """
    mini = 20
    maxi = 0
    for note in tab:
        if note < mini:
            mini = note
        if note > maxi:
            maxi = note
    return (mini, maxi)
\end{lstlisting}
\end{center}    

\end{frame}
\section{Exercice 2}
\begin{frame}[fragile]
    \frametitle{Exercice 2}

\begin{center}
\begin{lstlisting}[language=Python , basicstyle=\ttfamily\small, xleftmargin=2em, xrightmargin=2em]
def maxi_position(tab: list) -> tuple:
    """
    renvoie le max et sa première position dans le tableau
    """
    # On peut affecter plusieurs variables sur 1 ligne
    indice, note_max = 0, 0
    for i in range(len(tab)):
        if tab[i] > note_max:
            note_max = tab[i]
            indice = i
    return (indice, note_max)
\end{lstlisting}
\end{center}    

\end{frame}
\section{Exercice 3}
\begin{frame}[fragile]
    \frametitle{Exercice 3}

\begin{center}
\begin{lstlisting}[language=Python , basicstyle=\ttfamily\small, xleftmargin=2em, xrightmargin=1em]
def maxi_position_dernier(tab: list) -> tuple:
    """
    renvoie le max et sa dernière position dans le tableau
    """
    # On peut affecter plusieurs variables sur 1 ligne
    indice, note_max = 0, 0
    for i in range(len(tab)):
        # il suffit juste de modifier la comparaison
        if tab[i] >= note_max:
            note_max = tab[i]
            indice = i
    return (indice, note_max)
\end{lstlisting}
\end{center}    

\end{frame}
\section{Exercice 4}
\begin{frame}[fragile]
    \frametitle{Exercice 4}

\begin{center}
\begin{lstlisting}[language=Python , basicstyle=\ttfamily\small, xleftmargin=2em, xrightmargin=2em]
def maxi_nb(tab: list) -> int:
    """
    renvoie le nombre d'occurences du maximum
    """
    nb, note_max = 0, 0
    for note in tab:
        if note > note_max:
            # réinitialisation du max
            note_max = note
            nb = 1
        elif note == note_max:
            nb += 1
    return nb
\end{lstlisting}
\end{center}    

\end{frame}
\section{Exercice 5}
\begin{frame}[fragile]
    \frametitle{Exercice 5}

\begin{center}
\begin{lstlisting}[language=Python , basicstyle=\ttfamily\small, xleftmargin=2em, xrightmargin=2em]
def est_present(tab: list, note: int)->bool:
    """
    vérifie si note est dans le tableau
    """
    for n in tab:
        if n == note:
            return True
    # On est sorti de la boucle sans avoir trouvé note
    return False
\end{lstlisting}
\end{center}    
La fonction dépend de la taille du tableau. La complexité est \textbf{linéaire}.
\end{frame}
\section{Exercice 6}
\begin{frame}[fragile]
    \frametitle{Exercice 6}

\begin{center}
\begin{lstlisting}[language=Python , basicstyle=\ttfamily\small, xleftmargin=0em, xrightmargin=-6em]
def est_voyelle(lettre:str)->bool:
    """
    vérifie si lettre est une voyelle
    """    
    voyelles = ["a", "e", "i", "o", "u", "y"]
    for v in voyelles:
        if lettre == v:
            return True
    return False
\end{lstlisting}
\end{center}    

\end{frame}
\begin{frame}[fragile]

\begin{center}
\begin{lstlisting}[language=Python , basicstyle=\ttfamily\small, xleftmargin=0em, xrightmargin=-6em]
def compter_voyelles(mot: str) -> dict:
    """
    compte le nombre de chaque voyelles de mot
    """
    voyelles = {"a": 0, "e": 0, "i": 0, "o": 0, "u": 0, "y": 0}
    for lettre in mot:
        if est_voyelle(lettre):  # utilise la fonction précédente
            voyelles[lettre] += 1
    return voyelles
\end{lstlisting}
\captionof{code}{version 1}
\end{center}    

\end{frame}
\begin{frame}[fragile]

\begin{center}
\begin{lstlisting}[language=Python , basicstyle=\ttfamily\small, xleftmargin=0em, xrightmargin=-6em]
def compter_voyelles(mot: str)->dict:
    """
    compte le nombre de chaque voyelles de mot
    """
    voyelles = {"a": 0, "e": 0, "i": 0, "o": 0, "u": 0, "y": 0}
    for lettre in mot:
        # compare la lettre aux voyelles
        if lettre in voyelles.keys():
            voyelles[lettre] += 1
    return voyelles
\end{lstlisting}
\captionof{code}{méthode alternative}
\end{center}    

\end{frame}
\begin{frame}[fragile]
    \frametitle{}

\begin{center}
\begin{lstlisting}[language=Python , basicstyle=\ttfamily\small, xleftmargin=2em, xrightmargin=2em]
voyelles = compter_voyelles("orangeade")

for lettre, nb in voyelles.items():
    print(f"{lettre}: {nb}")
\end{lstlisting}
\captionof{code}{Affichage du dictionnaire dans le programme principal}
\label{CODE}
\end{center} 

\end{frame}
\begin{frame}[fragile]

\begin{center}
\begin{lstlisting}[language=Python , basicstyle=\ttfamily\small, xleftmargin=1em, xrightmargin=0em]
def max_voyelles(voyelles: dict) -> list:
    """
    parcourt le dict voyelles et renvoie
    celle qui a la plus grande valeur
    """
    maxi = 0
    lettres_maxi = []
    for lettre, nb in voyelles.items():
        if nb > maxi:
            maxi = nb
            # on réinitialise le tableau avec la nouvelle lettre max
            lettres_maxi = [lettre]
        elif nb == maxi:
            lettres_maxi.append(lettre)
    return lettres_maxi
\end{lstlisting}
\end{center}    
    
    \end{frame}
\end{document}