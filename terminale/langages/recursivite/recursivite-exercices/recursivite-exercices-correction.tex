\documentclass[a4paper,11pt]{article}
\input{/home/tof/Documents/Cozy/latex-include/preambule_lua.tex}
\newcommand{\showprof}{show them}  % comment this line if you don't want to see todo environment
\fancyhead[L]{Récursivité - correction exercices}
\newdate{madate}{10}{09}{2020}
\fancyhead[R]{Terminale - NSI} %\today
\fancyfoot[L]{~\\Christophe Viroulaud}
\fancyfoot[C]{\textbf{Page \thepage}}
\fancyfoot[R]{\includegraphics[width=2cm,align=t]{/home/tof/Documents/Cozy/latex-include/cc.png}}

\begin{document}
\begin{Form}
\begin{exo}
\begin{enumerate}
\item $$
somme(n) = \left\{
    \begin{array}{ll}
        0 & \mbox{si  }n=0\\
        \mbox{n + somme(n-1)} & \mbox{si }n>0\
    \end{array}
\right.
$$
\item Programme
\lstinputlisting[firstline=10,lastline=16]{"scripts/somme-entiers.py"}
\end{enumerate}
\end{exo}
\begin{exo}
\begin{enumerate}
\item $$
n! = \left\{
    \begin{array}{ll}
        1 & \mbox{si  }n=0\\
        \mbox{n × (n-1)!} & \mbox{si }n>0\
    \end{array}
\right.
$$
\item Programme
\lstinputlisting[firstline=10,lastline=16]{"scripts/factorielle.py"}
\end{enumerate}
\end{exo}
\begin{exo}
La première version affiche les termes à chaque appel.
\lstinputlisting[firstline=21,lastline=29]{"scripts/syracuse.py"}
Une seconde version enregistre les termes dans une liste et renvoie cette liste. Il faut noter l'utilisation d'une variable supplémentaire initialisée par défaut.
\lstinputlisting[firstline=10,lastline=19]{"scripts/syracuse.py"}
\begin{commentprof}
\paragraph{Conjecture de Syracuse:} quelle que soit la valeur de $u_0$ il existe un \emph{n} tel que $u_n=1$. Toujours pas prouvée à ce jour.\\
Il faut noter que la fonction ne renvoie rien à chaque appel cette fois.
\end{commentprof}
\end{exo}
\begin{exo}
\begin{enumerate}
\item Programme
\lstinputlisting[firstline=10,lastline=15]{"scripts/entiers.py"}
\item Programme
\lstinputlisting[firstline=19,lastline=25]{"scripts/entiers.py"}
\end{enumerate}
\end{exo}
\begin{exo}
Appliquons la méthode d'Euclide:
$$
pgcd(a,b) = \left\{
    \begin{array}{ll}
        b & \mbox{si } a=0\\
        pgcd(b\%a,b) & \mbox{sinon}\\
    \end{array}
\right.
$$
\lstinputlisting[firstline=10,lastline=16]{"scripts/pgcd.py"}
\end{exo}
\begin{exo}
\lstinputlisting[firstline=10,lastline=16]{"scripts/nombre-chiffres.py"}
\end{exo}
\begin{exo}
\begin{enumerate}
\item Fonction
\lstinputlisting[firstline=10,lastline=14]{"scripts/coef-binomiaux.py"}
\item Programme
\lstinputlisting[firstline=16,lastline=19]{"scripts/coef-binomiaux.py"}
\end{enumerate}
\end{exo}
\end{Form}
\end{document}