\documentclass[a4paper,11pt]{article}
\input{/home/tof/Documents/Cozy/latex-include/preambule_doc.tex}
\input{/home/tof/Documents/Cozy/latex-include/preambule_commun.tex}
\newcommand{\showprof}{show them}  % comment this line if you don't want to see todo environment
\setlength{\fboxrule}{0.8pt}
\fancyhead[L]{\fbox{\Large{\textbf{Encodage 02}}}}
\fancyhead[C]{\textbf{Exercices encodage caractère - correction}}
\newdate{madate}{10}{09}{2020}
%\fancyhead[R]{\displaydate{madate}} %\today
%\fancyhead[R]{Seconde - SNT}
\fancyhead[R]{Première - NSI}
%\fancyhead[R]{Terminale - NSI}
\fancyfoot[L]{\vspace{1mm}Christophe Viroulaud}
\AtEndDocument{\label{lastpage}}
\fancyfoot[C]{\textbf{Page \thepage/\pageref{lastpage}}}
\fancyfoot[R]{\includegraphics[width=2cm,align=t]{/home/tof/Documents/Cozy/latex-include/cc.png}}

\begin{document}
\begin{exo}
\begin{enumerate}
    \item Les caractères sont notés en hexadécimal.
    \item Attention le 20 est en hexadécimal; il s'agit donc en binaire de 0010 0000.
    \item En décimal $00100000 = 2^5 = 32$
    \item VIVE LES VACANCES
\end{enumerate}
\end{exo}
\begin{exo}
\begin{enumerate}
    \item U+0040 en binaire: 0100 0000
    \item Il suffit d'un octet pour encoder ce caractère en UTF-8.
    \item Le point de code de la lettre Ê est U+00CA. En binaire on a: 1100 1010. Il faut alors 2 octets pour encoder ce caractère en UTF-8.
    
    On utilise la suite d'octets: 110xxxxx 10xxxxxx et on remplace les \emph{x} par les chiffres binaires. On complète avec des 0:
\begin{center}
        110\textbf{00011} 10\textbf{001010}
    
\end{center}
\end{enumerate}
\end{exo}
\end{document}