\documentclass[a4paper,11pt]{article}
\input{/home/tof/Documents/Cozy/latex-include/preambule_lua.tex}
\newcommand{\showprof}{show them}  % comment this line if you don't want to see todo environment
\fancyhead[L]{Exercices}
\newdate{madate}{10}{09}{2020}
%\fancyhead[R]{\displaydate{madate}} %\today
%\fancyhead[R]{Seconde - SNT}
\fancyhead[R]{Première - NSI}
%\fancyhead[R]{Terminale - NSI}
\fancyfoot[L]{~\\Christophe Viroulaud}
\AtEndDocument{\label{lastpage}}
\fancyfoot[C]{\textbf{Page \thepage/\pageref{lastpage}}}
\fancyfoot[R]{\includegraphics[width=2cm,align=t]{/home/tof/Documents/Cozy/latex-include/cc.png}}

\begin{document}
\begin{Form}
\begin{exo}
Dans une classe de 10 élèves, on veut choisir au hasard, 2 délégués et 2 adjoints.
Écrire le programme en Python pour traiter le problème.
\end{exo}
\begin{exo}
Écrire un programme en python pour réviser ses tables de multiplication.

Le programme tire 2 entiers au hasard et demande à l'utilisateur le produit.

On interrogera 10 fois l'utilisateur et on donnera une note finale: 1 pt par bonne réponse et -1 sinon. 
\end{exo}
\begin{exo}
Un jeu de hasard se déroule de la façon suivante: On paie 2 euros pour jouer puis on lance 2 dés non truqués tétraédriques.
Si le joueur obtient un double, il récupère sa mise et reçoit la somme des points marqués.
Sinon il ne reçoit rien et perd sa mise. Écrire un programme pour simuler ce jeu. 
\end{exo}
\end{Form}
\end{document}