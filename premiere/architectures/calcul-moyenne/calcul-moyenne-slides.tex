\documentclass[svgnames,11pt]{beamer}
\input{/home/tof/Documents/Cozy/latex-include/preambule_commun.tex}
\input{/home/tof/Documents/Cozy/latex-include/preambule_beamer.tex}
%\usepackage{pgfpages} \setbeameroption{show notes on second screen=left}
\author[]{Christophe Viroulaud}
\title{Calcul de la moyenne\\Exercice commenté}
\date{\framebox{\textbf{ArchMat 04}}}
%\logo{}
\institute{Première - NSI}

\begin{document}
\begin{frame}
\titlepage
\end{frame}
\begin{frame}
    \frametitle{}

Pour faciliter le travail des enseignants, on se propose de construire un programme qui permet de calculer la moyenne des notes d'un élève.

\end{frame}
\begin{frame}
    \frametitle{Cahier des charges}

Le programme devra:
\begin{itemize}
    \item demander à l'utilisateur de rentrer huit notes,
    \item stocker chacune des notes dans la mémoire,
    \item calculer la somme des notes,
    \item calculer la moyenne,
    \item stocker en mémoire puis afficher la moyenne.
\end{itemize}

\end{frame}
\begin{frame}[fragile]
    \frametitle{Initialisation des paramètres}

\begin{center}
\begin{lstlisting}[language={[x86masm]Assembler} , basicstyle=\ttfamily\small, xleftmargin=2em, xrightmargin=2em]
MOV R0,#0
MOV R12,#8        //nombre de notes
MOV R11,#0        //compteur
MOV R10,#64       //adresse de stockage
\end{lstlisting}
\end{center}

\end{frame}
\begin{frame}[fragile]
    \frametitle{Récupération d'une note}

\begin{center}
\begin{lstlisting}[language={[x86masm]Assembler} , basicstyle=\ttfamily\small, xleftmargin=2em, xrightmargin=2em]
LDR R1,.InputNum
\end{lstlisting}
\end{center}
\begin{aretenir}[Commentaire]
Il faut effectuer cette opération huit fois.
\end{aretenir}
\end{frame}
\begin{frame}[fragile]
    \frametitle{Stockage d'une note}

\begin{center}
\begin{lstlisting}[language={[x86masm]Assembler} , basicstyle=\ttfamily\small, xleftmargin=2em, xrightmargin=2em]
STR R1,[R10]      //stockage de la note
ADD R10,R10,#4
\end{lstlisting}
\end{center}
\begin{aretenir}[Commentaire]
Les adresses vont de quatre en quatre dans cette mémoire.
\end{aretenir}
\end{frame}
\begin{frame}[fragile]
    \frametitle{Somme des notes}

\begin{center}
\begin{lstlisting}[language={[x86masm]Assembler} , basicstyle=\ttfamily\small, xleftmargin=2em, xrightmargin=2em]
ADD R0,R0,R1
\end{lstlisting}
\end{center}
\end{frame}
\begin{frame}[fragile]
    \frametitle{Boucle}

\begin{center}
\begin{lstlisting}[language={[x86masm]Assembler} , basicstyle=\ttfamily\small, xleftmargin=2em, xrightmargin=2em]
ADD R11,R11,#1
CMP R11,R12
BNE entreenote
\end{lstlisting}
\end{center}
\begin{aretenir}[Commentaire]
Il faut définir un label (\textbf{\texttt{entreenote}}) pour boucler.
\end{aretenir}
\end{frame}
\begin{frame}[fragile]
    \frametitle{Boucle complète}

\begin{center}
\begin{lstlisting}[language={[x86masm]Assembler} , basicstyle=\ttfamily\small, xleftmargin=2em, xrightmargin=2em]
    MOV R0,#0
    MOV R12,#8        //nombre de notes
    MOV R11,#0        //compteur
    MOV R10,#64       //adresse de stockage
entreenote:
    LDR R1,.InputNum
    STR R1,[R10]      //stockage de la note
    ADD R10,R10,#4
    ADD R0,R0,R1      //somme
    ADD R11,R11,#1
    CMP R11,R12
    BNE entreenote
\end{lstlisting}
\end{center}
\end{frame}
\begin{frame}[fragile]
    \frametitle{Moyenne}

\begin{center}
\begin{lstlisting}[language={[x86masm]Assembler} , basicstyle=\ttfamily\small, xleftmargin=2em, xrightmargin=2em]
LSR R0,R0,#3      //moyenne
STR R0,[R10]      //stockage moyenne
STR R0,.WriteUnsignedNum
HALT
\end{lstlisting}
\end{center}
\end{frame}
\begin{frame}[fragile]
    \frametitle{Code complet}

\begin{center}
\begin{lstlisting}[language={[x86masm]Assembler} , basicstyle=\ttfamily\small, xleftmargin=2em, xrightmargin=2em]
    MOV R0,#0
    MOV R12,#8        //nombre de notes
    MOV R11,#0        //compteur
    MOV R10,#64       //adresse de stockage
entreenote:
    LDR R1,.InputNum
    STR R1,[R10]      //stockage de la note
    ADD R10,R10,#4
    ADD R0,R0,R1      //somme
    ADD R11,R11,#1
    CMP R11,R12
    BNE entreenote
    LSR R0,R0,#3      //moyenne
    STR R0,[R10]      //stockage moyenne
    STR R0,.WriteUnsignedNum
    HALT
\end{lstlisting}
\end{center}
\end{frame}
\end{document}