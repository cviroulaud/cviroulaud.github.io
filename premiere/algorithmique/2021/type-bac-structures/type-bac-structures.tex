\documentclass[a4paper,11pt]{article}
\input{/home/tof/Documents/Cozy/latex-include/preambule_doc.tex}
\input{/home/tof/Documents/Cozy/latex-include/preambule_commun.tex}
\newcommand{\showprof}{show them}  % comment this line if you don't want to see todo environment
\setlength{\fboxrule}{0.8pt}
\fancyhead[L]{\fbox{\Large{\textbf{Structures}}}}
\fancyhead[C]{\textbf{Exercices type bac}}
\newdate{madate}{10}{09}{2020}
%\fancyhead[R]{\displaydate{madate}} %\today
%\fancyhead[R]{Seconde - SNT}
\fancyhead[R]{Première - NSI}
%\fancyhead[R]{Terminale - NSI}
\fancyfoot[L]{\vspace{1mm}Christophe Viroulaud}
\AtEndDocument{\label{lastpage}}
\fancyfoot[C]{\textbf{Page \thepage/\pageref{lastpage}}}
\fancyfoot[R]{\includegraphics[width=2cm,align=t]{/home/tof/Documents/Cozy/latex-include/cc.png}}

\begin{document}
\begin{exo}
    Soit le couple (note,coefficient):
    \begin{itemize}
        \item note est un nombre de type flottant (float) compris entre 0 et 20 ;
        \item coefficient est un nombre entier positif.
    \end{itemize}
    Les résultats aux évaluations d'un élève sont regroupés dans une liste composée de couples (note,coefficient).
    Écrire une fonction \textbf{moyenne} qui renvoie la moyenne pondérée de cette liste donnée en
    paramètre.

    Par exemple, l’expression moyenne([(15,2),(9,1),(12,3)]) devra renvoyer le
    résultat du calcul suivant :
    $$\dfrac{2 × 15 + 1 × 9 + 3 × 12}{2+1+3} = 12,5$$
\end{exo}
\begin{exo}
    Écrire une fonction \textbf{RechercheMinMax} qui prend en paramètre un tableau de nombres
    non triés tab, et qui renvoie la plus petite et la plus grande valeur du tableau sous la
    forme d’un dictionnaire à deux clés ‘min’ et ‘max’. Les tableaux seront représentés sous
    forme de liste Python.

    \begin{center}
        \begin{lstlisting}[language=Python]
>>> tableau = [0, 1, 4, 2, -2, 9, 3, 1, 7, 1]
>>> resultat = rechercheMinMax(tableau)
>>> resultat
{'min': -2, 'max': 9}
>>> tableau = []
>>> resultat = rechercheMinMax(tableau)
>>> resultat
{'min': None, 'max': None}
    \end{lstlisting}
        \captionof{code}{Exemples}
        \label{CODE}
    \end{center}

\end{exo}
\begin{exo}
    Un professeur de NSI décide de gérer les résultats de sa classe sous la forme d’un
    dictionnaire :
    \begin{itemize}
        \item les clefs sont les noms des élèves ;
        \item les valeurs sont des dictionnaires dont les clefs sont les types d’épreuves et les
        valeurs sont les notes obtenues associées à leurs coefficients.
    \end{itemize}
    Avec :
\lstinputlisting[firstline=1 ,lastline=12 ]{"scripts/exo-dico.py"}

L’élève dont le nom est Durand a ainsi obtenu au DS2 la note de 8 avec un coefficient 4.

Le professeur crée une fonction \textbf{moyenne} qui prend en paramètre le nom d’un de ces élèves et lui renvoie sa moyenne arrondie au dixième.
Compléter le code du professeur ci-dessous :
\lstinputlisting[firstline=15 ,lastline=26 ]{"scripts/exo-dico.py"}
\end{exo}
\end{document}