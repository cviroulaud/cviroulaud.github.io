\documentclass[svgnames,11pt]{beamer}
\input{/home/tof/Documents/Cozy/latex-include/preambule_commun.tex}
\input{/home/tof/Documents/Cozy/latex-include/preambule_beamer.tex}
%\usepackage{pgfpages} \setbeameroption{show notes on second screen=left}
\author[]{Christophe Viroulaud}
\title{Exercices types construits\\Correction}
\date{\framebox{\textbf{DonRep 09}}}
%\logo{}
\institute{Première - NSI}

\begin{document}
\begin{frame}
\titlepage
\end{frame}
\section{Exercice 1}
\begin{frame}
    \frametitle{Exercice 1}
Éléments de correction:
\begin{itemize}
    \item Un tuple n'est pas mutable (modifiable).
    \item Dans un dictionnaire, tenter de lire la valeur associée à une clé qui n'existe pas renvoie une erreur.
    \item Dans un dictionnaire, il est possible de créer un nouveau couple clé/valeur.
\end{itemize}

\end{frame}
\section{Exercice 2}
\begin{frame}[fragile]
    \frametitle{Exercice 2}

\begin{lstlisting}[language=Python , basicstyle=\ttfamily\small, xleftmargin=2em, xrightmargin=2em]
tab = [0, 1, 2, 3, 4, 5]

for i in range(len(tab)):
    tab[i] = tab[i]**2
\end{lstlisting}

\end{frame}
\section{Exercice 3}
\begin{frame}[fragile]
    \frametitle{Exercice 3}

\begin{lstlisting}[language=Python , basicstyle=\ttfamily\small, xleftmargin=2em, xrightmargin=2em]
def mini(t: tuple) -> int:
    """
    renvoie l'entier minimum de t
    """
    minimum = 101
    for elt in t:
        if elt < minimum:
            minimum = elt
    return minimum
\end{lstlisting}

\begin{center}
\begin{lstlisting}[language=Python , basicstyle=\ttfamily\small, xleftmargin=2em, xrightmargin=2em]
>>> tup = (17, 32, 8, 4, 35, 13)
>>> mini(tup)
4
\end{lstlisting}
\captionof{code}{Appel de la fonction}
\label{CODE}
\end{center}
\end{frame}
\section{Exercice 4}
\begin{frame}[fragile]
    \frametitle{Exercice 4}

\begin{lstlisting}[language=Python , basicstyle=\ttfamily\small, xleftmargin=2em, xrightmargin=2em]
def mediane(t: tuple) -> float:
    """
    calcule la médiane du tuple
    indications:
        le tuple est ordonné
        le tuple contient au moins 2 éléments
    """
    taille = len(t)
    milieu = taille//2
    if taille % 2 == 0:  # pair
        # la numérotation commence à 0
        med = (t[milieu-1]+t[milieu])/2
    else:  # impair
        med = t[milieu]
    return med
\end{lstlisting}


\end{frame}
\begin{frame}
    \frametitle{}

    \begin{aretenir}[Observations]
\begin{itemize}
    \item Le programme respecte l'algorithme donné.
    \item Dans le cas pair, que se passerait-il si la taille du tuple était inférieure à 2?
\end{itemize}
    \end{aretenir}

\end{frame}
\begin{frame}[fragile]
    \frametitle{}
    \begin{center}
\begin{lstlisting}[language=Python , basicstyle=\ttfamily\small, xleftmargin=2em, xrightmargin=2em]
salaire = (800, 830, 830, 950, 1002, 1100,
    1100, 1103, 1340, 1530, 1600)
salaire1 = (900, 950, 950, 960, 1050, 1060,
        1100, 1160, 1370, 1555)
print(mediane(salaire))
print(mediane(salaire1))
\end{lstlisting}
        \captionof{code}{Appel de la fonction}
        \label{CODE}
        \end{center}
    

\end{frame}

\section{Exercice 5}
\begin{frame}[fragile]
    \frametitle{Exercice 5}

\begin{lstlisting}[language=Python , basicstyle=\ttfamily\small, xleftmargin=2em, xrightmargin=2em]
def ecart_max(t: list) -> int:
    maxi = 0
    # attention à ne par sortir du tableau
    for i in range(len(t) - 1):
        ecart = t[i+1]-t[i]
        if ecart > maxi:
            maxi = ecart
    return maxi
\end{lstlisting}

\begin{aretenir}[Observation]
Il faut remarquer le générateur \textbf{\texttt{range(len(t) - 1)}}. La boucle s'arrête à l'avant-dernier élément du tableau. Ceci permet de ne pas dépasser la taille du tableau dans l'appel \textbf{\texttt{t[i+1]}}.
\end{aretenir}
\end{frame}
\section{Exercice 6}
\begin{frame}[fragile]
    \frametitle{Exercice 6}

\begin{lstlisting}[language=Python , basicstyle=\ttfamily\small, xleftmargin=2em, xrightmargin=2em]
def somme(t1: list, t2: list) -> list:
    tab = [0, 0, 0, 0, 0]
    for i in range(5):
        tab[i] = t1[i]+t2[i]
    return tab
\end{lstlisting}

\begin{center}
    \begin{lstlisting}[language=Python , basicstyle=\ttfamily\small, xleftmargin=2em, xrightmargin=2em]
t1 = [12, 17, 8, 10, 13]
t2 = [4, 18, 9, 11, 23]
print(somme(t1, t2))
\end{lstlisting}
    \captionof{code}{Appel de la fonction}
    \label{CODE}
    \end{center}
\end{frame}
\section{Exercice 7}
\begin{frame}[fragile]
    \frametitle{Exercice 7}

\begin{lstlisting}[language=Python , basicstyle=\ttfamily\small, xleftmargin=2em, xrightmargin=2em]
bibliotheque = [
    {"titre": "Il était deux fois",
        "auteur": "Franck Thilliez",
        "editeur": "Poche",
        "prix": 8.70},
    {"titre": "Fahrenheit 451",
        "auteur": "Ray Bradbury",
        "editeur": "Folio",
        "prix": 6.30}
]
\end{lstlisting}

\begin{center}
    \begin{lstlisting}[language=Python , basicstyle=\ttfamily\small, xleftmargin=2em, xrightmargin=2em]
for livre in bibliotheque:
    # livre contient un dictionnaire
    print(livre["auteur"])
\end{lstlisting}
    \captionof{code}{Le tableau \textbf{\texttt{bibliotheque}} contient des dictionnaires.}
    \label{CODE}
    \end{center}
\end{frame}
\section{Exercice 8}
\begin{frame}[fragile]
    \frametitle{Exercice 8}

\begin{lstlisting}[language=Python , basicstyle=\ttfamily\small, xleftmargin=2em, xrightmargin=2em]
groupe = [
    {"prenom": "Alice",
        "nom": "Durant",
        "notes": [12, 8, 10, 9.5]},
    {"prenom": "Bob",
        "nom": "Bois",
        "notes": [15, 17, 18, 14]},
    {"prenom": "John",
        "nom": "Doe",
        "notes": [10.5, 8, 16, 13.5]},
    {"prenom": "Jules",
        "nom": "Dupont",
        "notes": [12, 9, 17.5, 10]},
    {"prenom": "Alan",
        "nom": "Turing",
        "notes": [14, 18, 16, 19]},
]
\end{lstlisting}

\begin{center}
    \begin{lstlisting}[language=Python , basicstyle=\ttfamily\small, xleftmargin=2em, xrightmargin=2em]
for livre in bibliotheque:
    # livre contient un dictionnaire
    print(livre["auteur"])
\end{lstlisting}
    \captionof{code}{Le tableau \textbf{\texttt{bibliotheque}} contient des dictionnaires.}
    \label{CODE}
    \end{center}
\end{frame}
\begin{frame}[fragile]
    \frametitle{}

\begin{lstlisting}[language=Python , basicstyle=\ttfamily\small, xleftmargin=2em, xrightmargin=2em]
def moyenne(eleve: dict) -> float:
    """
    calcule la moyenne de l'élève
    """
    somme = 0
    for n in eleve["notes"]:
        somme += n
    
    nb_notes = len(eleve["notes"])
    return somme/nb_notes
\end{lstlisting}

\end{frame}
\begin{frame}[fragile]
    \frametitle{}

\begin{center}
\begin{lstlisting}[language=Python , basicstyle=\ttfamily\small, xleftmargin=2em, xrightmargin=2em]
# moyenne générale
somme_moy = 0
for eleve in groupe:
    # calcule la moyenne de chaque élève
    somme_moy += moyenne(eleve)

moyenne_generale = somme_moy/len(groupe)
print(moyenne_generale)
\end{lstlisting}
\captionof{code}{Calcul de la moyenne générale}
\label{CODE}
\end{center}    

\end{frame}
\end{document}