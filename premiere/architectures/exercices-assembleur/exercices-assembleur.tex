\documentclass[a4paper,11pt]{article}
\input{/home/tof/Documents/Cozy/latex-include/preambule_doc.tex}
\input{/home/tof/Documents/Cozy/latex-include/preambule_commun.tex}
\newcommand{\showprof}{show them}  % comment this line if you don't want to see todo environment
\setlength{\fboxrule}{0.8pt}
\fancyhead[L]{\fbox{\Large{\textbf{ArchMat 03}}}}
\fancyhead[C]{\textbf{Exercices assembleur}}
\newdate{madate}{10}{09}{2020}
%\fancyhead[R]{\displaydate{madate}} %\today
\fancyhead[R]{Première - NSI}
\fancyfoot[L]{\vspace{1mm}Christophe Viroulaud}
\AtEndDocument{\label{lastpage}}
\fancyfoot[C]{\textbf{Page \thepage/\pageref{lastpage}}}
\fancyfoot[R]{\includegraphics[width=2cm,align=t]{/home/tof/Documents/Cozy/latex-include/cc.png}}
%DODO revoir exo 2: init x, y et z au départ
\begin{document}
\begin{exo}
Écrire un programme qui:
\begin{itemize}
    \item stocke deux entiers dans la mémoire vive avec les labels \textbf{\texttt{val1}} et \textbf{\texttt{val2}},
    \item échange ses deux valeurs.
\end{itemize}
\end{exo}
\begin{exo}
    Traduire les instructions suivantes en assembleur:
    \begin{lstlisting}[caption=Un code en Python]
x = y + 42
if x == 42:
    z = 1
else:
    z = 2
\end{lstlisting}
\end{exo}
\begin{exo}
\begin{enumerate}
    \item Écrire puis exécuter le programme \ref{boucle}.
    \lstinputlisting[caption=Réaliser une boucle,label=boucle]{scripts/boucle.txt}
    \item Certaines notions non encore abordées sont présentes dans ce programme:
    \begin{itemize}
        \item Il répète un code plusieurs fois.
        \item Le label \textbf{\texttt{tab}} peut être assimilé à un tableau de données.
    \end{itemize} 
    Prendre le temps de comprendre le fonctionnement du programme. Observer notamment le contenu de la mémoire vive.
    \item Modifier le programme pour qu'il recherche et affiche dans la console la plus grande valeur de \textbf{\texttt{tab}} (on considère que le tableau ne contient que des entiers positifs ou nuls).
\end{enumerate}
\end{exo}
\begin{exo}
    \begin{enumerate}
        \item Que fait le programme \ref{prog1}?
              \lstinputlisting[firstline=2,label=prog1,caption=Que fait ce programme?]{scripts/somme-entier.txt}
        \item Les microprocesseurs Intel utilisent le langage d'assemblage x86. Ce langage a huit registres: eax, ebx, ecx, edx, esi, edi, esp et ebp. La documentation (très incomplète) ci-dessous présente plusieurs instructions de ce langage:
              \begin{center}
                  \begin{tabular}{|*{3}{>{\centering\arraybackslash}m{.3\textwidth}|}}
                      \hline
                      Instructions            & Utilisation                                                                                                                                                                                                                                                                              & Exemple      \\
                      \hline
                      mov destination, source & copie la source dans la destination                                                                                                                                                                                                                                                      & mov eax, 42  \\
                      \hline
                      add destination, source & ajoute la source à la destination                                                                                                                                                                                                                                                        & add eax, 10  \\
                      \hline
                      sub destination, source & soustraie la source à la destination                                                                                                                                                                                                                                                     & sub eax, 10  \\
                      \hline
                      cmp source1, source2    & effectue la soustraction source1-source2 et indique le résultat de comparaison dans des registres spéciaux (flag). Si le flag \emph{zf} vaut 1 c'est que \emph{source1} est égal à \emph{source2}. Si le flag \emph{sf} vaut 1 c'est que \emph{source1} est inférieure à \emph{source2}. & cmp eax, ebx \\
                      \hline
                      je adr                  & saute à l'adresse si le flag \emph{zf} vaut 1. L'adresse est une étiquette ou label.                                                                                                                                                                                                     & je label     \\
                      \hline
                      jne adr                 & saute à l'adresse si \emph{zf} vaut 0. L'adresse est une étiquette ou label.                                                                                                                                                                                                             & jne label    \\
                      \hline
                      jmp adr                 & saute à l'adresse. L'adresse est une étiquette ou label.                                                                                                                                                                                                                                 & jmp label    \\
                      \hline
                  \end{tabular}
              \end{center}
              Traduire le code \ref{prog1} dans le langage assembleur x86.
    \end{enumerate}
\end{exo}

\end{document}